\chapter{Wellen in einem Kanal\label{chapter:wellen}}
\lhead{Wellen in einem Kanal}
\begin{refsection}
\chapterauthor{Daniela Meier und Hansruedi Patzen}

Das Kapitel \ref{chapter:wellen} Wellen, besch"aftigt sich mit Wellen in einem 
parabolischen Kanal. Die dazugeh"orende Gleichung \ref{wellen:grundgleichung}, 
mit diversen Konfigurationen f"ur die Konstanten ${a,b,c} \in \mathbb{R}$, soll 
dazu n"aher betrachtet werden.

\begin{equation}
	y'' + (ax^2+bx+c)y
	=
	0
	\label{wellen:grundgleichung}
\end{equation}

Als ersten L"osungsansatz wird der im Kapitel \ref{chapter:potenzreihen} 
kennengelernte Ansatz \ref{potenzreihen:ansatz} verwendet.

Zuerst werden die Potenzreihen f"ur $y$ und $y''$ aufgestellt.

\begin{equation}
	y(x)
	=
	a_0 + a_1x + a_2x^2 + a_3x^3 + a_4x^4 + a_5x^5 + a_6x^6 + \dotsb
	\label{wellen:ersteableitung}
\end{equation}

\begin{equation}
	y''(x)
	=
	2a_2 + 3 \mathbin{\cdot} 2a_3x + 4 \mathbin{\cdot} 3a_4x^2 + 5 
	\mathbin{\cdot} 4a_5x^3 + 6 \mathbin{\cdot} 5a_6x^4 + \dotsb
	\label{wellen:zweiteableitung}
\end{equation}

Aus den beiden Gleichungen \ref{wellen:ersteableitung} und 
\ref{wellen:zweiteableitung} kann nun der Koeffizientenvergleich 
\ref{wellen:koeffizietenvergleich} erstellt werden.

\begin{equation}
	\begin{split}
		y''
		&=
		-(ax^2+bx+c)y \\
		2a_2 + 3 \mathbin{\cdot} 2a_3x + 4 \mathbin{\cdot} 3a_4x^2 + \dotsb
		&=
		-(ax^2+bx+c)(a_0 + a_1x + a_2x^2 + a_3x^3 + a_4x^4 + \dotsb)
	\end{split}
	\label{wellen:koeffizietenvergleich}
\end{equation}

\begin{equation}
	a_{k+2} = -\frac{1}{(k+2)(k+1)} (aa_{k-2}+ba_{k-1}+ca_k)
	\hspace{20pt}, a_{-2} = a_{-1} = 0
	\label{wellen:koeffizientengleichung}
\end{equation}

\begin{equation}
	y(x) = a_0 + a_1x 
	-\sum_{k=0}^{\infty}\frac{1}{(k+2)(k+1)}(aa_{k-2}+ba_{k-1}+ca_k)
	\hspace{20pt}, a_{-2} = a_{-1} = 0
	\label{wellen:ygleichung}
\end{equation}

\printbibliography[heading=subbibliography]
\end{refsection}