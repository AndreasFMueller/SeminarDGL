\chapter{Wellen in einem Kanal\label{chapter:wellen}}
\lhead{Wellen in einem Kanal}
\begin{refsection}
\chapterauthor{Daniela Meier und Hansruedi Patzen}

\begin{equation}
	y'' + (ax^2+bx+c)y = 0
	\label{wellen:grundgleichung}
\end{equation}

Dieses Kapitel besch"aftigt sich mit Ausbreitung von Wellen in einem 
parabolischen Kanal. Die Ausbreitung wird mittels Analyse der Wellengleichung 
n"aher betrachtet. In einem ersten Schritt wird erkl"art, wie die 
Gleichung \ref{wellen:grundgleichung} aus der Aufgabenstellung zustandegekommen 
ist.

Die allgemeine Differentialgleichung der Welle lautet (gem"ass 
\cite{wellen:smirnow2}):

\begin{equation*}
	\frac{\partial^2 u}{\partial t^2}
	=
	c^2
	\left(
		\frac{\partial^2 u}{\partial x^2} 
		+ \frac{\partial^2 u}{\partial y^2} 
		+ \frac{\partial^2 u}{\partial z^2}
	\right)
	\label{wellen:allgemeineDGL}
\end{equation*}

Da in der ausgew"ahlten Problemstellung nur eine zweidimensionale Welle 
betrachtet wird, fallen die y- und die z-Koordinate weg. Die Gleichung 
vereinfacht sich dadurch zu:

\begin{equation}
	\frac{\partial^2 u}{\partial t^2}
	=
	c^2
	\left(
		\frac{\partial^2 u}{\partial x^2} 
	\right)
	\label{wellen:allgemeineDGLvereinfacht}
\end{equation}

Somit ist die Funktion $u$ noch abh"angig von der Zeit $t$ und der 
Ortskoordinate 
$x$.

\begin{equation*}
	u = f(x,t)
\end{equation*}

Zur L"osung der vereinfachten Wellengleichung 
\ref{wellen:allgemeineDGLvereinfacht} wird die Theorie des Separationsansatzes 
verwendet. Gem"ass dem kann $u(x, t)$ in ein Produkt zweier Funktionen $X(x)$ 
und $T(t)$ umgewandelt werden.

\begin{equation}
	u (x,t) = X(x) T(t)
	\label{wellen:separierteFunktion}
\end{equation}

Ausgehend von der vereinfachten Grundgleichung der Welle 
\ref{wellen:allgemeineDGLvereinfacht} wird die Funktion 
\ref{wellen:separierteFunktion} auf der linken Seite zweimal partiell nach $t$ 
und auf der rechten Seite zweimal partiell nach $x$ abgeleitet. Daraus 
resultiert:

\begin{equation}
	T''(t) X(x) = c^2 X''(x)T(t)
\end{equation}

Diese Differentialgleichung wird nun durch $T(x)X(x)$ dividiert und somit 
werden die Variablen separiert und ein Term entsteht, welcher nur funktionieren 
kann, wenn beide Seiten der Gleichung konstant sind. 

\begin{equation*}
	\frac{T''(t)}{T(t)}
	=
	c^2 \frac{X''(x)}{X(x)} = \text{konstant} = \lambda
\end{equation*}

Aus dieser Gleichung entstehen nun zwei l"osbare Differentialgleichungen 
zweiter Ordnung.

\begin{equation*}
	T''= \lambda T(t)
\end{equation*}

und

\begin{equation}
	X'' = \frac{\lambda}{c^2}X(x)
	\label{wellen:DGLzweiterOrdnung}
\end{equation}

Bei der betrachteten Problemstellung, wird davon ausgegangen, dass die 
Zeit konstant ist. Aus diesem Grund wird ab hier nur weiter auf die Gleichung 
der Funktion $X(x)$ eingegangen. Wird die Gleichung 
\ref{wellen:DGLzweiterOrdnung} gleich Null gesetzt, ergibt sich:

\begin{equation*}
	X''(x) - \frac{\lambda}{c^2} X(x) = 0
\end{equation*}

Diese Gleichung widerspiegelt die Wellengleichung, die in der Fortsetzung 
betrachtet wird.

Mit dem Faktor $\frac{\lambda}{c^2}$ k"onnen verschiedene 
Geschwindigkeitsprofile betrachtet werden. Hier wird zuerst auf den speziellen 
Fall eines parabolischen Profils eingegangen. Die zu untersuchende 
Differentialgleichung ergibt sich hiermit zu der am Anfang definierten 
Gleichung \ref{wellen:grundgleichung}, mit den frei w"ahlbaren Variablen 
${a,b,c} \in \mathbb{R}$, sowie den Anfangsbedingungen $y(0) = a_0$ und $y'(0) 
= a_1$.

\section{Potenzreihenl"osung}
Als L"osungsansatz wird der im Kapitel \ref{chapter:potenzreihen} 
kennengelernte Potenzreihenansatz (\ref{potenzreihen:ansatz}) verwendet.

Zuerst werden die Potenzreihen f"ur $y$ und $y''$ aufgestellt.

\begin{equation}
	y(x)
	=
	\sum_{k = 0}^{\infty} a_{k}x^k
	=
	a_0 + a_1x + a_2x^2 + a_3x^3 + a_4x^4 + a_5x^5 + a_6x^6 + \dotsb
	\label{wellen:nullteableitung}
\end{equation}

\begin{equation*}
	y'(x)
	=
	\sum_{k=0}^{\infty} a_{k+1}(k+1)x^k
	=
	a_1 + 2a_2x + 3a_3x^2 + 4a_4x^3 + 5a_5x^4 + 6a_6x^5+ \dotsb
\end{equation*}

\begin{equation}
	y''(x)
	=
	\sum_{k = 0}^{\infty} a_{k+2}(k+1)(k+2)x^k
	=
	2a_2 + 3 \mathbin{\cdot} 2a_3x + 4 \mathbin{\cdot} 3a_4x^2 + 5 
	\mathbin{\cdot} 4a_5x^3 + 6 \mathbin{\cdot} 5a_6x^4 + \dotsb
	\label{wellen:zweiteableitung}
\end{equation}

Aus den beiden Gleichungen \ref{wellen:nullteableitung} und 
\ref{wellen:zweiteableitung} kann nun der Koeffizientenvergleich 
\ref{wellen:koeffizietenvergleich} erstellt werden.

\begin{equation}
	\begin{split}
		y''
		&=
		-(ax^2+bx+c)y \\
		2a_2 + 3 \mathbin{\cdot} 2a_3x + 4 \mathbin{\cdot} 3a_4x^2 + \dotsb
		&=
		-(ax^2+bx+c)(a_0 + a_1x + a_2x^2 + a_3x^3 + a_4x^4 + \dotsb) \\
		2a_2 + 3 \mathbin{\cdot} 2a_3x + 4 \mathbin{\cdot} 3a_4x^2 + \dotsb
		&=
		-aa_0x^2-aa_1x^3-aa_2x^4-\dotsb \\
		&\hspace{10pt}
		-ba_0x-ba_1x^2-ba_2^3-ba_3x^4-\dotsb \\
		&\hspace{10pt}
		-ca_0-ca_1x-ca_2x^2-ca_3x^3-ca_4x^4 - \dotsb
	\end{split}
	\label{wellen:koeffizietenvergleich}
\end{equation}

Daraus k"onnen nun die Resultate f"ur die verschiedenen $a_k$ bestimmt werden.

\begin{equation}
	\begin{split}
		a_2
		&=
		-\frac{1}{2}ca_0 \\
		a_3
		&=
		-\frac{1}{3 \cdot 2} (ba_0 + ca_1) \\
		a_4
		&=
		-\frac{1}{4 \cdot 3} (aa_0 + ba_1 + ca_2) \\
		&=
		-\frac{1}{4 \cdot 3} (aa_0 + ba_1 -\frac{1}{2}c^2a_0) \\
		a_5
		&=
		-\frac{1}{5 \cdot 4} (aa_1 + ba_2 + ca_3) \\
		&=
		-\frac{1}{5 \cdot 4} (aa_1 -\frac{1}{2}bca_0 -\frac{1}{3 \cdot 2} 
		c(ba_0 + ca_1)) \\
		&\hspace{5pt}\vdots
	\end{split}
	\label{wellen:aks}
\end{equation}

Hieraus l"asst sich nun eine allgemeine Rekursionsformel 
\ref{wellen:koeffizientengleichung} f"ur die verschiedenen $a_k$ herauslesen. 
Es gilt: $k \in \mathbb{N}$ und	$a_{k < 0} = 0$

\begin{equation}
	a_{k+2} = -\frac{1}{(k+2)(k+1)} (aa_{k-2}+ba_{k-1}+ca_k)
	\label{wellen:koeffizientengleichung}
\end{equation}

Nun kann man die L"osungsgleichung \ref{wellen:ygleichung} f"ur $y(x)$ 
aufstellen.

\begin{equation}
	y(x) = a_0 + a_1x 
	-\sum_{k=0}^{\infty}\frac{1}{(k+2)(k+1)}(aa_{k-2}+ba_{k-1}+ca_k)x^{k+2}
	\label{wellen:ygleichung}
\end{equation}

Um das Programmieren dieser Formel zu vereinfachen, formt man sie am besten von 
$a_{k+2}$ nach $a_k$ um. Damit erh"alt man die Gleichung 
\ref{wellen:koeffizientengleichungak}.
Bei der neuen $y(x)$-Gleichung \ref{wellen:ygleichungak} muss nun aber $k \in 
\mathbb{N} \backslash \{0, 1\}$ gelten.

\begin{equation}
	a_{k} = -\frac{1}{k(k-1)} (aa_{k-4}+ba_{k-3}+ca_{k-2})
	\label{wellen:koeffizientengleichungak}
\end{equation}

\begin{equation}
	y(x) = a_0 + a_1x 
	-\sum_{k=2}^{\infty}\frac{1}{k(k-1)}(aa_{k-4}+ba_{k-3}+ca_{k-2})x^k
	\label{wellen:ygleichungak}
\end{equation}

\section{Erkenntnisse bei der Variation von \texorpdfstring{$k_{max}$}{kmax}}

Es wird davon ausgegangen, dass sich mit der Variation von $k_{max}$ bei der 
Welle keine grossen Ver"anderungen einstellen. Die Werte werden immer ab einem 
gewissen Punkt ins Positive oder Negative explodieren, da das $x^k$ irgendwann 
dominiert. Mit $k_{max}$ wird die Anzahl der Summanden bezeichnet, die f"ur die 
Berechnung der Potenzreihe verwendet werden. 

\begin{equation*}
y(x) = a_0 + a_1x 
-\sum_{k=2}^{k_{max}}\frac{1}{k(k-1)}(aa_{k-4}+ba_{k-3}+ca_{k-2})x^k
\end{equation*}

Es werden die Startbedingungen gem"ass den Gleichungen 
\ref{wellen:Startbedingungen1} festgelegt.

\begin{equation}
	\begin{split}
		a_0 &= 1\\
		a_1 &= 0\\
		a &= 1\\
		b &= 0\\
		c &= -1
	\end{split}
	\label{wellen:Startbedingungen1}
\end{equation}

Auf folgenden Grafiken wird veranschaulicht, was mit der Welle passiert, wenn 
das $k_{max}$ jeweils um 30 vergr"ossert wird. Es wird dann "uberpr"uft, ob 
sich die Behauptung der kleinen Ver"anderungen als wahr herausstellt.

\noindent
\includegraphics[scale=0.35]{./wellen/octave/images/kmax/ak30wave.png}
\includegraphics[scale=0.35]{./wellen/octave/images/kmax/ak60wave.png}
\includegraphics[scale=0.35]{./wellen/octave/images/kmax/ak90wave.png}
\includegraphics[scale=0.35]{./wellen/octave/images/kmax/ak120wave.png}
\includegraphics[scale=0.35]{./wellen/octave/images/kmax/ak150wave.png}
\includegraphics[scale=0.35]{./wellen/octave/images/kmax/ak180wave.png}
\includegraphics[scale=0.35]{./wellen/octave/images/kmax/ak210wave.png}
\includegraphics[scale=0.35]{./wellen/octave/images/kmax/ak300wave.png}

Die Behauptung, dass der Wert irgendwann explodiert, ist, wie ersichtlich, 
wahr. Der Punkt wo die Werte gegen $\infty$ gehen verschiebt sich immer mehr 
nach rechts, zu gr"osseren $x$-Werten. Dies wird dadurch erkl"art, dass es 
immer l"anger geht, bis der $x^k$-Term dominiert. Je mehr der $k_{max}$-Wert 
gesteigert wird, umso besser wird die Approximation an die Welle. Ab einer 
gewissen Gr"osse von $k_{max}$ kann das Programm die Werte nicht mehr genau 
bestimmen, was die unregelm"assigen Ausschl"age in den letzten zwei Grafiken 
erkl"art.

Aufgrund dieser Messungen wird $x$ f"ur die weiteren Berechnungen auf $x \in 
[-8;8]$ und $k_{max}$ auf 180 beschr"ankt, damit die $y$-Werte nicht 
explodieren und keine unkontrollierten Messwerte entstehen.

Mit den in diesem Kapitel festgelegten Startbedingungen und Einschr"ankungen 
ergibt dies folgende Grafik:
\begin{center}
	\includegraphics[scale=0.5]{./wellen/octave/images/kmax/ak180-88wave.png}
\end{center}

\section{Erkenntnisse bei der Variation von \texorpdfstring{$a$}{a}, 
\texorpdfstring{$b$}{b}, \texorpdfstring{$c$}{c}, \texorpdfstring{$a_0$}{a0} 
und \texorpdfstring{$a_1$}{a1}}

Die Anfangsbedingungen $a_0$ und $a_1$ k"onnen beliebig festgelegt werden. Da 
sich die Wellen im betrachteten Fall um den Entwicklungspunkt $x_0=0$ bewegen, 
beschreibt das $a_0$ den y-Achsenabschnitt und das $a_1$ die Steigung im Punkt 
$y(0)$. Die gew"ahlten Werte von $a_0$ und $a_1$ sind jeweils in der Grafik 
ersichtlich.

Zu Beginn wird das vorgegebene, parabelf"ormige Geschwindigkeitsprofil in eine 
Konstante umgewandelt indem $a$ und $b$ gleich $0$ gesetzt werden. Folgende 
Gleichung wird also betrachtet:

\begin{equation*}
	y''+ cy = 0
\end{equation*}

oder

\begin{equation}
	y'' = -cy
	\label{wellen:lineareDGL}
\end{equation}

F"ur positive Werte von c ist die L"osung dieser Gleichung 
\ref{wellen:lineareDGL} bekanntlicherweise $C_1sin(\sqrt{c}x) + 
C_2cos(\sqrt{c}x)$. Die Konstanten $C_1$ und $C_2$ stellen sich ein, je nach 
dem, wie die Anfangsbedingungen $a_0$ und $a_1$ bestimmt werden. Als Beweis 
dient folgende Grafik:


F"ur negative Werte von c hat die Gleichung

\section{Graphische L"osung}

Hier sieht man zuerst die Ursprungsparabel $x^2-1$, danach die daraus 
entstehende Welle und schlussendlich noch eine "Uberlagerung der Parabel und 
der Welle.

\includegraphics[scale=0.6]{./wellen/octave/images/a01a10/parabola.png}

\includegraphics[scale=0.6]{./wellen/octave/images/a01a10/wave.png}

\includegraphics[scale=0.6]{./wellen/octave/images/a01a10/wavewithparabola.png}

\printbibliography[heading=subbibliography]
\end{refsection}