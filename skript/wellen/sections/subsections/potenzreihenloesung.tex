\subsection{\texorpdfstring{$y(x) = \sum_{k = 0}^{\infty} a_{k}x^k$}{y(x) = 
summe k = 0 bis unendlich ak xk}}
\label{subsec:wellen:Potenzreihenansatz}

Als n"achstes gehen wir zur"uck auf das parabolische Kanalprofil und suchen 
L"osungen f"ur die Differentialgleichung. Dazu wird der im Kapitel 
\ref{chapter:potenzreihen} kennengelernte Potenzreihenansatz angewandt.
Zuerst wird die Potenzreihe f"ur $y(x)$ aufgestellt und zweimal abgeleitet, da 
in der Gleichung (\ref{eq:wellen:pxdgl}) die zweite Ableitung vorkommt.
\begin{align}
	y(x)
	&=
	\sum_{k = 0}^{\infty} a_{k}x^k
	=
	a_0 + a_1x + a_2x^2 + a_3x^3 + a_4x^4 + a_5x^5 + a_6x^6 + \dotsb
	\label{eq:wellen:nullteableitung}
	\\
	y'(x)
	&=
	\sum_{k=0}^{\infty} a_{k+1}(k+1)x^k
	=
	a_1 + 2a_2x + 3a_3x^2 + 4a_4x^3 + 5a_5x^4 + 6a_6x^5+ \dotsb
	\\
	y''(x)
	&=
	\sum_{k = 0}^{\infty} a_{k+2}(k+1)(k+2)x^k
	=
	2a_2 + 3 \mathbin{\cdot} 2a_3x + 4 \mathbin{\cdot} 3a_4x^2 + 5 
	\mathbin{\cdot} 4a_5x^3 + 6 \mathbin{\cdot} 5a_6x^4 + \dotsb
	\label{eq:wellen:zweiteableitung}
\end{align}
Aus den beiden Gleichungen (\ref{eq:wellen:nullteableitung}) und
(\ref{eq:wellen:zweiteableitung}) und dem Umstellen der Anfangsgleichung in die 
Form
\begin{equation*}
	y'' = -(ax^2+bx+c)y
\end{equation*}
kann der Koeffizientenvergleich erstellt und L"osungen f"ur die $a_k$ gefunden 
werden. Statt die Koeffizienten reihenweise aufzustellen, wird der Vergleich 
mit Hilfe der Tabelle \ref{tab:wellen:koeffizietenvergleichtabelle} gemacht. 
Dies schafft auch bei komplizierten und langen Termen eine gute "Ubersicht und 
man erkennt auf den ersten Blick, welche $a_k$ von welchen anderen abh"angen.
\begin{table}
	\centering
	\begin{equation*}
		\begin{array}{r c r | c r | c r | c r c}
		y(x) & = &
		a_0 & + & a_1x & + & a_2x^2 & + \dotsb
		\\
		\hline&&&&&&&\\[-2ex]
		y''(x) & = &
		2\cdot1 a_2 & + & 3\cdot2 a_3x & + & 4\cdot3 a_4x^2 & + \dotsb
		\\
		& = &
		& & & + &- aa_0x^2 & + \dotsb
		\\
		& &
		& + &- ba_0x & + &- ba_1x^2 & + \dotsb
		\\
		& + &
		-ca_0 & + &- ca_1x & + &- ca_2x^2 & + \dotsb
		\\
		\hline&&&&&&&\\[-2ex]
		& &
		a_2 = -\frac{1}{2 \cdot 1}ca_0
		& & a_3 = -\frac{1}{3 \cdot 2}(ba_0+ca_1)
		& & a_4 = -\frac{1}{4 \cdot 3}(aa_0+ba_1+ca_2)
		& \dots
		\end{array}
	\end{equation*}
	\caption{Koeffizientenvergleich mittels Hilfstabelle.}
	\label{tab:wellen:koeffizietenvergleichtabelle}
\end{table}

Die Verschiebung der verschiedenen $a_k$ k"onnen folgendermassen erkl"art 
werden: Aufgrund der zweiten Ableitung werden die $a_k$ der zweiten Zeile um 
zwei Einheiten nach links geschoben. Die n"achste Verschiebung gibt es aufgrund 
des Polynomgrades. Da die Terme mit den Koeffizienten $a$ mindestens die Form 
$ax^2$ haben, verschieben sich diese um zwei Potenzen nach rechts. Die Terme 
mit den Koeffizienten $b$ haben mindestens die Form $bx^1$, daher die 
Verschiebung dieser um eine Potenz nach rechts. Einzig die Terme mit den 
Koeffizienten $c$ werden nicht verschoben, da sie mindestens als $cx^0$ 
vorhanden sind. Somit lassen sich die jeweiligen $a_k$ und ihre Abh"angigkeiten 
direkt aus den einzelnen Spalten ablesen.

Aus den in der Tabelle berechneten $a_k$ l"asst sich eine allgemeine 
Rekursionsformel
\index{Rekursionsformel}%
\begin{align*}
	a_{k+2} &= -\frac{1}{(k+2)(k+1)} (aa_{k-2}+ba_{k-1}+ca_k) \\
	\Leftrightarrow \qquad
	a_k &= -\frac{1}{k(k-1)} (aa_{k-4}+ba_{k-3}+ca_{k-2}), \qquad k \in 
	\mathbb{N} \setminus \{0, 1\}
	&&&a_{k<0} &= 0
\end{align*}
aufstellen.
