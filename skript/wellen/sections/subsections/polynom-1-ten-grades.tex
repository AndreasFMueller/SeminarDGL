\subsection{\texorpdfstring{$y''-xy = 0$}{y''-xy = 0}}
Die Koeffizienten der uns bereits bekannten Airy-Differentialgleichung
(Aufgaben~\ref{401} und \ref{402}, und das Beispiel auf
Seite~\pageref{komplex:airydgl})
k"onnen 
durch einfaches Einsetzen in die Formel (\ref{eq:wellen:allgemeineak}) 
berechnet werden:
\begin{equation*}
	\begin{split}
		a_k &= -\frac{1}{k(k-1)} ((-1) a_{k-2-1} + 
		0 a_{k-2-0})
		\\
		&= \frac{1}{k(k-1)} a_{k-3}, \qquad a_{k < 0} = 0.
	\end{split}
\end{equation*}
Diese Rekursionsformel deckt sich mit den bereits bekannten L"osungen.
Auch die genannten Konsequenzen werden in der Abbildung 
\ref{fig:wellen:airy-dgl} deutlich aufgezeigt. So "andert sich die Form von 
einer Schwingung in eine Exponentialfunktion, und zwar exakt bei der Nullstelle 
der Geraden.

%\begin{figure}
%	\includegraphics[width=1\hsize]{./wellen/images/allgemein/n1.pdf}
%	\caption{L"osung Airy-Differentialgleichung}
%	\label{fig:wellen:airy-dgl}
%\end{figure}
