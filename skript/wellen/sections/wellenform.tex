\section{Diskussion der Wellenform}
\label{sec:wellen:diskussionwellenform}

Wie schon im Kapitel \ref{sec:wellen:parametervariation} ersichtlich, 
wechselt die Welle ihre Form im bei den jeweiligen Nullstellen der Parabel.

Es stellt sich heraus, dass die L"osung der Gleichung 
\ref{eq:wellen:grundgleichung} eng mit der gefundenen L"osung der Gleichung 
\ref{eq:wellen:lineareDGL} verwandt ist. So k"onnen die L"osungen der Parabel 
als verschiedene $c$ Werte der linearen Differentialgleichung verstanden 
werden, welche dann in die L"osungsgleichung \ref{eq:wellen:loesunglinearedgl} 
eingesetzt werden k"onnen. Daraus ergibt es also f"ur negative Parabell"osungen 
eine kombination aus den hyperbolischen Funktionen $\sinh$ und $\cosh$ und f"ur 
positive eine Kombination aus $\sin$ und $\cos$.
