\section{Diskussion der Wellenform}
\label{sec:wellen:DiskussionWellenform}

Wie schon in den vorhergehenden Kapiteln ersichtlich war, wechselt die Welle 
ihre Form im ``Umschlagspunkt'', welcher die Schnittpunkte der Parabel mit der 
$x$-Achse bezeichnet.


Grunds"atzlich gibt sechs untschiedliche F"alle wie die Parabel liegen kann:
\begin{itemize}
	\item Der Schnittpunkt der Parabel mit der $y$-Achse ist im negativen 
	Bereich und die Parabel nach oben ge"offnet
	\item Der Schnittpunkt der Parabel mit der $y$-Achse ist im positiven 
	Bereich und die Parabel ist nach oben ge"offnet
	\item Der Schnittpunkt der Parabel mit der $y$-Achse liegt im Nullpunkt und 
	die Parabel ist nach oben ge"offnet
	\item Der Schnittpunkt der Parabel mit der $y$-Achse ist im negativen 
	Bereich und die Parabel nach unten ge"offnet
	\item Der Schnittpunkt der Parabel mit der $y$-Achse ist im positiven 
	Bereich und die Parabel nach unten ge"offnet
	\item Der Schnittpunkt der Parabel mit der $y$-Achse liegt im Nullpunkt und 
	die Parabel ist nach unten ge"offnet
\end{itemize}

Zu Beginn wird der erste Fall betrachtet, welcher beispielsweise die Parabel 
auf der Grafik im Kapitel \ref{subsec:wellen:Variationc} beschreibt. Diese 
Parabel hat 
zwei Schnittpunkte mit der $x$-Achse und der Scheitelpunkt liegt im negativen 
$y$-Bereich. Dies bedeutet, dass alle L"osungen der Parabel von $x=0$ bis zum 
Schnittpunkt der Parabel mit der $x$-Achse negativ sind. 

Wenn nun in der Differentialgleichung f"ur die Parabel ein beliebiger negativer 
Wert eingesetzt wird, resultiert dieselbe Gleichung 
\ref{eq:wellen:LSGnegcWerte} 
wie im Kapitel \ref{subsec:wellen:Eliminierungab}. Durch das, dass die Parabel 
durch 
einen Wert ersetzt wird, ist die Aufgabenstellung die Gleiche, wie wenn man $a$ 
und $b$ eliminieren und f"ur $c$ einen negativen Wert einsetzen w"urde. In 
diesem Bereich entsteht demzufolge gem"ass Gleichung 
\ref{eq:wellen:LSGhyperbolFunktion} eine hyperbolische Funktion. Welche von den 
zwei hyperbolischen Funktionen entsteht, l"asst sich mit verschiedenen 
Anfangsbedingungen $a_0$ und $a_1$ bestimmen wie man in den folgenden Grafiken 
sieht.

GRAFIK SINH COSH

Beim Schnittpunkt der Parabel mit der $x$-Achse "andert sich das Vorzeichen der 
L"osung der Parabel und es entsteht nicht die Gleichung 
\ref{eq:wellen:LSGnegcWerte} wie oben, sondern die Gleichung 
\ref{eq:wellen:lineareDGL}, welche f"ur positive $c$-Werte aufgestellt wurde. 

Daraus wird ersichtlich, dass sich ab diesem Punkt eine L"osung einstellt, 
welche aus einer "Uberlagerung von $\sin$ und $\cos$ ist. Daraus entsteht die 
regelm"assige, kontinuierliche Schwingung. 


Bei all den anderen F"allen kann das Gleiche beobachtet werden. Bei der Parabel 
die nach oben ge"offnet ist, passiert genau das Umgekehrte. Im Innenbereich der 
Parabel entsteht eine Schwingung und sobald die Parabel die $x$-Achse schneidet 
wandelt sich die Schwingung in eine hyperbolische Funktion um. 4

Bei den Parabeln, die die $x$-Achse im Nullpunkt ber"uhren entsteht nur eine 
Schwingung, respektive nur eine hyperbolische Funktion, je nach Wert von a, 
welcher angibt, ob die Parabel nach oben oder nach unten ge"offnet ist. 
Dasselbe geschieht auch, wenn die Parabel die $x$-Achse gar nicht schneidet. 