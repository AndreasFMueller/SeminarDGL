\section{Potenzreihenl"osung}
Als L"osungsansatz wird der im Kapitel \ref{chapter:potenzreihen} 
kennengelernte Potenzreihenansatz verwendet.

Zuerst wird die Potenzreihe f"ur $y(x)$ aufgestellt und zweimal abgeleitet.

\begin{align}
	y(x)
	&=
	\sum_{k = 0}^{\infty} a_{k}x^k
	=
	a_0 + a_1x + a_2x^2 + a_3x^3 + a_4x^4 + a_5x^5 + a_6x^6 + \dotsb
	\label{eq:wellen:nullteableitung}
	\\
	y'(x)
	&=
	\sum_{k=0}^{\infty} a_{k+1}(k+1)x^k
	=
	a_1 + 2a_2x + 3a_3x^2 + 4a_4x^3 + 5a_5x^4 + 6a_6x^5+ \dotsb
	\label{eq:wellen:ersteableitung}
	\\
	y''(x)
	&=
	\sum_{k = 0}^{\infty} a_{k+2}(k+1)(k+2)x^k
	=
	2a_2 + 3 \mathbin{\cdot} 2a_3x + 4 \mathbin{\cdot} 3a_4x^2 + 5 
	\mathbin{\cdot} 4a_5x^3 + 6 \mathbin{\cdot} 5a_6x^4 + \dotsb
	\label{eq:wellen:zweiteableitung}
\end{align}

Aus den beiden Gleichungen \ref{eq:wellen:nullteableitung} und
\ref{eq:wellen:zweiteableitung} und dem Umstellen der Anfangsgleichung in die 
Form
\begin{equation*}
	y'' = -(ax^2+bx+c)y
\end{equation*}
kann nun der Koeffizientenvergleich erstellt werden. 

Statt die Koeffizienten einfach reihenweise aufzustellen, wird der Vergleich 
mit Hilfe einer Tabelle gemacht.

\begin{equation}
	\begin{array}{r c r | c r | c r | c r c}
	y(x) & = &
	a_0 & + & a_1x & + & a_2x^2 & + \dotsb
	\\
	\hline&&&&&&&\\[-2ex]
	y''(x) & = &
	2\cdot1 a_2 & + & 3\cdot2 a_3x & + & 4\cdot3 a_4x^2 & + \dotsb
	\\
	& = &
	& & & + &- aa_0x^2 & + \dotsb
	\\
	& &
	& &- ba_0x & + &- ba_1x^2 & + \dotsb
	\\
	& &
	-ca_0 & + &- ca_1x & + &- ca_2x^2 & + \dotsb
	\\
	\hline&&&&&&&\\[-2ex]
	& &
	a_2 = -\frac{1}{2 \cdot 1}ca_0,
	& & a_3 = -\frac{1}{3 \cdot 2}(ba_0+c_a1),
	& & a_4 = -\frac{1}{4 \cdot 3}(aa_0+ba_1+ca_2),
	& \dots
	\end{array}
	\label{eq:wellen:koeffizietenvergleichtabelle}
\end{equation}

Die Verschiebung der verschiedenen $a_k$ in der Tabelle 
\ref{eq:wellen:koeffizietenvergleichtabelle} kann einfach erkl"art werden. 
Aufgrund der zweiten Ableitung werden die $a_k$ der zweiten Zeile um zwei nach 
links geschoben. Eine weitere Verschiebung gibt es wegen des Polynomgrades.

Da die $a$-Summanden mindestens die Form $ax^2$ haben, verschiebt sich die 
dritte Zeile um zwei nach rechts. Die $b$-Summanden haben mindestens Die Form 
$bx^1$, daher die Verschiebung der vierten Zeile um eins nach rechts. Einzig 
die f"unfte Zeile mit den $c$ wird nicht verschoben, da es mindestens als 
$cx^0$ vorhanden ist. Somit lassen sich die jeweiligen $a_k$ und ihre 
Abh"angigkeiten direkt aus den jeweiligen Spalten ablesen.

Aus den in der Tabelle berechneten $a_k$ l"asst sich nun eine allgemeine 
Rekursionsformel

\begin{equation*}
	\begin{split}
		a_{k+2} &= -\frac{1}{(k+2)(k+1)} (aa_{k-2}+ba_{k-1}+ca_k) \\
		\Leftrightarrow \qquad
		a_k &= -\frac{1}{k(k-1)} (aa_{k-4}+ba_{k-3}+ca_{k-2}), \qquad k \in 
		\mathbb{N} \setminus \{0, 1\}, a_{k<0} = 0
	\end{split}
\end{equation*}
herauslesen. 

Daraus kann nun die L"osungsgleichung
\begin{equation}
	y(x) = a_0 + a_1x 
	-\sum_{k=2}^{\infty}\frac{1}{k(k-1)}(aa_{k-4}+ba_{k-3}+ca_{k-2})x^k
	\label{eq:wellen:ygleichung}
\end{equation}
aufgestellt werden.
