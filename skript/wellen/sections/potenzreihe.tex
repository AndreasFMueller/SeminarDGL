\section{Potenzreihenl"osung}
Als L"osungsansatz wird der im Kapitel \ref{chapter:potenzreihen} 
kennengelernte Potenzreihenansatz (\ref{potenzreihen:ansatz}) verwendet.

Zuerst wird die Potenzreihe f"ur $y$ aufgestellt und zweimal abgeleitet.

\begin{equation}
	y(x)
	=
	\sum_{k = 0}^{\infty} a_{k}x^k
	=
	a_0 + a_1x + a_2x^2 + a_3x^3 + a_4x^4 + a_5x^5 + a_6x^6 + \dotsb
	\label{wellen:nullteableitung}
\end{equation}

\begin{equation*}
	y'(x)
	=
	\sum_{k=0}^{\infty} a_{k+1}(k+1)x^k
	=
	a_1 + 2a_2x + 3a_3x^2 + 4a_4x^3 + 5a_5x^4 + 6a_6x^5+ \dotsb
\end{equation*}

\begin{equation}
	y''(x)
	=
	\sum_{k = 0}^{\infty} a_{k+2}(k+1)(k+2)x^k
	=
	2a_2 + 3 \mathbin{\cdot} 2a_3x + 4 \mathbin{\cdot} 3a_4x^2 + 5 
	\mathbin{\cdot} 4a_5x^3 + 6 \mathbin{\cdot} 5a_6x^4 + \dotsb
	\label{wellen:zweiteableitung}
\end{equation}

Aus den beiden Gleichungen \ref{wellen:nullteableitung} und 
\ref{wellen:zweiteableitung} kann nun der Koeffizientenvergleich 
\ref{wellen:koeffizietenvergleich} erstellt werden.

\begin{equation}
	\begin{split}
		y''
		&=
		-(ax^2+bx+c)y \\
		2a_2 + 3 \mathbin{\cdot} 2a_3x + 4 \mathbin{\cdot} 3a_4x^2 + \dotsb
		&=
		-(ax^2+bx+c)(a_0 + a_1x + a_2x^2 + a_3x^3 + a_4x^4 + \dotsb) \\
		2a_2 + 3 \mathbin{\cdot} 2a_3x + 4 \mathbin{\cdot} 3a_4x^2 + \dotsb
		&=
		-aa_0x^2-aa_1x^3-aa_2x^4-\dotsb \\
		&\hspace{10pt}
		-ba_0x-ba_1x^2-ba_2^3-ba_3x^4-\dotsb \\
		&\hspace{10pt}
		-ca_0-ca_1x-ca_2x^2-ca_3x^3-ca_4x^4 - \dotsb
	\end{split}
	\label{wellen:koeffizietenvergleich}
\end{equation}

Daraus k"onnen nun die Resultate f"ur die verschiedenen $a_k$ bestimmt werden.

\begin{equation}
	\begin{split}
		a_2
		&=
		-\frac{1}{2}ca_0 \\
		a_3
		&=
		-\frac{1}{3 \cdot 2} (ba_0 + ca_1) \\
		a_4
		&=
		-\frac{1}{4 \cdot 3} (aa_0 + ba_1 + ca_2) \\
		&=
		-\frac{1}{4 \cdot 3} (aa_0 + ba_1 -\frac{1}{2}c^2a_0) \\
		a_5
		&=
		-\frac{1}{5 \cdot 4} (aa_1 + ba_2 + ca_3) \\
		&=
		-\frac{1}{5 \cdot 4} (aa_1 -\frac{1}{2}bca_0 -\frac{1}{3 \cdot 2} 
		c(ba_0 + ca_1)) \\
		&\hspace{5pt}\vdots
	\end{split}
	\label{wellen:aks}
\end{equation}

Hieraus l"asst sich nun eine allgemeine Rekursionsformel 
\ref{wellen:koeffizientengleichung} f"ur die verschiedenen $a_k$ herauslesen. 
Es gilt: $k \in \mathbb{N}$ und	$a_{k < 0} = 0$

\begin{equation}
	a_{k+2} = -\frac{1}{(k+2)(k+1)} (aa_{k-2}+ba_{k-1}+ca_k)
	\label{wellen:koeffizientengleichung}
\end{equation}

Nun kann man die L"osungsgleichung \ref{wellen:ygleichung} f"ur $y(x)$ 
aufstellen.

\begin{equation}
	y(x) = a_0 + a_1x 
	-\sum_{k=0}^{\infty}\frac{1}{(k+2)(k+1)}(aa_{k-2}+ba_{k-1}+ca_k)x^{k+2}
	\label{wellen:ygleichung}
\end{equation}

Um das Programmieren dieser Formel zu vereinfachen, formt man sie am besten von 
$a_{k+2}$ nach $a_k$ um. Damit erh"alt man die Gleichung 
\ref{wellen:koeffizientengleichungak}.
Bei der neuen L"osungsgleichung f"ur $y(x)$ \ref{wellen:ygleichungak} muss nun 
aber $k \in \mathbb{N} \backslash \{0, 1\}$ gelten.

\begin{equation}
	a_{k} = -\frac{1}{k(k-1)} (aa_{k-4}+ba_{k-3}+ca_{k-2})
	\label{wellen:koeffizientengleichungak}
\end{equation}

\begin{equation}
	y(x) = a_0 + a_1x 
	-\sum_{k=2}^{\infty}\frac{1}{k(k-1)}(aa_{k-4}+ba_{k-3}+ca_{k-2})x^k
	\label{wellen:ygleichungak}
\end{equation}