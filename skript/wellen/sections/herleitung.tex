\section{Herleitung Ausgangsformel}
In einem ersten Schritt wird erkl"art, wie die 
Gleichung \ref{eq:wellen:grundgleichung} aus der Aufgabenstellung 
zustandegekommen 
ist.

Die allgemeine Differentialgleichung der Welle lautet (gem"ass 
\cite{wellen:smirnow2}):

\begin{equation*}
	\frac{\partial^2 u}{\partial t^2}
	=
	c^2
	\left(
		\frac{\partial^2 u}{\partial x^2} 
		+ \frac{\partial^2 u}{\partial y^2} 
		+ \frac{\partial^2 u}{\partial z^2}
	\right)
	\label{eq:wellen:allgemeineDGL}
\end{equation*}

Da in der ausgew"ahlten Problemstellung nur eine zweidimensionale Welle 
betrachtet wird, fallen die y- und die z-Koordinate weg. Die Gleichung 
vereinfacht sich dadurch zu:

\begin{equation}
	\frac{\partial^2 u}{\partial t^2}
	=
	c^2
	\left(
		\frac{\partial^2 u}{\partial x^2} 
	\right)
	\label{eq:wellen:allgemeineDGLvereinfacht}
\end{equation}

Somit ist die Funktion $u$ noch abh"angig von der Zeit $t$ und der 
Ortskoordinate 
$x$.

\begin{equation*}
	u = f(x,t)
\end{equation*}

Zur L"osung der vereinfachten Wellengleichung 
\ref{eq:wellen:allgemeineDGLvereinfacht} wird die Theorie des 
Separationsansatzes 
angewendet. Gem"ass dem kann $u(x, t)$ in ein Produkt zweier Funktionen $X(x)$ 
und $T(t)$ umgewandelt werden.

\begin{equation}
	u (x,t) = X(x) T(t)
	\label{eq:wellen:separierteFunktion}
\end{equation}

Ausgehend von der vereinfachten Grundgleichung der Welle 
\ref{eq:wellen:allgemeineDGLvereinfacht} wird die Funktion 
\ref{eq:wellen:separierteFunktion} auf der linken Seite zweimal partiell nach 
$t$ 
und auf der rechten Seite zweimal partiell nach $x$ abgeleitet. Daraus 
resultiert:

\begin{equation}
	T''(t) X(x) = c^2 X''(x)T(t)
\end{equation}

Diese Differentialgleichung wird nun durch $T(x)X(x)$ dividiert und somit 
werden die Variablen separiert und ein Term entsteht, welcher nur funktionieren 
kann, wenn beide Seiten der Gleichung konstant sind. 

\begin{equation*}
	\frac{T''(t)}{T(t)}
	=
	c^2 \frac{X''(x)}{X(x)} = \text{konstant} = \lambda
\end{equation*}

Aus dieser Gleichung entstehen nun zwei l"osbare Differentialgleichungen 
zweiter Ordnung.

\begin{equation*}
	T''= \lambda T(t)
\end{equation*}

und

\begin{equation}
	X'' = \frac{\lambda}{c^2}X(x)
	\label{eq:wellen:DGLzweiterOrdnung}
\end{equation}

Bei der betrachteten Problemstellung, wird davon ausgegangen, dass die 
Zeit konstant ist. Aus diesem Grund wird ab hier nur weiter auf die Gleichung 
der Funktion $X(x)$ eingegangen. Wird die Gleichung 
\ref{eq:wellen:DGLzweiterOrdnung} gleich Null gesetzt, ergibt sich:

\begin{equation*}
	X''(x) - \frac{\lambda}{c^2} X(x) = 0
\end{equation*}

Diese Gleichung widerspiegelt die Wellengleichung, die in der Fortsetzung 
betrachtet wird.

Mit dem Faktor $\frac{\lambda}{c^2}$ k"onnen verschiedene 
Geschwindigkeitsprofile betrachtet werden. Hier wird unter anderem auf den 
speziellen Fall eines parabolischen Profils eingegangen. Die zu untersuchende 
Differentialgleichung ergibt sich hiermit zu der am Anfang definierten 
Gleichung \ref{eq:wellen:grundgleichung}, mit den frei w"ahlbaren Variablen 
${a,b,c} \in \mathbb{R}$, sowie den Anfangsbedingungen $y(0) = a_0$ und $y'(0) 
= a_1$.