\section{Allgemeines Polynom}
Aufgrund der bisherigen Beobachtungen ist es nun m"oglich, eine 
Allgemeine geltende L"osung f"ur diese Art der Differentialgleichung zu 
erstellen.

Hierzu wird das anf"angliche Polynom $ax^2 + bx + c$ mit

\begin{equation*}
\begin{split}
	\sum_{i=0}^{n}\lambda_ix^i =
	\lambda_nx^n + \lambda_{n-1}x^{n-1} + \lambda_{n-2}x^{n-2} + \dotsb + 
	\lambda_2x^2 + \lambda_1x + \lambda_0, \quad n \ge 0
\end{split}
\end{equation*}
ersetzt. Was dann folgende Gleichung liefert:

\begin{equation*}
	y''+\sum_{i=0}^{n}\lambda_ix^i y=0, \quad n \ge 0
\end{equation*}

Aufgrund der zweiten Ableitung von $y$, ist bekannt, dass f"ur die 
verschiedenen $a_k$ eine Abh"angigkeit "uber jeweils zwei Elemente besteht. 
Zus"atzlich erh"oht sich diese um den Grad des zu untersuchenden Polynoms. 
Die $a_k$ eines Polynoms $n$-ten Grades lassen sich also wie folgt berechnen:

\begin{equation*}
	a_k = -\frac{1}{k(k-1)}\sum_{i=0}^{n}a_{k-2-i)}\lambda_i, \quad n \ge 0, 
	a_{k-2-i < 0} =  0.
\end{equation*}

Eingesetzt in die bereits bekannte L"osungsgleichung ergibt sich nun:
\begin{equation}
	y(x) = a_0 + a_1x - \sum_{k=2}^{\infty}\frac{1}{k(k-1)}\sum_{i=0}^{n}
	\lambda_ia_{k-2-i}x^k, \quad n \ge 0, a_{k < 0} = 0
	\label{wellen:allgemein}
\end{equation}

\subsection{Konsequenzen}

Mit der Formel \ref{wellen:allgemein} gibt es nun ein einfaches Werkzeug mit 
dem man durch einfaches Einsetzen die Potenzreihenl"osung f"ur diese Art von 
Differentialgleichungen erh"alt.

Diese Formel hat aber auch noch weitergehende Konsequenzen. So kann man nun 
direkt aus der Form des gegebenen Polynoms ablesen, wie sich die 
Differentialgleichung jeweils verhalten wird. Soll heissen, positive L"osungen 
des Polynoms f"uhren zu einer Wellenform, die $\sin$ und $\cos$ enthalten. 
Negative L"osungen liefern hingegen eine Kombination aus $\sinh$ und $\cosh$.

\subsection{Beispiel: $n = 1$, Airy-Differentialgleichung}
Die Airy-Differentialgleichung

\begin{equation*}
	y''-xy = 0
\end{equation*}
ergibt nun in die allgemeine L"osungsformel \ref{wellen:allgemein} eingesetzt:

\begin{equation*}
\begin{split}
	y(x) &= a_0+a_1x-\sum_{k=2}^{\infty} \frac{1}{k(k-1)} (-1) a_{k-2-1} x^k
	\\
	&= a_0+a_1x+\sum_{k=2}^{\infty} \frac{1}{k(k-1)} a_{k-3} x^k,
	\qquad a_{k < 0} = 0
\end{split}
\end{equation*}

Graphisch betrachtet werden die genannten Konsequenzen deutlich erkennbar:

%TODO Grafik!

\subsection{Beispiel: $n = 4$}

Auch ein Polynom 4-ten Grades stellt kein Problem dar. Die 
Differentialgleichung:

\begin{equation*}
	y''+(ax^4+bx^3+cx^2+dx+e)y = 0
\end{equation*}
ergibt nach dem Einsetzen:

\begin{equation*}
\begin{split}
	y(x) &= a_0+a_1x-\sum_{k=2}^{\infty} \frac{1}{k(k-1)} (aa_{k-2-4} + 
	ba_{k-2-3} + ca_{k-2-2} + da_{k-2-1} +ea_{k-2-0})x^k
	\\
	&= 0+a_1x-\sum_{k=2}^{\infty} \frac{1}{k(k-1)} (aa_{k-6} + ba_{k-5} + 
	ca_{k-4} + da_{k-3} +ea_{k-2})x^k \qquad a_{k<0} = 0
\end{split}
\end{equation*}

Auch hier kann man Graphisch die "Uberg"ange zwischen $\sin$ und $\cos$ bei 
positiven und $\sinh$ und $\cosh$ bei negativen Polynoml"osungen klar erkennen.

\begin{center}
	\includegraphics[scale=0.7]{./wellen/images/allgemein/wave.png}
\end{center}