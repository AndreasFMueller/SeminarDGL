\section{Stolpersteine}

Bei der Berechnung der Welle mittels Potenzreihen gibt es mehrere Dinge, die 
beachtet werden m"ussen. Von besonderer Signifikanz sind vor allem die 
Genauigkeit der L"osungen f"ur die jeweiligen $x$, sowie der auch die Zeit, 
welche f"ur die Berechnung ben"otigt wird.


\subsubsection{Genauigkeit}
Gem"ass
\begin{equation*}
	y(x) = a_0 + a_1x 
	-\sum_{k=2}^{\textcolor{red}{\infty}}\frac{1}{k(k-1)} 
	(aa_{k-4}+ba_{k-3}+ca_{k-2})x^k
\end{equation*}
m"usste man f"ur jedes $x$ eine unendliche Anzahl Additionen ausf"uhren, was 
aber bedeutet, dass die Berechnung f"ur ein einzelnes $x$ unendlich lange 
dauern w"urde. Folglich muss man die maximale Anzahl $k$ einschr"anken.

\begin{equation*}
	y(x) = a_0 + a_1x 
	-\sum_{k=2}^{\textcolor{red}{k_{\text{max}}}}\frac{1}{k(k-1)} 
	(aa_{k-4}+ba_{k-3}+ca_{k-2})x^k
\end{equation*}

Das hat aber weitreichende Folgen: So werden je nach $k_{\text{max}}$ fr"uher 
oder sp"ater die L"osungswerte gegen $+\infty$ oder $-\infty$ gehen, was daran 
liegt, dass der Term $x^k$ einer finiten Reihe gegen"uber den $a_k$ beginnen 
wird zu dominieren.

\begin{figure}
	\includegraphics[scale=0.65]{./wellen/images/kmax/kmax.pdf}
	\caption{Auswirkung unterschiedlicher $k_{\text{max}}$ auf Wellenform}
	\label{fig:wellen:variablekmax}
\end{figure}

Die Abbildung \ref{fig:wellen:variablekmax} zeigt auf, wie sich die L"osung 
der Titelgleichung mit verschiedenen $k_{\text{max}}$ entwickelt. Es ist leicht 
zu erkennen, dass mit gr"osser werdenden $k$ auch der Punkt, an dem $x^k$ 
dominiert, sich weiter vom Entwicklungspunkt (hier = 0) entfernt. Es taucht 
noch ein weiteres Ph"anomen auf: Da die Werte von $a_k$ immer kleiner werden 
und die von $x^k$ immer gr"osser, kommt der Rechner irgendwann an seine Grenzen 
und beginnt unbrauchbare Resultate zu liefern. Diese Ungenauigkeit ist durch 
die bin"are Repr"asentation von Dezimalzahlen und den Limitierungen der 
verschiedenen Datentypen wie \texttt{double} oder \texttt{float} in einem 
Rechnersystem gegeben.

Diese Probleme treten nicht nur in diesem einen Fall des parabolischen Kanals 
auf, sondern sind allgemeing"ultig.

\subsubsection{Berechnungszeit}
Die Berechnung von Potenzreihenl"osung im Intervall
$[x_{\text{min}},x_{\text{max}}]$ kann "ausserst lange dauern. Auch hier muss 
Anzahl Berechnungen, in diesem Fall die Aufl"osung des Plots, eingeschr"ankt 
werden. 

\begin{algorithm}
	\floatname{algorithm}{Pseudocode}
	\begin{algorithmic}[1]
		\State $x \gets x_{\text{min}}$
		\For{$x \le x_{\text{max}}$}
			\State $a_{-2} \gets 0$
			\State $a_{-1} \gets 0$
			\State $a_0 \gets y(0)$
			\State $a_1 \gets y'(0)$
			\State $s_{\text{series}} \gets a_0 + a_1x$
			\State $k \gets 2$
			\For{$k \le k_{\text{max}}$}
				\State $a_k \gets -\cfrac{1}{k(k-1)}			
				(aa_{k-4}+ba_{k-3}+ca_{k-2})$
				\State $s_{\text{series}} \gets s_{\text{series}} + a_k x^k$
				\State $k \gets k + 1$
			\EndFor
			\State $x \gets x + x_{\text{step}}$
		\EndFor
	\end{algorithmic}
	\caption{Wellen Potenzreihenberechnung} 
	\label{alg:wellen:potenzreihenrechnung}
\end{algorithm}

Die Berechnung von Potenzreihenl"osungen im Intervall
$[x_{\text{min}},x_{\text{max}}]$ kann "ausserst lange dauern. Auch hier muss 
die Anzahl Berechnungen, in diesem Fall die Aufl"osung des Plots, 
eingeschr"ankt werden. Eine genauere Absch"atzung kann mit Hilfe des 
Pseudocodes \ref{alg:wellen:potenzreihenrechnung} gemacht werden. Der 
Algorithmus hat eine Laufzeit von
\begin{equation*}
	O
	\left(
		k_{\text{max}}\frac{x_{\text{max}}-x_{\text{min}}}{x_{\text{step}}}
	\right).
\end{equation*}
Da einzig vier Speicherpl"atze f"ur $a_{-2}$ bis $a_1$ ben"otigt werden, hat 
dieser Algorithmus einen Speicherverbrauch von
\begin{equation*}
	O
	\left(
		1
	\right).
\end{equation*}

\begin{algorithm}
	\floatname{algorithm}{Pseudocode}
	\begin{algorithmic}[1]
		\State $a_{-2} \gets 0$
		\State $a_{-1} \gets 0$
		\State $a_0 \gets y(0)$
		\State $a_1 \gets y'(0)$
		\State $k \gets 2$
		\For{$k \le k_{\text{max}}$}
			\State $a_k \gets -\cfrac{1}{k(k-1)} (aa_{k-4}+ba_{k-3}+ca_{k-2})$
			\State $k \gets k + 1$
		\EndFor
		\State $x \gets x_{\text{min}}$
		\For{$x \le x_{\text{max}}$}
			\State $s_{\text{series}} \gets a_0 + a_1x$
			\State $k \gets 2$
			\For{$k \le k_{\text{max}}$}
				\State $s_{\text{series}} \gets s_{\text{series}} + a_k x^k$
				\State $k \gets k + 1$
			\EndFor
			\State $x \gets x + x_{\text{step}}$
		\EndFor
	\end{algorithmic}
	\caption{Wellen Potenzreihenberechnung (Alternative)}
	\label{alg:wellen:potenzreihenrechnungalt}
\end{algorithm}

Die $a_k$ k"onnen auch bereits vorg"angig berechnet werden, was uns zum 
Pseudocode~\ref{alg:wellen:potenzreihenrechnungalt} f"uhrt. Dieser 
alternative Algorithmus hat eine Laufzeit von
\begin{equation*}
	O
	\left(k_{\text{max}} + 
		k_{\text{max}}\frac{x_{\text{max}}-x_{\text{min}}}{x_{\text{step}}}
	\right)
	=
	O
	\left(
		k_{\text{max}}
		\left(
			1+\frac{x_{\text{max}}-x_{\text{min}}}{x_{\text{step}}}
		\right)
	\right),
\end{equation*}
und da alle $a_k$ von $a_{-2}$bis $k_{\text{max}}$ bereits vorberechnet werden 
einen Speicherverbrauch von $O(k_{\text{max}})$.