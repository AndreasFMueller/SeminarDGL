\section{\texorpdfstring{$p(x) = \sum_{i=0}^{n}\lambda_ix^i$}{p(x) = summe i = 
0 bis n lambdai xi}}
\rhead{$p(x) = \sum_{i=0}^{n}\lambda_ix^i$}

Aufgrund der bisherigen Beobachtungen ist es nun m"oglich, eine 
allgemein geltende Potenzreihenl"osung f"ur diese Art Differentialgleichungen 
zu erstellen. Hierzu wird das anf"angliche parabolische Profil $ax^2 + bx + c$ 
durch das allgemeine Polynomprofil $n$-ten Grades
\begin{equation*}
	p(x) =
	\lambda_nx^n + \lambda_{n-1}x^{n-1} + \lambda_{n-2}x^{n-2} + \dotsb + 
	\lambda_2x^2 + \lambda_1x + \lambda_0 = \sum_{i=0}^{n}\lambda_ix^i, \quad n 
	\ge 0
\end{equation*}
ersetzt.

Bei der neuen Problemstellung handelt es sich immer noch um eine 
Differentialgleichung zweiter Ordnung. Es bleibt also eine Abh"angigkeit von 
mindestens zwei Elementen zwischen den verschiedenen $a_k$. Zus"atzlich erh"oht 
sich diese jeweils um den Index $i$ aufgrund der Verschiebung der $a_k$ nach 
rechts, wie im Abschnitt~\ref{subsec:wellen:Potenzreihenansatz} gezeigt wurde. 
Daraus folgt die Rekursionsformel
\index{Rekursionsformel}%
\begin{align}
	a_k &= -\frac{1}{k(k-1)}\sum_{i=0}^{n}a_{k-2-i}\lambda_i, \quad n \ge 0, 
	&a_{k-2-i < 0} &=  0
	\label{eq:wellen:allgemeineak}
\end{align}
f"ur ein Polynom $n$-ten Grades.

\subsubsection{Schlussfolgerungen}

Mit der Formel \eqref{eq:wellen:allgemeineak} gibt es ein einfaches 
Werkzeug mit dem man, allein durch Einsetzen der Parameter, die Koeffizienten 
der Potenzreihenl"osung f"ur diese Art von Differentialgleichungen erh"alt.

Sie erlaubt es uns zudem auch noch weitere Schl"usse zu ziehen. Man kann direkt 
aus der Form des gegebenen Polynomprofils ablesen, wie sich die 
Differentialgleichung verhalten wird. Soll heissen, positive Werte
f"uhren zu einer Wellenform, die Sinus und Cosinus enthalten, negative 
Werte liefern hingegen eine exponentielle Form aus Sinus Hyperbolicus und 
Cosinus Hyperbolicus.

\subsubsection{Berechnungsaufwand}

\begin{algorithm}
	\floatname{algorithm}{Pseudocode}
	\begin{algorithmic}[1]
		\State $x \gets x_{\text{min}}$
		\For{$x \le x_{\text{max}}$}
			\State $j \gets 1$
			\For{$j \le n$}
				\State $a_{-j} \gets 0$
			\EndFor
			\State $a_0 \gets y(0)$
			\State $a_1 \gets y'(0)$
			\State $s_{\text{pot}} \gets a_0 + a_1x$
			\State $k \gets 2$
			\For{$k \le k_{\text{max}}$}
				\State $s_{\text{polynom}} \gets 0$
				\State $i \gets 0$
				\For{$i \le n$}
					\State $s_{\text{polynom}} \gets 
					s_{\text{polynom}}+\lambda_i a_{k-2-i}$
					\State $i \gets i + 1$
				\EndFor
				\State $a_k \gets -\cfrac{1}{k(k-1)} s_{\text{polynom}}$
				\State $s_{\text{series}} \gets s_{\text{series}} + 
				a_k x^k$
				\State $k \gets k + 1$
			\EndFor
		\State $x \gets x + x_{\text{step}}$
		\State $y(x) \gets s_{\text{series}}$
		\EndFor
	\end{algorithmic}
	
	\caption{Allgemeine Potenzreihenberechnung} 
	\label{alg:wellen:allgemeinepotenzreihenrechnung}
\end{algorithm}

Der Pseudocode~\ref{alg:wellen:potenzreihenrechnung} kann mit kleinen 
"Anderungen an die neue L"osung angepasst werden. Es ergibt sich somit der 
Pseudocode~\ref{alg:wellen:allgemeinepotenzreihenrechnung} f"ur die allgemeine 
Potenzreihenl"osung dieser Differentialgleichung. Der allgemeine Algorithmus 
mit dem Pseudocode~\ref{alg:wellen:allgemeinepotenzreihenrechnung} hat eine 
Laufzeit von
\begin{equation*}
	O
	\left(
		nk_{\text{max}}\frac{x_{\text{max}}-x_{\text{min}}}{x_{\text{step}}}
	\right).
\end{equation*}
Wie bereits im Pseudocode~\ref{alg:wellen:potenzreihenrechnung} wird nur 
Speicher f"ur die Anfangs ben"otigten $a_k$ gebraucht. Somit ergibt sich einen 
Speicherverbrauch von $O(n)$.



\begin{algorithm}
	\floatname{algorithm}{Pseudocode}
	\begin{algorithmic}[1]
		\State $j \gets 1$
		\For{$j \le n$}
			\State $a_{-j} \gets 0$
		\EndFor
		\State $a_0 \gets y(0)$
		\State $a_1 \gets y'(0)$
		\State $k \gets 2$
		\For{$k \le k_{\text{max}}$}
			\State $s_{\text{polynom}} \gets 0$
			\State $i \gets 0$
			\For{$i \le n$}
				\State $s_{\text{polynom}} \gets s_{\text{polynom}}+\lambda_i 
				a_{k-2-i}$
				\State $i \gets i + 1$
			\EndFor
			\State $a_k \gets -\cfrac{1}{k(k-1)} s_{\text{polynom}}$
			\State $k \gets k + 1$
		\EndFor
		\State $x \gets x_{\text{min}}$
		\For{$x \le x_{\text{max}}$}
			\State $s_{\text{series}} \gets a_0 + a_1x$
			\State $k \gets 2$
			\For{$k \le k_{\text{max}}$}
				\State $s_{\text{series}} \gets s_{\text{series}} + a_k x^k$
				\State $k \gets k + 1$
			\EndFor
			\State $x \gets x + x_{\text{step}}$
		\EndFor
	\end{algorithmic}
	\caption{Allgemeine Potenzreihenberechnung (Alternative)} 
	\label{alg:wellen:allgemeinepotenzreihenrechnungalt}
\end{algorithm}
Alternativ kann man auch den Pseudocode 
\ref{alg:wellen:allgemeinepotenzreihenrechnungalt} verwenden. Dieser hat eine 
Laufzeit von
\begin{equation*}
	O
	\left(
		nk_{\text{max}} + k_{\text{max}} 
		\frac{x_{\text{max}}-x_{\text{min}}}{x_{\text{step}}}
	\right)
	=
	O
	\left(
		k_{\text{max}}
		\left(
			n+\frac{x_{\text{max}}-x_{\text{min}}}{x_{\text{step}}}
		\right)
	\right),
\end{equation*}
mit einem Speicherverbrauch von
\begin{equation*}
	O
	\left(
		\max(n, k_{\text{max}})
	\right).
\end{equation*}

Eine Implementation mit \texttt{octave} hat gezeigt, dass dieser alternative 
Algorithmus (Pseudocode \ref{alg:wellen:allgemeinepotenzreihenrechnungalt}) 
fast zehn Mal schneller ist. Dies liegt neben der zus"atzlichen inneren 
Schlaufe auch vor allem daran, dass beim anderen Algorithmus mit 
\texttt{modulo} gerechnet werden musste, um den Speicherverbrauch von $O(n)$ zu 
erm"oglichen, was zus"atzlichen Rechenaufwand bedeutet.


\subsection{\texorpdfstring{$y''-xy = 0$}{y''-xy = 0}}
Die bereits bekannte Airy-Differentialgleichung
\begin{align*}
	y''-xy = 0
\end{align*}
ergibt nun in die allgemeine L"osungsformel (\ref{eq:wellen:allgemeineloesung}) 
eingesetzt

\begin{equation*}
	\begin{split}
		y(x) &= a_0+a_1x-\sum_{k=2}^{\infty} \frac{1}{k(k-1)} ((-1) a_{k-2-1} + 
		0 
		a_{k-2-0}) x^k
		\\
		&= a_0+a_1x+\sum_{k=2}^{\infty} \frac{1}{k(k-1)} a_{k-3} x^k,
		\qquad a_{k < 0} = 0,
	\end{split}
\end{equation*}
was sich mit den bereits bekannten L"osung deckt.

Abbildung (\ref{fig:wellen:airy-dgl}) zeigt die genannten Konsequenzen deutlich 
auf.

\begin{figure}
	\includegraphics[scale=0.65]{./wellen/images/allgemein/n1.png}
	\caption{L"osung Airy-Differentialgleichung}
	\label{fig:wellen:airy-dgl}
\end{figure}

\subsection{\texorpdfstring{$y''+(ax^4+bx^3+cx^2+dx+e)y = 
0$}{y''-(ax4+bx3+cx2+dx+e)y = 0}}

Auch ein Polynom $4$-ten Grades stellt kein Problem mehr dar. In die 
Formel (\ref{eq:wellen:allgemeineak}) eingesetzt, ergibt

\begin{equation*}
	\begin{split}
		a_k &= -\frac{1}{k(k-1)} (aa_{k-2-4} + 
		ba_{k-2-3} + ca_{k-2-2} + da_{k-2-1} +ea_{k-2-0})
		\\
		&= -\frac{1}{k(k-1)} (aa_{k-6} + ba_{k-5} + 
		ca_{k-4} + da_{k-3} +ea_{k-2}), \qquad a_{k<0} = 0
	\end{split}
\end{equation*}

Auch hier k"onnen wir in der Abbildung \ref{fig:wellen:poly4-dgl} die 
"Uberg"ange zwischen Sinus und Cosinus bei positiven und Sinus Hyperbolicus und 
Cosinus Hyperbolicus bei negativen Polynoml"osungen deutlich sehen.

\begin{figure}
	\includegraphics[scale=0.65]{./wellen/images/allgemein/n4.pdf}
	\caption{Vergleich Polynom 2-ten und 4-ten Grades}
	\label{fig:wellen:poly4-dgl}
\end{figure}
