Wellen umgeben uns st"andig, selbst wenn wir uns dem nicht direkt bewusst sind. 
Sei es in Form von Lichtwellen, Schallwellen, Wasserwellen oder noch vielen 
anderen Formen. Die Wasserwellen, welche ans Ufer schlagen, sind wohl ein 
Beispiel, unter dem sich jeder etwas vorstellen kann. Dieses an sich sch"one 
Naturschauspiel kann auch destruktiv sein. So kann beispielsweise eine 
pl"otzliche Absenkung am Meeresgrund einen Tsunami ausl"osen.

Mit Hilfe der Titelgleichung wird in diesem Kapitel die Ausbreitung von Wellen 
untersucht. Dabei wird besonders auf das parabolische Kanalprofil eingegangen 
und gegen Ende des Kapitels eine Verallgemeinerung mit dem Profil 
$p(x)$ in der Differentialgleichung
\begin{equation}
	y'' + p(x) y = 0
	\label{eq:wellen:pxdgl}
\end{equation}
beschrieben.

Diese folgend untersuchten Gleichungen sind eine Ann"aherung und entsprechen 
nur bedingt der real vorkommenden Wellenausbreitung. Eine Wasserwelle 
kann sich beispielsweise "uberschlagen, was hier nicht abgebildet wird. 
Trotzdem kann anhand von den in diesem Kapitel besprochenen Modellrechnungen 
versucht werden, die Ausbreitung einer Welle in Abh"angigkeit von $p(x)$ zu 
verstehen.
