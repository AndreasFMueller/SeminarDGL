Wellen umgeben uns st"andig, selbst wenn wir uns dem nicht direkt bewusst sind. 
Sei es in Form von Lichtwellen, Schallwellen, Wasserwellen oder noch vielen 
anderen Formen.

Die Wasserwellen, welche ans Ufer schlagen, sind wohl ein Beispiel, bei dem 
sich jeder etwas vorstellen kann. Dieses an sich sch"one 
Naturschauspiel kann aber auch destruktiv sein, so kann eine pl"otzliche 
Absenkung am Meeresgrund beispielsweise einen Tsunami ausl"osen.

Mit Hilfe der Titelgleichung wird in diesem Kapitel die Ausbreitung von Wellen 
untersucht. Dabei wird besonders auf das parabolische Kanalprofil eingegangen 
und gegen Ende des Kapitels eine Verallgemeinerung mit dem Profil $p(x)$ 
beschrieben:

\begin{equation*}
y'' + p(x) y = 0
\end{equation*}

Diese folgend untersuchten Gleichungen sind nur eine Ann"aherung und 
entsprechen bedingt der real vorkommenden Wellenausbreitung. Eine Wasserwelle 
kann sich beispielsweise "uberschlagen, was hier nicht abgebildet wird. 
Trotzdem kann anhand von den folgenden Modellrechnungen versucht werden, die 
Ausbreitung einer Welle in Abh"angigkeit des Profils zu verstehen.
