\section{Einleitung}
Wellen umgeben uns st"andig, selbst wenn wir es uns dem nicht direkt bewusst 
sind. Sei es nun in Form von Lichtwellen, Schallwellen, Wasserwellen und vielen 
mehr.

Die Wasserwellen welche ans Ufer schlagen, sind wohl ein Beispiel, bei dem sich 
jeder etwas darunter vorstellen kann. Dieses an sich sch"one Naturschauspiel 
kann aber auch destruktiv sein, so kann eine pl"otzliche Absenkung am 
Meeresgrund beispielsweise einen Tsunami ausl"osen.

In diesem Kapitel wird anfangs die Gleichung \ref{eq:wellen:grundgleichung}, 
welche die Ausbreitung einer Welle in einem parabolischen Kanal beschreibt, 
genauer analysiert. Gegen Ende soll aber auch noch eine allgemeinere Kanalform 
untersucht werden.

Nat"urlich sind diese Gleichungen nur eine Ann"aherung und entsprechen nur 
bedingt der in der Realit"at vorkommenen Wellenausbreitung. So kann sich eine 
Wasserwelle zum Beispiel "uberschlagen, was hier so nicht abgebildet wird. 
Trotzdem kann anhand von diesen Modellrechnungen versucht werden, die 
Ausbreitung einer Welle nachzuvollziehen.
