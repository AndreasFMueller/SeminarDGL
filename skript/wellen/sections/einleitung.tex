\section{Einleitung}
Wellen sind etwas was uns st"andig umgibt, selbst wenn wir es uns dem nicht 
direkt bewusst sind. Sei es nun in Form von Licht, Schallwellen, Wasserwellen 
und viele mehr.

Die Wasserwellen welche ans Ufer schlagen, sind wohl ein Beispiel bei dem sich 
jeder etwas darunter vorstellen kann. Dieses an sich sch"one Naturschauspiel 
kann aber auch destruktiv sein, so kann eine pl"otzliche Absenkung am 
Meeresgrund zu Beispiel einen Tsunami ausl"osen.

In diesem Kapitel wird Anfangs die Gleichung \ref{eq:wellen:grundgleichung}, 
welche die Ausbreitung einer Welle in einem parabolischen Kanal beschreibt 
genauer analysiert. Gegen Ende soll aber auch noch eine allgemeinere Kanalform 
untersucht werden.

Nat"urlich sind diese Gleichungen nur eine Ann"aherung und entsprechen nur 
bedingt der in der Realit"at vorkommenen Wellenausbreitung. So kann eine 
Wasserwelle sich zum Beispiel "uberschlagen, was hier so nicht abgebildet wird. 
Trotzdme kann anhand von diesen Modelrechnungen versucht werden die Ausbreitung 
einer Welle nach zu vollziehen.