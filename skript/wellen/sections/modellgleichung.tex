\section{\texorpdfstring{$p(x)$}{p(x)}}
Gem"ass \cite{wellen:smirnow2} lautet die allgemeine Differentialgleichung der 
Welle:

\begin{equation*}
	\frac{\partial^2 u}{\partial t^2}
	=
	c^2
	\left(
		\frac{\partial^2 u}{\partial x^2} 
		+ \frac{\partial^2 u}{\partial y^2} 
		+ \frac{\partial^2 u}{\partial z^2}
	\right)
	\label{eq:wellen:allgemeineDGL}
\end{equation*}

Da es sich in der ausgew"ahlten Problemstellung nur um eine zweidimensionale 
Welle handelt, fallen die y- und z-Koordinate weg. Die Gleichung vereinfacht 
sich dadurch zu:

\begin{equation}
	\frac{\partial^2 u}{\partial t^2}
	=
	c^2
	\left(
		\frac{\partial^2 u}{\partial x^2} 
	\right)
	\label{eq:wellen:allgemeineDGLvereinfacht}
\end{equation}

Zur L"osung der vereinfachten Wellengleichung 
(\ref{eq:wellen:allgemeineDGLvereinfacht}) wird nun mittels Separation die von 
der Zeit $t$ und der Ortskoordinate $x$ abh"angigen Funktion $u(x,t)$ in ein 
Produkt zweier Funktionen $X(x)$ und $T(t)$ umgewandelt.

\begin{equation*}
	u (x,t) = X(x) T(t)
	\Rightarrow T''(t) X(x) = c^2 X''(x)T(t)
\end{equation*}

Nach einer Division durch $T(x)X(x)$ sind die Variablen separiert. Der neu 
entstandene Term kann nur dann funktionieren, wenn beide Seiten der Gleichung 
konstant sind.

\begin{equation*}
	\frac{T''(t)}{T(t)}
	=
	c^2 \frac{X''(x)}{X(x)} = \text{konstant} = \lambda
\end{equation*}

Es folgen nun also zwei l"osbare Differentialgleichungen zweiter Ordnung:

\begin{align*}
	T''(t) &= \lambda T(t) \\
	X''(x) &= \frac{\lambda}{c^2}X(x).
\end{align*}

Bei der betrachteten Problemstellung wird davon ausgegangen, dass die 
Zeit konstant ist. Aus diesem Grund wird ab hier nur noch auf die Gleichung 
der Funktion $X(x)$ weiter eingegangen.

\begin{equation*}
	X''(x) + p(x) X(x) = 0
\end{equation*}
widerspiegelt die Wellengleichung, die in der Fortsetzung betrachtet wird.

Mit $p(x)$ k"onnen verschiedene Geschwindigkeitsprofile betrachtet werden. In 
diesem Kapitel wird unter anderem auf den speziellen Fall eines parabolischen 
Profils eingegangen. Die zu untersuchende Differentialgleichung ergibt sich 
hiermit zu der Titelgleichung, mit den frei w"ahlbaren Variablen ${a,b,c} \in 
\mathbb{R}$.