\section{\texorpdfstring{$p(x)$}{p(x)}}
Als erster Schritt wird in diesem Abschnitt beschrieben, wo die Gleichung ihren 
Ursprung findet. Gem"ass \cite{wellen:smirnow2} lautet die allgemeine 
Differentialgleichung einer Welle
\begin{equation*}
	\frac{\partial^2 u}{\partial t^2}
	=
	c^2
	\left(
		\frac{\partial^2 u}{\partial x^2} 
		+ \frac{\partial^2 u}{\partial y^2} 
		+ \frac{\partial^2 u}{\partial z^2}
	\right).
	\label{eq:wellen:allgemeineDGL}
\end{equation*}
Da wir die Ausbreitung der Welle nur im Querprofil des Kanals betrachten, 
besteht lediglich noch eine Abh"angigkeit der $x$-Koordinate. Die Gleichung 
vereinfacht sich zu
\begin{equation}
	\frac{\partial^2 u}{\partial t^2}
	=
	c^2
	\left(
		\frac{\partial^2 u}{\partial x^2} 
	\right).
	\label{eq:wellen:allgemeineDGLvereinfacht}
\end{equation}

Zur L"osung der vereinfachten Wellengleichung 
(\ref{eq:wellen:allgemeineDGLvereinfacht}) wird mittels Separation die von der 
Zeit $t$ und der Ortskoordinate $x$ abh"angigen Funktion $u(x,t)$ in ein 
Produkt zweier Funktionen $X(x)$ und $T(t)$ umgewandelt.
\begin{equation*}
	u (x,t) = X(x) T(t)
	\Rightarrow T''(t) X(x) = c^2 X''(x)T(t)
\end{equation*}
Nach einer Division durch $T(x)X(x)$ sind die Variablen $x$ und $t$ separiert.
\begin{equation*}
	\frac{T''(t)}{T(t)}
	=
	c^2 \frac{X''(x)}{X(x)}
\end{equation*}
Der neu entstandene Term kann nur dann funktionieren, wenn beide Seiten der 
Gleichung konstant sind.
\begin{equation*}
	\frac{T''(t)}{T(t)}
	=
	c^2 \frac{X''(x)}{X(x)} = \text{konstant} = \lambda
\end{equation*}
Es folgen zwei analytisch l"osbare Differentialgleichungen zweiter Ordnung:
\begin{align*}
	T''(t) &= \lambda T(t) \\
	X''(x) &= \frac{\lambda}{c^2}X(x)
\end{align*}

In dieser Untersuchung wird lediglich ein einzelner Zeitpunkt betrachtet.
Deshalb wird ab hier nur noch auf die Gleichung der Funktion des Weges $X(x)$
weiter eingegangen.
\begin{equation*}
	X''(x) + p(x) X(x) = 0
\end{equation*}
widerspiegelt die Wellengleichung, die in der Fortsetzung betrachtet wird. Mit 
dem Profilquerschnitt $p(x)$ k"onnen verschiedene Kanalformen, in denen 
sich die Welle ausbreitet, betrachtet werden.

Wie bereits erw"ahnt, wird im Abschnitt \ref{sec:wellen:p(x)=parabel} auf den 
speziellen Fall eines parabolischen Profils eingegangen. Die zu untersuchende 
Differentialgleichung ergibt sich hiermit zu der Titelgleichung, mit den frei 
w"ahlbaren Parameter ${a,b,c} \in \mathbb{R}$, welche die Form des 
parabolischen Profils beschreiben.