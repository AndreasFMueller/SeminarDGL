\begin{algorithm}
	\floatname{algorithm}{Pseudocode}
	\begin{algorithmic}[1]
		\State $a_{-2} \gets 0$
		\State $a_{-1} \gets 0$
		\State $a_0 \gets y(0)$
		\State $a_1 \gets y'(0)$
		\State $k \gets 2$
		\For{$k \le k_{\text{max}}$}
			\State $a_k \gets -\cfrac{1}{k(k-1)} (aa_{k-4}+ba_{k-3}+ca_{k-2})$
			\State $k \gets k + 1$
		\EndFor
		\State $x \gets x_{\text{min}}$
		\For{$x \le x_{\text{max}}$}
			\State $s_{\text{series}} \gets a_0 + a_1x$
			\State $k \gets 2$
			\For{$k \le k_{\text{max}}$}
				\State $s_{\text{series}} \gets s_{\text{series}} + a_k x^k$
				\State $k \gets k + 1$
			\EndFor
			\State $x \gets x + x_{\text{step}}$
		\EndFor
	\end{algorithmic}
	\caption{Wellen Potenzreihenberechnung (Alternative)}
	\label{alg:wellen:potenzreihenrechnungalt}
\end{algorithm}

Die $a_k$ k"onnen auch bereits vorg"angig berechnet werden, was uns zum 
Pseudocode \ref{alg:wellen:potenzreihenrechnungalt} f"uhrt. Dieser 
alternative Algorithmus hat eine Laufzeit von
\begin{equation*}
	O
	\left(k_{\text{max}} + 
		k_{\text{max}}\frac{x_{\text{max}}-x_{\text{min}}}{x_{\text{step}}}
	\right)
	=
	O
	\left(
		k_{\text{max}}
		\left(
			1+\frac{x_{\text{max}}-x_{\text{min}}}{x_{\text{step}}}
		\right)
	\right),
\end{equation*}
und da alle $a_k$ von $a_{-2}$bis $k_{\text{max}}$ bereits vorberechnet werden 
einen Speicherverbrauch von
\begin{equation*}
	O
	\left(
		k_{\text{max}}
	\right).
\end{equation*}