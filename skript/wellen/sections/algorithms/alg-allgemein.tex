\begin{algorithm}
	\floatname{algorithm}{Pseudocode}
	\begin{algorithmic}[1]
		\State $x \gets x_{\text{min}}$
		\For{$x \le x_{\text{max}}$}
			\State $j \gets 1$
			\For{$j \le n$}
				\State $a_{-j} \gets 0$
			\EndFor
			\State $a_0 \gets y(0)$
			\State $a_1 \gets y'(0)$
			\State $s_{\text{pot}} \gets a_0 + a_1x$
			\State $k \gets 2$
			\For{$k \le k_{\text{max}}$}
				\State $s_{\text{polynom}} \gets 0$
				\State $i \gets 0$
				\For{$i \le n$}
					\State $s_{\text{polynom}} \gets 
					s_{\text{polynom}}+\lambda_i a_{k-2-i}$
					\State $i \gets i + 1$
				\EndFor
				\State $a_k \gets -\cfrac{1}{k(k-1)} s_{\text{polynom}}$
				\State $s_{\text{series}} \gets s_{\text{series}} + 
				a_k x^k$
				\State $k \gets k + 1$
			\EndFor
		\State $x \gets x + x_{\text{step}}$
		\State $y(x) \gets s_{\text{series}}$
		\EndFor
	\end{algorithmic}
	
	\caption{Allgemeine Potenzreihenberechnung} 
	\label{alg:wellen:allgemeinepotenzreihenrechnung}
\end{algorithm}

Der Pseudocode \ref{alg:wellen:potenzreihenrechnung} kann mit kleinen 
"Anderungen an die neue L"osung angepasst werden. Es ergibt sich somit der 
Pseudocode \ref{alg:wellen:allgemeinepotenzreihenrechnung} f"ur die allgemeine 
Potenzreihenl"osung dieser Differentialgleichung. Der allgemeine Algorithmus 
mit dem Pseudocode \ref{alg:wellen:allgemeinepotenzreihenrechnung} hat eine 
Laufzeit von
\begin{equation*}
	O
	\left(
		nk_{\text{max}}\frac{x_{\text{max}}-x_{\text{min}}}{x_{\text{step}}}
	\right).
\end{equation*}
Wie bereits im Pseudocode \ref{alg:wellen:potenzreihenrechnung} wird nur 
Speicher f"ur die Anfangs ben"otigten $a_k$ gebraucht. Somit ergibt sich einen 
Speicherverbrauch von
\begin{equation*}
	O
	\left(
		n
	\right).
\end{equation*}