\begin{algorithm}
	\floatname{algorithm}{Pseudocode}
	\begin{algorithmic}[1]
		\State $j \gets 1$
		\For{$j \le n$}
			\State $a_{-j} \gets 0$
		\EndFor
		\State $a_0 \gets y(0)$
		\State $a_1 \gets y'(0)$
		\State $k \gets 2$
		\For{$k \le k_{\text{max}}$}
			\State $s_{\text{pol}} \gets 0$
			\State $i \gets 0$
			\For{$i \le n$}
				\State $s_{\text{pol}} \gets s_{\text{pol}}+\lambda_i a_{k-2-i}$
				\State $i \gets i + 1$
			\EndFor
			\State $a_k \gets -\cfrac{1}{k(k-1)} s_{\text{pol}}$
			\State $k \gets k + 1$
		\EndFor
		\State $x \gets x_{\text{min}}$
		\For{$x \le x_{\text{max}}$}
			\State $s_{\text{pot}} \gets a_0 + a_1x$
			\State $k \gets 2$
			\For{$k \le k_{\text{max}}$}
				\State $s_{\text{pot}} \gets s_{\text{pot}} + a_k x^k$
				\State $k \gets k + 1$
			\EndFor
			\State $x \gets x + x_{\text{step}}$
		\EndFor
	\end{algorithmic}
	\caption{Allgemeine Potenzreihenberechnung (Alternative)} 
	\label{alg:wellen:allgemeinepotenzreihenrechnungalt}
\end{algorithm}

Alternativ kann man auch Pseudocode 
\ref{alg:wellen:allgemeinepotenzreihenrechnungalt} verwenden. Dieser hat eine 
Laufzeit von
\begin{equation*}
	\mathcal{O}
	\left(
		nk_{\text{max}} + k_{\text{max}} 
		\frac{x_{\text{max}}-x_{\text{min}}}{x_{\text{step}}}
	\right)
	=
	\mathcal{O}
	\left(
		k_{\text{max}}
		\left(
			n+\frac{x_{\text{max}}-x_{\text{min}}}{x_{\text{step}}}
		\right)
	\right),
\end{equation*}
mit einem Speicherverbrauch von
\begin{equation*}
	\mathcal{O}
	\left(
		\max(n, k_{\text{max}})
	\right).
\end{equation*}

Eine Implementation mit \texttt{octave} hat gezeigt, dass dieser alternative 
Algorithmus (Pseudocode \ref{alg:wellen:allgemeinepotenzreihenrechnungalt}) 
fast 10-Mal schneller ist. Dies liegt neben der zus"atzlichen inneren Schlaufe 
auch vor allem daran, dass beim anderen Algorithmus mit \texttt{modulo} 
gerechnet werden musste um den Speicherverbrauch zu optimieren, was 
zus"atzlichen Rechenaufwand bedeutet.