\begin{algorithm}
	\floatname{algorithm}{Pseudocode}
	\begin{algorithmic}[1]
		\State $x \gets x_{\text{min}}$
		\For{$x \le x_{\text{max}}$}
			\State $a_{-2} \gets 0$
			\State $a_{-1} \gets 0$
			\State $a_0 \gets y(0)$
			\State $a_1 \gets y'(0)$
			\State $s_{\text{series}} \gets a_0 + a_1x$
			\State $k \gets 2$
			\For{$k \le k_{\text{max}}$}
				\State $a_k \gets -\cfrac{1}{k(k-1)}			
				(aa_{k-4}+ba_{k-3}+ca_{k-2})$
				\State $s_{\text{series}} \gets s_{\text{series}} + a_k x^k$
				\State $k \gets k + 1$
			\EndFor
			\State $x \gets x + x_{\text{step}}$
		\EndFor
	\end{algorithmic}
	\caption{Wellen Potenzreihenberechnung} 
	\label{alg:wellen:potenzreihenrechnung}
\end{algorithm}

Die Berechnung von Potenzreihenl"osungen im Intervall
$[x_{\text{min}},x_{\text{max}}]$ kann "ausserst lange dauern. Auch hier muss 
die Anzahl Berechnungen, in diesem Fall die Aufl"osung des Plots, 
eingeschr"ankt werden. Eine genauere Absch"atzung kann mit Hilfe des 
Pseudocodes \ref{alg:wellen:potenzreihenrechnung} gemacht werden. Der 
Algorithmus hat eine Laufzeit von
\begin{equation*}
	O
	\left(
		k_{\text{max}}\frac{x_{\text{max}}-x_{\text{min}}}{x_{\text{step}}}
	\right).
\end{equation*}
Da einzig vier Speicherpl"atze f"ur $a_{-2}$ bis $a_1$ ben"otigt werden, hat 
dieser Algorithmus einen Speicherverbrauch von
\begin{equation*}
	O
	\left(
		1
	\right).
\end{equation*}