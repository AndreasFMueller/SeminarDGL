%
% komplex.tex -- Komplexe Differentialgleichungen
%
% (c) 2015 Prof Dr Andreas Mueller, Hochschule Rapperswil
%
\chapter{Komplexe Differentialgleichungen\label{chapter:komplexeanalysis}}
\lhead{}
\rhead{Komplexe Differentialgleichungen}
Die bisher betrachteten Differentialgleichungen waren immer f"ur
$x\in\mathbb R$ definiert.
Bei der L"osung mit Hilfe von Potenzreihen haben wir L"osungsfunktionen
gefunden, die man auch f"ur komplexe $x$-Werte auswerten kann.
Definiert man die Ableitung einer Funktionen einer komplexen Variablen
$z$ rein formal als
\[
\frac{d}{dz}z^n= nz^{n-1},
\]
dann kann man auch Potenzreihen in der Variablen $z$ formal differenzieren,
indem man jeden Term der Potenzreihe ableitet.
Und die mit der Potenzreihen-Methode gefunden L"osungen erf"ullen dann
auch die urspr"ungliche Differentialgleichung.
Dies ist aber eine rein formale "Uberlegung, da die Ableitung nach einer
komplexen Variablen noch gar nicht definiert ist.

\section{Komplex differenzierbare Funktionen}
Wir betrachten in diesem Kapitel komplexwertige Funktionen,
die ein einem Teilgebiet der komplexen Ebene definiert sind.
Ein {\em Gebiet} ist eine offene Teilmenge $\Omega\subset \mathbb C$.
{\em Offen} heisst, dass mit jedem Punkt $z_0\in\Omega$ eine Umgebung
\[
U=\{z\in\mathbb Z\,|\,|z-z_0|<\varepsilon\}
\]
ebenfalls in $\Omega$ enthalten ist, also $U\subset \Omega$ f"ur gen"ugen
kleines $\varepsilon$.
Sei also $f(z)$ eine in $\Omega\subset\mathbb C$ definierte
Funktion $f\colon\Omega\to\mathbb C$. 

Eine komplexwertige Funktion $f(z)$ kann betrachtet werden als zwei
reellwertige Funktionen von zwei Variablen $x$ und $y$:
\[
f(z)=\operatorname{Re}f(x+iy) + i \operatorname{Im}f(x+iy)
\]
Schreibt man
$\operatorname{Re}f(x+iy)=u(x,y)$
und
$\operatorname{In}f(x+iy)=v(x,y)$,
dann ist die komplexe Funktion vollst"andig durch reelle Funktionen
beschrieben.
Und nat"urlich wissen wir auch, was unter den Ableitungen der Funktionen 
$u(x,y)$ und $v(x,y)$ zu verstehen ist.
Der Funktion $f(z)$ entspricht eine Abbildung $\mathbb R^2\to\mathbb R^2$
\[
(x,y)\mapsto\begin{pmatrix}u(x,y)\\v(x,y)\end{pmatrix}.
\]
Die Ableitung einer solchen Funktion im Punkt $(x_0,y_0)$
ist eine lineare Abbildung von Vektoren, die in linearer N"aherung
den Funktionswert bei $f(z_0 + \Delta z)$ 
\[
\begin{pmatrix}
u(x+\Delta x, y +\Delta y)\\
v(x+\Delta x, y +\Delta y)
\end{pmatrix}
=
\begin{pmatrix}
\frac{\partial u}{\partial x}&\frac{\partial u}{\partial y}\\
\frac{\partial v}{\partial x}&\frac{\partial v}{\partial y}
\end{pmatrix}
\begin{pmatrix} \Delta x\\\Delta y \end{pmatrix}
+o(\Delta x, \Delta y).
\]
In dieser Sicht einer komplexen Funktion gibt es keine einzelne Zahl, die
die Funktion einer Ableitung "ubernehmen k"onnte, die Ableitung
ist eine $2\times 2$-Matrix.

\subsection{Komplexe Ableitung}
Die Ableitung einer Funktion einer reellen Variablen wird mit Hilfe des
Grenzwertes
\[
f'(x_0)=\lim_{x\to x_0}\frac{f(x)-f(x_0)}{x-x_0}
\]
definiert, oder als diejenige Zahl $f'(x_0)\in\mathbb R$ mit der Eigenschaft,
dass
\begin{equation}
f(x)=f(x_0)+f'(x_0)(x-x_0) + o(x-x_0)
\label{komplex:abldef}
\end{equation}
gilt.
Der Term $x-x_0$ und die Gleichung (\ref{komplex:abldef}) sind aber auch
f"ur komplexe Argument sinnvoll, wir definieren daher

\begin{definition}
Die komplexe Funktion $f(z)$ heisst im Punkt komplex differenzierbar
und hat die komplexe Ableitung $f'(z_0)\in\mathbb C$,
wenn 
\begin{equation}
f(z)=f(z_0) + f'(z_0)(z-z_0) +o(z-z_0)
\label{komplex:defkomplabl}
\end{equation}
gilt.
\end{definition}

\begin{beispiel}
Die Funktion $z\mapsto f(z)=z^n$ ist "uberall komplex differenzierbar
und hat die Ableitung $nz^{n-1}$.
Um dies nachzupr"ufen, m"ussen wir die Bedingung~(\ref{komplex:defkomplabl})
verifizieren.
Aus einer wohlbekannten Faktorisierung von $z^n - z_0^n$ k"onnen wir den
Differenzenquotienten finden:
\[
\frac{f(z)-f(z_0)}{z-z_0}
=
\frac{z^n-z_0^n}{z-z_0}
=
\frac{(z-z_0)(z^{n-1}+z^{n-2}z_0+z^{n-3}z_0^2+\dots+z^{n-1})}{z-z_0}
=
z^{n-1}+z^{n-2}z_0+z^{n-3}z_0^2+\dots+z^{n-1}
\]
Lassen wir jetzt $z$ gegen $z_0$ gehen, wird die rechte Seite
zu $nz_0^{n-1}$.
\end{beispiel}

\begin{beispiel}
Die Funktion $z\mapsto f(z)=\bar z=x-iy$ ist nicht differenzierbar.
Wenn $f(z)=\bar z$ differenzierbar w"are, dann m"usste es eine Zahl
$a\in\mathbb C$ geben, so dass 
\[
\bar z-\bar z_0=a(z-z_0)+o(z-z_0)
\]
gilt.
w"ahlen wir $z=z_0+x$ bzw.~$z=z_0+iy$, dann erhalten wir
\[
\begin{aligned}
z-z_0&=x:&
\bar z-\bar z_0&=x
&&\Rightarrow&
\bar z-\bar z_0&=1\cdot x
&&\Rightarrow&
a&=1
\\
z-z_0&=iy:&
\bar z-\bar z_0&=-iy
&&\Rightarrow&
\bar z-\bar z_0&=-1\cdot iy
&&\Rightarrow&
a&=-1
\end{aligned}
\]
Es ist also nicht m"oglich, eine einzige Zahl $a$ zu finden, die als
die Ableitung der Funktion $z\mapsto \bar z$ betrachtet werden k"onnte.
\end{beispiel}

Das letzte Beispiel zeigt, dass
selbst Funktionen, deren Real- und Imagin"arteil beliebig oft stetig
differenzierbare Funktionen sind, nicht komplex differenzierbar
sein m"ussen.
Komplexe Differenzierbarkeit ist eine wesentlich st"arkere Bedingung
an eine Funktion, komplex differenzierbare Funktionen bilden eine
echte Teilmenge aller Funktionen, deren Real- und Imagin"arteil
differenzierbar ist.

\subsection{Die Cauchy-Riemann-Differentialgleichungen}
Komplexe Funktionen k"onnen nur differenzierbar sein, wenn sich die vier
partiellen Ableitungen zu einer einzigen komplexen Zahl zusammenfassen
lassen.
Um diese Beziehung zu finden, gehen wir von einer komplexen Funktion
\[
f(x+iy) = u(x,y) + iv(x,y)
\]
aus, und berechnen die Ableitung auf zwei verschiedene Arten, indem
wir sowohl nach $x$ als auch nach $iy$ ableiten:
\begin{align*}
f'(z)&
=
\lim_{x\to 0}\frac{f(z+x)-f(z)}{x}
=
\frac{\partial u}{\partial x}+i\frac{\partial v}{\partial x}
\\
f'(z)&
=
\lim_{y\to 0}\frac{f(z+x)-f(z)}{iy}
=
\frac1{i}
\frac{\partial u}{\partial y}+\frac{\partial v}{\partial y}
=
\frac{\partial v}{\partial y}
-i
\frac{\partial u}{\partial y}.
\end{align*}
Dies ist nur m"oglich, wenn Real- und Imagin"arteile "ubereinstimmen.
Es folgt also

\begin{satz}
\label{komplex:satz:cauchy-riemann}
Real- und Imagin"arteil $u(x,y)$ und $v(x,y)$ einer
komplexe Funktion $f(z)$ mit $f(x+iy)=u(x,y)+iv(x,y)$
erf"ullen die Cauchy-Riemannschen Differentialgleichungen
\index{Cauchy-Riemann-Differentialgleichungen}
\begin{equation}
\begin{aligned}
\frac{\partial u}{\partial x}
&=
\frac{\partial v}{\partial y},
&
\frac{\partial u}{\partial y}
&=
-
\frac{\partial v}{\partial x}
\end{aligned}
\label{komplex:dgl:cauchy-riemann}
\end{equation}
\end{satz}

Leitet man die Cauchy-Riemann-Differentialgleichungen nochmals nach
$x$ und $y$ ab, erh"alt man
\begin{equation*}
\begin{aligned}
\frac{\partial^2 u}{\partial x^2}
&=
\frac{\partial^2 v}{\partial x\,\partial y},
&
\frac{\partial^2 u}{\partial x\,\partial y}
&=
-\frac{\partial^2 v}{\partial x^2},
&
\frac{\partial^2 u}{\partial y\,\partial x}
&=
\frac{\partial^2 v}{\partial y^2},
&
\frac{\partial^2 u}{\partial y^2}
&=
-\frac{\partial^2 v}{\partial y\,\partial x}.
\end{aligned}
\end{equation*}
Die erste und die letzte sowie die mittleren zwei k"onnen zu jeweils
einer Differentialgleichung f"ur die Funktionen $u$ und $v$ zusammengefasst
werden, n"amlich
\begin{equation*}
\frac{\partial^2 u}{\partial x^2}
+
\frac{\partial^2 u}{\partial y^2}
=
0
\qquad\text{und}\qquad
\frac{\partial^2 v}{\partial x^2}
+
\frac{\partial^2 v}{\partial y^2}
=
0
\end{equation*}

\begin{definition}
Der Operator 
\[
\Delta =
\frac{\partial^2}{\partial x^2}
+
\frac{\partial^2}{\partial y^2}
\]
heisst der {\em Laplace-Operator} in zwei Dimensionen
\index{Laplace-Operator}.
\end{definition}

\begin{definition}
Eine Funktion $h(x,y)$ von zwei Variablen heisst {\em harmonisch}, wenn sie
die Gleichung
\[
\Delta h=0
\]
erf"ullt.
\index{harmonische Funktion}
\end{definition}

\begin{satz}
Real- und Imagin"arteil einer komplexen Funktion sind harmonische Funktionen.
\end{satz}

Die Cauchy-Riemann-Differentialgleichungen schr"anken also einerseits stark
ein, welche Funktionen "uberhaupt als Real- und Imagin"arteil einer
komplex differenzierbaren Funktion in Frage kommen.
Andererseits koppeln sie auch Real- und Imagin"arteil stark zu sammen.

\begin{beispiel}
Von einer komplex differenzierbaren Funktion $f(z)$ sei nur der Realteil
$u(x,y)=x^3 -3xy^2$ bekannt.
Zun"achst kontrollieren wir, ob dies "uberhaupt ein Realteil sein kann,
indem wir nachrechnen, ob $u(x,y)$ harmonisch ist.
\begin{equation*}
\begin{aligned}
\frac{\partial u}{\partial x}
&=
3x^2-3y^2
&&\Rightarrow&
\frac{\partial^2 u}{\partial x^2}
&=
6x
\\
\frac{\partial u}{\partial y}
&=
-6xy
&&\Rightarrow&
\frac{\partial^2 u}{\partial y^2}
&=
-6x
\\
&&&&\Delta u&=\frac{\partial^2u}{\partial x^2}+\frac{\partial^2u}{\partial y^2}=6x-6x=0,
\end{aligned}
\end{equation*}
$u$ ist also harmonisch.
Um die Funktion $f$ zu finden, brauchen wir jetzt noch den Imagin"arteil.
Wir finden ihn mit Hilfe der Cauchy-Riemann-Differentialgleichungen.
Es gilt
\begin{equation}
\begin{aligned}
\frac{\partial v}{\partial x}
&=
-\frac{\partial u}{\partial y}=6xy,
&
\frac{\partial v}{\partial y}
&=
\frac{\partial u}{\partial x}=3x^2-3y^2
\end{aligned}
\label{komplex:crbeispiel}
\end{equation}
Aus der ersten Gleichung erh"alt man durch Integrieren nach $x$ 
\[
v(x,y)=-3x^2y + C(y),
\]
die Integrations-``Konstante'' ist eine Funktion, die aber nur von $y$
abh"angen darf.
Die zweite Cauchy-Riemann-Gleichung verwendet die Ableitung von $v$ nach $y$,
sie ist
\[
\frac{\partial v}{\partial y}=3x^2+C'(y).
\]
Aus der zweiten Gleichung von (\ref{komplex:crbeispiel}) liesst man
ab, dass
\[
C'(y)=-3y^2
\qquad\Rightarrow\qquad
C(y)=-y^3+k
\]
sein muss.
Damit ist $v$ bis auf eine Konstante bestimmt.
Die zugeh"orige Funktion $f(z)$ ist daher
\[
f(z)=f(x+iy)=x^3-3xy^2+i(3x^2y-y^3)+ik
=x^3 + 3x^2iy + 3x(iy)^2+(iy)^3+ik=z^3+ik.
\]
Wir haben die Funktion $f(z)$ bis auf eine Konstanten $ik$ 
aus ihrem Realteil rekonstruiert.
\end{beispiel}

\subsection{Wegintegrale und die Cauchy-Formel}
Das Finden einer Stammfunktion, die Integration, ist die Grundtechnik,
mit der man den "Ubergang von lokaler Information in Form von Ableitungen,
zu globaler Information "uber reelle Funktionen vollzieht.
Sie liefert aus der Steigung zwischen zwei Punkten $x_0$ und $x$ den
Funktionswert mittels
\[
f(x)=f(x_0)+\int_{x_0}^xf'(\xi)\,d\xi.
\]
Bei einer reellen Funktion gibt es nur eine Richtung, entlang der man
integrieren k"onnte.

Auch in der komplexen Ebene erwarten wir eine Formel
\[
f(z) = f(z_0) + \int_{z_0}^z f'(\zeta)\,d\zeta.
\]
In der komplexen Ebene gibt es aber beliebig viele Wege, mit denen die
Punkte $z_0$ und $z$ verbunden werden k"onnen. 
Der Wert von $f(z)$ muss also durch Integration entlang eines speziell
gew"ahlten Weges $\gamma$
\[
f(z) = f(z_0) + \int_{\gamma} f'(\zeta)\,d\zeta
\]
bestimmt werden.
Es muss also zun"achst gekl"art werden, wie ein solches Wegintegral
"uberhaupt zu verstehen und zu berechnen ist.
Dann gilt es zu untersuchen, inwieweit diese Konstruktion unabh"angig
von der Wahl des Weges ist.
F"ur komplex differenzierbare Funktionen wird sich eine sehr erfolgreiche
Theorie ergeben.

% XXX Ableitungsregeln

\subsubsection{Wegintegrale}
Ein Weg in der komplexen Ebene ist eine Abbildung 
\[
\gamma\colon [a,b]\to\mathbb C: t\mapsto \gamma(t).
\]
Wir verlangen f"ur unsere Zwecke zus"atzlich, dass $\gamma$ differenzierbar
ist.
Dann k"onnen wir f"ur jede beliebige Funktion das Wegintegral definieren.

\begin{definition}
Sei $\gamma\colon[a,b]\to\mathbb C$ ein Weg in $\mathbb C$ und $f(z)$
eine stetige komplexe Funktion, dann heisst
\[
\int_{\gamma} f(z)\,dz = \int_a^bf(\gamma(t)) \gamma'(t)\,dt
\]
das {\em Wegintegral} von $f(z)$ entlang der Kurve $\gamma$.
\index{Wegintegral}
\end{definition}

\begin{beispiel}
Wir berechnen als Beispiel das Wegintegral der Funktion $f(z)=1/z$ entlang
eines Halbkreises von $1$ zu $-1$. 
Es gibt zwei verschiedene solche Halbkreise:
\begin{equation*}
\begin{aligned}
\gamma_+(t)&=e^{it},&t&\in[0,\pi]
\\
\gamma_-(t)&=e^{-it},&t&\in[0,\pi]
\end{aligned}
\end{equation*}
Wir finden f"ur die Wegintegrale
\begin{align*}
\int_{\gamma_+}\frac1z\,dz
&=
\int_0^\pi \frac1{e^{it}}ie^{it}\,dt=i\int_0^\pi\,dt=i\pi
\\
\int_{\gamma_-}\frac1z\,dz
&=
-\int_0^\pi \frac1{e^{-it}}ie^{-it}\,dt=-i\int_0^\pi\,dt=-i\pi
\end{align*}
Das Wegintegral zwischen $1$ und $-1$ h"angt also mindestens f"ur diese
spezielle Funktion $f(z)=1/z$ von der Wahl des Weges ab.
\end{beispiel}

Wie Wahl der Parametrisierung der Kurve hat keinen Einfluss auf den 
Wert des Wegintegrals.

\begin{satz}
Seien $\gamma_1(t), t\in[a,b],$ und $\gamma_2(s),s\in[c,d]$
verschiedene Parametrisierungen
der gleichen Kurve, es gebe also eine Funktion $t(s)$ derart, dass
$\gamma_1(t(s))=\gamma_2(s)$.
Dann ist
\[
\int_{\gamma_1}f(z)\,dz
=
\int_{\gamma_2}f(z)\,dz.
\]
\end{satz}

\begin{proof}[Beweis]
Wir verwenden die Definition des Wegintegrales
\begin{align*}
\int_{\gamma_1} f(z)\,dz
&=
\int_a^b f(\gamma_1(t))\,\gamma_1'(t)\,dt
=
\int_c^d f(\gamma_1(t(s))\,\underbrace{\gamma_1'(t(s)) t'(s)}_{\textstyle
=\frac{d}{ds}\gamma_1(t(s))}\,ds
\\
&=
\int_c^d f(\gamma_2(s)\,\gamma_2'(s)\,ds
=
\int_{\gamma_2}f(z)\,dz
\end{align*}
Beim zweiten Gleichheitszeichen haben wir die Formel f"ur die
Variablentransformation $t=t(s)$ in einem Integral verwendet.
\end{proof}

Wir erwarten, dass das Wegintegral "ahnlich wie das Integral reeller
Funktionen eine Art ``Umkehroperation'' zur Ableitung ist.
Wir untersuchen daher den Fall, dass $f(z)$ eine komplexe Stammfunktion $F(z)$
hat, also $f(z)=F'(z)$.
Wir berechnen das Wegintegral entlang des Weges $\gamma$:
\begin{align*}
\int_{\gamma}f(z)\,dz
&=
\int_a^bf(\gamma(t))\,\gamma'(t)\,dt
=
\int_a^bF'(\gamma(t))\,\gamma'(t)\,dt
=
\int_a^b\frac{d}{dt}F(\gamma(t))\,dt
=
F(\gamma(a))-F(\gamma(b))
\end{align*}
Dies ist genau die Formel, die man als den Hauptsatz der Infinitesimalrechnung
kennt.
Trotzdem ist die Situation hier etwas anders.
In der reellen Infinitesimalrechnung war die Existenz einer Stammfunktion
durch das Integral gesichert, man konnte mit
\[
F(x)=\int_a^xf(\xi)\,d\xi
\]
immer eine Stammfunktion angeben.
Im komplexen Fall k"onnen wir nat"urlich auch versuchen, eine Stammfunktion
mit Hilfe von 
\[
F(z)=\int_{\gamma_z} f(\zeta)\,d\zeta
\]
zu definieren.
Dabei muss allerdings $\gamma_z$ ein Weg sein, der im Punkt $z$ endet,
und wir wissen noch nicht einmal, ob die Wahl des Weges eine Rolle
spielt.
Bevor wir also sicher sein k"onnen, dass eine Stammfunktion existiert,
m"ussen wir zeigen, dass das Wegintegral einer komplex differenzierbaren
Funktion zwischen zwei Punkten nicht von der Wahl des Weges abh"angt,
der die beiden Punkte verbindet.
Dazu ist notwendig, geschlossene Wege genauer zu betrachten.

\subsubsection{Geschlossene Wege}
\begin{definition}
Ein Weg $\gamma\colon[a,b]\to\mathbb C$ heisst {\em geschlossen}, wenn
$\gamma(a)=\gamma(b)$.
\index{geschlossener Weg}
Das Integral entlang eines geschlossenen Weges h"angt nicht von der
Parametrisierung ab und wird zur Verdeutlichung mit
\[
\int_{\gamma}f(z)\,dz
=
\oint_{\gamma}f(z)\,dz
\]
bezeichnet.
\end{definition}

\begin{beispiel}
Wir berechnen das Integral von $f(z)=z^n$ entlang des Einheitskreises,
den wir mit $\gamma(t)=e^{it},t\in[0,2\pi]$ parametrisieren.
Die Definition liefert:
\begin{align*}
\oint_{\gamma}f(z)\,dz
&=
\int_0^{2\pi}e^{int}ie^{it}\,dt
=
i\int_0^{2\pi}e^{i(n+1)t}\,dt
\end{align*}
F"ur $n=-1$ ist dies das Integral einer konstanten Funktion, also
\[
\oint_{\gamma}\frac1z\,dz=2\pi i.
\]
F"ur $n\ne -1$ kann man eine Stammfunktion von $e^{i(n+1)t}$
verwenden:
\[
\oint_{\gamma}f(z)\,dz
=
i\left[\frac1{i(n+1)}e^{i(n+1)t}\right]_0^{2\pi}
=0,
\]
weil $e^{i(n+1)t}$ periodisch ist mit Periode $2\pi$.
\end{beispiel}
Das Beispiel zeigt, dass ein Wegintegral der Potenzfunktionen,
aller Polynome und schliesslich aller konvergenten Potenzreihen
"uber einen geschlossenen Weg verschwinden.
Es zeigt aber auch, dass das Wegintegral "uber einen geschlossenen
Weg nicht zu verschwinden braucht, wie das Beispiel $f(z)=1/z$ 
zeigt.
Letztere Funktion unterscheidet sich von den Potenzfunktionen allerdings
dadurch, dass sie im Nullpunkt nicht definiert ist.

\begin{satz}
Sei $f(z)$ eine in einem zusammenh"angenden Gebiet $\Omega\subset\mathbb C$
definierte komplexe Funktion, f"ur die das Wegintegral "uber jeden
geschlossenen Weg verschwindet.
Dann hat $f(z)$ eine komplexe Stammfunktion $F(z)$.
\end{satz}

\begin{proof}[Beweis]
Wir w"ahlen einen beliebigen Punkt $z_0\in\Omega$ definieren die
komplexe Stammfunktion mit Hilfe des Wegintegrals
\[
F(z)=\int_{\gamma_z} f(\zeta)\,d\zeta,
\]
wobei $\gamma_z$ ein beliebiger Weg ist, der $z_0$ mit $z$ verbindet.

Wir m"ussen uns davon "uberzeugen, dass die Wahl des Weges keinen Einfluss
auf $F(z)$ hat.
Dazu seien $\gamma_1$ und $\gamma_2$ zwei verschiedene Wege, die
$z_0$ mit $z$ verbinden.
Da die Parametrisierung der Wege keinen Einfluss auf das Wegintegral haben,
nehmen wir an, dass beide Wege auf dem Interval $[0,1]$ definiert sind.

Jetzt konstruieren wir einen geschlossene Weg $\gamma$ durch die
Defintion:
\[
\gamma\colon[0,2]\to\mathbb C:t\mapsto
\begin{cases}
\gamma_1(t)&\qquad 0\le t\le 1\\
\gamma_2(2-t)&\qquad 1\le t\le 2
\end{cases}
\]
Der Weg $\gamma$ besteht aus $\gamma_1$ und dem in umgekehrter Richtung
durchlaufenen Weg $\gamma_2$, denn an der Stelle $t=1$ passen die
beiden Teilwege natlos zusammen: $\gamma_1(1)=\gamma_2(1)=\gamma_2(2-1)$.
Wegen $\gamma(2)=\gamma_2(2-2)=\gamma_2(0)=\gamma_1(0)$ ist der
Weg geschlossen.
Nach Voraussetzung ist verschwindet das Wegintegral "uber $\gamma$.
Es folgt
\begin{align*}
0
&=
\int_{\gamma}f(z)\,dz
\\
&=
\int_0^1 f(\gamma_1(t))\gamma_1'(t)\,dt
+ \int_1^2f(\gamma_2(2-t))\frac{d}{dt}\gamma_2(2-t)\,dt
\\
&=
\int_0^1 f(\gamma_1(t))\gamma_1'(t)\,dt
- \int_1^2f(\gamma_2(2-t))\gamma_2'(2-t)\,dt
\\
&=
\int_0^1 f(\gamma_1(t))\gamma_1'(t)\,dt
- \int_0^1f(\gamma_2(s))\gamma_2'(s)\,ds
\\
&=
\int_{\gamma_1}f(z)\,dz - \int_{\gamma_2}f(z)\,dz
\\
\Rightarrow\qquad
\int_{\gamma_2}f(z)\,dz&=\int_{\gamma_1}f(z)\,dz.
\end{align*}
Da die Wahl des Weges keine Rolle spielt, ist $F(z)$ wohldefiniert.
\end{proof}

Die Bedingung des eben bewiesenen Satzes ist nicht wirklich n"utzlich,
sie ist kaum nachpr"ufbar.
Es braucht also zus"atzliche Anstrengungen um gen"ugen viele
Funktionen zu finden, welche die Eigenschaft haben, dass Wegintegrale
"uber geschlossene Wege verschwinden.
Wir zielen dabei auf den folgenden Satz hin:
\begin{satz}[Cauchy]
Ist $f(z)$ eine in einem Gebiet $\Omega\subset\mathbb C$ definierte
komplex differenzierbare Funktion, und ist $\gamma$ ein im Gebiet
$\Omega$ auf einen Punkt zusammenziehbarer Weg, dann gilt
\[
\int_{\gamma}f(z)\,dz=0.
\]
Ist insbesondere $\Omega$ {\em einfach zusammenh"angend}
(d.~h.~jeder geschlossene Weg l"asst sich in einen Punkt zusammenziehen),
dann verschwindet das Wegintegral von $f(z)$ "uber jeden geschlossenen
Weg in $\Omega$.
\index{einfach zusammenhangend@einfach zusammenh\"angend}
\end{satz}

% XXX Wegunabh"angigkeit und Integrale über geschlossene Wege

\subsubsection{Cauchy-Integral}
Sei jetzt $f(z)$ eine komplex differenzierbare Funktion.
Dann ist auch die Funktion
\[
g(z)=\frac{f(z)}{z-a}
\]
komplex differenzierbar f"ur $z\ne a$.
Insbesondere ist der Wert des Wegintegrales von $g(z)$ entlang
eines geschlossenen Pfades um den Punkt $a$ unabh"angig von der Wahl
des Weges.
Zum Beispiel k"onnten wir das Wegintegral mit Hilfe eines kleinen Kreises
um $a$ mit Radius $r$ mit der Parametrisierung
\[
t\mapsto \gamma(t)=a+re^{it},\quad t\in[0,2\pi]
\]
berechnen.
Die Rechnung ergibt
\begin{align*}
\oint_\gamma \frac{f(z)}{z-a}\,dz
&=
\int_0^{2\pi} \frac{f(a+re^{it})}{re^{it}}ire^{it}\,dt
=
i\int_0^{2\pi} f(a+re^{it})\,dt
\end{align*}
Da $f(z)$ komplex differenzierbar ist, k"onnen wir $f(z)$ approximieren
durch $f(z)=f(a)+f'(a)(z-a)+o(z-a)$, also
\begin{align*}
\oint_{\gamma} \frac{f(z)}{z-a}\,dz
&=
i\int_0^{2\pi}f(a) + f'(a)re^{it}+o(r)\,dt
\\
&=
f(a)i\int_0^{2\pi}\,dt + irf'(a)\int e^{it}\,dt + i\int_0^{2\pi}o(r)\,dt
\\
&=
2\pi i f(a) + irf'(a)\underbrace{\left[\frac1{i}e^{it}\right]_0^{2\pi}}_{=0}+o(r)
\\
&=2\pi i f(a)+o(r).
\end{align*}
Da das Wegintegral einer komplex differenzierbaren Funktion aber anabh"angig
vom Weg und damit vom Radius $r$ sein muss, folgt
\[
\oint_\gamma \frac{f(z)}{z-a}\,dz=2\pi i f(a).
\]
Wir haben damit den folgenden Satz bewiesen:

\begin{satz}[Cauchy]
Ist $\gamma$ ein geschlossener Weg in der komplexen Ebene, die ein
Gebiet berandet, in dem die komplexe Funktion $f(z)$ komplex
differenzierbar ist, dann gilt
\[
f(a)=\frac{1}{2\pi i}\oint_{\gamma}\frac{f(z)}{z-a}\,dz.
\]
Insbesondere sind die Werte einer komplex differenzierbaren Funktion 
im Inneren eines Gebietes durch die Werte auf dem Rand bereits vollst"andig
bestimmt.
\end{satz}

\subsubsection{Analytische Funktionen}
Bisher haben wir Potenzreihen nur f"ur reelle Funktionen betrachtet,
nicht f"ur komplexe Funktionen.
Die Eigenschaften komplex differenzierbarer Funktionen gestatten
uns jetzt aber auch, mehr "uber deren Entwickelbarkeit in Potenzreihen
zu erfahren.
Wir definieren daher

\begin{definition}
Eine komplexe Funktion $f(z)$ heisst im Punkt $z_0$ {\em analytisch}, wenn
sie in eine konvergente Potenzreihe
\[
f(z)=\sum_{k=0}^n a_kz^k
\]
entwickelt werden kann.
\end{definition}

Offenbar sind analytische Funktionen im Punkt $z_0$ beleibig oft
differenzierbar, und es gilt
\[
a_k=\frac{f^{(k)}(z_0)}{k!},
\]
die Taylorreihe von $f(z)$ ist also die urspr"ungliche Potenzreihe.
In der reellenn Analysis gibt es Funktionen, deren Taylerreihen zwar
konvergieren, aber nicht mit der urspr"unglichen Funktion "ubereinstimmen.
Wir m"ochten verstehen, dass dies im Komplexen nicht so ist.

Sei als $f(z)$ eine komplex differenzierbare Funktion, als Definitionsgebiet
nehmen wir der Einfachheit halber einen Kreis vom Radius $r$ um den Nullpunkt,
sein Rand ist die Kurve $\gamma$.
Durch ableiten der Cachyschen Integralformel finden wir
\begin{align*}
f(z)
&=
\frac1{2\pi i}\oint_{\gamma}\frac{f(\zeta)}{\zeta-z}\,d\zeta
\\
f'(zG)
&=
\frac1{2\pi i}\oint_{\gamma}\frac{f(\zeta)}{(\zeta-z)^2}\,d\zeta
\\
f'' (z)
&=
\frac1{2\pi i}\oint_{\gamma}2\frac{f(\zeta)}{(\zeta-z)^3}\,d\zeta
\\
f'''(z)
&=
\frac1{2\pi i}\oint_{\gamma}2\cdot 3\frac{f(\zeta)}{(\zeta-z)^4}\,d\zeta
\\
&\vdots
\\
f^{(k)}(z)
&=
\frac{k!}{2\pi i}\oint_{\gamma}\frac{f(\zeta)}{(\zeta-z)^{k+1}}\,d\zeta
\end{align*}
Es folgt

\begin{satz}
Eine komplex differenzierbare Funktion ist beliebig oft differenzierbar.
\end{satz}

Es gilt aber noch mehr:

\begin{satz}
Eine komplex differenzierbare Funktion $f(z)$, die in einer Kreisscheibe
vom Radius $r$ um den Nullpunkt definiert ist, ist analytisch.
Ihre Potenzreihenentwicklung
\[
f(z)=\sum_{k=0}^na_kz^k
\]
hat die Koeffizienten
\[
a_k=\frac1{2\pi i}\int_{\gamma}\frac{f(z)}{z^{k+1}}\,dz.
\]
\end{satz}

\begin{proof}[Beweis]
Die Funktion
\[
g(z)=\sum_{k=0}^na_kz^k
\]
ist komplex differenzierbar.
\end{proof}













