%
% kalman.tex -- Anwendung stochastischer DGL auf Kalman Filter
%
\section{Kalman-Filter\label{section:kalman}}
\rhead{Kalman-Filter}
Der klassische diskrete Kalmanfilter l"ost das Filterproblem f"ur ein
diskretes, lineares dynamisches System.
Das System wird mit Hilfe einer linearen Differenzengleichung beschrieben,
in der ein diskreter stochastische Prozess sowohl bei der Systementwicklung
wie auch beim Messen Einfluss nimmt.
Der Filter soll aus den Messungen den Systemzustand rekonstruieren, soweit
dies von den stochastischen Komponenten nicht verunm"oglicht wird.
Der Kalman-Filer l"ost auf optimale Weise: jede andere L"osung hat
eine gr"ossere Fehlervarianz.
Eine detaillierte Darstellung des Filterproblems wird gegeben im
Skript \cite{skript:wrstat}, die dort verwendete Notation wird auch
im Abschnitt~\ref{skript:diskreter-kalman-filter} verwendet.

Es liegt nahe, nach einer entsprechenden L"osung f"ur kontinuierliche
Systeme zu fragen.
Zwar ist es in sehr vielen F"allen in der Praxis m"oglich, dank
Diskretisierung mit dem diskreten Kalman-Filter eine gute L"osung
zu finden.
Dies wird verunm"oglicht wenn die Abtastrate zu gross wird, und die
Rechenleistung f"ur den mathematisch relativ aufwendigen
Kalman-Filter nicht zur Verf"ugung steht.

Die Theorie der stochastischen Differentialgleichungen verallgemeinert
stochastische lineare Differenzengleichungen zu stochastischen
Differentialgleichungen, und erm"oglicht so, das Filter-Problem f"ur
ein kontinuierliches System zu formulieren.
Im Folgenden soll gezeigt werden, wie das Filterproblem formuliert
werden muss, und es soll die L"osung des Filterproblems skizziert werden.
Die L"osung ist nicht vollst"andig, sie wird in einigen F"allen nur durch
Analogie dargestellt, f"ur die Details wird auf die weiterf"uhrende
Literatur verwiesen.
Eine gute Einf"uhrung in das Filterproblem ist das Buch \cite{skript:filter}
von Fridstedt, Jain und Krylow.
Eine detaillierte Untersuchung findet man in Kapitel~6 des Klassikers
\cite{skript:oksendal} von \O{}ksendal.

%
% Beschreibung des diskreten Kalman-Filters zum späteren Vergleich
% des stetigen Problems
%
\subsection{Der diskrete Kalman-Filter\label{skript:diskreter-kalman-filter}}
\index{diskretes Filterproblem}
Das klassische diskrete Filterproblem beschreibt ein System mit Hilfe
eines $n$-dimensionalen Vektors $x_k$ zu diskreten Zeitpunkten, spezifiziert
durch den Index $k$.
Die zeitliche Entwicklung wird durch ein Matrix $\varphi_k$
gem"ass
\begin{equation}
x_{k+1} = \varphi_k x_k + u_k
\label{stochastisch:diskrete-entwicklungs-gleichung}
\end{equation}
beschreiben, jedoch nur bis auf einen normalverteilten Fehler $u_k$.
Gemessen wird aber nicht $x_k$, sondern nur ein Teil oder eine
Linearkombination $H_kx_k$ der Komponenten von $x_k$, und auch wieder
nur mit einem unvermeidbaren normalverteilten Fehler $w_k$:
\[
z_k=H_kx_k + w_k.
\]
Die Aufgabe des Filterproblems besteht darin, die Werte $x_k$ mit m"oglichst
kleinem Fehler zu sch"atzen.
Die gesch"atzten Werte werden mit $\hat x_k$ bezeichnet.

Die L"osung besteht jeweils aus zwei Schritten. 
Zun"achst wird $\varphi_k$ dazu verwendet, eine Sch"atzung $\hat x_{k+1|k}$
des Systemszustand f"ur die Zeit $k+1$ nur auf Grund der zur Zeit
$k$ zur Verf"ugung stehenden Informationen zu bestimmen.
Die Matrix $\varphi_k$ liefert diese Information mittels
\[
\hat x_{k+1|k} = \varphi_k \hat x_k.
\]
Im zweiten Schritt verwendet man die Messung, um den Fehler zu korrigieren.
Dazu wird zun"achst berechnet, was gemessen werden m"usste, wenn
$\hat x_{k+1|k}$ bereits die richtige L"osung w"are.
Die Abweichung der vorhergesagten Messung $H_{k+1}\hat x_{k+1|k}$
von der tats"achlichen Messung $z_{k+1}$ ist
$z_{k+1}-H_{k+1}\hat x_{k+1|k}$r, daraus wird die Korrektur mit Hilfe
der Gain-Matrix $K_{k+1}$ nach
\index{Gain-Matrix}%
\[
\hat x_{k+1}
=
x_{k+1|k} + K_{k+1}(z_{k+1} - H_{k+1} \hat x_{k+1|k})
=
(I-K_{k+1}H_{k+1}) \varphi_{k}\hat x_k + K_{k+1}z_{k+1}
\]
berechnet.
Der Fehler der Sch"atzung $\hat x_{k+1}$ kann mit Hilfe der
Fehler-Kovarianzmatrizen $Q_k=E(u_ku^t_k)$ der Systemfehler $u_k$
\index{Fehler-Kovarianz}
und $R_k=E(w_kw_k^t)$ der Messfehler $w_k$ bestimmt werden.
Die Fehlerkovarianz ist
\begin{align*}
P_{k+1|k}
&=
\varphi_k P_k \varphi_k^t + Q_k
&
&\text{Fehler der Vorhersage}
\\
P_{k+1}
&=
(I-K_{k+1}H_{k+1})P_{k+1|k}(I-K_{k+1}H_{k+1})^t + K_{k+1}R_{k+1}K_{k+1}^t
&
&\text{Fehler nach Korrektur}
\end{align*}
Die optimale L"osung wird f"ur die {\em Kalman-Matrix}
\index{Kalman-Matrix}
\[
K_{k+1}
=
P_{k+1|k}H_{k+1}^t(H_{k+1}P_{k+1|k}H_{k+1}^t + R_{k+1})^{-1}
\]
als Gain-Matrix gefunden, und der Sch"atzfehler ist
\[
P_{k+1}=(I-K_{k+1}H_{k+1})(\varphi_{k}P_k\varphi_k^t+Q_k).
\]

Die L"osung des Filterpoblem in dieser Form ist nicht direkt f"ur ein
kontinuierliches System verallgemeinerungsf"ahig, denn Vorhersage und 
Korrektur erfolgen in dieser zeitlichen Reihenfolge.
Bei einem kontinuierlichen System ist dies nicht m"oglich.
Stattdessen muss Korrektur und Messung sozusagen gleichzeitig erfolgen,
und wir m"ussen erst noch ein mathematische Formulierung daf"ur finden.

%
% Formulierung des stetigen Filterproblems, insbesondere die Formulierung
% als stochastisches Differentialgleichugnssystems, aber auch das Problem
% der bedingten Erwartung.
%
\subsection{Das kontinuierliche Filterproblem}
Wir suchen jetzt eine kontinuierliche Entsprechung f"ur die diskrete
Entwicklungsgleichung~\eqref{stochastisch:diskrete-entwicklungs-gleichung}.
Gesucht ist der zeitabhn"agige $n$-dimensional Systemzustand $X(t)$. 
Die Entwicklung des Systems wird durch eine lineare Differentialgleichung
\[
\frac{dX(t)}{dt}
=
F(t) X(t)
\]
mit einer zeitabh"angigen $n\times n$-Matrix $F(t)$ beschrieben.
Die Entwicklung wird nun aber nicht exakt eingehalten, den Fehler
modellieren wir mit Hilfe eines Wiener-Prozesses.
Dazu m"ussen wir die Differentialgleichung jedoch wieder in Inkrementform
schreiben:
\begin{equation}
dX(t) = F(t) X(t)\,dt + C(t)\, dU(t),
\label{stochastisch:kontinuierliche-zeitentwicklung}
\end{equation}
worin $U(t)$ ein $n$-dimensionaler Wiener-Prozess ist und $C(t)$
eine $n\times n$-Matrix.

Nun ist aber nicht $X(t)$ bekannt, sondern nur eine $m$-dimensionale
Messung $Z(t)$, die jedoch auch nicht exakt ist.
Die Messung h"angt linear vom Systemzustand ab, es gibt also eine
$m\times n$-Matrix $G(t)$ mit 
\[
Z(t) = G(t) X(t).
\]
Der Matrix $G(t)$ entspricht im diskreten Filterproblem die Matrix $H_k$.

Wir m"ussen jetzt aber noch den Messfehler modellieren.
Dieser sollte im Wesentlichen ein Wiener-Prozess sein, wir k"onnen daher
den Messprozess durch die stochastische Differentialgleichung
\begin{equation}
dZ(t)
= 
G(t)X(t) \,dt +  D(t)\,dV(t)
\label{stochastisch:kontinuierlicher-messprozess}
\end{equation}
beschreiben,
worin $V(t)$ ein $m$-dimensionaler Wiener-Prozess ist und $D(t)$ 
eine $m\times m$-Matrix.

Den Prozess $X(t)$ k"onnen wir grunds"atzlich nicht kennen, sondern
nur eine Sch"atzung $\hat X(t)$ davon.
Die Filter-Aufgabe besteht also darin, einen Prozess $\hat X(t)$ zu
finden mit den folgenden Eigenschaften:
\begin{enumerate}
\item
Die Sch"atzung $\hat X(t)$ ist im Mittel richtig:
$E(\hat X(t)) = E(X(t))$.
\item
$\hat X(t)$ h"angt nur von Messungen $Z(s)$ f"ur $s\le t$ ab.
\item
Unter allen Sch"atzprozessen $Y(t)$ mit den Eigenschaften 1 und 2 ist
$\hat X(t)$ derjenige, f"ur den der quadratische Fehler 
$E((Y(t)-\hat X(t))^2)$  minimal wird.
\end{enumerate}
Im Folgenden soll versucht werden, eine L"osung des Filterproblems zu
finden.

%
% Lösung des Filterproblems nach Kalman-Bucy, insbesondere die Entwicklung
% des Fehlers (Riccati-Differentialgleichung)
%
\subsection{L"osung des Filterproblems\label{stochastisch:loesung-filterproblem}}
In diesem Abschnitt beschreiben wir die L"osung des Filterproblems
f"ur den eindimensionalen Fall ($n=m=1$).

Ohne die Information aus dem Messprozess $Z(t)$ steht f"ur die 
Sch"atzung nur noch die
Entwicklungsgleichung~\eqref{stochastisch:kontinuierliche-zeitentwicklung}
zur Verf"ugung.
Ausserdem ist der Systemfehler unbekannt, so dass nur noch die gew"ohnliche
lineare Differentialgleichung
\begin{equation}
d\hat X(t) = F(t)\hat X(t)\,dt
\label{stochastisch:homogene-entwicklung}
\end{equation}
bleibt.

Die Gleichung~\eqref{stochastisch:homogene-entwicklung} muss erweitert
werden um einen Term, der die Abweichung der Messung misst.
Ohne die Messfehler m"usste
\[
dZ(t) = G(t)\hat X(t)\,dt
\]
sein.
Als Basis f"ur die Korrektur muss die Differenz zur tats"achlichen
Messung verwendet werden, also
\begin{equation}
dN(t)
=
dZ(t) - G(t)\hat X(t)\,dt
=
G(t)(X(t)-\hat X(t))\,dt + D(t)\,dV(t)
.
\label{stochastisch:innovations-prozess}
\end{equation}
Die Korrektur muss linear von \eqref{stochastisch:innovations-prozess}
abh"angen, die stochastische Differentialgleichung f"ur $\hat X(t)$
hat daher die Form
\begin{align*}
d\hat X(t)
&=
F(t)\hat X(t)\,dt + K(t)\,dN(t),
\\
&=
F(t)\hat X(t)\,dt
+
K(t)\,dZ(t) - K(t)G(t)\hat X(t)\,dt
\\
&=
(F(t)-K(t)G(t))\hat X(t)\,dt
+
K(t)\,dZ(t)
\end{align*}
mit einer noch zu bestimmenden Funktionen $K(t)$.
$N(t)$ heisst der Innovations-Prozess.
Die Funktion $K(t)$ muss so bestimmt werden, dass die Bedingungen
1--3 an den Sch"atzprozess erf"ullt sind.

Der Sch"atzfehler zur Zeit $t$ ist der Erwartungswert
\[
S(t) = E((X(t)-\hat X(T))^2).
\]
Man kann zeigen, dass $S(t)$ die gew"ohnliche Differentialgleichung
\begin{equation}
\frac{dS}{dt}
=
2F(t)S(t) - \frac{G(t)^2}{D(t)^2}S(t)^2 +C(t)^2
\label{stochastisch:ricatti}
\end{equation}
erf"ullt.
\eqref{stochastisch:ricatti} ist eine Riccati-Differentialgleichung.
Mit $S(t)$ kann das Filterproblem gel"ost werden, es gilt
\begin{equation}
d\hat X(t)
=
\biggl(F(t)-\frac{G(t)^2S(t)}{D(t)^2}\biggr)\hat X(t)\,dt 
+
\frac{G(t)S(t)}{D(t)^2}\,dZ(t)
\end{equation}






