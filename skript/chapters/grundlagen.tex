%
% grundlagen.tex -- Grundlagen ueber Differentialgleichungen
%
% (c) 2015 Prof Dr Andreas Mueller, Hochschule Rapperswil
%
\chapter{Grundlagen der Theorie der gew"ohnlichen Differentialgleichungen
\label{chapter:grundlagen}}
\lhead{}
\rhead{Grundlagen}
\section{Differentialgleichungen\label{section:differentialgleichungen}}
Eine gew"ohnliche Differentialgleichung f"ur eine reellwertige
Funktion $y(x)$ stellt einen Zusammenhang her zwischen der Funktion
und ihren Ableitungen.
Wir schreiben die Ableitungen als $y'$, $y''$, $y'''$ und $y^{(n)}$
f"ur die $n$-te Ableitung.
Wir lassen oft das Argument der Funktion weg.
Beispiele von Differentialgleichungen sind
\begin{align*}
y'&=-Ny
&&\text{Ordnung: $1$}
\\
y''&=-\omega^2 y
&&\text{Ordnung: $2$}
\\
x^2y''+xy'+(x^2-n^2)y&=0
&&\text{Ordnung: $2$}
\end{align*}
Die Abh"angigkeit kann in expliziter Form als
\begin{equation}
y^{(n)}=f(x,y,y',\dots,y^{(n-1)})
\label{grundlagen:explizit}
\end{equation}
oder in impliziter Form
\[
F(x,y,y',\dots,y^{(n)})=0
\]
gegeben sein.
Die Ordnung einer Differentialgleichung ist die h"ochste vorkommende
Ableitung.

Insbesondere in Anwendungen in der Physik ist die Zeit die
unabh"angige Variable.
Die abh"angige Variable ist dann zum Beispiel die Ortskoordinate
$x(t)$ und wir bezeichnen ihre Ableitungen mit $\dot{x}(t)$ f"ur
die Geschwindigkeit, $\ddot{x}(t)$ f"ur die Beschleunigung.
Dieses Beispiel suggeriert auch, dass die abh"angige Variable 
ein Vektor sein kann, den man als den Ortsvektor eines Teilchens
interpretieren kann.
Die Funktion $f(t,x,\dots,x^{(n-1)})$ ist dann auch vektorwertig, und
alle Argumente ausser dem ersten von $f$ sind vektorwertig.

Eine Differentialgleichung $n$-ter Ordnung f"ur eine skalare Funktion
kann in eine Vektor-Differentialgleichung erster Ordnung f"ur eine
$n$-dimensionale vektorwertige Funktion umgewandelt werden.
Ist $y(x)$ die gesuchte Funktion in der
Differentialgleichung~(\ref{grundlagen:explizit}), dann kann man
den Vektor
\[
u(x)=\begin{pmatrix}
y(x)\\y'(x)\\\vdots\\y^{(n-1)}(x)
\end{pmatrix}
\in\mathbb R^n
\]
bilden.
Er erf"ullt die Differentialgleichung
\begin{equation}
\frac{d}{dx}\begin{pmatrix}
y\\y'\\\vdots\\y^{(n-1)}
\end{pmatrix}
=
\begin{pmatrix}
y'\\y''\\\vdots\\y^{(n)}
\end{pmatrix}
=
\begin{pmatrix}
y'\\y''\\\vdots\\f(x,y,y',\dots,y^{(n-1)}.
\end{pmatrix}
\label{grundlagen:vektordgl}
\end{equation}
Der Vektor auf der rechten Seite h"ang nur von $x$, der Funktion $y$
und ihren Ableitungen bis zur $n-1$-ten Ordnung ab, also von $u$, man
kann (\ref{grundlagen:vektordgl}) daher als
\begin{equation}
\frac{d}{dx}u=\tilde{f}(x,u)
\end{equation}
schreiben.

\section{Anfangswertprobleme\label{section:anfangswertprobleme}}

\section{Randwertprobleme\label{section:randwertprobleme}}

\section{Analytische L"osungsverfahren\label{section:analytischeverfahren}}
\subsection{Separation der Variablen}
Differentialgleichungen erster Ordnung lassen sich oft durch sogenannte
Trennung der Variablen auf die Berechnung von Integralen reduzieren.
Dank der Schreibweise der Ableitung als Differentialquotient wird
dieser L"osungsweg sehr suggestiv.
Wir betrachten als Beispiel die Differentialgleichung
\[
y'=-Ny.
\]
Schreibt man die Ableitung als Differentialquotient, wird daraus die
Gleichung
\[
\frac{dy}{dx}=-Ny.
\]
Durch Division durch $y$ und formale Multiplikation mit $dx$ wird daraus
die formale Gleichung
\begin{equation}
\frac{dy}{y}=-N\,dx.
\label{grundlagen:separiert}
\end{equation}
In dieser Gleichung kommt die Variable $y$ nur auf der linken, die Variable
$x$ nur auf der rechten Seite vor.
Man sagt, die Variablen seien {\em separiert}.
\index{separiert}
\index{Variablen, Separation der}
Man beachte, dass die Gleichung (\ref{grundlagen:separiert}) nur eine
formale Bedeutung haben kann, die Symbole $dy$ und $dx$ sind ja keine Zahlen,
mit denen man algebraische Operationen durchf"uhren k"onnte.
Mit etwas Vorsicht angewandt f"uhrt dieser Kalk"ul aber nicht auf
Widerspr"uche.

Wir integrieren jetzt beide Seiten von (\ref{grundlagen:separiert}), und
erhalten 
\[
\int\frac1y\,dy=-N\int\,dx
\]
Beide Integrale lassen sich in geschlossener Form auswerten:
\[
\log|y|=-Nx+C.
\]
Aufgel"ost nach $y$ ergibt sich
\[
y=\pm e^{C}e^{-Nx},
\]
wobei die beiden Vorzeichen $\pm$ das Betragszeichen in der Stammfunktion
von $\frac1y$ reflektieren.
Man kann den Faktor $\pm e^{C}$ in eine neue Konstante $a$ zusammenfassen,
und erh"alt somit als L"osung der urspr"unglichen Differentialgleichung
die Familie
\[
y(x)=ae^{-Nx}
\]
von Funktionen.
Der Paramter $a$ muss mit Hilfe der Anfangsbedingung festgelegt werden.

\subsection{Lineare Differentialgleichungen}
Eine Differentialgleichung der Form
\begin{equation}
a_n(x)y^{(n)}+a_{n-1}(x)y^{(n-1)}+\dots+a_2(x)y''+a_1(x)y'+a_0(x)=f(x)
\label{grundlagen:linearedgl}
\end{equation}
heisst {\em lineare Differentialgleichung}.
\index{lineare Differentialgleichung}
Ist $f(x)=0$, nennt man die Differentialgleichung {\em homogen}, die
Funktion $f(x)$ wird auch die {\em Inhomogenit"at} genannt.
\index{homogen Differentialgleichung}
\index{Inhomogenitat@Inhomogenit\"at}
Die L"osungsmenge einer homogenen linearen Differentialgleichung
bildet einen Vektorraum: jede Linearkombination von L"osungen
ist wieder eine L"osung.
Seien zum Beispiel $y_1(x)$ und $y_2(x)$ L"osungen der Differentialgleichung
(\ref{grundlagen:linearedgl}).
Wir m"ochten zeigen, dass
$y(x)=\alpha y_1(x)+\beta y_2(x)$ eine L"osung ist.
Die Ableitungen selbst sind linear:
\begin{align*}
y^{(k)}&=\alpha y_1^{(k)}(x)+\beta y_2^{(k)}(x).
\end{align*}
Setzt man dies in die Differentialgleichung ein, erh"alt man
\begin{align*}
a_ny^{(n)}+\dots+a_1y'+a_0y
&=
a_n\alpha y_1^{(n)}+a_n\beta y_2^{(n)}+\dots+a_1\alpha y_1'+a_1\beta y_2'
+ a_0\alpha y_1+a_0\beta y_2
\\
&=
\alpha(\underbrace{a_ny_1^{(n)}+\dots+a_1y_1'+a_0y_1}_{=0})
+
\beta(\underbrace{a_ny_2^{(n)}+\dots+a_1y_2'+a_0y_2}_{=0})=0,
\end{align*}
die Linearkombination $y$ erf"ullt also die homogene Differentialgleichung
ebenfalls.


\subsection{Variation der Konstanten}
\subsection{Laplace-Transformation}
