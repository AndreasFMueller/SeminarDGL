%
% potenzreihen.tex -- Lösung von Differentialgleichungen mit Potenzreihen
%
% (c) 2015 Prof Dr Andreas Mueller, Hochschule Rapperswil
%
\chapter{Potenzreihen-Methode\label{chapter:potenzreihen}}
\lhead{}
\rhead{Potenzreihen-Methode}
Die meisten Differentialgleichungen k"onnen nicht in geschlossener
Form gel"ost werden.
Viele davon sind aber von generischer Bedeutung, sie treten
in verschiedenen Anwendungen immer wieder auf, und verdienen daher
genauer studiert zu werden.
In solchen F"allen kann man die L"osungsfunktionen einfach also
neue interessante Funktionen definieren, ihnen einen Namen geben
und sie verwenden wie zum Beispiel die trigonometrischen Funktionen.
Um sie aber auch berechnen zu k"onnen, braucht man irgend eine 
effiziente Darstellung.
Potenzreihen haben sich daf"ur als besonders geeignet herausgestellt.

In diesem Kapitel wird gezeigt, wie Differentialgleichungen mit Potenzreihen
gel"ost werden k"onnen.
Es zeigt, dass die trigonometrischen Funktionen als Resultat
dieser L"osungsmethode gefunden werden, und es verallgemeinert die
Methode soweit, dass auch kompliziertere Funktionen wie die
Bessel-Funktionen damit verstanden werden k"onnen.

\section{Analytische L"osungen
\label{section:potenzreihen:analytisch}}

\section{Trigonometrische Funktionen
\label{section:potenzreihen:trigo}}

\section{Verallgemeinerte Potenzreihen
\label{section:potenzreihen:verallgemeinert}}
Viele Differentialgleichungen in der Anwendungen sind von der Form
\begin{equation}
x^2y''+p(x)xy'+q(x)y=0,
\label{potenzreihen:verallgemeinert-dgl}
\end{equation}
wobei $p(x)$ und $q(x)$ beliebige Funktionen von $x$ sind.
Das h"auffige Auftreten der Kombinationen $x^2y''$ und $xy'$ in Anwendungen
wird durch ein Dimensionsargument verst"andlich.
Schreiben wir $[u]$ f"ur die Masseinheit einer Gr"osse, dann gilt
\begin{equation*}
\begin{aligned}
\biggl[\frac{dy}{dx}\biggr]
&=
\frac{[y]}{[x]}
&
&\Rightarrow&
[xy']&=[y]\\
\biggl[\frac{d^2y}{dx^2}\biggr]
&=
\frac{[y]}{[x]^2}
&
&\Rightarrow&
[x^2y'']&=[y]
\end{aligned}
\end{equation*}
Wenn zwischen den Termen $x^2y''$, $xy'$ und $y$ irgend ein
gesetzm"assiger Zusammenhang bestehen soll, dann ist die wohl
einfachste Form davon eine lineare Beziehung mit dimensionslosen
Koeffizienten wie in~(\ref{potenzreihen:verallgemeinert-dgl}).

Die Differentialgleichung~(\ref{potenzreihen:verallgemeinert-dgl})
liefert f"ur den Punkt $x=0$ keinen Wert von $y''$.
Eine L"osungsfunktion $y(x)$, die die Differentialgleichung f"ur
$x\ne 0$ erf"ullt, erf"ullt sie f"ur alle $x$, wenn $y(0)=0$ der
wenn $q(0)=0$ ist. 
Insbesondere haben die Ableitungen an der Stelle $x=0$ keinen
Einfluss darauf, ob die Differentialgleichung erf"ullt ist.
Man spricht auch von einer ausserwesentlichen Singularit"at an
der Stelle $x=0$.
Wir k"onnen allerdings nicht mehr annehmen, dass die L"osung als
Potenzreihe in $x$ geschrieben werden kann, denn eine solche
L"osung w"urde beliebige Ableitungen im Punkt $x=0$ haben, was 
die Differentialgleichung nicht garantieren kann.

Wir suchen jetzt eine L"osung der
Differentialgleichung~(\ref{potenzreihen:verallgemeinert-dgl})
in Form einer verallgemeinerten Potenzreihe der Form
\begin{equation}
y(x)=x^\varrho\sum_{k=0}^\infty a_kx^k.
\label{potenzreihen:verallgemeinert}
\end{equation}
Ausser den Koeffizienten $a_k,\;k\ge 0$ ist auch der Exponent
$\varrho$ zu bestimmen.
Wir k"onnen $y(x)$ und seine Ableitungen auch als
\begin{equation}
\left.
\begin{aligned}
y(x)
&=
\sum_{k=0}^\infty a_kx^{\varrho+k}
\\
y'(x)
&=
\sum_{k=0}^\infty (\varrho+k)a_kx^{\varrho+k-1}
\\
y''(x)
&=
\sum_{k=0}^\infty (\varrho+k)(\varrho+k-1)a_kx^{\varrho+k-2}
\end{aligned}
\right\}
\label{potenzreihen:verallgemeinert-ableitungen}
\end{equation}
schreiben, und vermeiden so eine explodierende Zahl von Termen, die
die Produktregel hervorbringen w"urde.
Wir setzen (\ref{potenzreihen:verallgmeinert-ableitungen})
in die Differentialgleichung ein und erhalten
\begin{equation}
\sum_{k=0}^\infty (\varrho+k)(\varrho+k-1)a_kx^{\varrho+k}
+p(x)
\sum_{k=0}^\infty (\varrho+k)a_kx^{\varrho+k}
+q(x)
\sum_{k=0}^\infty a_kx^{\varrho+k}
=0
\label{potenzreihen:verallgemeinert-reihen}
\end{equation}
Offenbar m"ussen wir jetzt zus"atzliche Annahmen "uber die
Funktionen $p(x)$ und $q(x)$ machen.
Wir nehmen an, dass sich beide als Potenzreihen schreiben lassen,
wir setzen
\begin{align*}
p(x)&=\sum_{k=0}^\infty p_kx^k
&
q(x)&=\sum_{k=0}^\infty q_kx^k
\end{align*}
Einsetzen in die Gleichungen~(\ref{potenzreihen:verallgemeinert-reihen})
ergibt
\begin{equation}
\sum_{k=0}^\infty (\varrho+k)(\varrho+k-1)a_kx^{\varrho+k}
+\sum_{s=0}^\infty
\sum_{k=0}^\infty p_s(\varrho+k)a_kx^{\varrho+k+s}
+
\sum_{s=0}^\infty
\sum_{k=0}^\infty q_sa_kx^{\varrho+k+s}
=0.
\end{equation}


