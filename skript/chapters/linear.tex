%
% linear.tex -- L"osung linearer Differentialgleichungen
%
% (c) 2015 Prof Dr Andreas Mueller, Hochschule Rapperswil
%
\chapter{Differentialgleichungen und lineare Algebra\label{chapter:linear}}
\lhead{}
\rhead{Lineare Differentialgleichungen}
Es lohnt sich, lineare Differentialgleichungen unter Zuhilfenahme
der linearen Algebra etwas genauer zu untersuchen.
Gel"ost werden soll das lineare Differentialgleichungssystem
\begin{equation}
\frac{d}{dt}x = A(t)x + f(t)\qquad x(0)=x_0
\label{linear:gleichung}
\end{equation}
gel"ost werden.
Darin ist $A(t)$ eine $n\times n$-Matrix, wir nehmen der Einfachheit
halber an, dass die Matrixelemente $a_{kl}(t)$ von $A(t)$ differenzierbare
Funktionen von $t$ sind.
Die Funktion $f(t)$ hat $n$-dimensionale Vektoren als Werte.
Falls $f(t)=0$ ist, ist die Gleichung~(\ref{linear:gleichung}) 
homogen.

\section{Konstante Koeffizienten}
In diesem Abschnitt nehmen wir an, dass $A(t)$ konstant ist, wir schreiben
einfach $A$ f"ur den konstanten Wert von $A(t)$.
\subsection{Diagonalisierbare Koeffizientenmatrix}
Besonders einfach w"are (\ref{linear:gleichung}) zu l"osen, wenn $A$
Diagonalform h"atte,
\[
A=\begin{pmatrix}
\lambda_1&         &      &         \\
         &\lambda_2&      &         \\
         &         &\ddots&         \\
         &         &      &\lambda_n
\end{pmatrix}.
\]
Durch Wechsel der Basis kann man diese Situation immer erreichen, wenn
$A$ diagonalisierbar ist.

In diesem Fall zerf"allt das Gleichungssystem in $n$ unabh"angige
lineare Differentialgleichungen erster Ordnung
\begin{align*}
\dot{x}_1&=\lambda_1x_1 + f_1(t)\\
\dot{x}_2&=\lambda_2x_2 + f_2(t)\\
&\;\vdots\\
\dot{x}_n&=\lambda_nx_n + f_n(t).
\end{align*}
Die homogenen Gleichungen l"assen sich sofort l"osen:
\[
\dot{x}_i=\lambda_ix_i
\qquad
\Rightarrow
\qquad
x_i(t)=x_{0i}e^{\lambda_it}.
\]
Daraus kann man die L"osung des inhomogenen Systems mit Hilfe der
Variation der Konstanten l"osen:
\[
x_i(t)=C_i(t)e^{\lambda_i t}
\qquad
\Rightarrow
\qquad
\dot{x}_i(t)
=
\dot{C}_i(t)e^{\lambda_i t}+C_i(t)\lambda_ie^{\lambda_i t}
=
\lambda_i x_i(t) + \dot{C}_i(t)e^{\lambda_i t}
\]
Die $C_i(t)$ m"ussen also so gew"ahlt werden, dass
$\dot{C}_i(t)=f_i(t)e^{-\lambda_i t}$, es folgt
\[
C_i(t)=x_{0i}+\int_0^t C_i(\tau)e^{-\lambda_i \tau}\,d\tau
\]
und damit die L"osung
\[
x_i(t)
=
\biggl(x_{0i}+\int_0^t C_i(\tau)e^{-\lambda_i \tau}\,d\tau\biggr)
e^{\lambda_i t}.
\]
Diese L"osung hatten wir fr"uher bereits gefunden, doch wollen wir
dies auch noch in Matrixform schreiben.
Dazu setzen wir
\[
e^{At}=\begin{pmatrix}
e^{\lambda_1 t}&               &      &               \\
               &e^{\lambda_1 t}&      &               \\
               &               &\ddots&               \\
               &               &      &e^{\lambda_n t}
\end{pmatrix}.
\]
Mit dieser Schreibweise kann die L"osung geschrieben werden als
\[
x(t)
=
e^{At}
\biggl(x_0 + \int_0^t e^{-A\tau}f(\tau)\,d\tau\biggr)
\]

\subsection{Matrix-Exponentialfunktion}
Die L"osung im vorangegangenen Abschnitt konnte mit Hilfe der
Matrix-Exponentialfunktion besonders kompakt formuliert werden.
Allerdings war diese nur f"ur diagonalisierbare Matrizen definiert.
Wir k"onnten $e^A$ aber auch "uber die Potenzreihe definieren
\[
e^A=E+A+\frac12A+\frac1{3!}A^3+\frac1{4!} A^4+\dots
\]
Die Ableitung von $e^{At}$ nach $t$ ist
\begin{align*}
\frac{d}{dt}e^{At}
&=
\frac{d}{dt}\biggl(E+At+\frac1{2!}t^2A^2+\frac1{3!}t^3A^3+\dots\Biggr)
\\
&=
A+tA^2+\frac12t^2A^3+\frac1{3!}t^3A^4+\dots
\\
&=
\biggl(E+tA+\frac12t^2A^2+\frac1{3!}A^3+\dots\biggr)A
=e^{tA}A=Ae^{At}
\end{align*}
Die Matrix-Funktion $t\mapsto X(t)=e^{At}$ erf"ullt also die
Matrix-Differentialgleichung
\begin{equation}
\frac{d}{dt} X = AX,
\label{linear:matrix-dgl}
\end{equation}

\subsection{Beliebige Koeffizientenmatrix}
Mit der Matrix-Exponentialfunktion kann jetzt auch das homogene
Differentialgleichungssystem gel"ost werden.
Dazu multiplizieren wir~(\ref{linear:matrix-dgl}) auf der rechten Seite
$x_0$ 
\[
\frac{d}{dt}X(t)x_0 = A X(t)x_0,
\]
Damit ist die Funktion
\[
x(t)=X(t)x_o = e^{At}x_0
\]
eine L"osung der homogenen Gleichung.

Auch das Verfahren der Variation der Konstanten kann durchgef"uhrt werden.
Dazu ersetzen wir $x_0$ durch einen Vektor $C(t)$, und leiten den Ansatz
\[
x(t)=e^{At}C(t)
\]
nach $t$ ab
\[
\frac{d}{dt}x(t)
=
Ae^{At}C(t) + e^{At}\dot{C}(t)
=
A x(t) + e^{At}\dot{C}(t)
\]
Damit $x(t)$ eine L"osung ist, muss gelten
\[
\begin{aligned}
e^{At}\dot{C}(t)
&=
f(t)
&&\text{und}
&
C(0)0=x_0
\end{aligned}
\]
Diese Differentialgleichung kann durch Integration gel"ost werden,
wir erhalten
\begin{align*}
\dot{C}(t)
&=
e^{-At}f(t)
\\
C(t)
&=
x_0+\int_0^t e^{-A\tau}f(\tau)\,d\tau
\end{align*}
und damit als L"osung der inhomogenen Differentialgleichung
\[
x(t)
=
e^{At}\biggl(
x_0+\int_0^t e^{-A\tau}f(\tau)\,d\tau
\biggr).
\]
Die fr"uher gefundene Formel f"ur die L"osung der Differentialgleichung
(\ref{linear:gleichung}) f"ur den Spezialfall diagonalisierbarer 
Matrizen gilt daher f"ur beliebige Matrizen.

\section{Spezialf"alle}

\section{Zeitabh"angige Koeffizienten}
Die L"osung der Differentialgleichung~(\ref{linear:gleichung}) war
deshalb einfach, weil wir die Matrixdifferentialgleichung
\[
\frac{d}{dt} X=AX
\qquad
\text{mit}
\qquad
X(0)=E
\]
mit der Exponentialfunktion sofort l"osen k"onnten.
F"ur eine beliebige Funktion $A(t)$ k"onnen wir nicht mehr mit der
Exponentialfunktion arbeiten.
Wir k"onnen uns aber daran erinnern, dass die Exponentialfunktion
eine L"osung einer Matrixdifferentialgleichung war.
Wir suchen daher $X(t)$ als L"osung der Differentialgleichung
\[
\frac{d}{dt}X(t)=A(t)X(t)
\qquad\text{und}\qquad
X(0)=E.
\]
Die inverse Matrix $X(t)^{-1}$ kann man ebenfalls als L"osung einer
Differentialgleichung finden.
Die Ableitung der Beziehung
\[
X(t)X(t)^{-1}=E
\]
ist
\[
0
=
\frac{d}{dt}X(t)X(t)^{-1}
=
\frac{d}{dt} X(t) X(t)^{-1}
+
X(t) \frac{d}{dt}X(t)^{-1}
=
A(t)X(t)X^{-1}(t)
+
X(t) \frac{d}{dt}X(t)^{-1}
=
A(t)+ X(t)\frac{d}{dt}X(t)^{-1}
\]
Die Inverse $X(t)^{-1}$ erf"ullt daher die Differentialgleichung
\[
\frac{d}{dt}X(t)^{-1}= X(t)^{-1}A(t).
\]
Die Inverse kann also genauso als L"osung einer Differentialgleichung
gefunden werden wie $X(t)$.

Dann ist $x(t)=X(t)x_0$ wieder eine L"osung der homogenen Gleichung.
Ausserdem k"onnen wir wieder versuchen, die L"osung der inhomogenen
Gleichung mit Variation der Konstanten zu finden.
Dazu setzen wir $x(t)=X(t)C(t)$ an, und berechnen die Ableitung
\[
\frac{d}{dt}x(t)
=
\dot{X}(t)C(t)+X(t)\dot{C}(t)
=
A(t)X(t)C(t)+X(t)\dot{C}(t)
=
A(t)x(t)+X(t)\dot{C}(t)
\]
Wieder muss $X(t)\dot{C}(t)=f(t)$ gelten, wenn $x(t)$ eine L"osung sein
soll.
Die Gleichung
\[
\frac{d}{dt}C(t)
=
X(t)^{-1}f(t)
\]
Kann durch Integration gel"ost werden:
\[
C(t)
=x_0+\int_0^t X(t)^{-1}(\tau)f(\tau)\,d\tau,
\]
so dass die L"osung der inhomogenen Differntialgleichung
\[
x(t)=X(t)\biggl(x_0+\int_0^t X(\tau)^{-1} f(\tau)\,d\tau\biggr)
\]
wird.




