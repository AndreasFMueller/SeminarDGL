%
% stochastisch.tex -- Kapitel ueber stochastische Differentialgleichungen
%
\chapter{Stochastische Differentialgleichungen\label{chapter:stochastisch}}
\lhead{Stochastische Differentialgleichungen}
\rhead{}
In vielen Anwendungen wird die Bewegung eines Systems auch von
zuf"alligen Einfl"ussen bestimmt, die man oft auch Rauschen nennt.
Die Natur des Rauschen bedeutet, dass aufeinanderfolgende inkremente
v"ollig unkorreliert sind, w"ahrend Inkremente einer differenzierbaren
Funktion voneinander abh"angig sind.
Die L"osung einer Differentialgleichung unter Einfluss von Rauschen 
kann daher niemals eine differenzierbare Funktion sein, und sie kann
niemals eine L"osung der Differentialgleichung im bisher verwendeten
Sinn sein.
Um der Idee einen mathematischen Sinn zu geben, der auch erlaubt,
solche Differentialgleichungen zu l"osen und in Anwendungen
einzusetzen, muss daher zuerst gekl"art werden, was Rauschen genau ist.
Anschliessend muss das Konzept einer Differentialgleichung so formuliert
werden, dass es auch f"ur nicht differenzierbare Funktionen und Rauschen
anwendbar ist.

%
% Ein Modell f"ur Rauschen
%
\section{Modell f"ur Rauschen: der Wiener-Prozess}

%
%
%
\section{Stochastische Differentialgleichungen}


%
%
%
\section{Kalman-Filter}





