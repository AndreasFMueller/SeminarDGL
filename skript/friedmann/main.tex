\chapter{Friedmann-Gleichung\label{chapter:thema}}
\lhead{Friedmann-Gleichung}
\begin{refsection}
\chapterauthor{Andri Hartmann und Tobias Schuler}
\printbibliography[heading=subbibliography]
\section{Expansion des Universums}
Seit langem stellte man sich die Frage, was das Universum ist und wie es sich verh\"{a}lt. Als 1912 Albert Einstein die Relativit\"{a}tstheorie herleitete, konnten gewisse Fragen beantwortet werden. Unter anderem bewies Edwin Hubble mit der kosmologischen Rotverschiebung im Jahre 1929, dass das Universum expandiert.
\subsection{Kosmologische Rotverschiebung}
Die Kosmologische Rotverschiebung ist nicht zu verwechseln mit dem Dopplereffekt. Grund daf\"{u}r ist, dass sich die Galaxien nicht in der Raumzeit voneinander entfernen, sondern sich der Raum ausdehnt. 
\section{Friedmann-Gleichung}
\begin{equation}
\left(\frac{a'}{a}\right) ^ 2 = \frac{8 \pi G}{3} \rho - \frac{k c^2}{a^2} + \frac{\Lambda c^2}{3}
\end{equation}
\end{refsection}

