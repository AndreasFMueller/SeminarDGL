\chapter{Richtcharakteristik kreisf"ormiger Antennen\label{chapter:kreis}}
\lhead{Richtcharakteristik kreisf"ormiger Antennen}
\begin{refsection}
\chapterauthor{Kevin Cina und Benjamin R"aber}
\\
%
%
%
In diesem Kapitel geht es um die \emph{Richtcharakteristik von kreisf\"ormigen Antennen}.
Vorweg genommen bekommt man die \emph{Besselfunktion} als L\"osung.
Die Herleitung ist aber allgemein f\"ur aller zylindrische K\"orper anwendbar.
F\"ur das bessere Verst\"andnis wird zuerst der Weg von einer rechteckigen Pauke zu einer kreisf\"ormigen Pauke gemacht und erst danach kommen wir zum Thema \emph{Richtcharakteristik von kreisf\"ormigen Antennen}.
%
%
%
%Potenzreihenherleitung
\section{Potenzreihenherleitung der Besselfunktion}
Dieser Abschnitt befasst sich mit der Potenzherleitung der Besselfunktion. 
Bei der Herleitung werden Methoden, die in den vorherigen Kapitel erl\"autert wurden, 
verwendet. Zuerst m\"uss eine Differenzialgleichung aufgestellt werden.
Die Besselfunktion kommt aus der Wellengleichung.
%
Die Differenzialgleichung für die Besselfunktion lautet wiefolgt:
\begin{align}
	r^2 \cdot R''\left( r \right)
	+
	r \cdot R' \left( r \right)
	+
	\left( r^2 - n^2 \right) \cdot R \left( r \right)
	=
	0
	\label{eq:bessel_dgl}
\end{align}
Um die Differenzialgleichung zu l\"osen, w\"ahlen wir den Potenzreihenansatz mit der folgenden Potenzreihe und deren Ableitungen:
\begin{align*}
	R \left( r \right)
	&=
	r^{\sigma}
	\sum_{k=0}^{\infty} a_k \cdot r^k
\\
	R'\left( r \right)
	&=
	\sigma \cdot r^{\sigma - 1}
	\sum_{k=0}^{\infty} a_k \cdot r^k
	+
	r^{\sigma}
	\sum_{k=0}^{\infty} a_k \cdot k \cdot r^{k - 1}
\\
	R'' \left( r \right)
	&=
	\sigma \cdot \left( \sigma - 1 \right) \cdot r^{\sigma - 2}
	\sum_{k=0}^{\infty} a_k \cdot r^k
	+
	2 \cdot \sigma \cdot r^{\sigma - 1}
	\sum_{k=0}^{\infty} a_k \cdot k \cdot r^{k - 1}
	+
	r^{\sigma}
	\sum_{k=0}^{\infty} a_k \cdot k \cdot \left( k - 1 \right) \cdot r^{k - 2}	
\end{align*}
Nun kann man die Potenzreihen in die Differenzialgleichung \ref{eq:bessel_dgl} einsetzen und bekommt dann:
\begin{align*}
	\sigma \cdot \left( \sigma - 1 \right) \cdot r^{\sigma}
	\sum_{k=0}^{\infty} a_k \cdot r^k
	+
	2 \cdot \sigma \cdot r^{\sigma}
	\sum_{k=0}^{\infty} a_k \cdot k \cdot r^k
	+
	r^{\sigma}
	\sum_{k=0}^{\infty} a_k \cdot k \cdot \left( k - 1 \right) \cdot r^k
	+ \\
	\sigma \cdot r^{\sigma}
	\sum_{k=0}^{\infty} a_k \cdot r^k
	+
	r^{\sigma}
	\sum_{k=0}^{\infty} a_k \cdot k \cdot r^k
	+\\
	r^{\sigma}
	\sum_{k=0}^{\infty} a_k \cdot r^{k + 2}
	-
	n^2 \cdot r^{\sigma}
	\sum_{k=0}^{\infty} a_k \cdot r^k
	= & \text{ } 0
\end{align*}
Gleiche Summenzeichen zusammenfassen:
\begin{align*}
	\left(
	\sigma \cdot \left( \sigma - 1 \right)
	+
	\sigma
	-
	n^2
	\right)
	\cdot r^{\sigma}
	\sum_{k=0}^{\infty} a_k \cdot r^k
	+ \\
	\left(	
	2 \cdot \sigma
	+
	1
	\right)
	\cdot r^{\sigma}
	\sum_{k=0}^{\infty} a_k \cdot k \cdot r^k
	+ \\
	r^{\sigma}
	\sum_{k=0}^{\infty} a_k \cdot k \cdot \left( k - 1 \right) \cdot r^k
	+ \\
	r^{\sigma}
	\sum_{k=0}^{\infty} a_k \cdot r^{k + 2}
	= & \text{ } 0
\end{align*}
Nun muss man eine Indexgleichung für die Unbekannte $\sigma$ aufstellen, um diese zu bestimmen.
Dazu eignet sich eine Indexgleichung f\"ur den Koeffizienten $a_0$:
\begin{align*}
	\left[ \sigma \cdot \left( \sigma -1 \right) + \sigma - n^2 \right] \cdot a_0 &= 0 && \left| :a_0 \right. \\
	\sigma \cdot \left( \sigma -1 \right) + \sigma - n^2 &= 0 && \left| \text{Ausmultiplizieren} \right. \\
	\sigma ^2 - \sigma + \sigma -n^2 &= 0 && \left| \text{Vereinfachen} \right.\\
	\sigma ^2 - n^2 &= 0 && \left| +n^2 \right.\\
	\sigma ^2 &= n^2 && \left| \sqrt{\centerdot} \right. \\
	\sigma &= \pm n \text{ und } \\
	\sigma &= \pm 0 \left( \text{Doppelte Nullstelle} \right)
\end{align*}
Wir beschr\"anken uns in diesem Kapitel nur mit der L\"oesung $\sigma = \pm n$. Die L\"osung f\"ur $\sigma = \pm 0$ wird im Kapitel \nameref{chapter:komplex} genauer behandelt.
Nun kann man $n$ f\"ur $\sigma$ in die Gleichung einsetzen und erh\"alt:
\begin{align*}
	\left(
	n \cdot \left( n - 1 \right)
	+
	n
	-
	n^2
	\right)
	\cdot r^{n}
	\sum_{k=0}^{\infty} a_k \cdot r^k
	+ \\
	\left(	
	2 \cdot n
	+
	1
	\right)
	\cdot r^{n}
	\sum_{k=0}^{\infty} a_k \cdot k \cdot r^k
	+ \\
	r^{n}
	\sum_{k=0}^{\infty} a_k \cdot k \cdot \left( k - 1 \right) \cdot r^k
	+ \\
	r^{n}
	\sum_{k=0}^{\infty} a_k \cdot r^{k + 2}
	= & \text{ } 0
\end{align*}
Als n\"achstes fasst man alles in eine einzige Summe zusammen und vereinfacht diese:
\begin{align*}
	r^n
	\sum_{\textcolor{red}{k=2}}^{\infty}
	\left[ n \cdot \left( n - 1 \right) \cdot a_k
	+
	2 \cdot n \cdot k \cdot a_k
	+
	k \cdot \left( k - 1 \right) \cdot a_k
	+
	n \cdot a_k
	+
	k \cdot a_k
	+
	a_{k-2}
	-
	n^2 \cdot a_k
	\right]
	\cdot r^k
	= 0 \\
	%
	r^n
	\sum_{k=2}^{\infty}
	\left[
	2 \cdot n \cdot k \cdot a_k
	+
	k^2 \cdot a_k
	+
	a_{k - 2}
	\right]
	\cdot r^k
	= 0
\end{align*}
Weil $r \neq 0$ sein muss, muss $ \left( 2 \cdot n \cdot k + k^2 \right) \cdot a_k + a_{k - 2} = 0$ sein.
Nun kann man eine Rekursion der Koeffizienten $a_k$ berechnen.
\begin{align}
	a_k
	=
	\frac
	{
		-a_{k - 2}
	}{
		k \cdot \left( 2 \cdot n + k \right)	
	}
	\label{eq:bessel_koeffreq}
\end{align}
Da $k \geq 0$ sein muss, sind nur gerade Koeffizienten m\"oglich. Somit kann man f\"ur $k$ nun $2k$ in die Gleichung \ref{eq:bessel_koeffreq} einsetzen und erh\"alt:
\begin{align*}
	a_{2k}
	&=
	\frac
	{
		-a_{2k - 2}
	}{
		2k \cdot \left( 2 \cdot n + 2k \right)	
	} \\
	&=
	\frac
	{
		-a_{2k - 2}
	}{
		4k \cdot \left( n + k \right)	
	} \\
	&=
	\frac
	{
		-a_0
	}{
		4^k \cdot {k}! \cdot {\left( n + k \right)}!
	} \\
	&=
	\frac
	{
		-a_0
	}{
		2^{2k} \cdot {k}! \cdot {\left( n + k \right)}!
	}
\end{align*}
Setzt man nun f\"ur $a_0 = \frac{1}{2^n \cdot {n}!}$ ein, so kommt man auf die Form:
\begin{align}
	J_n \left( r \right)
	&= \nonumber
	\sum_{k=0} ^{\infty}
	\frac
	{
		\left( - 1 \right) ^k \cdot r ^{2k+n}
	}{
		2^{2k+n} \cdot {k}! \cdot { \left( k + n \right) }!
	} \\
	&=
	\sum_{k=0} ^{\infty}
	\frac
	{
		\left( - 1 \right) ^k \cdot 
		\frac
		{
			r ^{2k+n}
		}{
			2^{2k+n}
		}
	}{
		{k}! \cdot { \left( k + n \right) }!
	} \\
	&=
	\sum_{k=0} ^{\infty}
	\frac
	{
		\left( - 1 \right) ^k \cdot 
		\left(		
		\frac
		{
			r
		}{
			2
		} \right) ^{2k+n}
	}{
		{k}! \cdot { \left( k + n \right) }!
	}
	\label{eq:bessel_summenformel}
\end{align}

\begin{figure}
	\includegraphics[scale=0.75]{kreis/besselfunction.png}
	\label{img:besselfunction}
	\caption[Besselfunktion]{Besselfunktion geplottet}
\end{figure}

\printbibliography[heading=subbibliography]
\end{refsection}

