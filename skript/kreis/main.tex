\chapter{Richtcharakteristik (kreisf"ormiger) Lautsprecher \label{chapter:kreis}}
\lhead{Richtcharakteristik (kreisf"ormiger) Lautsprecher}
\begin{refsection}
\chapterauthor{Kevin Cina und Benjamin R"aber}
\\
%
%
%%Text ist für mich noch nicht zufriedenstellend
%In diesem Kapitel geht es um die \emph{Richtcharakteristik von kreisf\"ormigen Antennen}.
%Vorweg genommen bekommt man die \emph{Besselfunktion} als L\"osung.
%Die Herleitung ist aber allgemein f\"ur aller zylindrische K\"orper anwendbar.
%F\"ur das bessere Verst\"andnis wird zuerst der Weg von einer rechteckigen Pauke zu einer kreisf\"ormigen Pauke gemacht und erst danach kommen wir zum Thema \emph{Richtcharakteristik von kreisf\"ormigen Antennen}.
%
%
%
%Potenzreihenherleitung
\section{Potenzreihenherleitung der Besselfunktion}
Das Ziel dieses Abschnittes ist, die Besselfunktion mit der Potenzreihen-Methode die in Kapitel \ref{section:potenzreihen:verallgemeinert} erl\"autert wurde, herzuleiten.
\dots \\
%Dieser Abschnitt befasst sich mit der Potenzherleitung der Besselfunktion. 
%Bei der Herleitung werden Methoden, die in den vorherigen Kapitel  erl\"autert wurden, 
%verwendet. Zuerst m\"uss eine Differenzialgleichung aufgestellt werden.
%Die Besselfunktion kommt aus der Wellengleichung.
%
\dots So kommt man auf die Gleichung \ref{eq:bessel_dgl}, welche grosse \"Ahnlichkeit mit der Gleichung \ref{potenzreihen:verallgemeinert-bessel} aufweisst.
%Die Differenzialgleichung f\"ur die Besselfunktion lautet wiefolgt:
\begin{align}
	r^2 \, R'' \left( r \right)
	+
	r \, R' \left( r \right)
	+
	\left( r^2 - n^2 \right) \, R \left( r \right)
	=
	0
	\label{eq:bessel_dgl}
\end{align}
Im Folgenden wird nun die Gleichung \ref{eq:bessel_dgl}, mithilfe der Potenzreihen-Methode aus dem Kapitel \ref{section:potenzreihen:verallgemeinert}, gel\"ost und der L\"osungsweg aufgezeigt.
\subsection*{Vorgehensweise}
\begin{enumerate}
	\item Die Potenzreihe und deren Ableitungen berechnet.
	\item Die Potenzreihen in die Differenzialgleichung \ref{eq:bessel_dgl} eingesetzen, um
	\item die Indexgleichung f\"ur $\varrho$ zu l\"osen, welche
	\item eine Rekursion der Koeffizienten erm\"oglicht.
	\item Bestimmen der Koeffizienten f\"ur
	\item die Besselfunktion mit ganzzahlige Parametern \ref{eq:bessel_summenformel}.
%	\item Allgemeine Besselfunktion \ref{eq:bessel_summenformel:allgemein}
\end{enumerate}


%Um die Differenzialgleichung zu l\"osen, w\"ahlen wir den Potenzreihenansatz mit der folgenden Potenzreihe und deren Ableitungen:
\subsection*{Potenzreihe und deren Ableitungen}

\begin{normalsize}
	Zum l\"osen der Differenzialgleichung \ref{eq:bessel_dgl} w\"ahlten wir den Ansatz der Potenzreihen-Methode.
	F\"ur den Ansatz ben\"otigen wir zuerst eine allgemeine Potenzreihe wie \ref{eq:bessel:potenzreihe:allgemein}.
	Da in der Differenzialgleichung die erste und zweite Ableitung darin vorkommen,
	muss man die Potenzreihe noch ableiten, was wiederum eine Potenzreiche ergibt,
	wie \ref{eq:bessel:potenzreihe:ersteableitung} und \ref{eq:bessel:potenzreihe:zweiteableitung} zeigen.
\end{normalsize}
\\
\begin{align}
	R \left( r \right)
	&=
	r^{\varrho}
	\sum_{k=0}^{\infty} a_k \, r^k
	\label{eq:bessel:potenzreihe:allgemein}
\\
	R'\left( r \right)
	&=
	\varrho \, r^{\varrho - 1}
	\sum_{k=0}^{\infty} a_k \, r^k
	+
	r^{\varrho}
	\sum_{k=0}^{\infty} a_k \, k \, r^{k - 1}
	\label{eq:bessel:potenzreihe:ersteableitung}
\\
	R'' \left( r \right)
	&=
	\varrho \, \left( \varrho - 1 \right) \, r^{\varrho - 2}
	\sum_{k=0}^{\infty} a_k \, r^k
	+
	2 \, \varrho \, r^{\varrho - 1}
	\sum_{k=0}^{\infty} a_k \, k \, r^{k - 1}
	+
	r^{\varrho}
	\sum_{k=0}^{\infty} a_k \, k \, \left( k - 1 \right) \, r^{k - 2}	
	\label{eq:bessel:potenzreihe:zweiteableitung}
\end{align}
\\
\begin{normalsize}
	Wie man sieht,
	wird die Potenzreihe mit zunehmender Ableitung immer gr\"osser und un\"ubersichtlicher.
	Zudem verringert sich bei jeder Ableitung der Grad der Potenz um 1.
	Wenn man die Potenz als Stelle nimmt,
	dann kann man auch sagen,
	dass sich pro Ableitung,
	alle Koeffizienten um eine Stelle nach links verschieben.
	Dank dieser Verschiebung ist es m\"oglich,
	eine Abh\"anigkeit bzw. eine Rekursion der Koeffizienten zu erreichen.
	Mit der Rekursion ist es wiederum m\"oglich,
	nur wenige der Koeffizienten selber zu definieren m\"ussen,
	was den Freiheitsgrad reduziert und somit die Berechnung vereinfacht.
	
	\begin{description}[style = nextline, leftmargin = \parindent, labelindent = \parindent]
	%[style = multiline, labelwidth = 3.5cm, leftmargin = 3.5cm, itemsep = 1cm]
		\item[Wieso ist es ein Vorteil, den Freiheitsgrad zu reduzieren?]
		\dots Parameter reduzieren \dots Anzahl M\"oglichkeiten reduzieren \dots \\
	\end{description}
		
\end{normalsize}

\subsection*{Potenzreihen in Differenzialgleichung \ref{eq:bessel_dgl} einsetzen}
%Nun kann man die Potenzreihen in die Differenzialgleichung \ref{eq:bessel_dgl} einsetzen und bekommt dann:

\begin{normalsize}
	
\end{normalsize}

\begin{align*}
	\overbrace{
		\varrho \, \left( \varrho - 1 \right) \, r^{\varrho}
		\sum_{k=0}^{\infty} a_k \, r^k
		+
		2 \, \varrho \, r^{\varrho}
		\sum_{k=0}^{\infty} a_k \, k \, r^k
		+
		r^{\varrho}
		\sum_{k=0}^{\infty} a_k \, k \, \left( k - 1 \right) \, r^k
	}^{r^2 \, R''\left( r \right)}
	+ \\
	\overbrace{
		\varrho \, r^{\varrho}
		\sum_{k=0}^{\infty} a_k \, r^k
		+
		r^{\varrho}
		\sum_{k=0}^{\infty} a_k \, k \, r^k
	}^{r \, R' \left( r \right)}
	+\\
	\overbrace{
	r^{\varrho}
		\sum_{k=0}^{\infty} a_k \, r^{\textcolor{red}{k + 2}}
		-
		n^2 \, r^{\varrho}
		\sum_{k=0}^{\infty} a_k \, r^k
	}^{\left( r^2 - n^2 \right) \, R \left( r \right)}
	= & \, 0
\end{align*}

%Gleiche Summenzeichen zusammenfassen:

\begin{align*}
	\left(
	\varrho \, \left( \varrho - 1 \right)
	+
	\varrho
	-
	n^2
	\right)
	\, r^{\varrho}
	\sum_{k=0}^{\infty} a_k \, r^k
	+ \\
	\left(	
	2 \, \varrho
	+
	1
	\right)
	\, r^{\varrho}
	\sum_{k=0}^{\infty} a_k \, k \, r^k
	+ \\
	r^{\varrho}
	\sum_{k=0}^{\infty} a_k \, k \, \left( k - 1 \right) \, r^k
	+ \\
	r^{\varrho}
	\sum_{k=0}^{\infty} a_k \, r^{\textcolor{red}{k + 2}}
	= & \, 0
\end{align*}

\subsection*{Indexgleichung f\"ur $\varrho$}

%Nun muss man eine Indexgleichung für die Unbekannte $\varrho$ aufstellen, um diese zu bestimmen.
%\ref{section:potenzreihen:verallgemeinert}
%Dazu eignet sich eine Indexgleichung f\"ur den Koeffizienten $a_0$:

\begin{align*}
%	\left( \varrho \, \left( \varrho -1 \right) + \varrho - n^2 \right) \, a_0 &= 0 && \left| :a_0 \right. \\
	\varrho \, \left( \varrho -1 \right) + \varrho - n^2 &= 0 && \left| \text{Ausmultiplizieren} \right. \\
	\varrho ^2 - \varrho + \varrho -n^2 &= 0 && \left| \text{Vereinfachen} \right.\\
	\varrho ^2 - n^2 &= 0 && \left| +n^2 \right.\\
	\varrho ^2 &= n^2 && \left| \sqrt{\centerdot} \right. \\
	\varrho &= \pm n \text{ und } \\
	\varrho &= \pm 0 %\left( \text{Doppelte Nullstelle} \right)
\end{align*}

%Wir beschr\"anken uns in diesem Kapitel nur mit der L\"oesung $\varrho = \pm n$. Die L\"osung f\"ur $\varrho = \pm 0$ wird im Kapitel \ref{chapter:komplex} genauer behandelt.
%Nun kann man $n$ f\"ur $\varrho$ in die Gleichung einsetzen und erh\"alt:

\begin{align*}
	\left(
	n \, \left( n - 1 \right)
	+
	n
	-
	n^2
	\right)
	\, r^{n}
	\sum_{k=0}^{\infty} a_k \, r^k
	+ \\
	\left(	
	2 \, n
	+
	1
	\right)
	\, r^{n}
	\sum_{k=0}^{\infty} a_k \, k \, r^k
	+ \\
	r^{n}
	\sum_{k=0}^{\infty} a_k \, k \, \left( k - 1 \right) \, r^k
	+ \\
	r^{n}
	\sum_{k=0}^{\infty} a_k \, r^{\textcolor{red}{k + 2}}
	= & \, 0
\end{align*}

%Als n\"achstes fasst man alles in eine einzige Summe zusammen und vereinfacht diese:

\begin{align*}
	r^n
	\sum_{\textcolor{red}{k=2}}^{\infty}
	\left( n \, \left( n - 1 \right) \, a_k
	+
	2 \, n \, k \, a_k
	+
	k \, \left( k - 1 \right) \, a_k
	+
	n \, a_k
	+
	k \, a_k
	+
	a_{k-2}
	-
	n^2 \, a_k
	\right)
	\, r^k
	= 0 \\
	%
	r^n
	\sum_{k=2}^{\infty}
	\left(
	2 \, n \, k \, a_k
	+
	k^2 \, a_k
	+
	a_{k - 2}
	\right)
	\, r^k
	= 0
\end{align*}

%Weil $r \neq 0$ sein muss, muss $ \left( 2 \, n \, k + k^2 \right) \, a_k + a_{k - 2} = 0$ sein.
%Nun kann man eine Rekursion der Koeffizienten $a_k$ berechnen.

\begin{align}
	a_k
	=
	\frac
	{
		-a_{k - 2}
	}{
		k \, \left( 2 \, n + k \right)	
	}
	\label{eq:bessel_koeffreq}
\end{align}

%Da $k \geq 0$ sein muss, sind nur gerade Koeffizienten m\"oglich. Somit kann man f\"ur $k$ nun $2k$ in die Gleichung \ref{eq:bessel_koeffreq} einsetzen und erh\"alt:

\begin{align*}
	a_{2k}
	&=
	\frac
	{
		-a_{2k - 2}
	}{
		2k \, \left( 2 \, n + 2k \right)	
	} \\
	&=
	\frac
	{
		-a_{2k - 2}
	}{
		4k \, \left( n + k \right)	
	} \\
	&=
	\frac
	{
		-a_0
	}{
		4^k \, {k}! \, {\left( n + k \right)}!
	} \\
	&=
	\frac
	{
		-a_0
	}{
		2^{2k} \, {k}! \, {\left( n + k \right)}!
	}
\end{align*}

%Setzt man nun f\"ur $a_0 = \frac{1}{2^n \, {n}!}$ ein, so kommt man auf die Form:

\begin{align}
	J_n \left( r \right)
	&= \nonumber
	\sum_{k=0} ^{\infty}
	\frac
	{
		\left( - 1 \right) ^k \, r ^{2k+n}
	}{
		2^{2k+n} \, {k}! \, { \left( k + n \right) }!
	} \\
	&= \nonumber
	\sum_{k=0} ^{\infty}
	\frac
	{
		\left( - 1 \right) ^k \, 
		\frac
		{
			r ^{2k+n}
		}{
			2^{2k+n}
		}
	}{
		{k}! \, { \left( k + n \right) }!
	} \\
	&=
	\sum_{k=0} ^{\infty}
	\frac
	{
		\left( - 1 \right) ^k \, 
		\left(		
		\frac
		{
			r
		}{
			2
		} \right) ^{2k+n}
	}{
		{k}! \, { \left( k + n \right) }!
	}
	\label{eq:bessel_summenformel}
\end{align}

%%	Die Gleichung ist für ganzzahlige Koeffizienten.
%%	Übergang zu Gammafunktion erläutern für allgemeine Formel
%$\Gamma \left( \alpha \right) = {\left( \alpha - 1 \right) }!$
%\begin{align}
%	J_{\pm n} \left( r \right)
%	&=
%	\sum_{k=0} ^{\infty}
%	\frac
%	{
%		\left( - 1 \right) ^k \, 
%		\left(		
%		\frac
%		{
%			r
%		}{
%			2
%		} \right) ^{2k+n}
%	}{
%		{k}! \, \Gamma \left( k + n + 1 \right)
%	}
%	\label{eq:bessel_summenformel:allgemein}
%\end{align}


\begin{figure}
	\begin{center}
		\includegraphics[scale=0.5]{kreis/besselfunction.pdf}
		\label{img:besselfunction}
		\caption[Besselfunktion]{Besselfunktion geplottet}
	\end{center}
\end{figure}

\newpage

\subsection{Veranschaulichung der Besselfunktion mit Beispielen}
\begin{itemize}
	\item Lautsprecher
	\item Antenne
	\item Lichtbrechung im Fernglas
	\item \dots
\end{itemize}

\printbibliography[heading=subbibliography]
\end{refsection}

