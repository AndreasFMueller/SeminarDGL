\section{Wellengleichung}

Wie im Kapitel 9 bereits gesehen, lassen sich Wellen durch Differentialgleichungen beschreiben. Allgemein können beliebige Wellen, z.B. Schallwellen oder Elektromagnetische Wellen, durch folgende Gleichung beschrieben werden:

\begin{equation}
\frac{1}{c^2} \frac{\partial^2 u}{\partial t^2} = \frac{\partial^2 u}{\partial x^2} + \frac{\partial^2 u}{\partial y^2} + \frac{\partial^2 u}{\partial z^2}
\end{equation}

Diese Gleichung lässt sich mit Hilfe des Laplace-Operators verkürzt wie folgt schreiben:

\begin{equation}
\frac{1}{c^2} \frac{\partial^2 u}{\partial t^2} = \Delta u
\end{equation}

Durch Umformen der Gleichung erhalten wir folgendes:

\begin{equation}
\frac{1}{c^2} \frac{\partial^2 u}{\partial t^2} - \Delta u = 0
\end{equation}

In diesem Kapitel wollen wir uns auf Rotationssymmetrische Anwendungen der Wellengleichung beschränken. Da solche Probleme besser in Zylinderkoordinaten beschrieben werden können, transformieren wir die Wellengleichung. Wir erhalten folgende Gleichung:

\begin{equation}
\frac{1}{c^2} \frac{\partial^2 u}{\partial t^2} = \frac{1}{r} \frac{\partial}{\partial r}(r \frac{\partial u}{\partial r}) + \frac{1}{r^2} \frac{\partial^2 u}{\partial \phi^2} + \frac{\partial^2 u}{\partial z^2} 
\end{equation}

\subsection[Lösung durch Variablenseparation]{Lösung durch Variablenseparation}

Wir wollen nun die Wellengleichung in Zylinderkoordinaten lösen. Dazu verwenden wir die Separationsmethode.
Bei der Separationsmethode versuchen wir die Funktion $u(r, \phi, t)$ durch ein Produkt der Funktionen darzustellen, welche jeweils nur von einer Variable abhängen: $R(r)\Phi(\phi)T(t)$
\\Somit erhalten wir folgende Gleichung:

\begin{equation}
\frac{1}{c^2} R(r) \Phi(\phi) T''(t) = \frac{1}{r} \frac{\partial}{\partial r}(r R'(r)) \Phi(\phi) T(t) + \frac{1}{r^2} R(r) \Phi''(\phi) T(t)
\end{equation}

Wir teilen nun durch die einzelnen Funktionen, damit die Funktionen alleine stehen:

\begin{equation}
\frac{1}{c^2} R(r) \Phi(\phi) T''(t) = \frac{1}{r}(R'(r) + R''(r)) \Phi(\phi) T(t) + \frac{1}{r^2} R(r) \Phi''(\phi) T(t)
\Bigl\lvert
:R(r) : T(t) :\Phi(\phi)
\end{equation}

Somit erhalten wir:

\begin{equation}
\frac{1}{c^2} R(r) \Phi(\phi) T''(t) = \frac{1}{r} (R'(r) + R''(r)) \Phi(\phi) T(t) + \frac{1}{r^2} R(r) \Phi''(\phi) T(t)
\end{equation}

Fixiert man nun eine Variable, stellt man fest, dass auf beiden Seiten eine Konstante stehen muss. Wir erhalten also:

\begin{equation}
\frac{1}{c^2}
\frac{T''(t)}{T(t)} = 
\frac{1}{r} 
\frac{R'(r) + R''(r)}{R(r)} + 
\frac{1}{r^2}
\frac{\Phi''(\phi)}{\Phi(\phi)} = \mu
\end{equation}

Wir erhalten somit eine Orts- und eine Zeitgleichung:

\begin{equation}
\frac{1}{c^2} 
\frac{T''(t)}{T(t)} = 
\mu
,\quad
\frac{1}{r} \frac{R'(r) + R''(r)}{R(r)} + 
\frac{1}{r^2} \frac{\Phi''(\phi)}{\Phi(\phi)} = 
\mu
\end{equation}

Setzen wir die Konstante in die Gleichung ein und wiederholen den obigen Schritt, so erhalten wir eine weitere Konstante:

\begin{equation}
\frac{1}{r} \frac{R'(r) + R''(r)}{R(r)} - \mu =
-\frac{1}{r^2} \frac{\Phi''(\phi)}{\Phi(\phi)} = n
\end{equation}

Formt man diese Gleichung nun um, so erhält man die Bessel'sche DGL:

\begin{equation}
r^2 R''(r) + r R'(r) - (\mu r^2 + n^2)R(r) = 0
\label{eq:besselsche_dgl}
\end{equation}
