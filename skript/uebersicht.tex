%
% uebersicht.tex -- Uebersicht ueber die Seminar-Arbeiten
%
% (c) 2015 Prof Dr Andreas Mueller, Hochschule Rapperswil
%
\chapter*{"Ubersicht}
\lhead{"Ubersicht}
\rhead{}
\label{skript:uebersicht}
Im zweiten Teil kommen die Teilnehmer des Seminars selbst zu Wort.
Sie zeigen Anwendungsbeispiele f"ur die im ersten
Teil entwickelte Theorie der gew"ohnlichen Differentialgleichungen.
Eine breite Vielfalt von Arbeiten vertieft einzelne Aspekte der Theorie,
untersucht spezielle Differentialgleichungen im Detail
oder erm"oglicht das bessere Verst"andnis interessanter Anwendungen.

{\em Daniela Meier} und {\em Hansruedi Patzen} untersuchen eine spezielle
Differentialgleichung zweiter Ordnung mit Hilfe der Potenzreihenmethode
und entwickeln einiges an Intuition, wie L"osungen einer solchen Gleichung
aussehen m"ussen.
Sie decken auch Schwierigkeiten bei der numerischen Berechnung der
Potenzreihenl"osung auf.

{\em Kevin Cina} und {\em Benjamin R"aber} untersuchen das zylindersymmetrische
Wellenausbreitungsproblem, leiten die Besselsche Differentialgleichung
her, und l"osen sie mit Hilfe eines Potenzreihenansatz.
Sie finden so die Besselfunktionen, ausser im Falle $\nu =0$.
Diesen Fall untersuche {\em Stefan Kull} und {\em Roy Seitz}, sie konstruieren
die mit dem reinen Potenzreihenansatz nicht zug"angliche zweite linear
unabh"angige L"osung.
Ihre Darstellung verwendet einen originellen Operator zur Formalisierung
der analytische Fortsetzung.

{\em Pascal Horat} und {\em Matthias Kn"opfel} untersuchen das Problem
der Schrittl"ange bei der numerischen L"osung gew"ohnlicher
Differentialgleichung.
Aus den Lehren an einem Beispielproblem leiten sie einen eigenen
Schrittl"angensteuerungsalgorithmus ab und untersuchen seine Leistung.

{\em Simon Sch"afer} und{\em Tibor Schneider} entwickeln die
Differentialgleichungen f"ur die Lichtbrechung in der Atmosph"are, und
simulieren sie f"ur verschiedene Atmosph"arenmodelle.
Eine weitere Anwendung untersuchen {\em Max Obrist} und {\em Martin Stypinski}.
Sie modellieren die Ausbreitung ansteckender Krankheiten, und erweitern
ihr Modell auch auf die bevorstehende Zombie-Apokalypse.

{\em Andri Hartmann} und {\em Tobias Schuler} untersuchen die 
Friedmann-Gleichung

Als abschliessendes Beispiel zeigt {\em Reto Christen}, wie ein Transformator
simuliert werden kann. 
Seine L"osung zeichnet sich durch besonders hohe Rechengeschwindigkeit aus,
welche dank eines vertieften Verst"andnisses der Mathematik des Problems
erreicht werden konnte.











