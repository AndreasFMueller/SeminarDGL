%
% skript.tex -- Skript ueber Differentialgleichungen
%
% (c) 2014 Prof. Dr. Andreas Mueller, HSR
%
\documentclass{book}
\usepackage{etex}
\usepackage{geometry}
\geometry{papersize={170mm,240mm},total={140mm,200mm},top=21mm,bindingoffset=10mm}
\usepackage[english,ngerman]{babel}
\usepackage{times}
\usepackage{amsmath,amscd}
\usepackage{amssymb}
\usepackage{amsfonts}
\usepackage{amsthm}
\usepackage{graphicx}
\usepackage{fancyhdr}
\usepackage{textcomp}
\usepackage[all]{xy}
\usepackage{txfonts}
\usepackage{alltt} 
\usepackage{verbatim}
\usepackage{paralist}
\usepackage{makeidx}
\usepackage{array}
\usepackage[colorlinks=true]{hyperref}
\usepackage{tikz}
\usepackage{pgfplots}
\usepackage{pgfplotstable}
\usepackage{pdftexcmds}
%\usepackage{pgfmath}
\usepackage{placeins}
\usepackage{subfigure}
\usepackage[autostyle=false,english=american]{csquotes}
\usepackage{float}
\usepackage{enumitem}
\usepackage{wasysym}
\usepackage{environ}
\usepackage{pifont}
\usepackage{feynmp}
\usepackage{appendix}
\usetikzlibrary{calc,intersections,through,backgrounds,graphs,positioning,shapes,arrows,fit}
\usetikzlibrary{patterns,decorations.pathreplacing}
\usetikzlibrary{decorations.pathreplacing}
\usetikzlibrary{external}
\usepackage[europeanvoltages,
            europeancurrents,
            europeanresistors,   % rectangular shape
            americaninductors,   % "4-bumbs" shape
            europeanports,       % rectangular logic ports
            siunitx,             % #1<#2>
            emptydiodes,
            noarrowmos,
            smartlabels]         % lables are rotated in a smart way
           {circuitikz}          %
\usepackage{siunitx}
\usepackage{tabularx}
\usetikzlibrary{arrows}

\usepackage{algpseudocode}
\usepackage{algorithm}

\usepackage{listings}
\lstdefinestyle{Matlab}{
  numbers=left,
  belowcaptionskip=1\baselineskip,
  breaklines=true,
  frame=L,
  xleftmargin=\parindent,
  language=Matlab,
  showstringspaces=false,
  basicstyle=\footnotesize\ttfamily,
  keywordstyle=\bfseries\color{green!40!black},
  commentstyle=\itshape\color{purple!40!black},
  identifierstyle=\color{blue},
  stringstyle=\color{orange},
  numberstyle=\ttfamily\tiny
}
\lstdefinelanguage{Maxima}{
  keywords={addrow,addcol,zeromatrix,ident,augcoefmatrix,ratsubst,diff,ev,tex,%
    with_stdout,nouns,express,depends,load,submatrix,div,grad,curl,matrix,%
    invert,lambda,facsum,expand,false,then,if,else,subst,%
    rootscontract,solve,part,assume,sqrt,integrate,abs,inf,exp,sin,cos,sinh,cosh},
  sensitive=true,
  comment=[n][\itshape]{/*}{*/}
}
\lstdefinestyle{Maxima}{
  numbers=left,
  belowcaptionskip=1\baselineskip,
  breaklines=true,
  frame=L,
  xleftmargin=\parindent,
  language=Maxima,
  showstringspaces=false,
  basicstyle=\footnotesize\ttfamily,
  keywordstyle=\bfseries\color{green!40!black},
  commentstyle=\itshape\color{purple!40!black},
  identifierstyle=\color{blue},
  stringstyle=\color{orange},
  numberstyle=\ttfamily\tiny
}
\usepackage{caption}
\usepackage[mode=buildnew]{standalone}
\usepackage[backend=bibtex]{biblatex}
\addbibresource{references.bib}
% Bibresources für jeden einzelnen Artikel
%\addbibresource{thema/main.bib}
\AtEndDocument{\clearpage\ifodd\value{page}\else\null\clearpage\fi}
\makeindex
%\pgfplotsset{compat=1.12}
\setlength{\headheight}{15pt} % fix headheight warning
\DeclareGraphicsRule{*}{mps}{*}{}
\begin{document}
\pagestyle{fancy}
\frontmatter
\newcommand\HRule{\noindent\rule{\linewidth}{1.5pt}}
\begin{titlepage}
\vspace*{\stretch{1}}
\HRule
\vspace*{5pt}
\begin{flushright}
{
\LARGE
Mathematisches Seminar\\
\vspace*{20pt}
\Huge
Differentialgleichungen%
}
\vspace*{5pt}
\end{flushright}
\HRule
\begin{flushright}
\vspace{60pt}
\Large
Leitung: Andreas M"uller\\
\vspace{40pt}
\Large
Reto~Christen,
Kevin~Cina,
Andri~Hartmann,
Pascal~Horat %,
Matthias~Kn"opfel,
Stefan Kull,
Daniela~Meier,
Max~Obrist %,
Hansruedi~Patzen,
Benjamin~R"aber,
Simon~Schaefer %,
Tibor~Schneider,
Tobias~Schuler,
Roy~Seitz,
Martin~Stypinski
\end{flushright}
\vspace*{\stretch{2}}
\begin{center}
Hochschule f"ur Technik, Rapperswil, 2016
\end{center}
\end{titlepage}
\hypersetup{
    linktoc=all,
    linkcolor=blue
}
\newcounter{beispiel}
\newenvironment{beispiele}{
\bgroup\smallskip\parindent0pt\bf Beispiele\egroup

\begin{list}{\arabic{beispiel}.}
  {\usecounter{beispiel}
  \setlength{\labelsep}{5mm}
  \setlength{\rightmargin}{0pt}
}}{\end{list}}
\newcounter{uebungsaufgabe}
% environment fuer uebungsaufgaben
\newenvironment{uebungsaufgaben}{
\begin{list}{\arabic{uebungsaufgabe}.}
  {\usecounter{uebungsaufgabe}
  \setlength{\labelwidth}{2cm}
  \setlength{\leftmargin}{0pt}
  \setlength{\labelsep}{5mm}
  \setlength{\rightmargin}{0pt}
  \setlength{\itemindent}{0pt}
}}{\end{list}\vfill\pagebreak}
\newenvironment{teilaufgaben}{
\begin{enumerate}
\renewcommand{\labelenumi}{\alph{enumi})}
}{\end{enumerate}}
% Loesung
\def\swallow#1{
%nothing
}
\NewEnviron{loesung}[1][L"osung]{%
\begin{proof}[#1]%
\renewcommand{\qedsymbol}{$\bigcirc$}
\BODY
\end{proof}
}
\NewEnviron{bewertung}{%
\begin{proof}[Bewertung]%
\renewcommand{\qedsymbol}{}
\BODY
\end{proof}
}
\NewEnviron{diskussion}{%
\begin{proof}[Diskussion]%
\renewcommand{\qedsymbol}{}
\BODY
\end{proof}
}
\NewEnviron{hinweis}{%
\begin{proof}[Hinweis]%
\renewcommand{\qedsymbol}{}
\BODY
\end{proof}
}
\def\keineloesungen{%
\RenewEnviron{loesung}{\relax}
\RenewEnviron{bewertung}{\relax}
\RenewEnviron{diskussion}{\relax}
}
\newenvironment{beispiel}{%
\begin{proof}[Beispiel]%
\renewcommand{\qedsymbol}{$\bigcirc$}
}{\end{proof}}

\input{linsys.tex}
\allowdisplaybreaks

\lhead{Inhaltsverzeichnis}
\rhead{}
\tableofcontents
\newtheorem{satz}{Satz}[chapter]
\newtheorem{hilfssatz}[satz]{Hilfssatz}
\newtheorem{definition}[satz]{Definition}
\newtheorem{annahme}[satz]{Annahme}
\renewcommand{\floatpagefraction}{0.75}
\mainmatter
%
% vorwort.tex -- Vorwort zum Buch zum Seminar
%
% (c) 2015 Prof Dr Andreas Mueller, Hochschule Rapperswil
%
\chapter*{Vorwort}
\lhead{Vorwort}
\rhead{}
Dieses Buch entstand im Rahmen des Mathematischen Seminars
im Fr"uhjahrssemester 2016 an der Hochschule f"ur Technik Rapperswil.
Die Teilnehmer, Studierende der Abteilungen f"ur Elektrotechnik,
Informatik und Bauingenieurwesen der
HSR, erarbeiteten nach einer Einf"uhrung in das Themengebiet jeweils
einzelne Aspekte des Gebietes in Form einer Seminararbeit, "uber
deren Resultate sie auch in einem Vortrag informierten. 

Im Fr"uhjahr 2016 war das Thema des Seminars ``Differentialgleichungen''.
Die Einf"uhrung bestand aus einigen Vorlesungsstunden, deren
Inhalt im ersten Teil dieses Skripts zusammengefasst ist.
Es ging darum, die zum Teil aus dem Analysis-Unterricht bekannte
Theorie der Differentialgleichungen zu vertiefen, mit anderen Gebieten
wie zum Beispiel der komplexen Analysis zu verkn"upfen und sie
auf die Analyse einiger relevanter Praxisprobleme anzuwenden.
Dabei ging es nicht um die analytische L"osung von Differentialgleichungen,
die meisten Differentialgleichungen lassen sich ohnehin nicht in
geschlossener Form l"osen.
Einzelne Differentialgleichungen wurden untersucht, weil sie Anlass
zu einer wichtigen Familie von Funktionen geben, zum Beispiel die
Bessel- und Airy-Funktionen.
In anderen Beispielen ging es um die Schwierigkeiten, die bei einer
numerischen L"osung zu meistern sind.
Besonders anspruchsvoll sind jedoch "Uberlegungen zum Verhalten der
L"osung f"ur lange Zeiten, zum Beispiel Stabilit"at, das Auftreten
von Schwingungen bei der Hopf-Bifurkation oder der "Ubergang zum
Chaos.

Im zweiten Teil dieses Skripts kommen dann die Teilnehmer selbst zu Wort.
Ihre Arbeiten wurden jeweils als einzelne
Kapitel mit meist nur typographischen "Anderungen "ubernommen.
Diese weiterf"uhrenden Kapitel sind sehr verschiedenartig.
Eine "Ubersicht und Einf"uhrung befindet sich in der Einleitung
zum zweiten Teil auf Seite~\pageref{skript:uebersicht}.

In einigen Arbeiten wurde auch Code zur Demonstration der 
besprochenen Methoden und Resultate geschrieben, soweit
m"oglich und sinnvoll wurde dieser Code im Github-Repository
dieses Kurses\footnote{\url{https://github.com/AndreasFMueller/SeminarDGL.git}}
abgelegt, in anderen F"allen verweisen die Artikel selbst auf
das zugeh"orige Code-Repository.

Im genannten Repository findet sich auch der Source-Code dieses
Skriptes, es wird hier unter einer Creative Commons Lizenz
zur Verf"ugung gestellt.


\part{Grundlagen}
%\keineloesungen
\begin{refsection}
\section{Einleitung}
Wellen umgeben uns st"andig, selbst wenn wir es uns dem nicht direkt bewusst 
sind. Sei es nun in Form von Lichtwellen, Schallwellen, Wasserwellen und vielen 
mehr.

Die Wasserwellen welche ans Ufer schlagen, sind wohl ein Beispiel, bei dem sich 
jeder etwas darunter vorstellen kann. Dieses an sich sch"one Naturschauspiel 
kann aber auch destruktiv sein, so kann eine pl"otzliche Absenkung am 
Meeresgrund beispielsweise einen Tsunami ausl"osen.

In diesem Kapitel wird anfangs die Titelgleichung, welche die Ausbreitung einer 
Welle in einem parabolischen Kanal beschreibt, genauer analysiert. Gegen Ende 
soll aber auch noch eine allgemeinere Kanalform untersucht werden.

Nat"urlich sind diese Gleichungen nur eine Ann"aherung und entsprechen nur 
bedingt der in der Realit"at vorkommenen Wellenausbreitung. So kann sich eine 
Wasserwelle zum Beispiel "uberschlagen, was hier so nicht abgebildet wird. 
Trotzdem kann anhand von diesen Modellrechnungen versucht werden, die 
Ausbreitung einer Welle nachzuvollziehen.

%\input{kapitel.tex}
%
% grundlagen.tex -- Grundlagen ueber Differentialgleichungen
%
% (c) 2015 Prof Dr Andreas Mueller, Hochschule Rapperswil
%
\chapter{Grundlagen der Theorie der gew"ohnlichen Differentialgleichungen
\label{chapter:grundlagen}}
\lhead{}
\rhead{Grundlagen}
\section{Differentialgleichungen\label{section:differentialgleichungen}}
Eine gew"ohnliche Differentialgleichung f"ur eine reellwertige
Funktion $y(x)$ stellt einen Zusammenhang her zwischen der Funktion
und ihren Ableitungen.
Wir schreiben die Ableitungen als $y'$, $y''$, $y'''$ und $y^{(n)}$
f"ur die $n$-te Ableitung.
Wir lassen oft das Argument der Funktion weg.
Beispiele von Differentialgleichungen sind
\begin{align*}
y'&=-Ny
&&\text{Ordnung: $1$}
\\
y''&=-\omega^2 y
&&\text{Ordnung: $2$}
\\
x^2y''+xy'+(x^2-n^2)y&=0
&&\text{Ordnung: $2$}
\end{align*}
Die Abh"angigkeit kann in expliziter Form als
\begin{equation}
y^{(n)}=f(x,y,y',\dots,y^{(n-1)})
\label{grundlagen:explizit}
\end{equation}
oder in impliziter Form
\[
F(x,y,y',\dots,y^{(n)})=0
\]
gegeben sein.
Die Ordnung einer Differentialgleichung ist die h"ochste vorkommende
Ableitung.

Insbesondere in Anwendungen in der Physik ist die Zeit die
unabh"angige Variable.
Die abh"angige Variable ist dann zum Beispiel die Ortskoordinate
$x(t)$ und wir bezeichnen ihre Ableitungen mit $\dot{x}(t)$ f"ur
die Geschwindigkeit, $\ddot{x}(t)$ f"ur die Beschleunigung.
Dieses Beispiel suggeriert auch, dass die abh"angige Variable 
ein Vektor sein kann, den man als den Ortsvektor eines Teilchens
interpretieren kann.
Die Funktion $f(t,x,\dots,x^{(n-1)})$ ist dann auch vektorwertig, und
alle Argumente ausser dem ersten von $f$ sind vektorwertig.

Eine Differentialgleichung $n$-ter Ordnung f"ur eine skalare Funktion
kann in eine Vektor-Differentialgleichung erster Ordnung f"ur eine
$n$-dimensionale vektorwertige Funktion umgewandelt werden.
Ist $y(x)$ die gesuchte Funktion in der
Differentialgleichung~(\ref{grundlagen:explizit}), dann kann man
den Vektor
\[
u(x)=\begin{pmatrix}
y(x)\\y'(x)\\\vdots\\y^{(n-1)}(x)
\end{pmatrix}
\in\mathbb R^n
\]
bilden.
Er erf"ullt die Differentialgleichung
\begin{equation}
\frac{d}{dx}\begin{pmatrix}
y\\y'\\\vdots\\y^{(n-1)}
\end{pmatrix}
=
\begin{pmatrix}
y'\\y''\\\vdots\\y^{(n)}
\end{pmatrix}
=
\begin{pmatrix}
y'\\y''\\\vdots\\f(x,y,y',\dots,y^{(n-1)}.
\end{pmatrix}
\label{grundlagen:vektordgl}
\end{equation}
Der Vektor auf der rechten Seite h"ang nur von $x$, der Funktion $y$
und ihren Ableitungen bis zur $n-1$-ten Ordnung ab, also von $u$, man
kann (\ref{grundlagen:vektordgl}) daher als
\begin{equation}
\frac{d}{dx}u=\tilde{f}(x,u)
\end{equation}
schreiben.

\section{Anfangswertprobleme\label{section:anfangswertprobleme}}

\section{Randwertprobleme\label{section:randwertprobleme}}

\section{Analytische L"osungsverfahren\label{section:analytischeverfahren}}
\subsection{Separation der Variablen}
Differentialgleichungen erster Ordnung lassen sich oft durch sogenannte
Trennung der Variablen auf die Berechnung von Integralen reduzieren.
Dank der Schreibweise der Ableitung als Differentialquotient wird
dieser L"osungsweg sehr suggestiv.
Wir betrachten als Beispiel die Differentialgleichung
\[
y'=-Ny.
\]
Schreibt man die Ableitung als Differentialquotient, wird daraus die
Gleichung
\[
\frac{dy}{dx}=-Ny.
\]
Durch Division durch $y$ und formale Multiplikation mit $dx$ wird daraus
die formale Gleichung
\begin{equation}
\frac{dy}{y}=-N\,dx.
\label{grundlagen:separiert}
\end{equation}
In dieser Gleichung kommt die Variable $y$ nur auf der linken, die Variable
$x$ nur auf der rechten Seite vor.
Man sagt, die Variablen seien {\em separiert}.
\index{separiert}
\index{Variablen, Separation der}
Man beachte, dass die Gleichung (\ref{grundlagen:separiert}) nur eine
formale Bedeutung haben kann, die Symbole $dy$ und $dx$ sind ja keine Zahlen,
mit denen man algebraische Operationen durchf"uhren k"onnte.
Mit etwas Vorsicht angewandt f"uhrt dieser Kalk"ul aber nicht auf
Widerspr"uche.

Wir integrieren jetzt beide Seiten von (\ref{grundlagen:separiert}), und
erhalten 
\[
\int\frac1y\,dy=-N\int\,dx
\]
Beide Integrale lassen sich in geschlossener Form auswerten:
\[
\log|y|=-Nx+C.
\]
Aufgel"ost nach $y$ ergibt sich
\[
y=\pm e^{C}e^{-Nx},
\]
wobei die beiden Vorzeichen $\pm$ das Betragszeichen in der Stammfunktion
von $\frac1y$ reflektieren.
Man kann den Faktor $\pm e^{C}$ in eine neue Konstante $a$ zusammenfassen,
und erh"alt somit als L"osung der urspr"unglichen Differentialgleichung
die Familie
\[
y(x)=ae^{-Nx}
\]
von Funktionen.
Der Paramter $a$ muss mit Hilfe der Anfangsbedingung festgelegt werden.

\subsection{Lineare Differentialgleichungen}
Eine Differentialgleichung der Form
\begin{equation}
a_n(x)y^{(n)}+a_{n-1}(x)y^{(n-1)}+\dots+a_2(x)y''+a_1(x)y'+a_0(x)=f(x)
\label{grundlagen:linearedgl}
\end{equation}
heisst {\em lineare Differentialgleichung}.
\index{lineare Differentialgleichung}
Ist $f(x)=0$, nennt man die Differentialgleichung {\em homogen}, die
Funktion $f(x)$ wird auch die {\em Inhomogenit"at} genannt.
\index{homogen Differentialgleichung}
\index{Inhomogenitat@Inhomogenit\"at}
Die L"osungsmenge einer homogenen linearen Differentialgleichung
bildet einen Vektorraum: jede Linearkombination von L"osungen
ist wieder eine L"osung.
Seien zum Beispiel $y_1(x)$ und $y_2(x)$ L"osungen der Differentialgleichung
(\ref{grundlagen:linearedgl}).
Wir m"ochten zeigen, dass
$y(x)=\alpha y_1(x)+\beta y_2(x)$ eine L"osung ist.
Die Ableitungen selbst sind linear:
\begin{align*}
y^{(k)}&=\alpha y_1^{(k)}(x)+\beta y_2^{(k)}(x).
\end{align*}
Setzt man dies in die Differentialgleichung ein, erh"alt man
\begin{align*}
a_ny^{(n)}+\dots+a_1y'+a_0y
&=
a_n\alpha y_1^{(n)}+a_n\beta y_2^{(n)}+\dots+a_1\alpha y_1'+a_1\beta y_2'
+ a_0\alpha y_1+a_0\beta y_2
\\
&=
\alpha(\underbrace{a_ny_1^{(n)}+\dots+a_1y_1'+a_0y_1}_{=0})
+
\beta(\underbrace{a_ny_2^{(n)}+\dots+a_1y_2'+a_0y_2}_{=0})=0,
\end{align*}
die Linearkombination $y$ erf"ullt also die homogene Differentialgleichung
ebenfalls.


\subsection{Variation der Konstanten}
\subsection{Laplace-Transformation}

%
% numerik.tex -- numerische Lösung von gewöhnlichen Differentialgleichungen
%
% (c) 2015 Prof Dr Andreas Mueller, Hochschule Rapperswil
%
\chapter{Numerische L"osung\label{chapter:numerik}}
\lhead{}
\rhead{Numerische L"osung}
\index{Numerische Loesung@Numerische L\"osung}
Im Kapitel~\ref{chapter:grundlagen} waren wir in der Lage, f"ur einige
einfache Differentialgleichungen eine L"osung in geschlossener Form
zu finden.
Zum Beispiel konnten wir lineare Differentialgleichungen mit Hilfe
der Exponentialfunktion l"osen.
Dieses Bild tr"ugt allerdings.
Die meisten Differentialgleichungen k"onnen nicht in geschlossener
Form gel"ost werden.
Wir k"onnen daher nicht erwarten, dass wir die L"osungen beliebiger
Differentialgleichungen einfach dadurch verstehen, dass wir
L"osungsfunktionen diskutieren.
Stattdessen bleiben uns nur die folgenden zwei M"oglichkeiten:
\begin{enumerate}
\item
Wir l"osen die Differentialgleichung mit Hilfe eines Computers,
und studieren den Verlauf der L"osungsfunktionen oder die Abh"angigkeit
von Parameter oder Anfangsbedingungen durch Vergleich verschiedener
numerisch gefundener L"osungen.
\item
Wir entwickeln Methoden, mit denen sich Aussagen "uber den Verlauf der
L"osungskurven studieren lassen, ohne dass man sie berechnet haben muss.
Nat"urlich kann man nicht erwarten, dass eine solche Methode genaue
Aussagen dar"uber erlaubt, wann eine L"osungskurve wo genau durchgehen
wird.
Es werden nur qualitative Aussagen m"oglich sein, zum Beispiel ob
Gleichgewichtsl"osungen stabil sind, ob es periodische L"osungen gibt
und ob L"osungskurven zu den periodischen L"osungen konvergieren.
\end{enumerate}
In diesem Kapitel entwickeln wir Methoden, Differentialgleichungen 
numerisch zu l"osen.

\section{Grundprinzip}
\lhead{Grundprinzip}
\begin{figure}
\centering
\includegraphics{chapters/images/numerik-2.pdf}
\caption{Lineare Approximation von $y(x+\Delta x)$ durch Information,
die am Punkt $x$ verf"ugbar ist.
\label{numerik:lineareapproximation}}
\end{figure}
Wir versuchen die Differentialgleichung
\begin{equation}
y'=-\alpha y,\qquad y(0)=y_0
\label{numerik:expdgl}
\end{equation}
numerisch zu l"osen. 
Dazu unterteilen wir die $x$-Achse in diskrete Abschnitte der L"ange $h$,
und bezeichnen die Teilpunkte mit $x_k=kh$.
Das Ziel ist jetzt, $y(x_k)$ n"aherungsweise zu berechnen.
Wir schreiben $y_k$ f"ur die N"aherungswerte von $y(x_k)$.
Die Ableitung liefert eine lineare Approximation f"ur $y(x)$,
n"amlich
\[
y(x+\Delta x)\simeq y(x) + y'(x)\cdot\Delta x
\]
(Abbildung~\ref{numerik:lineareapproximation}).
F"ur die Punkte $x_k$ bedeutet das
\[
y(x_{k+1})\simeq y(x_{k})+y'(x_k)h.
\]
Die Differentialgleichung liefert Werte f"ur $y'(x_k)$ aus $x_k$ und $y(x_k)$,
damit k"onnen wir aus dieser Approximation ein allgemeines
N"aherungsverfahren f"ur die L"osung einer Differentialgleichung
konstruieren.

\begin{satz}[Euler-Verfahren]
\index{Euler-Verfahren}
Die Differentialgleichung
\begin{equation}
y'=f(x,y),\qquad y(0)=y_0
\label{numerik:eulerdgl}
\end{equation}
und die Schrittweite $h$ definieren eine Folge 
\[
y_{\mathstrut k}=y_{k-1} + h\cdot f(x_{k-1}, y_{k-1}),\quad k>0,
\]
mit $x_k=kh$,
die eine N"aherung f"ur die Funktionswerte $y(x_k)$ der L"osung $y(x)$
der Differentialgleichung~(\ref{numerik:eulerdgl}) ist.
\end{satz}

Dieses Verfahren ist nicht besonders gut, wie wir im Folgenden zeigen
wollen.
Die Diskussion soll uns aber zeigen, worauf bei der Weiterentwicklung
des Verfahrens geachtet werden muss.

Im vorliegenden Beispiel liefert die
Differentialgleichung~(\ref{numerik:expdgl})
den Wert $y'(x_k)=-\alpha y(x_k)$ f"ur die Ableitung,
woraus wir die Rekursionsformel
\[
y_{k+1}=y_k - \alpha y_k \dot h.
\]
gewinnen.
Die Rekursionsgleichung kann in diesem Fall exakt gel"ost werden,
und wir finden
\begin{equation}
y(x_{k+1}) = y(x_k)-\alpha y(x_k) h=(1-\alpha h) y(x_k)=\dots
=(1-\alpha h)^{k+1}y_0
\label{numerik:rekursion}
\end{equation}
f"ur die N"aherung $y_k$ der Funktionswerte $y(x_k)$.
%Angewendet auf eine beliebige Differentialgleichung, ist dieses
%einfache numerische Verfahren bekannt als das {\em Euler-Verfahren}.
%Es ist nicht besonders genau, aber soll in diesem Abschnitt dazu
%dienen, die Anforderungen an ein gutes numerisches Verfahren
%zu illustrieren.



Wir m"ochten $y(x)$ f"ur einen ganz bestimmten $x$-Wert berechnen.
Dazu unterteilen wir das Intervall $[0,x]$ in $n$ Teilschritte der
Breite $x/n$, und wenden die Formel~(\ref{numerik:rekursion}) an:
\[
y(x)=y(x_n)=(1-\alpha h)^n y_0=\biggl(1+\frac{-\alpha x}{n}\biggr)^n y_0.
\]
F"ur eine grosse Zahl von Teilschritten erhalten wir so tats"achlich die
korrekte L"osung:
\[
\lim_{n\to\infty}y_0\biggl(1+\frac{-\alpha x}n\biggr)^n=y_0 e^{-\alpha x}.
\]
\begin{figure}
\centering
\includegraphics{chapters/images/numerik-1.pdf}
\caption{Approximationen der L"osung der Differentialgleichung $y'=-\alpha y$
mit verschiedener Anzahl Schritte (rot) n"ahern sich f"ur wachsendes
$n$ der exakten L"osung (blau).
\label{numerik:approximation}}
\end{figure}%
Abbildung~\ref{numerik:approximation} zeigt, wie die
durch~(\ref{numerik:rekursion}) gegebenen Approximationen mit zunehmendem
$n$ der exakten L"osung $y(x)=e^{-\alpha x}$ n"aher kommen.

Wir k"onnen auch den Fehler des numerischen Verfahrens berechnen.
Bei der Schrittweite $h$ ist der Fehler von $y_k$ die Differenz
\[
y(x_k)-y_k
=
y_0e^{-\alpha kh}-y_0(1-\alpha h)^k
=
y_0((e^{-\alpha h})^k - (1-\alpha h)^k)
=
y_0e^{-\alpha hk}\biggl(
1-\biggl(\frac{1-\alpha h}{e^{-\alpha h}}\biggr)^k
\biggr).
\]
Man beachte, dass der Z"ahler $1-\alpha h$ die Approximation
$y_1$ ist, als eine Approximation von $e^{-\alpha h}$, dem Nenner.
Schreiben wir
\[
q=\frac{1-\alpha h}{e^{-\alpha h}},
\]
f"ur den Quotienten zwischen der Approximation und dem korrekten Wert,
dann ist sicher immer $q<1$.
Den Fehler k"onnen wir jetzt schreiben
\[
y(x_k)-y_k = y_0e^{-\alpha hk}(1-q^k) = y(x_k)(1-q^k).
\]
Der relative Fehler des Verfahrens ist also
\[
\frac{y(x_k)-y_k}{y(x_k)}=(1-q^k).
\]
\begin{figure}
\centering
\includegraphics{chapters/images/numerik-3.pdf}
\caption{Relativer Fehler des Euler-Verfahrens f"ur die Differentialgleichung
(\ref{numerik:expdgl}) in Abh"angigkeit von der Anzahl $k$ der Schritte.
\label{numerik:relfehler}}
\end{figure}%
Ganz unabh"angig von der Schrittweite $h$ wird der relative Fehler
des Verfahrens immer gegen 1 streben, der Fehler wird also von der
gleichen Gr"ossenordnung wie die berechneten Resultate.

Die Abbildung~\ref{numerik:relfehler} zeigt, dass zu Beginn des Verfahrens
der relative Fehler ungef"ahr linear mit der Anzahl der Schritt zunimmt.
Um eine angemessene Genauigkeit "uber einen gr"osseren Bereich
zu erreichen, muss das Euler-Verfahren also sehr viel kleinere Schritte
und eine entsprechend gr"ossere Anzahl von Schritten ausf"uhren,
die entsprechend viel Rechenzeit ben"otigen.

Ein praktisch n"utzliches Verfahren muss also anstreben, mit einer
sehr viel kleineren Anzahl von Schritten eine viel gr"ossere Genaugikeit
der Approximation zu erreichen.

\section{Fehler-Entwicklung numerischer L"osungen}
\lhead{Fehler-Entwicklung}
Wir betrachten wieder die Differentialgleichung~(\ref{numerik:eulerdgl})
und versuchen, den Fehler eines N"aherungsverfahrens zu bestimmen,
welches Schritte der Gr"osse $h$ durchf"uhrt, um den Wert $y(x)$
zu approximieren.

Das Euler-Verfahren verwendet Schritte der Form
\[
y_{k+1}=y_{k\mathstrut} + hf(x_{k\mathstrut},y_{k\mathstrut}).
\]
In jedem einzelnen Schritt entsteht ein Fehler, dessen Gr"osse wir
aus der Taylor-Entwicklung
\[
y(x+\Delta x)=
y(x) + y'(x)\cdot \Delta x + R(x) \Delta x^2
\]
absch"atzen k"onnen.
Die Funktion $R(x)$ ist beschr"ankt und beschreibt den verbleibenden
Fehler.
Um $y(x)$ zu approximieren, m"ussen $n=x/h$ Schritte der Schrittweite
$h$ durchgef"uhrt werden, von denen jeder einen Fehler
von der Gr"ossenordnung $R(x)h^2$ hat.
Der Gesamtfehler ist daher von der Gr"ossenordnung
\[
y(x)-y_n=O\biggl(R(x)h^2\frac{x}h\biggr)=O(h),
\]
er ist also von erster Ordnung in $h$.
Um eine zus"atzliche Stelle Genauigkeit zu erhalten, muss man also zehnmal
so viele Schritte von zehnmal kleinerer Gr"osse durchf"uhren,
wodurch auch wieder Rundungsfehler eingef"uhrt werden.

K"onnte man den Fehler des Einzelschrittes wesentlich verkleinern, w"urde
auch die Abh"angigkeit des Fehlers des Verfahrens vorteilhafter.
W"are der Fehler des Einzelschrittes $O(h^k)$ statt $O(h^2)$, dann
w"are der Gesamtfehler des Verfahrens nur noch $O(h^{k-1})$.
F"ur $k=3$ bedeutet dies, dass eine Halbierung der Schrittweite
zwar doppelt so viele Schritte braucht, aber auch, dass in jedem
Schritt nur ein Achtel des Fehlers auftritt.
Der Gesamtfehler ist also nur ein Viertel.
Mit zehnmal mehr Arbeit kann man also nicht nur eine Stelle an
Genauigkeit gewinnen, sondern gleich deren zwei.

Man nennt ein Verfahren, bei dem der Gesamt-Fehler von der Gr"ossenordnung
$O(h^k)$ ist, von einem Verfahren $k$-ter Ordnung.
Das Euler-Verfahren ist also ein Verfahren erster Ordnung oder ein
lineares Verfahren.
In der Praxis werden Verfahren bis zu vierter und f"unfter Ordnung
verwendet, so dass eine zehnmal kleinere Schrittweite zu gleich
vier Stellen Genauigkeitsgewinn f"uhren.
Das Ziel der kommenden Abschnitte muss daher sein, einfach
berechnebare Approximationen der Funktion mit m"oglichst geringen
Einzelschrittfehlern zu finden.

\section{Einschritt-Verfahren\label{section:numerik:einschritt}}
\lhead{Einschritt-Verfahren}
Die relativ geringe Genauigkeit des Eulerschrittes beruht darauf,
dass die zu Beginn des Schrittes berechnete Ableitung $f(x_k,y_k)$
nur f"ur das linke Ende des Intervalls $[x_k, x_k+h]$ zutrifft,
weiter rechts im Intervall wird die Abweichung immer gr"osser.
Eine m"ogliche L"osung des Problems k"onnte darin bestehen, statt
nur einer linearen N"aherung zus"atzliche Glieder der Taylorreihe
\begin{equation}
y(x+\Delta x)
=
y(x)
+
y'(x)\cdot \Delta x
+
\frac12 y''(x)\cdot \Delta x^2
+
\frac16 y'''(x)\cdot \Delta x^3
+
o(\Delta x^3)
\label{numerik:taylor}
\end{equation}
zu verwenden.
In (\ref{numerik:taylor}) werden h"ohere Ableitungen von $y(x)$ ben"otigt,
w"ahrend die Differentialgleichung nur die erste Ableitung liefert.
Die h"oheren Ableitungen wurden aber bereits im
Abschnitt~\ref{grundlagen:hoehere-ableitungen} berechnet.

Wir untersuchen, wie sich das Verfahren f"ur die Beispiel-Gleichung
(\ref{numerik:expdgl}) anwenden l"asst.
Dort gilt
\begin{equation*}
\begin{aligned}
y'(x)&=f(x,y)=-\alpha y
\\
\Rightarrow\qquad
\frac{\partial f}{\partial x}&=0&\frac{\partial f}{\partial y}&=-\alpha
\end{aligned}
\end{equation*}
Alle zweiten Ableitungen verschwinden.
Die Gleichungen werden damit einfach:
\begin{align*}
y''(x)&=-\alpha f(x,y)=\alpha^2 y
\\
y'''(x)&=\alpha^2f(x,y)=-\alpha^3 y.
\end{align*}
Statt der linearen Approximation sollte daher die kubische Approximation
\begin{equation}
y_{k+1}
=
y_{k}-\alpha h y_k +\frac12\alpha^2 h^2 y_k -\frac16 \alpha^3h^3 y_k
=
y_{k}\underbrace{\biggl(1-\alpha h +\frac12\alpha^2h^2 -\frac16 \alpha^3h^3\biggr)}_{\simeq e^{-\alpha h}}
\label{numerik:kubisch}
\end{equation}
verwendet werden.
Dass man hier mit einer gr"osseren Genauigkeit rechnen darf ist schon daran
erkennbar, dass der Klammerausdruck auf der rechten Seite eine viel
bessere Approximation von $e^{-\alpha x}$ ist also der Faktor
$(1-\alpha h)$ im Euler-Verfahren.
Genauer erwarten wir, dass wir hier ein kubisches Verfahren konstruiert haben.

\begin{table}
\centering
\begin{tabular}{|r|c|r|r|r|}
\hline
$i$&$x$&$e^{-\alpha x}$&Euler&kubisch\\
\hline
 1 & 0.1 & 0.95122942 & 0.\underline{95}000000 & 0.\underline{951229}17 \\
 2 & 0.2 & 0.90483742 & 0.\underline{90}250000 & 0.\underline{904836}93 \\
 3 & 0.3 & 0.86070798 & 0.\underline{85}737500 & 0.\underline{860707}28 \\
 4 & 0.4 & 0.81873075 & 0.\underline{81}450625 & 0.\underline{818729}87 \\
 5 & 0.5 & 0.77880078 & 0.\underline{77}378094 & 0.\underline{778799}73 \\
 6 & 0.6 & 0.74081822 & 0.\underline{73}509189 & 0.\underline{74081}702 \\
 7 & 0.7 & 0.70468809 & 0.\underline{6}9833730 & 0.\underline{70468}675 \\
 8 & 0.8 & 0.67032005 & 0.\underline{6}6342043 & 0.\underline{67031}859 \\
 9 & 0.9 & 0.63762815 & 0.\underline{6}3024941 & 0.\underline{63762}660 \\
10 & 1.0 & 0.60653066 & 0.\underline{5}9873694 & 0.\underline{60652}902 \\
\hline
\end{tabular}
\caption{N"aherungswerte f"ur die L"osung $e^{-\alpha x}$ der
Beispieldifferentialgleichung (\ref{numerik:expdgl}) nach dem Euler-Verfahren
und nach dem kubischen Verfahren (\ref{numerik:kubisch}) mit einer
Schrittweite von 0.1. Unterstrichen ist jeweils die Stellen, die nach
Rundung auf die angegebene Anzahl stellen mit dem exakten Wert "ubereinstimmt.
\label{numerik:euler-kubisch}}
\end{table}%
In Tabelle~\ref{numerik:euler-kubisch} werden die Resultate des
kubischen Verfahrens denen des Euler-Verfahrens gegen"ubergestellt.
Im ersten Schritt ist der Fehler des Euler-Verfahrens kleiner als $10^{-2}$,
was einer Einheit in der zweiten Nachkommastelle entspricht.
Der Fehler des kubischen Verfahrens ist kleiner als $10^{-6}$, eine
Einheit in der sechsten Nachkommastelle, ungef"ahr die von einem
kubischen Verfahren zu erwartende Verbesserung.
Nach zehn Rechenschritten liefert das Euler-Verfahren dank Rundung
gerade noch eine korrekte Stelle, w"ahrend das kubische Verfahren immer noch
gerundet f"unf korrekte Stellen gibt.

Es wurde bereits darauf hingewiesen, dass die Terme f"ur die Ableitungen
sehr kompliziert werden.
noch viel gravierender ist allerdings, dass auch die partiellen Ableitungen
von $f$ nach $x$ und $y$ bekannt sein m"ussen.
Es ist zwar im Prinzip m"oglich, diese zu berechnen, der Rechenaufwand 
daf"ur kann aber so erheblich sein, dass er den Genauigkeitsgewinn
leicht wieder zunichte machen kann.
Praktisch n"utzliche Verfahren m"ussen daher danach streben,
die h"oheren Ableitungen von $y(x)$ ausschliesslich aus Funktionswerten
von $f(x,y)$ zu berechnen.

Wir m"ochten aber weiterhin nur $y_{k+1}$ ausschliesslich aus $x_k$ und $y_k$
berechnen, also in einem einzelnen Schritt der Form
\[
y_{k+1}=y_k + h\, F(x_k, y_k, h).
\]
Die Funktion $F(x,y,h)$ heisst die {\em Inkrement-Funktion}
\index{Inkrement-Funktion}
des Verfahrens.
F"ur das Euler-Verfahren ist $F(x,y,h)=f(x,y)$.
Es soll also eine Inkrement-Funktion gefunden werden, bei der $y(x+\Delta x)$
durch $y(x) + \Delta x\cdot F(x,y,\Delta x)$ bis auf Terme h"oherer
Ordnung approximiert werden kann.

\subsection{Quadratische Verfahren}
Ein quadratisches Verfahren verwendet eine Inkrement-Funktion $F(x,y,h)$,
welche
\[
y(x+h)=y(x)+hF(x,y,h)+O(h^3)
\]
erf"ullt.
Aus den einleitenden Bemerkungen von~\ref{section:numerik:einschritt}
folgt, dass dieses Ziel m"oglicherweise dadurch erreicht werden kann,
dass man Werte von $f$ f"ur verschiedene $x$ geeignet miteinander
kombiniert.
Ein denkbarer Ansatz daf"ur ist
\[
F(x,y,h)=af(x,y) + bf(x+\alpha h, y +\beta hf(x,y)),
\]
oder anders ausgedr"uckt: Man f"uhrt zuerst etwas "ahnliches wie einen
Eulerschritt durch, um zum Punkt $(x+\alpha h,y+\beta hf(x,y))$ zu
gelangen.
Dort berechnet man den Wert von $f$, und bildet dann einen geeigneten
Mittelwert davon  mit $f(x,y)$.
Durch geeignete Wahl von $a$, $b$, $\alpha$ und $\beta$ sollte es m"oglich
sein, dass die Inkrement-Funktion einen Fehler h"ochstens dritter Ordnung
hat, womit wir dann ein Integrationsverfahren zweiter Ordnung gewonnen
h"atten.

Wir m"ussen jetzt die Parameter $a$, $b$, $\alpha$ und $\beta$ bestimmen.
Da wir mit dem "ubereinstimmen der ersten zwei Ableitungen
nur zwei Bedingungen haben, k"onnen wir nicht erwarten, dass wir
eine eindeutige L"osung finden werden.
Vielmehr werden einzelne Parameter frei w"ahlbar sein, es wird eine
ganze Familie von quadratischen L"osungsverfahren entstehen, parametrisiert
durch eine der Variablen $a$, $b$, $\alpha$ und $\beta$.

Wir berechnen nun $F(x,y,h)$ bis zur zweiten Ordnung, damit wird 
$y(x+h)$ bis zur dritten Ordnung ausdr"ucken k"onnen.
\begin{align*}
f(x+\alpha h, y + \beta h f(x,y))
&=
f(x,y)+\alpha h\frac{\partial f(x,y)}{\partial x}
+ \beta h \frac{\partial f(x,y)}{\partial y} + O(h^2)
\end{align*}
\begin{align}
F(x,y,h)
&=
af(x,y) + bf(x+\alpha h, y + \beta h f(x,y))
\notag
\\
&=
(a+b)f(x,y) + \biggl(\alpha b\frac{\partial f(x,y)}{\partial x}
+ \beta b\frac{\partial f(x,y)}{\partial y} f(x,y))\biggr)h+O(h^2)
\label{numerik:inkrementF}
\end{align}
Damit dies bis zur zweiten Ordnung mit dem Inkrement zwischen $x$ und $x+h$
"ubereinstimmt, muss~(\ref{numerik:inkrementF}) mit der Taylorreihe
von $y(x)$ "ubereinstimmen, also mit
\begin{equation}
\frac{y(x+h)-y(x)}{h}=y'(x) + \frac12y''(x)h + O(h^2)
=f(x,y) + \frac12\frac{\partial f(x,y)}{\partial x}
+\frac12\frac{\partial f(x,y)}{\partial y}f(x,y) + O(h^2),
\label{numerik:ytaylor}
\end{equation}
wobei wir f"ur $y''(x)$ die Gleichung (\ref{grundlagen:2abl}) verwendet haben.
Durch Koeffizientenvergleich finden wir die Bedingungen
\[
\begin{aligned}
a+b&=1,&
\alpha b&=\frac12,&
\beta b&=\frac12.
\end{aligned}
\]
Einzig $b$ kommt in allen drei Gleichungen vor, und bestimmt den Wert der
jeweiligen anderen Variablen:
\[
\begin{aligned}
a&=1-b,&\alpha&= \beta=\frac{1}{2b}.
\end{aligned}
\]
Jeder Wert von $b$ zwischen $0$ und $1$ liefert ein Verfahren mit quadratischer
Genauigkeit.

Der Parameterwert $b=1$ f"uhrt auf $\alpha=\beta=1$ und $a=0$, die
Rekursionsformel ist in diesem Falle
\begin{equation}
y_{k+1}=y_{k}+hf\biggl(x_k+\frac{h}2,y_k+\frac{h}2 f(x_k,y_k)\biggr).
\label{numerik:improved-euler}
\end{equation}
Das Verfahren f"uhrt also erst einen halben Eulerschritt zum Punkt
$(x_k+\frac12h,y_k+\frac{h}2f(x_k,y_k))$ durch, berechnet dort mit Hilfe
von $f$ die Steigung, die dann f"ur einen Eulerschritt der L"ange $h$
verwendet wird.u
Daher heisst dieses Verfahren auch das {\em verbesserte Euler-Verfahren}.
\index{Euler-Verfahren!verbessertes}

Verwendet man $b=\frac12$, folgt zun"achst $a=\frac12$ und $\alpha=\beta=1$.
Daraus erh"alt man die Rekursionsformel
\begin{equation}
y_{k+1}=y_k+\frac{h}2\biggl(
f(x_k,y_k) + f(x_k+h, y_k + hf(x_k,y_k))
\biggr)
\label{numerik:simplified-runge-kutta}
\end{equation}
In diesem Verfahren f"uhrt man also zuerst einen Eulerschritt der L"ange
$h$ durch, mit dem man zum Punkt $(x_k+h, y_k+hf(x_k,y_k))$ gelangt.
Dort berechnet mit mit Hilfe von $f$ die Steigung.
Das arithmetische Mittel dieser Steigung mit der im Euler-Verfahren
verwendeten Steigung $f(x_k,y_k)$ im Punkt $x_k$ wird dann als
Steigung f"ur einen Eulerschritt verwendet.
Statt eines einzigen Steigungswertes werden hier also zwei Steigungswerte
von den Enden des Intervalls $[x_k,x_k+1]$ gemittelt.
Wegen der "Ahnlichkeit dieses Vorgehens mit dem sp"ater zu besprechenden
Runge-Kutte-Verfahren heisst diese Verfahren auch das {\em
vereinfachte Runge-Kutta-Verfahren}.
\index{Runge-Kutta-Verfahren!vereinfachtes}

\subsection{Runge-Kutta-Verfahren\label{subsection:numerik:runge-kutta}}
\index{Runge-Kutta-Verfahren}
Das {\em Runge-Kutta-Verfahren} erweitert die Inkrement-Funktion derart,
dass der Einzelschritt bis zur f"unften Ordnung mit der Taylorreihe von
$y(x)$ "ubereinstimmt.
So entsteht ein Verfahren vierter Ordnung, es stellt einen guten Kompromiss
zwischen Genauigkeit und Rechenaufwand dar.

Da vier Ableitungen korrekt dargestellt werden m"ussen, ist zu erwarten,
dass vier verschiedene Werte von $f$ an verschiedenen Punkten $(x,y)$
ausgewertet und geeignet miteinander kombiniert werden m"ussen.
Genauer: Man bestimmt zuerst die Werte
\begin{align*}
k_1&=f(x_k,y_k)\\
k_2&=f\biggl(x_k+\frac{h}2,y_k+\frac{h}2k_1\biggr)\\
k_3&=f\biggl(x_k+\frac{h}2,y_k+\frac{h}2k_2\biggr)\\
k_4&=f(x_k+h, y_k+hk_3)
\end{align*}
und setzt diese dann zusammen, um den n"achsten Wert $y_{k+1}$
zu berechnen:
\begin{equation}
y_{k+1} = y_k + h\frac{1}6(k_1 + 2k_2 + 2k_3 + k_4).
\label{numerik:runge-kutta-rekursion}
\end{equation}
Man kann die Formeln wie folgt interpretieren.
Zuerst wird ein halber Eulerschritt mit der Steigung $k_1=f(x_k,y_k)$,
durchgef"uhrt, und und am Zielpunkt die Steigung $k_2$ ermittelt.
Mit dieser Steigung wird dann erneut ein halber Schritt von $(x_k,y_k)$
aus durchgef"uhrt, und am Zielpunkt erneut die Steigung $k_3$ ermittelt.
Damit f"uhrt man einen ganzen Schritt aus, an dessen Zielpunkt man die
Steigung $k_4$ findet.
Diese vier Steigungen werden jetzt gewichtet gemittelt, wobei
$k_2$ und $k_3$ doppeltes Gewicht erhalten, und mit dieser
Steigung wird ein ganzer Schritt vorgenommen.

Die Formeln f"ur die $k_i$ sowie (\ref{numerik:runge-kutta-rekursion})
k"onnen ganz "ahnlich wie das verbesserte Euler-Verfahren bzw.~das
vereinfachte Runge-Kutta-Verfahren begr"undet werden.
Der Aufwand daf"ur ist aber betr"achtlich, so dass wir auf die
detaillierte Darstellung dieser Herleitung verzichten wollen.

\begin{table}
\centering
\begin{tabular}{|r|c|r|r|r|r|r|}
\hline
$i$& $x$ & $y(x)=e^{-\alpha x}$&Euler&verbessert&vereinfacht&Runge-Kutta\\
\hline
 0 & 0.0 & 1.00000000 & 1.000 & 1.00000000 & 1.00000000 & 1.0000000000 \\
 1 & 0.1 & 0.95122942 & 0.\underline{95}0 & 0.\underline{9512}5000 & 0.\underline{9512}5000 & 0.\underline{95122942}71 \\
 2 & 0.2 & 0.90483742 & 0.\underline{90}2 & 0.\underline{9048}7656 & 0.\underline{9048}7656 & 0.\underline{9048374}229 \\
 3 & 0.3 & 0.86070798 & 0.\underline{85}7 & 0.\underline{8607}6383 & 0.\underline{8607}6383 & 0.\underline{8607079}834 \\
 4 & 0.4 & 0.81873075 & 0.\underline{81}4 & 0.\underline{8188}0159 & 0.\underline{8188}0159 & 0.\underline{8187307}620 \\
 5 & 0.5 & 0.77880078 & 0.\underline{77}3 & 0.\underline{7788}8502 & 0.\underline{7788}8502 & 0.\underline{7788007}936 \\
 6 & 0.6 & 0.74081822 & 0.\underline{73}5 & 0.\underline{7409}1437 & 0.\underline{7409}1437 & 0.\underline{7408182}327 \\
 7 & 0.7 & 0.70468809 & 0.\underline{69}8 & 0.\underline{704}79480 & 0.\underline{704}79480 & 0.\underline{7046881}031 \\
 8 & 0.8 & 0.67032005 & 0.\underline{6}63 & 0.\underline{670}43605 & 0.\underline{670}43605 & 0.\underline{6703200}606 \\
 9 & 0.9 & 0.63762815 & 0.\underline{6}30 & 0.\underline{637}75229 & 0.\underline{637}75229 & 0.\underline{6376281}672 \\
10 & 1.0 & 0.60653066 & 0.\underline{5}98 & 0.\underline{606}66187 & 0.\underline{606}66187 & 0.\underline{6065306}762 \\
\hline
\end{tabular}
\caption{Vergleich der Genauigkeit der verbesserten numerischen Verfahren.
Unterstrichen jeweils die nach Rundung korrekten Stellen der L"osung.
\label{numerik:genauigkeit}}
\end{table}


\begin{table}
\centering
\begin{tabular}{|l|l|c|r|>{$}r<{$}|}
\hline
Verfahren                           &$h$  &Schritte&$y_n$&\text{Fehler}\\
\hline
Euler-Verfahren                     &0.025&  40    & 0.\underline{60}462232 &  0.00190834 \\
verbessertes Euler-Verfahren        &0.05 &  20    & 0.\underline{6065}6285 & -0.00003219 \\
vereinfachtes Runge-Kutta-Verfahren &0.05 &  20    & 0.\underline{6065}6285 & -0.00003219 \\
Runge-Kutta-Verfahren               &0.1  &  10    & 0.\underline{6065306}7 & -0.00000001 \\
\hline
\end{tabular}
\caption{Vergleich der verschiedenen Verfahren bei gleichbleibendem 
Rechenaufwand.
Die Schrittweite wurde jeweils so angepasst, dass in allen Verfahren bis
zum Wert $x=1$ die gleiche Anzahl von Auswertungen der Funktion $f$
notwendig wurde.
\label{numerik:vergleich-aufwand}}
\end{table}

Die Tabelle~\ref{numerik:genauigkeit} demonstriert die "uberragende
Genauigkeit des Runge-Kutta-Verfahrens.
Trotz der relativ grossen Schrittweite von $h=0.1$ erreicht das
Verfahren nach zehn Schritten eine Genauigkeit von sieben signifikanten
Stellen.
Da in jedem Schritt die Funktion $f$ viermal ausgewertet werden muss,
ist der Rechenaufwand mit dem Runge-Kutta-Verfahren viermal gr"osser
als im Euler-Verfahren, letzteres kann aber mit nur einer signifikanten
Stelle kaum als brauchbar bezeichnet werden.
Passt man in jedem Verfahren die Schrittweite so an, dass f"ur die
Berechnung der N"aherung f"ur $y(1)$ immer gleich viele Auswertungen
der Funktion $f(x,y)$ n"otig sind, ergeben sich die Resultate in
Tabelle~\ref{numerik:vergleich-aufwand}.
Bei gleichem Rechenaufwand ist das Runge-Kutta-Verfahren um viele
Gr"ossenordungen pr"aziser.
Es gibt daher eigentlich keinen praktischen Grund, "uberhaupt je etwas
anderes als das Runge-Kutta-Verfahren zu verwenden.


\section{Mehrschritt-Verfahren}
\lhead{Mehrschritt-Verfahren}
In den Einschritt-Verfahren wurde wiederholt die Funktion $f$ ausgewertet,
um die Inkrement-Funktion f"ur einen einzigen Schritt zu bestimmen.
Das Ziel dabei war, $y(x+h)$ in "Ubereinstimmung mit der Taylorreihe
bis zu m"oglichst hoher Ordnung zu bestimmen.
Im Runge-Kutta-Verfahren wurden dabei halbe Eulerschritte durchgef"uhrt,
man hat also eigentlich die Aufl"osung nochmals halbiert, um die
Inkrement-Funktion zu ermitteln.
Diese Zwischenwerte geben dem Verfahren die Information "uber die
h"oheren Ableitungen der Funktionen.

Sobald einige Werte der L"osung berechnet sind, l"asst sich die Kr"ummung
der L"osungskurve auch aus diesen Werten ablesen.
Es sollte daher auch m"oglich sein, aus mehreren bereits
ermittelten Werten $y_{n\mathstrut},y_{n+1},\dots,y_{n+s-1}$
den n"achsten Wert $y_{n+s\mathstrut}$ mit der verlangten Genauigkeit
zu berechnen.
Der Vorteil eines solchen Vorgehens ist, dass f"ur jeden Schritt nur 
eine einzige Auswertung der Funktion $f$ n"otig ist,
nicht mehrere wie bei den besprochenen Einschritt-Verfahren.

Als Beispiel versuchen wir daher ein Verfahren aufzubauen, welches
$y_{n+2}$ aus den bereits berechneten Werten $y_{n\mathstrut}$ und
$y_{n+1}$ berechnet.
Wir nehmen dabei an, dass $y_{n\mathstrut}$ und $y_{n+1}$ exakt
sind.
Der neue Datenpunkt soll mit Hilfe eines Ausdrucks der Form
\begin{equation}
y_{n+2}=y_{n+1} + h(af(x_{n+1},y_{n+1}) + b f(x_{n\mathstrut},y_{n\mathstrut}))
\label{numerik:zweischrittansatz}
\end{equation}
gefunden werden.
Die N"aherung kann wieder mit Hilfe der Ableitungen alleine
durch Werte bei $x_{n+1}$ ausgedr"uckt werden:
\begin{align*}
y_{n+2}
&=
y_{n+1}+h(af(x_{n+1}, y_{n+1}) + bf(x_{n+1}-h, y_{n\mathstrut}))
\\
&=
y_{n+1}+haf(x_{n+1}, y_{n+1}) + hbf(x_{n+1}-h, y_{n+1} - h f(x_{n+1},y_{n+1}) + O(h^2))
\\
&=
y_{n+1}+haf(x_{n+1}, y_{n+1}) + hb
\biggl(
f(x_{n+1},y_{n+1})
-h \frac{\partial f(x_{n+1},y_{n+1})}{\partial x}
\\
&\qquad
-h
\frac{\partial f(x_{n+1},y_{n+1})}{\partial y}
f(x_{n+1},y_{n+1})
+ 
\frac{\partial f(x_{n+1},y_{n+1})}{\partial y}
O(h^2)
\biggr)
\\
&=
y_{n+1}
+ (a+b)hf(x_{n+1},y_{n+1})
- bh^2\biggl(
\frac{\partial f(x_{n+1},y_{n+1})}{\partial x}
+
\frac{\partial f(x_{n+1},y_{n+1})}{\partial y}
f(x_{n+1},y_{n+1})
+O(h^3)
\biggr)
\end{align*}
Sie muss bis zur zweiten Ordnung mit der Taylorreihe "ubereinstimmen:
\begin{align*}
y(x_{n+2})
&=
y_{n+1} + hy'(x_{n+1}) + \frac12h^2 y''(x_{n+1})+O(h^3)
\\
&=
y_{n+1}+hf(x_{n+1},y_{n+1})+\frac12h^2\biggl(
\frac{\partial f(x_{n+1},y_{n+1})}{\partial x}
+
\frac{\partial f(x_{n+1},y_{n+1})}{\partial y}
f(x_{n+1},y_{n+1})
\biggr)
\end{align*}
Vergleicht man Koeffizienten, findet man
\[
\begin{aligned}
a+b&=1&-b&=\frac12&&\Rightarrow&a=\frac32
\end{aligned}
\]
Aus der Formel (\ref{numerik:zweischrittansatz}) wird somit die
Iterationsformel
\begin{equation}
y_{n+2}=y_{n+1}+h\biggl(\frac32f(x_{n+1},y_{n+1})
- \frac12 f(x_{n\mathstrut},y_{n\mathstrut})\biggr)
\end{equation}
Diese Rekursionsformel definiert ein quadratisches Verfahren, das
{\em Adams-Bashforth-Verfahren} mit $s=2$.
\index{Adams-Bashforth-Verfahren}

Das Verfahren kann "ahnlich wie das Runge-Kutta-Verfahren auf h"ohere
Ordnung erweitert werden.
Man findet nach einiger Rechnung
\begin{align*}
s&=1\colon&
y_{n+1}
&=
y_n+hf(x_n,y_n)
\\
s&=2\colon&
y_{n+2}
&=
y_{n+1}+h\biggl(\frac32f(x_{n+1},y_{n+1})-\frac12f(x_n,y_n)\biggr)
\\
s&=3\colon&
y_{n+3}
&=
y_{n+2}+h\biggl(\frac{23}{12}f(x_{n+2},y_{n+2})-\frac43f(x_{n+1},y_{n+1})+\frac{5}{12}f(x_n,y_n)\biggr)
\\
s&=4\colon&
y_{n+4}
&=
y_{n+3}+h\biggl(\frac{55}{24}f(x_{n+3},y_{n+3})
	-\frac{59}{24}f(x_{n+2},y_{n+2})
	+\frac{37}{24}f(x_{n+1},y_{n+1})
	-\frac{3}{8}f(x_n,y_n)
\biggr)
\end{align*}
Es ist also m"oglich, ausgehend von dieser Idee Verfahren beliebig hoher
Ordnung zu produzieren.

\begin{table}
\centering
\begin{tabular}{|r|c|r|r|r|r|}
\hline
$i$& $x$ & $y(x)=e^{-\alpha x}$&Euler&Adams-Bashforth&Runge-Kutta\\
\hline
 0 & 0.0 & 1.00000000 & 1.00000000 & 1.00000000 & 1.0000000000 \\
 1 & 0.1 & 0.95122942 & 0.\underline{95}000000 & 0.\underline{9512}8178 & 0.\underline{95122942}71 \\
 2 & 0.2 & 0.90483742 & 0.\underline{90}250000 & 0.\underline{904}93564 & 0.\underline{90483742}29 \\
 3 & 0.3 & 0.86070798 & 0.\underline{85}737500 & 0.\underline{860}84752 & 0.\underline{86070798}34 \\
 4 & 0.4 & 0.81873075 & 0.\underline{81}450625 & 0.\underline{818}90734 & 0.\underline{81873076}20 \\
 5 & 0.5 & 0.77880078 & 0.\underline{77}378094 & 0.\underline{779}01048 & 0.\underline{77880079}36 \\
 6 & 0.6 & 0.74081822 & 0.\underline{73}509189 & 0.\underline{741}05738 & 0.\underline{74081823}27 \\
 7 & 0.7 & 0.70468809 & 0.\underline{69}833730 & 0.\underline{704}95334 & 0.\underline{7046881}031 \\
 8 & 0.8 & 0.67032005 & 0.\underline{6}6342043 & 0.\underline{670}60827 & 0.\underline{6703200}606 \\
 9 & 0.9 & 0.63762815 & 0.\underline{63}024941 & 0.\underline{637}93648 & 0.\underline{6376281}672 \\
10 & 1.0 & 0.60653066 & 0.\underline{59}873694 & 0.\underline{606}85645 & 0.\underline{6065306}762 \\
\hline
\end{tabular}
\caption{Vergleich der Genauigkeit der Verfahren von Euler,
Adams-Bashforth und Runge-Kutta.
Als Startwerte f"ur das Adams-Bashforth-Verfahren wurden die
Werte $y(-h)=e^{-\alpha h}$ und $y(0)=1$ verwendet, um keine zus"atzlichen
Fehler aus der Durchf"uhrung des ersten Schrittes hinzuzuf"ugen.
\label{numerik:genauigkeit-adams-bashforth}}
\end{table}

In der Tabelle~\ref{numerik:genauigkeit-adams-bashforth} wird
das Adams-Bashforth-Verfahren verglichen mit dem lineare Euler-Verfahren 
und dem Verfahren vierter Ordnung von Runge-Kutta.
Die Verbesserung der Genauigkeit des Adams-Bashforth-Verfahrens
gegen"uber dem Euler-Ver\-fah\-ren ist konsistent damit, dass
das Adams-Bashforth-Verfahren ein quadratisches Verfahren ist.

Nachteilig an den Mehrschritt-Verfahren ist die Notwendigkeit,
gen"ugend viele Werte $y_{n},\dots,y_{n+s-1}$ mit ausreichend
hoher Genauigkeit zu bestimmen, bevor das Mehrschritt-Verfahren
seine Schritte der Ordnung $s$ beginnen kann.
Solange diese Werte nicht zur Verf"ugung stehen, kann ein Mehrschritt-Verfahren
nur Schritte niedrigerer Ordnung als $s$ durchf"uhren.

Bei einem Einschritt-Verfahren kann in jedem Schritt die Schrittweite $h$
ver"andert werden, zum Beispiel f"ur Bereiche von $x$-Werten, in denen
die Steigung von $y(x)$ sehr rasch "andert.

F"ur die Beispiel-Differentialgleichung (\ref{numerik:expdgl}) k"onnen
wir das Adams-Bashforth-Verfahren zweiter Ordnung ($s=2$) vollst"andig
analysieren.
Die Rekursionsformel wird zu
\[
y_{n+2}=y_{n+1}+h\biggl(\frac32 (-\alpha y_{n+1})-\frac12(-\alpha y_n)\biggr)
=
\biggl(1-\frac32\alpha h\biggr)
y_{n+1}
+\frac{\alpha h}{2}
y_{n\mathstrut}
\]
Dies ist eine Differenzengleichung mit konstanten Koeffizienten, man kann
sie mit Hilfe eines Potenzansatzes l"osen. 
Wir nehmen also an, dass $y_n=\lambda^n$, und setzen dies in die
Rekursionsformel ein.
Ausserdem k"urzen wir $\alpha h/2=\delta$  ab.
Wir erhalten
\[
\lambda^{n+2}-(1-3\delta)\lambda^{n+1}-\delta\lambda^n=0.
\]
Nach Division durch $\lambda^n$ erhalten wir die quadratische Gleichung
\[
\lambda^2-(1-3\delta )\lambda-\delta=0
\]
f"ur $\lambda$ mit den L"osungen
\[
\lambda_\pm
=
\frac12(1-3\delta) \pm \frac12\sqrt{(1-3\delta)^2+4\delta}.
\]
Da $\delta$ klein ist, wird $\lambda_-$ ebenfalls klein sein,
w"ahrend $\lambda_+$ n"aher bei $1$ sein wird.
Der dominante Einfluss auf die L"osung r"uhrt also von $\lambda_+$ her.
Um diesen Unterschied genauer zu verstehen, verwenden wir eine
lineare Approximation der Wurzel auf der rechten Seite von $\lambda_\pm$:
\begin{align*}
\sqrt{1+x}
&=
1+\frac{x}{2}-\frac{x^2}{4}+\frac{3x^3}{8}-\dots
\\
\sqrt{x}
&=
\sqrt{x_0+x-x_0}
=
\sqrt{x_0}\sqrt{1+\frac{x-x_0}{x_0}}
=
\sqrt{x_0}\biggl(1+\frac12\frac{x-x_0}{x_0}-\frac14\frac{(x-x_0)^2}{x_0^2}+\dots\biggr)
\\
&=
\sqrt{x_0}+\frac12\frac{x-x_0}{\sqrt{x_0}}-\frac14\frac{(x-x_0)^2}{\sqrt{x_0}^3}+\dots
\end{align*}
Wir verwenden diese Approximation mit $x_0=(1-3\delta)^2$ und $x-x_0=-4\delta$
\begin{align*}
\sqrt{(1-3\delta)^2+4\delta}
&=
(1-3\delta)\biggl(1+\frac12\frac{4\delta}{(1-3\delta)^2}
-\frac14\frac{16\delta^2}{(1-3\delta)^4}+\dots\biggr)
\\
&=(1-3\delta)+\frac12\frac{4\delta}{1-3\delta}
-\frac14\frac{16\delta^2}{(1-3\delta)^3}+\dots
\\
&=1-3\delta+2\delta(1+3\delta)-4\delta^2+O(\delta^3)
\\
&=1-3\delta+2\delta + 2\delta^2+O(\delta^3)
\\
&=1-3\delta+2\delta + \frac12(2\delta)^2+O(\delta^3)
\end{align*}
Damit k"onnen wir jetzt $\lambda_+$ bis zur zweiten Ordnung berechnen:
\begin{align*}
\lambda_+
&=
\frac12\biggl((1-3\delta)+ (1-3\delta)+2\delta+\frac12(2\delta)^2\biggr)
+O(\delta^3)
\\
&=
1-2\delta+\frac12(2\delta)^2+O(\delta^3)
\\
&=e^{-2\delta}+O(\delta^3).
\end{align*}
Die exakte L"osung erf"ullt $y_{n+1}=e^{-2\delta}y_n$, der Faktor
$\lambda_+$ stimmt bis auf Terme mindestens dritter Ordnung mit 
$e^{-2\delta}$ "uberein.
Damit ist erneut best"atigt, dass wir es mit einem quadratischen Verfahren zu
tun haben.

Wir k"onnen auch $\lambda_-$ berechnen, und erhalten
\[
\lambda_-=-\delta-2\delta^2+O(\delta^3).
\]
Da $\delta$ klein ist, ist eine Komponente der L"osung bereits nach
drei Schritten kleiner als $O(\delta^3)$, und spielt daher im Vergleich
zu den von $\lambda_+$ herr"uhrenden L"osungen in dritter Ordnung keine
Rolle.

\section{Software}
Die im letzten Abschnitt entwickelten numerischen Verfahren zur L"osung
einer Differentialgleichung kommen ausschliesslich mit Auswertungen der
Funktion $f$ aus, die Ableitungen der Funktion $f$ m"ussen nicht bekannt
sein.
Es sollte also ein Leichtes sein, eine Softwarebibliothek zur
Verf"ugung zu stellen, mit der eine beliebige gew"ohnliche
Differentialgleichung gel"ost werden kann.
Als Input braucht es nur die Funktion $f$ und die Anfangsbedingungen.

Als Beispiel wollen wir in diesem Abschnitt die Differentialgleichung
\[
y''+y=\sin \frac{x}{10},\qquad y(0)=y'(0)=0
\]
in verschiedenen Programmierumgebungen l"osen.
Als erstes bringen wir die Differentialgleichung wieder in die Standardform
einer Vektordifferentialgleichung erste Ordnung:
\begin{equation}
\frac{d}{dt}Y
=
\frac{d}{dx}\begin{pmatrix}y_1\\y_2\end{pmatrix}
=
\begin{pmatrix}
y_2\\
-y_1+\sin\frac{x}{10}
\end{pmatrix}
=
f(x,Y)
\label{numerik:beispieldgl}
\end{equation}
Ein numerisches Verfahren braucht also als Input eine Anfangsbedingung
sowie die Funktion $f$.
Ausserdem muss es M"oglichkeiten bereitstellen, wie man den Gang der
Rechnung beeinflussen kann, z.~B.~um die $x$-Werte anzugeben, f"ur die
die $Y(x)$ bestimmt werden sollen, oder um Genauigkeitsziele zu erreichen.

\subsection{Octave}
In Octave steht eine einzige Funktion \texttt{lsode} zur Verf"ugung, welche
auf zuverl"assige Art Differentialgleichungen l"ost.
Der Anwender muss eine Implementation der Funktion $f$ zur Verf"ugung
stellen, allerdings werden die Argument in der umgekehrten Reihenfolge
zu der erwarte, die wir in diesem Skript bisher verwendet haben.
F"ur die Beispieldifferentialgleichung (\ref{numerik:beispieldgl})
kann man sie zum Beispiel so definieren:
\verbatiminput{chapters/examples/octave-dgl-f.m}

Beim Aufruf der Funktion \texttt{lsode} muss man den {\em Namen}
der Funktion, die Anfangsbedingung, sowie einen Vektoren mit $x$-Werten,
f"ur die man die L"osung ausgegeben haben m"ochte, als Argumente
"ubergeben.
Der erste Wert im $x$-Vektor muss der $x$-Wert f"ur die Anfangsbedingung
sein, in unserem Fall also $0$.
Um die Werte von $y(x)$ f"ur ganzzahlige Werte von $x$ zu erhalten,
muss man also die Befehle
\verbatiminput{chapters/examples/octave-dgl-sol.m}
ausf"uhren.
Als R"uckgabewert erh"alt man eine Matrix, die in jeder Zeile die
Werte von $y(x)$ und $y'(x)$ zum entsprechenden Wert von $x$
aus dem \texttt{x}-Argument enth"alt.
Die Resultate sind zusammen mit den Werten der exakten
L"osung~(\ref{grundlagen:numerik-beispiel-loesung}) in der dritten Spalte 
in der Tabelle~\ref{numerik:octave-resultate} zusammengestellt.
Es ist gut erkennbar, wie der Fehler anf"anglich langsam ansteigt,
dann aber unter Kontrolle bleibt.
Die Dokumentation der Funktion \texttt{lsode} beschreibt, wie man mit
Hilfe von Optionen ihr Verhalten und insbesondere die Gr"osse der
Fehler weiter beeinflussen kann.
\begin{table}
\centering
\begin{tabular}{|>{$}r<{$}|>{$}r<{$}|>{$}r<{$}|>{$}r<{$}|}
\hline
    x&  y_{\text{numerisch}}(x)&y_{\text{exakt}}(x) & \text{Fehler}\\
\hline
    0&  0.00000000&  0.00000000&  0.00000000\\
    1&  0.00158525&  0.00158528&  0.00000003\\
    2&  0.01090682&  0.01090678&  0.00000003\\
    3&  0.02858723&  0.02858716&  0.00000007\\
    4&  0.04756207&  0.04756212&  0.00000004\\
    5&  0.05957416&  0.05957437&  0.00000020\\
    6&  0.06276426&  0.06276444&  0.00000018\\
    7&  0.06337942&  0.06337932&  0.00000010\\
    8&  0.07002849&  0.07002811&  0.00000037\\
    9&  0.08576626&  0.08576594&  0.00000032\\
   10&  0.10528405&  0.10528416&  0.00000010\\
  100&  0.84661503&  0.84661930&  0.00000427\\
 1000& -0.55228836& -0.55234514&  0.00005678\\
 2000&  0.90392063&  0.90373523&  0.00018540\\
 3000& -0.99018339& -0.99032256&  0.00013917\\
 4000&  0.75185982&  0.75202340&  0.00016358\\
 5000& -0.25298074& -0.25252044&  0.00046030\\
 6000& -0.30093757& -0.30056348&  0.00037408\\
 7000&  0.76905079&  0.76889872&  0.00015207\\
 8000& -1.00327790& -1.00396748&  0.00068958\\
 9000&  0.88860437&  0.88793099&  0.00067338\\
10000& -0.50337856& -0.50335983&  0.00001873\\
\hline
\end{tabular}
\caption{Exakte und numerische L"osung der Beispieldifferentialgleichung
berechnet mit der Funktion \texttt{lsode} von Octave.
\label{numerik:octave-resultate}}
\end{table}

\subsection{GNU Scientific Library}
W"ahrend Octave dem Benutzer die Wahl eines geeigneten Verfahrens abnimmt
und ihm "uberhaupt wenig Kontrolle "uber den Gang der Rechnung gibt,
kann ein Programmierer durch den Einsatz der GNU Scientific Library (GSL) die
volle Kontrolle "uber alle Aspekte der Iteration erhalten.
Der Preis ist eine wesentlich h"ohere Komplexit"at.
Ziel dieses Abschnitts ist, ein einfaches Beispielprogramm zu
zeigen, welches als Basis eigener Programme dienen kann.
Es verwendet eine Runge-Kutta-Verfahren achter Ordnung.

Die Funktionen zum L"osen von gew"ohnlichen Differentialgleichungen
der GSL haben alle das Pr"afix \texttt{gsl\_odeiv2\_}. 
Zun"achst braucht es nat"urlich wieder eine Implementation der
Funktion $f$. 
Die GSL "ubergibt zwei Arrays, im einen findet die Funktion die aktuellen
$Y$-Werte, im anderen soll sie die Werte der Ableitung zur"uckgeben.
F"ur die Beispiel-Differentialgleichung (\ref{numerik:beispieldgl})
sieht der Code wie folgt aus:
\verbatiminput{chapters/examples/dgl-f.c}
Der Parameter \texttt{params} dient dazu, der Funktion zus"atzliche
Parameter zu "ubergeben.
In unserem Fall ist das nur die Zahl $\omega$.
Da \texttt{params} ein \texttt{void}-Pointer ist, kann eine beliebige
Struktur zur Parameter"ubergabe verwendet werden.

Die Differentialgleichung wird beschrieben durch eine Struktur vom Typ
\texttt{gsl\_odeiv2\_system}, welche ausser Zeigern auf die Funktion
und die Parameter-Struktur auch noch die Dimension der Vektoren enth"alt.
Es kann auch noch ein Funktionszeiger f"ur eine eine Funktion "ubergeben
werden, die die Jacobi-Matrix berechnet, in unserem Beispiel wird dies
jedoch nicht ben"otigt.

Die eigentliche wird von einer ``driver''-Funktion durchgef"uhrt.
Diese sorgt im wesentlichen f"ur die Wahl der Schrittweite, verwaltet
Datenstrukturen, und ruft die Funktionen auf, die die einzelnen Schritte
durchf"uhren.
Die Treiber-Funktion f"uhrt die einzelnen Schritte (im Sinne der
in Abschnitt~\ref{section:numerik:einschritt} besprochenen
Einschritt-Verfahren) mit
Hilfe der Schritt-Funktionen durch, von denen die Bibliothek eine
ganze Reihe bereitstellt.
Die Funktion \texttt{gsl\_odeiv2\_step\_rk4} ist das klassische
Runge-Kutta-Verfahren vierter Ordnung, welches in
Abschnitt~\ref{subsection:numerik:runge-kutta}
beschrieben wurde.
Im Beispielverfahren verwenden wir \texttt{gsl\_odeiv2\_step\_rk8pd},
das Runge-Kutta Prince-Dormand Verfahren achter Ordnung.
F"ur Aufgaben allgemeiner Art ebenfalls sehr gut geeignet ist das
Runge-Kutta-Fehlberg-Verfahren f"unfter Ordnung mit dem Namen
\texttt{gsl\_odeiv2\_step\_rkf45}.
Diese Datenstrukturen werden mit dem Code
\verbatiminput{chapters/examples/dgl-init.c}
initialisiert.
Durch Austausch des zweiten Arguments der Driver-Allozierungs-Funktion
kann man leicht das Verfahren wechseln und so Zeitaufwand und Genauigkeit
f"ur verschiedene L"osungsverfahren vergleichen.

Um die Rechnung durchzuf"uhren, muss jetzt die Driver-Funktion so oft
angewendet werden, wie man Punkt der L"osungskurve ausgeben will.
Dazu dient die Funktion \texttt{gsl\_odeiv2\_driver\_apply}. 
An Argument braucht sie den eben initialisierten Driver, den aktuellen
$x$-Wert, den $x_{\text{next}}$-Wert, f"ur den der n"achste Punkt
ausgegeben werden soll, sowie einen Vektor, in dem der aktuelle Anfangswert
f"ur $Y(x)$ "ubergeben und $Y(x_{\text{next}})$ zur"uckgegeben wird.
$x$ wird als Referenz "ubergeben, wenn die Funktion zur"uckkehrt,
findet man dort den neuen aktuellen Wert von $x$, also im Erfolgsfall
$x_{\text{next}}$.
In unserem Fall brauchen wir $X(x)$ f"ur ganzzahlige $x$, die folgende
Schleife bewerkstelligt dies:
\verbatiminput{chapters/examples/dgl-loop.c}

\begin{table}
\centering
\begin{tabular}{|>{$}r<{$}|>{$}r<{$}|>{$}r<{$}|>{$}r<{$}|}
\hline
    x&  y_{\text{numerisch}}(x)&y_{\text{exakt}}(x) & \text{Fehler}\\
\hline
    1&   0.01584477&   0.01584477&  -0.00000000\\
    2&   0.10882786&   0.10882787&  -0.00000000\\
    3&   0.28425071&   0.28425071&  -0.00000001\\
    4&   0.46979656&   0.46979656&  -0.00000000\\
    5&   0.58112927&   0.58112926&   0.00000001\\
    6&   0.59856974&   0.59856972&   0.00000002\\
    7&   0.58436266&   0.58436265&   0.00000001\\
    8&   0.62466692&   0.62466694&  -0.00000001\\
    9&   0.74961115&   0.74961117&  -0.00000003\\
   10&   0.90492231&   0.90492232&  -0.00000001\\
   20&   0.82626555&   0.82626556&  -0.00000000\\
   30&   0.24234669&   0.24234664&   0.00000005\\
   40&  -0.83971103&  -0.83971092&  -0.00000011\\
   50&  -0.94210772&  -0.94210787&   0.00000015\\
   60&  -0.25144906&  -0.25144893&  -0.00000013\\
   70&   0.58545210&   0.58545205&   0.00000005\\
   80&   1.09974465&   1.09974456&   0.00000009\\
   90&   0.32597838&   0.32597860&  -0.00000023\\
  100&  -0.49836792&  -0.49836823&   0.00000031\\
  200&   1.01037929&   1.01037877&   0.00000052\\
  300&  -0.89702589&  -0.89702630&   0.00000041\\
  400&   0.83859090&   0.83859101&  -0.00000010\\
  500&  -0.21777633&  -0.21777543&  -0.00000090\\
  600&  -0.31235410&  -0.31235239&  -0.00000171\\
  700&   0.72675906&   0.72676124&  -0.00000219\\
  800&  -1.09422992&  -1.09422790&  -0.00000202\\
  900&   0.80223761&   0.80223872&  -0.00000111\\
 1000&  -0.59500324&  -0.59500363&   0.00000040\\
 2000&  -0.97606658&  -0.97606187&  -0.00000471\\
 3000&  -1.03200392&  -1.03199479&  -0.00000914\\
 4000&  -0.79047800&  -0.79047372&  -0.00000428\\
 5000&  -0.37269297&  -0.37270218&   0.00000921\\
 6000&   0.08785158&   0.08783273&   0.00001886\\
 7000&   0.49827647&   0.49826463&   0.00001184\\
 8000&   0.80219750&   0.80220742&  -0.00000992\\
 9000&   0.94568799&   0.94571577&  -0.00002778\\
10000&   0.86607968&   0.86610200&  -0.00002232\\
\hline
\end{tabular}
\caption{L"osungen der Beispieldifferentialgleichung (\ref{numerik:beispieldgl})
mit Hilfe der GNU Scientific Library (GSL).
\label{numerik:gsl-resultate}}
\end{table}

Man kann die Funktion $f$ im Programm nat"urlich auch mit einem Z"ahler
ausstatten und damit herausfinden, wie viele Aufrufe der Funktion
f"ur die numerische L"osung ben"otigt werden.
Es stellt sich heraus, dass f"ur das erste Intervall von $0$ bis $1$
die Funktion $f$ 131 mal aufgerufen wird, hier versucht die Bibliothek
die optimale Schrittweite $h$ zu bestimmen.
In allen folgenden Intervallen der L"ange $1$ von $n$ bis $n+1$ werden nur
noch jeweils 13 Aufrufe der Funktion ben"otigt.
Verwendet man stattdessen das Runge-Kutta-Fehlberg-Verfahren,
werden pro Intervall 18 Auswertungen der Funktion $f$ ben"otigt,
und die Genauigkeit sinkt auf zwei Stellen nach dem Komma.

\section{Randwertprobleme\label{section:numerik:randwertprobleme}}
Die bisher beschriebenen Verfahren gehen von einer Anfangsbedingung
aus, und berechnen die dadurch eindeutig festgelegte L"osungskurve.
Randwertproblem, beschrieben in Abschnitt~\ref{section:randwertprobleme},
verkn"upfen dagegen Werte von einzelnen Komponenten von $Y$ an den
R"andern eines Intervalls.

Wir betrachten zwei prototypische Randwertprobleme, die auch gleich
zwei v"ollig verschiedene L"osungsverfahren motivieren.

\newtheorem{aufgabe}{Aufgabe}[chapter]
\begin{aufgabe}
\label{numerik:aufgabe-ball}
Mit einem nur der Schwerkraft unterworfenen Ball, der im Ursprung des
Koordinatensystems geworfen wird, soll ein Ziel im Punkt $P$ getroffen
werden.
In welcher Richtung und mit welcher Anfangsgeschwindigkeit muss er geworfen
werden?
\end{aufgabe}

Um das Problem einfach zu halten, modellieren wir diese Aufgabe wie
folgt.
Der Ball der Masse $m$ bewegt sich in der $x$-$y$-Ebene, wobei die
Schwerkraft in negativer $y$-Richtung zeigt.
Das Newtonsche Gesetz liefert die Differentialgleichung zweiter Ordnung
\begin{equation}
m\frac{d^2}{dt^2}\begin{pmatrix}x\\y\end{pmatrix}
=
\begin{pmatrix}
0\\-mg
\end{pmatrix}
\label{numerik:ball-dgl}
\end{equation}
Die Masse $m$ kann herausgek"urzt werden.
Gesucht ist eine L"osung so, dass die Bahn durch die Punkte $(0,0)$
und $P=(p,0)$ geht.

Genau genommen ist dies nicht ein Randwertproblem wie in 
Abschnitt~\ref{section:randwertprobleme}, denn es wird nicht verlangt,
dass der Ball zu einer bestimmten Zeit $t$ beim Punkt $P$ eintrifft.
Die Differentialgleichung bedeutet aber, dass die Horizontalgeschwindigkeit
des Balls konstant ist (die horizontale Beschleunigung ist immer $0$).
Ist $v_x$ die Horizontalgeschwindigkeit, dann erreicht der Ball zur
Zeit $t_1=p/v_x$ die $x$-Koordinate des Ziels.
Gesucht ist also die anf"angliche Vertikalgeschwindigkeit, die man
dem Ball geben muss, dass zur Zeit $p/v_x$ die $y$-Komponente
der L"osung den Wert $0$ hat.
In dieser Form liegt ein Randwertproblem wie in
Abschnitt~\ref{section:randwertprobleme} vor.

Die L"osungen der Differentialgleichung~\ref{numerik:ball-dgl} sind aus
dem Physik-Unterricht bekannt:
es gilt
\begin{equation}
\begin{pmatrix}x(t)\\y(t)\end{pmatrix}
=
\begin{pmatrix}v_xt\\ v_yt-\frac12gt^2\end{pmatrix}
\end{equation}
Damit l"asst sich auch das Randwertproblem l"osen.
F"ur $t=v_x/p$ muss $y(t)=0$ sein, also
\begin{align}
y(t)=y\biggl(\frac{p}{v_x}\biggr)
=v_y\frac{p}{v_x}-\frac12g\biggl(\frac{p}{v_x}\biggr)^2&=0
\notag
\\
\Rightarrow\qquad
v_y
&=
\frac{v_x}p\frac12g\frac{p^2}{v_x^2}=\frac{gp}{2v_x}.
\label{numerik:ball-bedingung}
\end{align}
Offenbar gibt es zu jedem $v_x$ einen passenden Wert von $v_y$,
mit dem das Ziel getroffen wird.

Die Differentialgleichung (\ref{numerik:ball-dgl}) ist nicht in einer
Form, die der numerischen L"osung zug"anglich ist.
Wir schreiben Sie daher als Differentialgleichung erster Ordnung 
f"ur vierdimensionale Vektoren:
\begin{align}
\frac{d}{dt}Y
=
\frac{d}{dt}\begin{pmatrix}x\\y\\\dot x\\\dot y\end{pmatrix}
&=
\begin{pmatrix}\dot x\\\dot y\\ 0\\ -g\end{pmatrix}.
\label{numerik:ball-dgl-1}
\end{align}
Gesucht ist eine L"osung, die die Randbedingungen
\begin{equation}
Y(0)
=
\begin{pmatrix}0\\0\\v_x\\\color{red}v_y\end{pmatrix},
\qquad
Y\biggl(\frac{p}{v_x}\biggr)
=
\begin{pmatrix}p\\0\\\color{red}?\\\color{red}?\end{pmatrix}
\label{numerik:ball-dgl-2}
\end{equation}
erf"ullt.
Darin stehen die roten Eintr"age f"ur Werte, die nicht vorgegeben sind.
Aus der Symmetrie des Problems kann man nat"urlich auch die Endgeschwindigkeit
ablesen.
Zu bestimmen ist also $v_y$ so, dass die L"osungskurve durch den Punkt
$(p,0)$ geht.

Wird statt der Horizontalkomponenten der Anfangsgeschwindigkeit die
gesamte Anfangsgeschwindigkeit $v_0$ vorgegeben, dann muss der
Winkel gefunden werden, unter dem der Ball geworfen werden muss,
um das Ziel zu trefen.
Bei der Elevation $\alpha$ sind die Komponenten der Anfangsgeschwindigkeit
$v_x=v_0\cos\alpha$ und $v_y=v_0\sin\alpha$. 
Setzt man dies in die Bedingung~(\ref{numerik:ball-bedingung}) ein,
findet man
\begin{align*}
v_0 \sin \alpha &=\frac{gp}{2v_0\cos\alpha}
\\
2\sin\alpha\cos\alpha&=\frac{gp}{v_0^2}
\\
\sin2\alpha&=\frac{gp}{v_0^2}
\\
\alpha&= \frac12 \arcsin\frac{gp}{v_0^2}
\end{align*}
Im Nenner rechts steht im wesentlichen die kinetische Energie,
je mehr kinetische Energie der Ball zu Beginn hat, desto kleiner
ist der Winkel, man trifft das Ziel mit einer sehr flachen Bahn.
Kleine Winkel reichen auch f"ur geringe Schwerkraft ($g$ klein)
und kurze Distanzen ($p$ klein).
Die maximale Distanz wird erreicht, wenn das Argument des Arcussinus
den Wert $1$ erreicht, gr"osser darf $p$ nicht werden, weil es sonst
keine L"osung mehr f"ur $\alpha$ gibt.
Die Maximaldistanz ist daher
\[
p_{\text{max}} = \frac{v_0^2}{g}.
\]

\begin{aufgabe}
\label{numerik:aufgabe-seil}
Ein Seil ist zwischen zwei Punkten aufgeh"angt, welche Form nimmt es
allein unter der Wirkung seines Eigengewichtes an?
\end{aufgabe}

\subsection{Schiess-Verfahren\label{numerik:schiess-verfahren}}
Wenn man experimentell versucht, ein Ziel zu treffen, dann wird man
in wiederholten Versuchen die Richtung anpassen, so dass man dem Ziel
immer n"aher kommt.
Der $y$-Wert zur Zeit $p/v_x$ h"angt von der Vertikalgeschwindigkeit ab,
wir bezeichnen ihn mit $h(v_y)$.
Man ver"andert also $v_y$, bis die Gleichung $h(v_y)=0$ erf"ullt ist.
Um das Randwertproblem zu l"osen, muss man also die Gleichung $h(v_y)=0$
numerisch l"osen.

Man kann dies zum Beispiel dadurch machen, dass man nach zwei Werten von $v_y$
sucht, so dass die zum einen geh"orige Bahn unter dem Punkt $P$ durchgeht,
w"ahrend der Ball im anderen Fall dar"uber hinwegfliegt.
Durch wiederholte Halbierung des Intervalls kann man dann den korrekten
Wert f"ur $v_y$ immer genauer eingrenzen%
\footnote{%
Tats"achlich wird dieses Verfahren in der Artillerie verwendet.
Der Schiesskommandant beobachtet die einschlagenden Granaten und kommandiert
"Anderungen der Anfangs-Elevation an die Gesch"utzbatterien.
Dabei sucht er Einschl"age, die aus seiner Perspektive vor bzw.~hinter
dem Ziel liegen, und halbiert dann das Intervall, bis die Einschl"age dem
Ziel gen"ugend nahe kommen.}.
Der Nachteil dieses Verfahrens ist, dass mit jedem Schritt die Genauigkeit
nur um in Bit ansteigt, es sind also sehr viele Iterationen notwendig.

Schnellere Konvergenz kann mit dem Newton-Verfahren erreicht werden,
welches in Anhang~\ref{chapter:newton} beschrieben wird.
F"ur die Anwendung des Newton-Verfahrens auf das Randwert-Problem
ist die Bestimmung der Steigung der Funktion n"otig, die die Abweichung
der Kurve von der Randbedingung am rechten Rand angibt.
Wir m"ussen also berechnen, wie schnell sich $y(p/v_x)$ "andert,
wenn $v_y$ ver"andert wird.
Dies ist die Ableitung
\[
h'(v_y)= \frac{\partial y}{\partial v_y},
\]
ein Eintrag der Jacobi-Matrix.
In Abschnitt~\ref{grundlagen:XXX} wurde gezeigt, wie man auch f"ur
die Jacobi-Matrix eine Differentialgleichung aufstellen kann, die
man nat"urlich ebenfalls mit den fr"uher beschriebenen numerischen
Bibliotheken l"osen kann.

Das Randwertproblem kann daher mit folgendem Algorithmus numerisch gel"ost
werden.
\begin{enumerate}
\item Beginne mit einer Sch"atzung f"ur $v_y$
\item Finde numerisch die L"osung des Anfangswertproblems mit $v_y$
als anf"angliche Vertikalgeschwindigkeit.
Berechnet dabei auch die Jacobi-Matrix
\item Lese die $h(v_y)$ aus der L"osung zur Zeit $p/v_x$ ab, und $h'(v_y)$
aus der Jacobi-Matrix und verwendet den Newton-Algorithmus
(\ref{numerik:ball-newton}), um eine verbesserte Sch"atzung von $v_y$ 
zu bekommen.
\item Wiederhole Schritte 2 und 3 bis die Randbedingung f"ur $t=p/v_x$
gen"ugend genau erf"ullt ist.
\item Die L"osung des Anfangswertproblems mit diesem $v_y$ ist die
L"osung des gestellten Randwertproblems.
\end{enumerate}

\begin{beispiel}
\begin{figure}
\centering
\includegraphics{chapters/images/randwert-1.pdf}
\caption{L"osungen des Anfangswertproblems~(\ref{numerik:ball-dgl-1}) und
(\ref{numerik:ball-dgl-2}).
Das Newton-Verfahren korrigiert $v_y$ derart, dass $h(v_y)=0$ wird.
So wird die L"osung des Randwertproblems (rot) gefunden.
\label{numerik:randwert-bild}}
\end{figure}
Wir f"uhren den eben skizzierten Algorithmus f"ur das Ball-Problem durch.
Um die Jacobi-Matrix zu berechnen, m"ussen wir die Ableitung von $f$ berechnen:
\begin{equation}
\frac{\partial f(x,y)}{\partial y}
=
\begin{pmatrix}
0& 0& 1& 0\\
0& 0& 0& 1\\
0& 0& 0& 0\\
0& 0& 0& 0
\end{pmatrix}.
\end{equation}
Da die rechte Seite nicht von $y$ abh"angt, k"onnen wir die Gleichung f"ur
die Jacobi-Matrix ganz unabh"angig von $y$ l"osen.
Da $F$ so einfach ist, kann man das Matrizenprodukt direkt ausrechnen, 
so wird die Differentialgleichung f"ur $J$
\begin{equation}
\begin{pmatrix}
J'_{11}&J'_{12}&J'_{13}&J'_{14}\\
J'_{21}&J'_{22}&J'_{23}&J'_{24}\\
J'_{31}&J'_{32}&J'_{33}&J'_{34}\\
J'_{41}&J'_{42}&J'_{43}&J'_{44}
\end{pmatrix}
=
\begin{pmatrix}
0& 0& 1& 0\\
0& 0& 0& 1\\
0& 0& 0& 0\\
0& 0& 0& 0
\end{pmatrix}
\begin{pmatrix}
J_{11}&J_{12}&J_{13}&J_{14}\\
J_{21}&J_{22}&J_{23}&J_{24}\\
J_{31}&J_{32}&J_{33}&J_{34}\\
J_{41}&J_{42}&J_{43}&J_{44}
\end{pmatrix}
=
\begin{pmatrix}
J_{31}&J_{32}&J_{33}&J_{34}\\
J_{41}&J_{42}&J_{43}&J_{44}\\
     0&     0&     0&     0\\
     0&     0&     0&     0
\end{pmatrix}
\end{equation}
\begin{table}
\centering
\begin{tabular}{|>{$}r<{$}|>{$}r<{$}|>{$}r<{$}|>{$}r<{$}|>{$}r<{$}|>{$}r<{$}|>{$}r<{$}|>{$}r<{$}|}
\hline
n&    v_y&    t& x(t)&      y(x)&\displaystyle\frac{\partial^{\mathstrut}y}{\partial v_y}&v_{y,\text{new}}&\Delta\\
\hline
0& 7.0000&  2.5& 20.0&-13.156250&  2.5& 12.26250000& -5.2625000000\\
1&12.2625&  2.5& 20.0& -0.000004&  2.5& 12.26250145& -0.0000014458\\
2&12.2625&  2.5& 20.0&  0.000000&  2.5& 12.26250143&  0.0000000204\\
\hline
\end{tabular}
\caption{Newton-Algorithmus f"ur das Ball-Problem, Resultate der numerischen
Rechnung.
$v_y$ wird in drei Schritten mit einer Genauigkeit von mehr als 10 Stellen
gefunden.
\label{numerik:newton-resultate}}
\end{table}%
Daraus kann man ablesen, dass die Elemente $J_{3j}$ und $J_{4j}$ sich
nicht "andern, sie bleiben also konstant.
Aber auch in den ersten zwei Zeilen k"onnen sich nur die Elemente $J_{13}$
und $J_{24}$ "andern, die Differentialgleichungen f"ur diese Elemente
sind
\begin{align*}
J'_{13}&=1\\
J'_{24}&=1
\end{align*}
oder in Matrixform:
\begin{equation}
J(x) = \begin{pmatrix}
1&0&x&0\\
0&1&0&x\\
0&0&1&0\\
0&0&0&1
\end{pmatrix}
\end{equation}
Die L"osung $J_{13}(x)=x$ und $J_{24}=x$.
Damit haben wir die n"otige Information, um den Newton-Algorithmus
durchzuf"uhren.
In Tabelle~\ref{numerik:newton-resultate}
sind die Resultate der numerischen Rechnung zusammengestellt.
Es zeigt sich, dass der korrekte Wert f"ur $v_y$ in drei Iterationen
mit 10 Stellen Genauigkeit gefunden werden kann.
Damit ist das Randwertproblem numerisch gel"ost.
\end{beispiel}

%
% potenzreihen.tex -- Lösung von Differentialgleichungen mit Potenzreihen
%
% (c) 2015 Prof Dr Andreas Mueller, Hochschule Rapperswil
%
\chapter{Potenzreihen-Methode\label{chapter:potenzreihen}}
\lhead{}
\rhead{Potenzreihen-Methode}
\section{Analytische L"osungen}
\section{Trigonometrische Funktionen}

%
% linear.tex -- L"osung linearer Differentialgleichungen
%
% (c) 2015 Prof Dr Andreas Mueller, Hochschule Rapperswil
%
\chapter{Lineare Differentialgleichungen\label{chapter:linear}}
\lhead{}
\rhead{Lineare Differentialgleichungen}
Es lohnt sich, lineare Differentialgleichungen unter Zuhilfenahme
der linearen Algebra etwas genauer zu untersuchen.
\section{Definition}
\section{L"osungsmenge}
\section{Normalformen}
\subsection{Diagonalisierung}
\subsection{Jordan-Normalform}


%
% geometry.tex -- L"osung linearer Differentialgleichungen
%
% (c) 2016 Prof Dr Andreas Mueller, Hochschule Rapperswil
%
\chapter{Geometrische Eigenschaften\label{chapter:geometrie}}
\rhead{}
\lhead{Geometrische Eigenschaften}
Die Geometrie schr"ankt die m"oglichen Bahnen eines
Differentialgleichungssystems bereits wesentlich ein.
In einem eindimensionalen autonomen System sind keine
Schwingungen m"oglich, in einem zweidimensionalen
System gibt es keine chaotischen Bewegungen.
Ziel dieses Kapitels ist, in diese geometrische
Denkweise einzuf"uhren.
Das Buch \cite{skript:hirsch} f"uhrt diesen Ansatz weiter bis zu
einer Einf"uhrung in chaotische Bewegung.

\section{Autonome Systeme}
\rhead{Autonome Systeme}
\begin{figure}
\centering
\includegraphics{chapters/images/geometrie-13.pdf}
\caption{Entwicklung des Systems~(\ref{geometrie:harvest-equation})
mit $a=5$ und $h=0.8$
\label{geometrie:harvest-graph}}
\end{figure}%
Ein eindimensionales System k"onnen wir schreiben als
\[
y'=f(x,y).
\]
Die L"osungskurven dieses Systems k"onnen wir in einem $x$-$y$-Diagramm
als Graphen darstellen.
Zum Beispiel k"onnen wir die Entwicklung des Systems
\begin{equation}
y' = ay(1-y)-h(1+\sin 2\pi x)
\label{geometrie:harvest-equation}
\end{equation}
wie in Abbildung~\ref{geometrie:harvest-graph} darstellen.
Dieses nicht-autonome System hat zwei Grenzzyklen: 
Anfangswerte oberhalb der roten Kurve f"uhren zu L"osungskurven, die
sich f"ur zunehmendes $x$ der blauen Kurve ann"ahern.
Anfangswerte unterhalb der blauen Kurve f"uhren zu L"osungskurven, die
sich f"ur abnehmendes $x$ der roten Kurve n"ahern.

Startwerte in der N"ahe der roten Kurve sind nicht stabil,
die Entwicklung f"ur zunehmende $x$ f"uhrt den Wert immer weiter von
der roten Kurve weg.
Lag der Startwert unterhalb der roten Kurve, wird die L"osung gegen
$-\infty$ divergieren.
Alle anderen L"osungen konvergieren gegen die blaue Kurve.

Offenbar ist es ziemlich schwierig, das
System~(\ref{geometrie:harvest-equation}) "uber gr"ossere $x$-Intervalle
zu verstehen. 
Zum Beispiel h"angt das Verhalten davon ab, ob der Startwert zwischen den
farbigen Kurven liegt oder ausserhalb, und diese h"angt vom Start-$x$ ab.

In Kapitel~\ref{chapter:grundlagen} wurde beschrieben, wie jedes
Differentialgleichungssystem, m"oglicherweise nach Erweiterung um eine
explizite Zeitkoordinate, zu einem autonomen System gemacht und damit
durch ein Vektorfeld ersetzt werden kann,
und wie die L"osungen als Bahnen eines Teilchens in einem zeitlich
unver"anderlichen Vektorfelde verstanden werden k"onnen.
Daraus ergibt sich auch, dass das Verhalten der L"osungen einer
Differentialgleichungen studiert werden kann, ohne dass man
die Abh"angigkeit von $x$ im Detail kennt.
Die Bahnen zerlegen den Raum und k"onnen sich nicht schneiden,
so schr"ankt die Geometrie des Raumes die M"oglichkeiten f"ur das
Verhalten der L"osung "uber lange Zeiten ein.
Besonders ausgepr"agt sind diese Einschr"ankungen im zweidimensionalen
Raum.
Eine geschlossene Kurve in der Ebene unterteilt diese in zwei Bereiche,
keine L"osung kann vom einen Bereich in den anderen f"uhren.

Ein Parameter in der Differentialgleichung modifiziert das Vektorfeld,
und kann damit Eigenschaften von L"osungen "uber lange Zeiten ver"andern.
Zum Beispiel k"onnen periodische Bahnen sich aufl"osen und zu Spiralbahnen
werden.

Wir gehen in diesem Kapitel immer von einem autonomen System
\[
\frac{d}{dx}y(x)=f(y),
\]
wobei das Vektorfeld $f(y)$ nicht von $x$ abh"angt.
Uns interessieren die geometrische Eigenschaften der L"osungskurven, 
soweit sie sich direkt aus den Eigenschaften des Vektorfeldes ableiten
lassen.
Insbesondere interessieren uns also spezielle Punkte des Vektorfeldes,
zum Beispiel Nullstellen.


%
% Spezielle Punkte und Bahnen
%
\section{Spezielle Punkte und Bahnen}
\rhead{Spezielle Punkte und Bahnen}
Das Verhalten der L"osungskurven "uber lange Zeiten wird wesentlich
beeinflusst von Punkten, in denen die Bewegung auf der Bahn
zum Stillstand kommt oder sogar die Richtung umkehrt, und von Punkten,
in die ein Punkt periodisch zur"uckkehrt.

%
% Kritische Punkte
%
\subsection{Kritische Punkte}
Wir betrachten zun"achst einen Punkt $y_0$ in dem $f(y_0)\ne 0$.
Dann gilt $f(y)\ne 0$ auch f"ur Punkte $y$, die gen"ugend nahe an 
$y_0$ sind.
Eine L"osungskurve durch $y_0$ hat im Punkt $y_0$ die Tangentenrichtung
$f(y_0)$.
In einer gen"ugend kleinen Umgebung von $y_0$ werden sich verschiedene
L"osungskurven nicht schneiden.
Ein solcher Schnittpunkt w"are ein Anfangsbedingung f"ur zwei
verschiedene L"osungen, aber der Eindeutigkeitssatz f"ur die
L"osung einer Differentialgleichung sagt, dass es zu jeder 
Anfangsbedingung nur eine L"osung geben kann.
Dies bedeutet, dass in einer Umgebung des Punktes $y_0$ nichts
Spannendes passiert, die Bahnen sehen ungef"ahr aus wie in
Abbildung~\ref{geometrie:parallelebahnen}.
\begin{figure}
\centering
\includegraphics{chapters/images/geometrie-12.pdf}
\caption{Verlauf der Bahnen eines autonomen Differentialgleichungssystems
in der N"ahe eines Punktes $y_0$ mit $f(y_0)\ne 0$.
\label{geometrie:parallelebahnen}}
\end{figure}

Von besonderem Interesse sind daher Punkte, in denen das Vektorfeld
verschwindet:

\begin{definition}
$y$ heisst {\em kritischer Punkt} des Vektorfeldes $f$, wenn $f(y)=0$.
\end{definition}

Dass in einem kritischen Punkt das Vektorfeld verschwindet bedeutet nicht,
dass die Bahn nicht dar"uber hinweg fortgesetzt werden kann,
wie das folgende Beispiel zeigt.

\begin{beispiel}
Das eindimensionale Gleichungssystem
\[
y'=\sqrt{|y|}
\]
hat einen kritischen Punkt bei $y=0$.
Die Funktion
\begin{equation}
y(x)=\frac14x^2\operatorname{sign}(x)
\label{geometrie:1dimkritloesung}
\end{equation}
ist eine L"osungsfunktion, denn
\begin{align*}
y'(x)&=\frac12|x|\\
\sqrt{\left|\frac14x^2\operatorname{sign}(x)\right|}
&=
\sqrt{\frac14x^2}
=
\frac12|x|=y'(x).
\end{align*}
Trotzdem erreicht $y(x)$ jeden beliebigen Punkt der $y$-Achse, denn 
\[
x=2\sqrt{|y|}\operatorname{sign}(y)
\]
ist die Umkehrfunktion von $y(x)$. 
Die L"osungskurve bewegt sich also "uber den kritischen Punkt hinweg.
\begin{figure}
\centering
\includegraphics{chapters/images/geometrie-1.pdf}
\caption{Bahnen das Differentialgleichungssystems $y'=\sqrt{|y|}$
k"onnen den kritischen Punkt bei $y=0$ traversieren.
Die L"osung~(\ref{geometrie:1dimkritloesung}) ist etwas fetter rot
eingezeichnet.
L"osungen k"onnen aber auch beliebig lange im Punkt $y$ verweilen,
wie die blaue Kurve illustriert.
\label{geometrie:1dimkrit}}
\end{figure}
Abbildung~\ref{geometrie:1dimkrit} zeigt die Bahnen.
Die L"osung~(\ref{geometrie:1dimkritloesung}) ist nicht die einzige.
Andere L"osungen bestehen jeweils aus einem Ast der Kurve mit positiven
$y$ und einem mit negativen $y$, dazwischen kann die L"osung beliebig lange
im kritischen Punkt verweilen.
F"ur den Punkt $y=0$ ist also der Eindeutigkeitssatz verletzt, zur
Anfangsbedingung $y=0$ gibt es beliebig viele L"osungen.
\end{beispiel}

%
% Linearisierung
%
\subsection{Linearisierung}
In einem kritischen Punkt $y_0$ verschwinden alle Komponenten von $f(y_0)$.
Falls $f$ stetig differenzierbar ist, k"onnen wird $f$ in einer Umgebung
des kritischen Punktes in erster Ordnung mit Hilfe der Ableitung
approximieren:
\[
f(y)=\frac{\partial f}{\partial y}(y-y_0) + o(|y-y_0|)
\]
Die partielle Ableitung bezeichnet im mehrdimensionalen Fall
die Jacobi-Matrix, sie ist die $n\times n$-Matrix bestehend aus allen
partiellen Ableitungen der Komponenten von $f$ nach den Variablen $y_i$.
In einer Umgebung eines kritischen Punktes ist die Gestalt der Bahnen also
im wesentlichen durch die Jacobi-Matrix bestimmt.

\begin{beispiel}
Wir betrachten als Beispiel die eindimensionale ($n=1$) Differentialgleichung
\[
y'=f(y)=y,
\]
die in $y_0=0$ einen kritischen Punkt hat.
Die Jacobi-Matrix ist konstant: $f'(y)=1$.
L"osungskurven mit Anfangsbedingungen $>0$ werden daher anwachsen,
w"ahrend L"osungskurven mit Anfangsbedingungen $<0$ abnehmen werden.
Die L"osungskurven werden daher vom kritischen Punkt ``abgestossen''.

W"ahlen wir stattdessen das System
\[
y'=f(y)=-y,
\]
dann ist die Jacobi-Matrix $f'(y)=-1$, und wir k"onnen analog schliessen,
dass L"osungskurven immer zum kritischen Punkt hinstreben.
\end{beispiel}

\begin{beispiel}
Das Beispiel im letzten Abschnitt verwendet
\[
y'=f(y)=\sqrt{|y|}
\]
ist im kritischen Punkt $y=0$ nicht stetig differenzierbar, daher ist
eine lineare Approximation nicht m"oglich.
Das fr"uher beobachtete Verhalten, dass sich L"osungskurven auf der
einen Seite vom kritischen Punkt entfernen, auf der anderen aber ann"ahern,
kann nur auftreten, wenn auch $f'(0)=0$ gilt, so dass wir $f$ in zweiter
Ordnung als
\[
y'=f(y)=\frac12 f''(0)y^2.
\]
Diese Differentialgleichung hat die Funktion
\[
y(x)=-\frac{2}{f''(0)x+C}
\]
als L"osung.
Je nach Wert von $C$ bekommen wir eine L"osung, f"ur $x>0$ anwachsen
oder abfallen, wenigstens f"ur ein kleines Intervall.
Die L"osung n"ahert sich dem kritischen Punkt, oder entfernt sich, je
nachdem auf welcher Seite die L"osungskurve beginnt.
\end{beispiel}

%
% Eindimensionale Systeme
%
\section{Eindimensionale Systeme}
\rhead{Eindimensionale Systeme}
Wir untersuchen in diesem Abschnitt das Verhalten eindimensionaler autonomer
Systeme, und konzentrieren uns dabei auf Eigenschaften, die allein
auf Grund der geringen Dimension erschliessen lassen.
Ein solches System hat die Form
\[
y'=f(y),\quad y\in\mathbb R
\]
und kann sofort mit Hilfe von Separation der Variablen gel"ost werden:
\[
\int\frac{dy}{f(y)} = t+C,
\]
was nat"urlich nur funktioniert, wenn $f$ konstantes Vorzeichen hat.
Daraus folgt, dass ein Startwert $y_0$, f"ur den $f(y_0)>0$ ist,
zu einer monoton wachsenden L"osung f"uhrt, w"ahrend $f(y_0)<0$
bedeutet, dass die L"osung monoton f"allt.
Erf"ullt ein Startwert $f(y_0)=0$, dann ist $y(x)=y_0$ eine
L"osung der Differentialgleichung.
Die kritischen Punkte von $f$ strukturieren also wie erwartet die
L"osungsmenge.
Wir suchen eine "ubersichtliche Visualisierung dieser Beobachtung
und wollen verstehen, wie sich die Struktur in Abh"angigkeit
von einem Parameter in der Differentialgleichung ver"andern kann.

%
% Mögliche Phasendiagramme in einer Dimension
%
\subsection{Phasenportraits}
Seien jetzt $y_1,y_2,\dots$ Nullstellen der Funktion $f(y)$.
Dann sind die Funktion $y(x)=y_i$ L"osungen der Differentialgleichung
$y'=f(y)$.
Die Bewegung zwischen den Nullstellen wird vollst"andig dominiert durch
das Vorzeichen von $f(y)$ zwischen den Nullstellen.
Wir k"onnen dies in einem sogenannten Phasen-Portrait visualisieren.
Abbildung~\ref{geometrie:phasenportrait} zeigt die L"osungen der
Differentialgleichung
\begin{equation}
y'=f(y)=y^3-y
\label{geometrie:cube}
\end{equation}
Die Nullstellen dieser Funktion sind $-1$, $0$ und $1$.
\begin{figure}
\centering
\includegraphics{chapters/images/geometrie-14.pdf}
\caption{Fluss der Differentialgleichung~(\ref{geometrie:cube}) und
Phasenportrait
\label{geometrie:phasenportrait}}
\end{figure}
Die exakte Abh"angigkeit von $x$ ist oft nicht wichtig, entscheidend
ist nur die Tatsache, dass Punkte im blauen Gebiet sich mit zunehmendem
$x$ nach unten bewegen, w"ahrend Punkte im roten Gebiet sich nach
oben bewegen.
Diese Information wird auch durch das Phasenportrait am linken
Rand der Abbildung~\ref{geometrie:phasenportrait} dargestellt.
Der Vortail eines Phasenportraits gegen"uber der Darstellung
der L"osung ist, dass sie mit einer Dimension auskommt.

%
% Mögliche Änderungen (Bifurkationen) der Phasendiagramme
%
\subsection{Bifurkationen\label{geometrie:subsection:bifurkationen}}
Betrachten wir jetzt eine Differentialgleichung, die ausserdem von einem
Parameter $b$ abh"angt:
\[
y'=f(y,b).
\]
Das Phasenportrait wird sich ver"andern, wenn $b$ variert, wir
m"ochten die verschiedenen dabei m"oglichen "Uberg"ange verstehen
k"onnen.
Es interessiert vor allem kleine "Anderungen in der Umgebung eines
Wertes $b_0$.
Dabei kommt es vor allem darauf an, dass mit Nullstellen passiert, und
welches Vorzeichen $f$ links und rechts von der Nullstelle hat.
Wir gehen daher davon aus, dass $f$ f"ur $b_0$ an der Stelle $y=0$
hat, dass also $a_0(b_0)=0$ ist.

Wenn $a_1(b_0)\ne0$ ist, also wenn $f'(0, b_0)\ne 0$ ist, dann 
gilt dies auch in einer Umgebung von $b_0$, und daher wird die
Nullstelle in erster N"aherung nur verschoben:
\[
y_0(b) \simeq -\frac{f(0,b)}{f'(0,b)},
\]
Das Phasenportrait erlebt also keine grundds"atzlichen "Anderung.

\subsubsection{Sattel-Knoten-Bifurkation}
Wir betrachten jetzt den Fall $f(0,b_0)=0$ und $f'(0,b_0)=0$, die Funktion
$f$ ist f"ur $b=b_0$ quadratisch, mit einer Nullstelle im Scheitel.
Variert $b$, wird zwar der Graph von $y\mapsto f(y,b)$ seine
Gestalt "andern, aber er wird weiterhin "ahnlich wie eine Parabel
aussehen.
Etwas interessantes passiert nur, wenn der Scheitel von die horizontale
Achse "uberquert, wenn also beim Durchgang von $b$ durch den Wert $b_0$
die Zahl der Nullstellen von $0$ auf $2$ steigt oder umgekehrt.
Dies bedeutet, dass 
\[
f''(y_0,b_0)\ne 0
\qquad
\text{und}
\qquad
\frac{\partial f}{\partial b}(y_0, b_0)\ne 0.
\]
Modellhaft k"onnen wir dies durch das System
\[
f(y,b)=y^2+b
\]
wiedergeben.
Es hat f"ur $b<0$ die Nullstellen $\pm\sqrt{-b}$, f"ur $b>0$ jedoch
keine Nullstellen.
Das zugeh"orige Phasenportrait ist in Abbildung~\ref{geometrie:saddle-node}
dargestellt.
Bei dieser Art von Bifurkation entstehen neue kritische Punkte
paarweise, davon ist jeweils einer stabil, der andere instabil.
Diese Art der Bifurkation heisst {\em Sattel-Knoten-}
oder {\em Saddle-Node-Bifurkation}.
\index{Sattel-Knoten-Bifurkation}
\index{Saddle-Node-Bifurkation}

\begin{figure}
\centering
\includegraphics{chapters/images/bifurkation-1.pdf}
\caption{Phasendiagramm f"ur die Sattel-Knoten-Bifurkation in
Abh"angigkeit vom Parameter $b$.
\label{geometrie:saddle-node}}
\end{figure}

\subsubsection{Heugabel-Bifurkation}
\begin{figure}
\centering
\includegraphics{chapters/images/bifurkation-2.pdf}
\caption{Phasendiagramm f"ur die Heugabel-Bifurkation in Abh"angigkeit vom
Parameter $b$.
\label{geometrie:pitchfork}}
\end{figure}
Eine andere Art von Bifurkatione zeigt das Modell
\[
f(y,b)=y^3-by.
\]
F"ur $b<0$ hat die Funktion $y\mapsto f(y,b)$ nur die eine
reelle Nullstelle $y=0$.
F"ur $b>0$ dagegen liegen die drei verschiedenen Nullstellen
$-\sqrt{b}$, $0$  und $\sqrt{b}$.
Das zugeh"orige Phasendiagramm ist in Abbildung~\ref{geometrie:pitchfork}
dargestellt, diese Art von Bifurkation heisst {\em Heugabel-}
oder {\em Pitchformk-Bifurkation}.
\index{Heugabel-Bifurkation}
\index{Pitchfork-Bifurkation}
Bei dieser Art von Bifurkation werden aus einer Nullstelle deren drei.
War die Nullstelle instabil (wie in Abbildung~\ref{geometrie:pitchfork}),
entstehen zwei neue instabile Nullstellen, die bisherige Nullstellen 
wird stabil.
War die eine Nullstelle stabil, entstehen zwei stabile Nullstellen, die
eine Nullstelle wird instabil.

\subsubsection{Transkritische Bifurkation}
\begin{figure}
\centering
\includegraphics[width=\hsize]{chapters/images/bifurkation-3.pdf}
\caption{Phasendigramm f"ur die transkritische Bifurkation in Abh"angigkeit
vom Parameter $b$.
\label{geometrie:transkritisch}}
\end{figure}
Das Standard-Modell f"ur die sogenannten {\em transkritische Bifurkation} ist 
\index{transkritische Bifurkation}
\[
f(y,b)=y^2+yb = y(y+b)
\]
(Abbildung~\ref{geometrie:transkritisch}).
F"ur $b=0$ hat dieses System einen kritischen Punkt bei $y=0$.
Eine L"osungskurve, die bei negativem $y$ beginnt, wird immer n"aher
an den Punkt $y=0$ herankommen.
Liegt der Anfangswert jedoch im Gebiet $y>0$, wird sich die L"osungskurve
immer weiter von $y=0$ entfernen.
Der Fixpunkt bei $y=0$ ist also weder stabil noch instabil.

F"ur $b\ne 0$ entsteht ein weiterer kritischer Punkt bei $y=-b$.
Da f"ur $y\to\pm\infty$ die Funktion $y\mapsto f(y,b)$ immer positiv
ist, ist der rechte der beiden Fixpunkte immer stabil, der
linke immer instabil.


%
% Zweidimensionale Systeme
%
\section{Zweidimensionale Systeme}
\rhead{Zweidimensionale Systeme}
In diesem Abschnitt betrachten wir die geometrischen Einschr"ankungen, denen
zweidimensionale autonome Systeme unterliegen.
Wieder sind Punkte von besonderem Interesse, in denen $f$ verschwindet.
Bei eindimensionalen Systemen teilen solche Punkte den $y$-Bereich in
verschiedene Intervall, in denen die Bewegung nur in jeweils
eine Richtung erfolgen kann, damit ist es leicht zu entscheiden,
ob ein solcher kritischer Punkt stabil oder instabil ist.
In zwei Dimensionen legen die Punkte alleine den Charakter noch nicht
fest, insbesondere ist es ja auch m"oglich, dass eine Bahn einen solche
Punkt umschliesst.

\subsection{Nullklinen}
Kritische Punkte von $f$ sind solche, in denen beide Komponenten
des Vektors $f(y)$ verschwinden.
Die Gleichungen
\[
f_1(y)=0
\qquad\text{und}\qquad
f_2(y)=0
\]
beschreiben zwei Kurvenscharen in der Ebene, die sogenannten 
{\em Nullklinen}.
\index{Nullkline}
Die Schnittpunkte von Nullklinen sind die kritischen Punkte.

Da auf einer Nullkline jeweils eine der Komponenten von $f(y)$ verschwindet,
schneiden L"osungskurven Nullklinen immer entweder horizontal oder
vertikal.
Ausserdem trennen die Nullklinen Gebiete unterschiedlicher Bewegungsrichtung
entlang einer Achse. 
Die Nullkline $f_1(y)=0$ trennt Gebiete, in denen sich eine L"osungskurve
nach rechts (zu gr"osseren $y_1$ hin) bewegt ($f_1(y)>0$) von Gebieten,
in denen sich die L"osungskurve nach links ($f_1(y)<0$) bewegt.
Desgleichen trennt die Nullkline $f_2(y)=0$ Gebiete mit Bewegung ``nach oben''
von Gebieten mit Bewegung ``nach unten''.
Allein aus dieser Information kann man sich bereits ein recht gutes
qualitives Bild "uber den Verlauf der L"osungen verschaffen, wie wir
im Folgenden an zwei Beispiel illustrieren wollen.

\begin{beispiel}
\begin{figure}
\centering
\includegraphics{chapters/images/nullklinen-1.pdf}
\caption{Nullklinen der Differentialgleichung~(\ref{geometrie:nullklinen-dgl1}),
die $y_1$-Nullklinen in rot, die $y_2$-Nullklinen in blau.
Kritische Punkte sind die Schnittpunkte verschiedenfarbiger Nullklinen.
\label{geometrie:nullklinen1}}
\end{figure}
\begin{figure}
\centering
\includegraphics{chapters/images/nullklinen-2.pdf}
\caption{Vektorfeld der Differentialgleichung~(\ref{geometrie:nullklinen-dgl1}),
es best"atigt die Resultate der qualitiativen Diskussion aus
Abbildung~\ref{geometrie:nullklinen1}.
\label{geometrie:nullklinen-fluss}}
\end{figure}
\begin{figure}
\centering
\includegraphics{chapters/images/nullklinen-3.pdf}
\caption{Vektorfeld der Differentialgleichung~(\ref{geometrie:nullklinen-dgl1})
in einer Umgebung des instabilen kritischen Punktes $(\frac12,\frac12)$.
\label{geometrie:nullklinen-instabil}}
\end{figure}
\begin{figure}
\centering
\includegraphics{chapters/images/nullklinen-4.pdf}
\caption{Vektorfeld der Differentialgleichung~(\ref{geometrie:nullklinen-dgl1})
in einer Umgebung des stabilen kritischen Punktes $(2,0)$.
\label{geometrie:nullklinen-stabil}}
\end{figure}
Wir betrachten das nichtlineare System 
\begin{equation}
\frac{d}{dx} \begin{pmatrix}y_1\\y_2\end{pmatrix}
=
\begin{pmatrix}
2y_1\biggl(1-\displaystyle\frac{y_1}2\biggr)-3y_1y_2\\
y_2(1-y_2)-y_1y_2
\end{pmatrix}.
\label{geometrie:nullklinen-dgl1}
\end{equation}
Die direkte L"osung ist ziemlich aussichtslos, wir versuchen daher mit
Hilfe der Nullklinen ein qualitatives Bild
(Abbildung~\ref{geometrie:nullklinen1}) zu erhalten.

Die $y_2$-Nullkline hat die Gleichung
\[
0=y_2(1-y_2)-y_1y_2=y_2(1-y_2-y_1)
\qquad\Rightarrow\qquad
y_2=0
\quad\text{oder}\quad
y_1+y_2=1.
\]
L"osungskurven schneiden also die beiden Geraden $y_2=0$ (die $y_1$-Achse)
und $y_1+y_2=1$ horizontal.
Da die $y_1$-Achse bereits horizontal ist, bedeutet dies, dass sich die
L"osungskurven der $y_1$-Achse anschmiegen.

Die $y_1$-Nullkline hat die Gleichung
\[
0=2y_1\biggl(1-\frac{y_1}2\biggr)-3y_1y_2=y_1(2-y_1-3y_2),
\qquad\Rightarrow\qquad
y_1=0
\quad\text{oder}\quad
y_1+3y_2=2.
\]
Wir schliessen wieder, dass die L"osungskurven die $y_2$-Achse und
die Gerade $y_1+3y_2=3$ vertikal schneiden.
Da die $y_2$-Achse schon vertikal ist, m"ussen sich auch dort
die L"osungskurven anschmiegen.

Wir k"onnen aus den Nullklinen auch die kritischen Punkte ableiten.
Da sind zun"achst die Punkte auf den Achsen, also zum Beispiel
der Schnittpunkt der $y_2$-Nullkline $y_2=0$ mit der $y_1$-Nullkline
$y_1+3y_2=2$, also $(2,0)$, oder der Schnittpunkt der
$y_1$-Nullkline $y_1=0$ mit der $y_2$-Nullkline $y_1+y_2=0$, also $(0,1)$.
Ausserdem ist nat"urlich $(0,0)$ ein kritischer Punkt.
ein vierter kritischer Punkt entsteht als Schnittpunkt
der $y_1$-Nullkline $y_1+3y_2=2$ mit der $y_2$-Nullkline $y_1+y_2=1$,
also als L"osung des linearen Gleichungssystems
\[
\begin{linsys}{2}
y_1&+&3y_2&=&2\phantom{.}\\
y_1&+& y_2&=&1,
\end{linsys}
\]
es hat die L"osung $(\frac12,\frac12)$.
Die kritischen Punkte sind also
\begin{equation}
(0,0),\quad
(2,0),\quad
(0,1)\quad\text{und}\quad
\biggl(\frac12,\frac12\biggr).
\label{geometrie:nullklinen-krit}
\end{equation}

An den aus den Nullklinen l"asst sich auch die Bewegungsrichtung der
L"osungen im Bezug auf die kritischen Punkte ablesen.
\label{geometrie:nullklinen-stabilitaet}
Aus dem Gebiet oben rechts und unten links in der N"ahe des Ursprungs
bewegen sich die L"osungen zun"achst auf den kritischen Punkt
bei $(\frac12,\frac12)$ zu, weichen dann aber ab in Richtung auf die
kritischen Punkte $(0,1)$ und $(2,0)$ zu.
Die kritischen Punkte $(0,0)$ und $(\frac12,\frac12)$ sind also
instabil, w"ahrend $(2,0)$ und $(0,1)$ stabil sind.
Dies wird auch von dem genaueren Vektorfeld in
Abbildung~\ref{geometrie:nullklinen-fluss} best"atigt.
In Abschnitt~\ref{geometrie:umgebung-kritisch} wird gezeigt, dass man 
die Bewegung in der Umgebung eines kritischen Punktes mit Hilfe der
Eigenwerte der Ableitungen klassifizieren kann.
\end{beispiel}


\begin{beispiel}
Das {\em Fitzhugh-Nagumo-Modell} wird verwendet, um das Verhalten eines Neurons
zu simulieren.
\index{Fitzhugh-Nagumo-Modell}
\begin{figure}
\centering
\includegraphics{chapters/images/nullklinen-5.pdf}
\caption{Nullklinen des Fitzhugh-Nagumo-Modells bei nur einem kritischen Punkt,
$a=-0.8$, $b=0.9$.
\label{geometrie:nullklinen-fh-1}}
\end{figure}
\begin{figure}
\centering
\includegraphics{chapters/images/nullklinen-6.pdf}
\caption{Nullklinen des Fitzhugh-Nagumo-Modells mit drei kritischen Punkten,
$b=4$, $a=0$.
\label{geometrie:nullklinen-fh-2}}
\end{figure}
Es verwendet das Differentialgleichungssystem
\begin{equation}
\begin{aligned}
    \dot v&= v-\frac13v^3-w\\
\tau\dot w&= v-a-bw.
\end{aligned}
\label{geometrie:fitzhugh-dgl}
\end{equation}
Wir versuchen, uns wieder mit Hilfe der Nullklinen ein Bild von den
L"osungskurven zu verschaffen.
Die $v$-Nullkline hat die Gleichung
\[
0=v-\frac13v^3-w
\qquad\Rightarrow\qquad
w=v-\frac13v^3 = v(1-{\textstyle\frac1{\sqrt{3}}}v)(1+{\textstyle\frac1{\sqrt{3}}}v)
\]
Diese kubische Parabel hat im Nullpunkt die Steigung $1$.
Die $w$-Nullkline ist die Gerade
\[
0=v-a-bw
\qquad\Rightarrow\qquad
v=bw+a
\qquad\Rightarrow\qquad
w = \frac{v-a}{b}.
\]
Wenn die Steigung $1/b$ dieser Geraden gr"osser als $1$ ist, dann schneidet
die Gerade die kubische Parabel nur in einem Punkt, es gibt dann nur
einen kritischen Punkt.
Ist die Steigung $1/b<1$, gibt es f"ur nicht zu grosses $|a|$ drei
Schnittpunkte.

In Abbildung~\ref{geometrie:nullklinen-fh-1} sind die Nullklinen des
Fitzhugh-Nagumo-Modells mit nur einem kritischen Punkt dargestellt.
Die L"osungskurven bewegen sich in Spiralen im Gegenurzeigersinn
um den Fixpunkt herum.
In diesem Fall erlauben die Nullklinen keine abschliessende Beurteilung
der Stabilit"at des Fixpunktes.
Die Untersuchung der Eigenwerte der Jacobi-Matrix, die im nächsten
Abschnitt erkl"art wird, erm"oglicht die Entscheidung, und wird in
der Fortsetzung dieses Beispiels auf Seite~\pageref{geometrie:fh-fortsetzung}
durchgef"uhrt.

In Abbildung~\ref{geometrie:nullklinen-fh-2} sind die Nullklinen des
Fitzhugh-Nagumo-Modells dargstellt mit drei kritschen Punkten.
Man kann sofort ablesen, dass $(0,0)$ ein instabiler kritischer Punkt ist.
Die L"osungskurven, die von diesem Punkt abgestossen werden, n"ahern sich
einem der beiden anderen kritischen Punkte und bewegen sich
in einer Spirale um diesen Punkt herum.
Da sich die L"osungskurven nicht schneiden d"urfen kann man folgern,
dass die beiden anderen kritschen Punkte stabil sein m"ussen,
die L"osungskurven werden sich ihnen in Spiralbahnen n"ahern.
\label{geometrie:fh-diskussion}
\end{beispiel}

%
% Bewegung in der Umgebung eines kritischen Punktes
%
\subsection{Bewegung in der Umgebung eines kritischen Punktes
\label{geometrie:umgebung-kritisch}}
Wir gehen jetzt davon aus, dass 
\[
y'=f(y)
\]
einen kritischen Punkt hat, wir k"onnen die Koordinaten immer so w"ahlen,
dass der kritische Punkt im Nullpunkt des Koordinatensystems liegt.
Die Bewegung in unmittelbarer Umgebung des Nullpunktes kann dann approximiert
werden durch die Bewegung des linearisierten Systems
\[
y'=\frac{\partial f(0)}{\partial y}y
\]
Die m"oglichen Bewegungsformen in der Umgebung des kritischen Punktes
sind also bestimmt durch die Jacobi-Matrix.
Jede beliebige $2\times 2$-Matrix kann auch tats"achlich als Jacobi-Matrix
vorkommen, denn das System
$
y'=Ay
$
hat die Matrix $A$ als Jacobi-Matrix.

Wir interessieren uns im Moment nur f"ur eine qualitative Beschreibung
der L"osungen, wir k"onnen also immer eine Koordinatentransformation
vornehmen, um die Situation zu vereinfachen.
Die gesuchte qualitative Klassifizierung von zweidimensionalen
Differentialgleichungssystemen l"auft also auf eine Klassifizierung
von reellen $2\times 2$-Matrizen bis auf Koordinatentransformation
hinaus.

Eine solche Klassifikation kann auf der Basis von Eigenwerten und
Eigenvektoren erfolgen.
Dazu ben"otigen wir eine "Ubersicht "uber die Eigenwerte einer
Matrix
\[
A=\begin{pmatrix}a&b\\c&d\end{pmatrix},
\]
die wir als Nullstellen des charakteristischen Polynoms bestimmen
k"onnen.
Das charakteristische Polynom ist
\[
\chi_A(\lambda)
=
\det(A-\lambda E)
=
\left|\,\begin{matrix}a-\lambda&b\\c&d-\lambda\end{matrix}\,\right|
=
(a-\lambda)(d-\lambda)-bc
=
\lambda^2-(a+b)\lambda + ad-bc,
\]
es ist bestimmt durch die Spur und die Determinante der Matrix
\[
\begin{aligned}
\det A&=ad -bc,
&
\operatorname{Spur}A&=a+d
&
&\Rightarrow
&
\chi_A(\lambda)&=\lambda^2-\lambda \operatorname{Spur}A+\det A
\end{aligned}
\]
Die L"osungsformel f"ur die quadratische Gleichung liefert die
Eigenwerte
\[
\lambda_{1,2}
=
\frac{\operatorname{Spur}A}2\pm\sqrt{\Delta}
\qquad
\qquad
\text{mit}\quad
\Delta = \biggl(\frac{\operatorname{Spur}A}2\biggr)^2 - \det A
\]
Falls die Diskriminanten $\Delta > 0$ ist, sind die beiden Eigenwerte
verschieden, und folglich gibt es zwei verschiedene Eigenvektoren,
die Matrix $A$ kann diagonalisiert werden mit Diagonalelementen
$\lambda_{1,2}$.  Das Differentialgleichungssystem zerf"allt dann
in zwei unabh"angige eindimensionale Systeme
\begin{align*}
y_1'&= \lambda_1 y_1\\
y_2'&= \lambda_2 y_2,
\end{align*}
die auch durch
\begin{align*}
y_1(x)&=y_{10} e^{\lambda_1 x}\\
y_2(x)&=y_{20} e^{\lambda_2 x}
\end{align*}
sofort gel"ost werden k"onnen.
F"ur die Diskussion der Form der L"osungskurven brauchen wir aber die
Abh"angigkeit der beiden Koordinaten untereinander, nicht von $x$.
Wir stellen daher $y_2$ als Funktion von $y_1$ dar:
\[
y_1=y_{10} e^{\lambda_1 x}
\qquad\Rightarrow\qquad
x=\frac1{\lambda_1}\log\frac{y_1}{y_{10}}
\qquad\Rightarrow\qquad
y_2
=
y_{20} e^{\frac{\lambda_2}{\lambda_1}\log\frac{y_1}{y_{10}}}
=
y_{20}\biggl(\frac{y_1}{y_{10}}\biggr)^{\frac{\lambda_2}{\lambda_1}}
=
Cy^{\frac{\lambda_2}{\lambda_1}}
\]
\begin{figure}
\centering
\begin{tabular}{ccc}
\includegraphics{chapters/images/geometrie-2.pdf}&%
\includegraphics{chapters/images/geometrie-3.pdf}&%
\includegraphics{chapters/images/geometrie-4.pdf}\\
$\displaystyle \frac{\lambda_2}{\lambda_1}>1$&%
$\displaystyle \frac{\lambda_2}{\lambda_1}=1$&%
$\displaystyle \frac{\lambda_2}{\lambda_1}<1$
\end{tabular}
\caption{L"osungskurven des linearisierten Systems f"ur
$\frac{\lambda_2}{\lambda_1}>0$.
\label{geometrie:posportraits}}
\end{figure}

\begin{figure}
\centering
\begin{tabular}{ccc}
\includegraphics{chapters/images/geometrie-5.pdf}&%
\includegraphics{chapters/images/geometrie-6.pdf}&%
\includegraphics{chapters/images/geometrie-7.pdf}\\
$\displaystyle \frac{\lambda_2}{\lambda_1}>-1$&%
$\displaystyle \frac{\lambda_2}{\lambda_1}=-1$&%
$\displaystyle \frac{\lambda_2}{\lambda_1}<-1$
\end{tabular}
\caption{L"osungskurven des linearisierten Systems f"ur
$\frac{\lambda_2}{\lambda_1}<0$.
\label{geometrie:negportraits}}
\end{figure}
Die Gestalt der L"osungskurven sind also im Wesentlichen durch den
Quotienten $\lambda_2/\lambda_1$ bestimmt.
Die Abbildung~\ref{geometrie:posportraits} zeigt die L"osungskurven
f"ur positive Werte des Quotienten, w"ahrend
Abbildung~\ref{geometrie:negportraits} die L"osungskurven f"ur
negative Werte des Quotienten zeigt.
F"ur positive Werte von $\lambda_2/\lambda_1$ bewegen sich die 
Punkte entweder immer auf den kritischen Punkt zu, oder entfernen
sich.
F"ur negative Werte von $\lambda_2/\lambda_1$ n"ahert sich ein
Punkt in der N"ahe der einen Achse zun"achst immer mehr dem kritischen
Punkt, um sich dann in Richtung der anderen Achse zu entfernen.

Die Matrix $A$ muss jedoch nicht diagonalisierbar sein, wenn
$\lambda_1=\lambda_2$.
Wenn die Matrix nur einen Eigenvektor hat, dann kann die Matrix
durch eine geeignete Koordinatentransformation in die Form
\[
\begin{pmatrix}
\lambda&      1\\
      0&\lambda
\end{pmatrix}
\]
bringen.
Die Differentialgleichungen lauten in diese Fall
\begin{align*}
y_1'&=\lambda y_1 + y_2\\
y_2'&=\lambda y_2
\end{align*}
Die zweite Gleichung kann wie vorher gel"ost werden, die L"osung f"ur $y_2$
ist
\[
y_2(x)=y_{20}e^{\lambda x},
\]
dies k"onnen wir in die erste Gleichung einsetzen, sie lautet jetzt
\[
y_1' = \lambda y_1 + y_{20}e^{\lambda x},
\]
dies ist wieder eine lineare Differentialgleichung, diesmal jedoch
eine inhomogene. 
Die L"osung der homogenen Gleichung ist $Ce^{\lambda x}$, die L"osung
der inhomogenen Gleichung kann durch Variation der Konstanten gefunden
werden, also $y_1(x)=C(x)e^{\lambda x}$.
Setzen wir dies in die Differentialgleichung 
\[
y_1'(x)
=
C'(x)e^{\lambda x}+C(x)\lambda e^{\lambda x}
=
\lambda y_1(x) + C'(x)e^{\lambda x}
=
\lambda y_1(x) + y_{20}e^{\lambda x}
\]
ein.
Diese Gleichung kann nur erf"ullt sein, wenn
\[
C'(x)=y_{20}
\qquad\Rightarrow\qquad
C(x)=y_{20}x+y_{10},
\]
die L"osung der Gleichung ist also
\[
y(x)=\begin{pmatrix}
y_{20}x+y_{10}\\
y_{20}
\end{pmatrix}e^{\lambda x}.
\]
\begin{figure}
\centering
\includegraphics{chapters/images/geometrie-8.pdf}
\caption{L"osungskurven f"ur den Fall nicht diagonalisierbarer Jacobi-Matrix
mit zwei gleichen Eigenwerten.
\label{geometrie:jnf-kurven}}
\end{figure}%
In Abbildung~\ref{geometrie:jnf-kurven} sind die L"osungskurven dargestellt.
F"ur $\lambda >0$ streben die L"osungskurven gegen den Nullpunkt, aber auf
eine Art, sie im Grenzfall die $y_1$-Achse ber"uhren.

Falls die Diskriminante $\Delta$ negativ ist, gibt es keine reellen
Eigenwerte, also auch keine reellen Eigenvektoren.
Wir k"onnen aber trotzdem eine Basis finden, in der die Geometrie
dar Bahnkurven leichter verst"andlich ist.
Dazu schreiben wir 
\[
\alpha = \frac{\operatorname{Spur}A}2=\frac{a+d}2
\qquad
\text{und}
\qquad
\beta = \sqrt{-\Delta},
\]
und betrachten die beiden Vektoren
\[
w_1 = \begin{pmatrix}0\\\beta\end{pmatrix}.
\qquad\text{und}\qquad
w_2 = \begin{pmatrix}b\\\alpha-a\end{pmatrix}
\]
Wir berechnen die Wirkung der Matrix $A$ auf diesen beiden Vektoren,
und zerlegen jeweils das Resultat wieder in $w_1$ und $w_2$:
\begin{align*}
Aw_1
&=
\begin{pmatrix}a&b\\c&d\end{pmatrix}
\begin{pmatrix}0\\\beta\end{pmatrix}
=
\begin{pmatrix}
\beta b\\
\beta d
\end{pmatrix}
=
v
\begin{pmatrix}0\\\beta\end{pmatrix}
+
\beta
\begin{pmatrix}b\\\alpha-a\end{pmatrix}
\\
Aw_2
&=
\begin{pmatrix}a&b\\c&d\end{pmatrix}
\begin{pmatrix}b\\\alpha-a\end{pmatrix}
=
\begin{pmatrix}
ab-ab+b\alpha\\
bc-ad+\alpha d
\end{pmatrix}
=
\alpha
\begin{pmatrix}b\\\alpha-a\end{pmatrix}
+u
\begin{pmatrix}0\\\beta\end{pmatrix}
\end{align*}
Wir m"ussen nur noch die Konstanten $u$ und $v$ bestimmen:
\begin{align*}
\beta d
&=
v\beta
+
\alpha\beta
-
\beta a
&&\Rightarrow&
v
&=
\alpha a+d-\alpha=2\alpha-\alpha=\alpha
\\
-\det A +\alpha d
&=
\alpha^2- \alpha a+u\beta
&&\Rightarrow&
u\beta&=-\det A +\alpha(a+d)-\alpha^2
=\alpha^2-\det A
=\Delta=-\beta^2
\end{align*}
Aus der zweiten Gleichung folgt $ u=-\beta$.
Damit haben wir die Wirkung der Matrix $A$ auf den Vektoren $w_1$ und $w_2$
bestimmt, und wir k"onnen daraus die Matrix von $A$ in der Basis
$\{w_1,w_2\}$ ablesen, wir bezeichnen sie mit $A'$
\[
\begin{aligned}
Aw_1&=\alpha w_1 + \beta w_2\\
Aw_2&=-\beta w_1 + \alpha w_2
\end{aligned}
\qquad\Rightarrow\qquad
A'=\begin{pmatrix}
\alpha&\beta\\
-\beta&\alpha
\end{pmatrix}
\]
Dies ist die gesuchte Form der Matrix, in der sich die L"osungskurven
leichter beschreiben lassen.
Eine L"osung daf"ur l"asst sich angeben, wenn man ber"ucksichtigt, dass
$A'$ einer Drehmatrix "ahnelt.
Wir vermuten daher, dass die L"osungskurve im wesentlichen den kritischen
Punkt umkreist, m"oglicherweise mit einer "Anderung des Abstandes
zum kritischen Punkt, und schreiben daher
\begin{equation}
y(x)
=
\begin{pmatrix}
r_0e^{ux}\cos(vx+\delta_0)\\
r_0e^{ux}\sin(vx+\delta_0)
\end{pmatrix}
=
r_0e^{ux}
\begin{pmatrix}
\cos(vx+\delta_0)\\
\sin(vx+\delta_0)
\end{pmatrix}
,
\label{geometrie:rotsol}
\end{equation}
wobei wird $r_0$ und $\delta_0$ so w"ahlen, dass
\[
y_{10}=r_0\cos\delta_0
\qquad\text{und}\qquad
y_{20}=r_0\sin\delta_0.
\]
Setzen wir jetzt den Ansatz~(\ref{geometrie:rotsol}) in die
Differentialgleichung ein, erhalten wir
\begin{align*}
y'(x)
&=
\begin{pmatrix}
r_0ue^{ux}\cos(vx+\delta_0)-r_0e^{ux}v\sin(vx+\delta_0)\\
r_0ue^{ux}\sin(vx+\delta_0)+r_0e^{ux}v\cos(vx+\delta_0)
\end{pmatrix}
\\
&=
\begin{pmatrix}
 \alpha&\beta\\
-\beta &\alpha
\end{pmatrix}
\begin{pmatrix}
r_0e^{ux}\cos(vx+\delta_0)\\
r_0e^{ux}\sin(vx+\delta_0)
\end{pmatrix}
=
\begin{pmatrix}
r_0e^{ux}\alpha\cos(vx+\delta_0)+r_0e^{ux}\beta\sin(vx +\delta_0)\\
-r_0e^{ux}\beta\cos(vx+\delta_0)+r_0e^{ux}\alpha\sin(vx+\delta_0)
\end{pmatrix}
\end{align*}
Diese Gleichung ist genau dann korrekt, wenn 
\[
u=\alpha
\qquad\text{und}\qquad
v=-\beta.
\]
Die Zahlen $\alpha$ und $\beta$ charakterisieren also wieder die L"osung.
F"ur $\alpha < 0$ n"ahern sich die L"osungen dem kritischen Punkt, f"ur
$\alpha>0$ entfernen sie sich.
Die Zahl $\beta$ ist die Winkelgeschwindigkeit, mit der die L"osung
um den kritischen Punkt rotiert.
Die L"osungskurven sind daher Spiralen um den kritischen Punkt, sie
sind in Abbildung~\ref{geometrie:rotkurv} dargestellt.
\begin{figure}
\centering
\begin{tabular}{ccc}
\includegraphics{chapters/images/geometrie-9.pdf}&%
\includegraphics{chapters/images/geometrie-11.pdf}&%
\includegraphics{chapters/images/geometrie-10.pdf}\\
$\alpha > 0$&$\alpha = 0$&$\alpha < 0$
\end{tabular}
\caption{L"osungskurven des linearisierten Systems im Falle $\Delta < 0$
sind Spiralen um den kritischen Punkt
\label{geometrie:rotkurv}}
\end{figure}

\begin{beispiel}
Wir kehren nochmals zum Beispiel~(\ref{geometrie:nullklinen-dgl1})
von Seite~\pageref{geometrie:nullklinen-dgl1} zur"uck.
Die kritischen Punkte wurden in (\ref{geometrie:nullklinen-krit}) bereits
zusammengestellt.
Die Ableitung von $f$, also die Jacobi-Matrix, ist
\[
\frac{\partial f}{\partial y}
=
\frac{\partial}{\partial y}
\begin{pmatrix}
2y_1-y_1^2-3y_1y_2\\
y_2-y_2^2-y_1y_2
\end{pmatrix}
=
\begin{pmatrix}
2-2y_1-3y_2 & -3y_1\\
-y_2        &1-2y_2-y_1
\end{pmatrix}
\]
In den vier kritischen Punkten finden wir die folgenden Matrizen und
Eigenwerte
\begin{align*}
&(0,0)
	&\frac{\partial f}{\partial y}&=\begin{pmatrix}2&0\\0&1\end{pmatrix}
		&&\lambda_1=2,\;\lambda_2=1\\
&(2,0)
	&\frac{\partial f}{\partial y}&=\begin{pmatrix}-2&-6\\0&-1\end{pmatrix}
		&&\lambda_1=-2,\;\lambda_2=-1\\
&(0,1)
	&\frac{\partial f}{\partial y}&=\begin{pmatrix}-1&0\\-1&-1\end{pmatrix}
		&&\lambda_1=-1,\;\lambda_2=-1\\
&\textstyle(\frac12,\frac12)
	&\frac{\partial f}{\partial y}&=\begin{pmatrix}-\frac12&-\frac32\\-\frac12&-\frac12\end{pmatrix}
		&&\lambda_1=\frac{\sqrt{3}-1}2,\;\lambda_2=-\frac{\sqrt{3}+1}2
\end{align*}
Nur bei den Punktein $(2,0)$ und $(0,1)$ sind beide Eigenwerte negativ,
nur diese beiden Punkte sind stabil, wie bereits die  Diskussion der
Nullklinen auf Seite~\pageref{geometrie:nullklinen-stabilitaet}
gezeigt hat.
\end{beispiel}

%
% Komplexe Eigenwerte
%
\subsection{Komplexe Eigenwerte}
Die Darstellung im vorangegangenen Abschnitt war darum bem"uht,
komplexe Zahlen zu vermeiden.
Die Darstellung im Falle $\Delta<0$ wurde dadurch unn"otig verkompliziert,
in diesem Abschnitt soll gezeigt werden, wie die Formeln f"ur die Vektoren
$w_1$ und $w_2$ mit Hilfe komplexer Zahlen hergeleitet werden k"onnen.

Zun"achst halten wir fest, dass im Falle $\Delta<0$ zwei konjugiert
komplexe Eigenwerte
$
\lambda= \alpha + i\beta
$
und
$
\overline{\lambda}= \alpha - i\beta
$
existieren.
Nehmen wir an, dass $v$ ein Eigenvektor zum Eigenwert $\lambda$ ist,
dann ist $\overline{v}$, dessen Komponenten die konjugiert komplexen
Komponenten von $v$ sind, ein Eigenvektor zum Eigenwert $\overline{\lambda}$.
Grund daf"ur ist die Tatsache, dass die Matrix $A$ nur reelle Matrixelemente
hat, also gilt
\[
A\overline{v}
=
\overline{Av}
=
\overline{\lambda v}=\overline{\lambda}\overline{v}.
\]
Der Vektor $v$ ist nat"urlich nicht geeignet f"ur eine reelle Beschreibung
der L"osungskurven des linearisierten Systems.
Wir konstruieren daher die Vektoren
\begin{equation}
\begin{aligned}
w_1&=\frac{i}2(v-\overline v)
&&\qquad
&
v&=-iw_1+w_2
\\
w_2&=\frac12(v+\overline v)
&&\qquad
&
\overline{v}&=iw_1+w_2
\end{aligned}
\label{geometrie:wv}
\end{equation}
und untersuchen, wie die Matrix $A$ darauf wirkt:
\begin{align*}
Aw_1
&=
\frac{i}2(Av-A\overline v)
=
\frac{i}2(\lambda v-\overline{\lambda}\overline{v})
\\
Aw_2
&=
\frac12(Av+A\overline{v})
=
\frac12(\lambda v+\overline{\lambda}\overline{v}).
\end{align*}
Setzen wir die Darstellungen von $v$ und $\overline{v}$ durch $w_i$ aus 
(\ref{geometrie:wv}) ein, und erhalten:
\begin{align*}
\frac{i}2(\lambda v-\overline{\lambda}\overline{v})
&=
\frac{i}2(\lambda(-iw_1+w_2) -\overline{\lambda}(iw_1+w_2))
=
\frac{1}2(\lambda+\overline{\lambda}) w_1
+
\frac{i}2(\lambda-\overline{\lambda}) w_2
=\alpha w_1-\beta w_2
\\
\frac12(\lambda v+\overline{\lambda}\overline{v})
&=
\frac12(\lambda(-iw_1+w_2)+\overline{\lambda}(iw_1+w_2))
=
-\frac{i}2(\lambda-\overline{\lambda}) w_1
+
\frac12(\lambda+\overline{\lambda}) w_2.
=\beta w_1+\alpha w_2
\end{align*}
Verwendet man also $\{w_1,w_2\}$ als Basis, dann bekommt die Matrix die
Form
\begin{equation}
A'=\begin{pmatrix}
\alpha&-\beta\\
\beta &\alpha
\end{pmatrix}
\label{geometrie:drehmatrix}
\end{equation}
Auf Grund der Konstruktion haben die Vektoren $w_1$ und $w_2$ reelle
Komponenten, $w_1$ ist der Realteil des Vektors $v$, $w_2$ ist
der Imagin"arteil.
Damit haben wir ein Rezept, wie wir eine Basis von reellen Vektoren
konstruieren k"onnen, in denen das System die
Form~(\ref{geometrie:drehmatrix}) hat.

Die Komponenten eines Eigenvektors $v$ erf"ullen die Gleichung
\[
(a-\lambda)v_1 + bv_2=0
\]
eine L"osung daf"ur ist
\[
v=\begin{pmatrix}
-b\\
a-\alpha-i\beta
\end{pmatrix},
\]
dessen Real- und Imagin"arteile
\[
\begin{pmatrix}
-b\\a-\alpha
\end{pmatrix}
\qquad\text{und}\qquad
\begin{pmatrix}
0\\\beta
\end{pmatrix},
\]
dies sind die Vektoren, die im vorangegangenen Abschnitt aus dem "Armel
gesch"uttelt worden waren, um die Matrix des Systems in die
Form~(\ref{geometrie:drehmatrix}) zu bringen.

\begin{beispiel}
\label{geometrie:fh-fortsetzung}
Wir wenden obige Analyse auf das
Fitzhugh-Nagumo-Modell~(\ref{geometrie:fitzhugh-dgl}) von
Seite~\pageref{geometrie:fitzhugh-dgl} an.
Um die Diskussion etwas zu vereinfachen, untersuchen wir nur den Fall
$\tau = 1$.
Wir m"ussen die Jacobi-Matrix in einem kritischen Punkt bestimmen, sie ist
\begin{equation}
J=
\begin{pmatrix}
1-v^2 &  -1 \\
  1   &  -b
\end{pmatrix}.
\end{equation}
Das charakteristische Polynom ist
\begin{align*}
\det(J-\lambda E)
&=
\left|
\begin{matrix}
1-v^2-\lambda&-1\\
1&-b-\lambda
\end{matrix}
\right|
\\
&=
(1-v^2-\lambda)(-b-\lambda)+1
\\
&=
(\lambda+v^2-1)(\lambda+b)+1
\\
&=
\lambda^2 + (\underbrace{v^2 + b - 1}_{\textstyle p})\lambda
+ \underbrace{b(v^2 - 1)+1}_{\textstyle q}.
\end{align*}
Es hat die Nullstellen
\begin{equation}
\lambda_{1,2}
=
-\frac{p}{2}\pm\sqrt{\frac{p^2}4-q}
=
-\frac{v^2+b-1}2\pm\sqrt{\frac{(v^2+b-1)^2}4-b(v^2-1)-1}.
\label{geometrie:fn-eigenwerte}
\end{equation}
Der erste Term in der Wurzel ist das Quadrat des Terms vor der Wurzel.
Ohne $q$ in der Wurzel g"abe es einen Eigenwert mit dem gleichen Vorzeichen
wie $p$, der andere ist $0$.
Die Vorzeichen von $p$ und $q$ bestimmen also weitgehen, ob ein 
kritischer Punkt stabil oder instabil ist.

\begin{enumerate}
\item
Wenn $q<0$ ist, dann ist die Wurzel gr"osser als $p/2$, die beiden
Eigenwerte haben verschiedenes Vorzeichen, und der kritische Punkt
ist instabil unabh"angig vom Vorzeichen von $p$. 
Dieser Fall tritt ein, wenn 
\[
b(v^2 -1) + 1 < 0
\qquad\Rightarrow\qquad
1-\frac1b > v^2.
\]
F"ur $b<1$ wird die linke Seite negativ, dann kann dieser Fall also gar nicht
eintreten.
\item
Ist $q>0$ und von gen"ugend grossem absolutem Betrag,
dann wird der Radikand negativ, beide Eigenwerte
sind komplex und der kritische Punkt ist genau dann stabil, wenn $p>0$ ist.
Ist $q>0$, aber nicht von gen"ugend grossem absolutem Betrag, dann
bleibt der Radikand positiv.
In diesem Fall haben beide Eigenwerte das gleiche Vorzeichen wie $p$,
auch in diesem Fall hat man also Stabilit"at genau dann, wenn $p>0$ ist.
\end{enumerate}
Wir k"onnen die Resultate der obigen Diskussion in der folgenden
Entscheidungstabelle zusammenfassen:
\begin{center}
\begin{tabular}{|>{$}c<{$}|>{$}c<{$}|l|}
\hline
q=b(v^2-1)+1 & p= v^2 + b -1 &           \\
\hline
     <0      &               &  instabil \\
     >0      &       <0      &  instabil \\
     >0      &       >0      &  stabil   \\
\hline
\end{tabular}
\end{center}

F"ur $b>1$ und $a=0$ ist $(0,0)$ ein kritischer Punkt, und es gilt $q=-b+1<0$
und $p=b-1>0$, wir sind daher im Fall~1, der Ursprung ist instabil.
Es ist aber auch $p=v^2+b-1>v^2\ge 0$, der Fall $p<0$ kann also 
gar nicht eintreten, ein solcher kritischer Punkt muss also immer
stabil sein.
Dies deckt sich mit den Resultaten der Diskussion von
Seite~\pageref{geometrie:fh-diskussion}.
\end{beispiel}

%
% Hopf-Bifurkation
%
\subsection{Hopf-Bifurkation}
\begin{figure}
\centering
\includegraphics{chapters/images/hopf-1.pdf}
\caption{Fluss f"ur $b<0$, der kritische Punkt ist stabil,
Bahnkurven konvergieren gegen $0$.
\label{geometrie:hopf1}}
\end{figure}%
\begin{figure}
\centering
\includegraphics{chapters/images/hopf-2.pdf}
\caption{Fluss f"ur $b=0$, der kritische Punkt ist immer noch stabil,
die Bahnkurven n"ahern sich jedoch nicht mehr exponentiell schnell
dem Nullpunkt.
\label{geometrie:hopf2}}
\end{figure}%
\begin{figure}
\centering
\includegraphics{chapters/images/hopf-3.pdf}
\caption{Fluss f"ur $b>0$, der Nullpunkt ist nicht mehr stabil, daf"ur
ist der Kreis mit Radius $\sqrt{b}$ eine stabile periodische Bahn (blau),
gegen die alle Bahnkurven exponentiell schnell konvergieren.
\label{geometrie:hopf3}}
\end{figure}%
\begin{figure}
\centering
\begin{tabular}{ccc}
\includegraphics{chapters/images/hopf-4.pdf}&%
\includegraphics{chapters/images/hopf-5.pdf}&%
\includegraphics{chapters/images/hopf-6.pdf}%
\end{tabular}
\caption{Vorzeichen von $\dot r$ in Abh"angigkeit von $b$.
Punkte mit $\dot r <0$ sind blau gef"arbt, Punkte mit $\dot r >0$ rot.
\label{geometrie:hopfvorzeichen}}
\end{figure}%
Die in Abschnitt~\ref{geometrie:subsection:bifurkationen} untersuchten
Bifurkationen eindimensionaler Differentialgleichungen k"onnen in
analoger Form auch bei zweidimensionalen Differentialgleichungen auftreten.
Sie sind jedoch immer eindimensionale Bifurkationen, die entlang der
durch die Eigenvektoren der Linearisierung gegebenen Richtungen
auftreten.

Die h"ohere Dimensionszahl erlaubt aber auch eine Bifurkation, bei der
ein stabiler Fixpunkt in stabil wird und einen stabilen Zyklus ``abwirft''
(Abbildungen~\ref{geometrie:hopf1}, \ref{geometrie:hopf2} and
\ref{geometrie:hopf3}).
Sie heisst die Hopf-Bifurkation.
\index{Hopf-Bifurkation}
Wir betrachten dazu das System 
\begin{equation}
\begin{aligned}
\dot r      &= r(b-r^2)\\
\dot \varphi&= -1
\end{aligned}
\label{geometrie:hopfsystem}
\end{equation}
in Polarkoordinaten.
Offenbar ist $r=0$ ein Fixpunkt.
F"ur $b>0$ gibt es ausserdem eine periodische Bahn mit $r=\sqrt{b}$
(Abbildung~\ref{geometrie:hopf3}).
Wir wollen die Stabilit"at des Fixpunktes sowie der periodischen Bahn
untersuchen.
Die Abbildung~\ref{geometrie:hopfvorzeichen} fasst die f"ur das Bahnverhalten
entscheidenden Vorzeichen in den drei F"allen $b<0$, $b=0$ und $b>0$
zusammen.

In Polarkoordinaten beschreibt die Gleichung f"ur $r$ eine
Heugabel-Bifurkation.
Der kritische Punkt $r$ ist f"ur $b<0$ stabil, er wird f"ur $b>0$
instabil, daf"ur entstehen zwei neue stabile kritische Punkte
$r=\pm\sqrt{b}$.

Unsere bisherige Theorie zur Beurteilung von Fixpunkten ging von
kartesischen Koordinaten aus, wir f"uhren daher die Analyse auch noch
in kartesischen Koordinaten durch.
Die Umrechnungsformeln von Polarkoordinaten in kartesische Koordinaten
und ihre Ableitungen
\[
\begin{aligned}
x&=r\cos\varphi&&\qquad&\dot x&=\dot r\cos\varphi-r\sin\varphi\cdot\dot\varphi\\
y&=r\sin\varphi&&\qquad&\dot y&=\dot r\sin\varphi+r\cos\varphi\cdot\dot\varphi
\end{aligned}
\]
erlauben uns,
das System~(\ref{geometrie:hopfsystem}) in kartesische Koordinaten
umzurechnen:
\begin{equation}
\begin{aligned}
\dot x&=r(b-r^2)\frac{x}{r}+y=(b-x^2-y^2)x+y\\
\dot y&=r(b-r^2)\frac{y}{r}-x=(b-x^2-y^2)y-x.
\end{aligned}
\label{geometrie:hopf-kartesisch}
\end{equation}
Zur Beurteilung der Stabilit"at des Nullpunktes berechnen wir die
Jacobi-Matrix
\[
J(x,y)=
\begin{pmatrix}
b-3x^2-y^2&1-2xy\\
-1-2xy&b-x^2-3y^2
\end{pmatrix}
\quad\Rightarrow\quad
J(0,0)=\begin{pmatrix}
b&1\\-1&b
\end{pmatrix}.
\]
Die Matrix $J(0,0)$ hat das charakteristische Polynom
\[
(b-\lambda)^2+1=0
\]
mit den Nullstellen
\[
\lambda=b\mp i.
\]
Stabilit"at wird durch das Vorzeichen des Realteils der Eigenwerte
bestimmt, wir lesen daher ab, dass der kritische Punkt $0$ stabil
ist f"ur $b<0$ und instabil f"ur $b>0$.

\section{"Ubungsaufgaben}
\rhead{"Ubungsaufgaben}
\uebungsaufgabe{601}
\uebungsaufgabe{602}
\uebungsaufgabe{603}


%
% komplex.tex -- Komplexe Differentialgleichungen
%
% (c) 2015 Prof Dr Andreas Mueller, Hochschule Rapperswil
%
\chapter{Komplexe Differentialgleichungen\label{chapter:komplexeanalysis}}
\rhead{}
\lhead{Komplexe Differentialgleichungen}
Die bisher betrachteten Differentialgleichungen waren immer f"ur
$x\in\mathbb R$ definiert.
Bei der L"osung mit Hilfe von Potenzreihen haben wir L"osungsfunktionen
gefunden, die man auch f"ur komplexe $x$-Werte auswerten kann.
Definiert man die Ableitung einer Funktionen einer komplexen Variablen
$z$ rein formal als
\[
\frac{d}{dz}z^n= nz^{n-1},
\]
dann kann man auch Potenzreihen in der Variablen $z$ formal differenzieren,
indem man jeden Term der Potenzreihe ableitet.
Und die mit der Potenzreihen-Methode gefunden L"osungen erf"ullen dann
auch die urspr"ungliche Differentialgleichung.
Dies ist aber eine rein formale "Uberlegung, da die Ableitung nach einer
komplexen Variablen noch gar nicht definiert ist.

%
% Komplex differenzierbare Funktion
%
\section{Komplex differenzierbare Funktionen}
\rhead{Komplex differenzierbare Funktionen}
Wir betrachten in diesem Kapitel komplexwertige Funktionen,
\index{komplexwertige Funktion}%
die ein einem Teilgebiet der komplexen Ebene definiert sind.
Ein {\em Gebiet} ist eine offene Teilmenge $\Omega\subset \mathbb C$.
\index{Gebiet}%
{\em Offen} heisst, dass mit jedem Punkt $z_0\in\Omega$ eine Umgebung
\index{offen}%
\index{Umgebung}%
\[
U=\{z\in\mathbb Z\,|\,|z-z_0|<\varepsilon\}
\]
ebenfalls in $\Omega$ enthalten ist, also $U\subset \Omega$ f"ur gen"ugen
kleines $\varepsilon$.
Sei also $f(z)$ eine in $\Omega\subset\mathbb C$ definierte
Funktion $f\colon\Omega\to\mathbb C$. 

Eine komplexwertige Funktion $f(z)$ kann betrachtet werden als zwei
reellwertige Funktionen von zwei Variablen $x$ und $y$:
\[
f(z)=\operatorname{Re}f(x+iy) + i \operatorname{Im}f(x+iy)
\]
Schreibt man
$\operatorname{Re}f(x+iy)=u(x,y)$
und
$\operatorname{In}f(x+iy)=v(x,y)$,
dann ist die komplexe Funktion vollst"andig durch reelle Funktionen
beschrieben.
Und nat"urlich wissen wir auch, was unter den Ableitungen der Funktionen 
$u(x,y)$ und $v(x,y)$ zu verstehen ist.
Der Funktion $f(z)$ entspricht eine Abbildung $\mathbb R^2\to\mathbb R^2$
\index{Abbildung}%
\[
(x,y)\mapsto\begin{pmatrix}u(x,y)\\v(x,y)\end{pmatrix}.
\]
Die Ableitung einer solchen Funktion im Punkt $(x_0,y_0)$
ist eine lineare Abbildung von Vektoren, die in linearer N"aherung
\index{lineare Naherung@lineare N\"aherung}
\index{Naherung@N\"aherung, lineare}
den Funktionswert bei $f(z_0 + \Delta z)$ 
\[
\begin{pmatrix}
u(x+\Delta x, y +\Delta y)\\
v(x+\Delta x, y +\Delta y)
\end{pmatrix}
=
\begin{pmatrix}
\frac{\partial u}{\partial x}&\frac{\partial u}{\partial y}\\
\frac{\partial v}{\partial x}&\frac{\partial v}{\partial y}
\end{pmatrix}
\begin{pmatrix} \Delta x\\\Delta y \end{pmatrix}
+o(\Delta x, \Delta y).
\]
In dieser Sicht einer komplexen Funktion gibt es keine einzelne Zahl, die
die Funktion einer Ableitung "ubernehmen k"onnte, die Ableitung
ist eine $2\times 2$-Matrix.

%
% Definition der komplexen Ableitungen
%
\subsection{Komplexe Ableitung}
Die Ableitung einer Funktion einer reellen Variablen wird mit Hilfe des
Grenzwertes
\[
f'(x_0)=\lim_{x\to x_0}\frac{f(x)-f(x_0)}{x-x_0}
\]
definiert, oder als diejenige Zahl $f'(x_0)\in\mathbb R$ mit der Eigenschaft,
dass
\begin{equation}
f(x)=f(x_0)+f'(x_0)(x-x_0) + o(x-x_0)
\label{komplex:abldef}
\end{equation}
gilt.
Der Term $x-x_0$ und die Gleichung \eqref{komplex:abldef} sind aber auch
f"ur komplexe Argument sinnvoll, wir definieren daher

\begin{definition}
Die komplexe Funktion $f(z)$ heisst im Punkt $z_0$ komplex differenzierbar
und hat die komplexe Ableitung $f'(z_0)\in\mathbb C$, wenn 
\index{komplex differenzierbar}%
\index{komplexe Ableitung}%
\index{Ableitung!komplexe}%
\begin{equation}
f(z)=f(z_0) + f'(z_0)(z-z_0) +o(z-z_0)
\label{komplex:defkomplabl}
\end{equation}
gilt.
\end{definition}

\begin{beispiel}
Die Funktion $z\mapsto f(z)=z^n$ ist "uberall komplex differenzierbar
und hat die Ableitung $nz^{n-1}$.
Um dies nachzupr"ufen, m"ussen wir die Bedingung~\eqref{komplex:defkomplabl}
verifizieren.
Aus einer wohlbekannten Faktorisierung von $z^n - z_0^n$ k"onnen wir den
Differenzenquotienten finden:
\begin{align*}
\frac{f(z)-f(z_0)}{z-z_0}
&=
\frac{z^n-z_0^n}{z-z_0}
=
\frac{(z-z_0)(z^{n-1}+z^{n-2}z_0+z^{n-3}z_0^2+\dots+z^{n-1})}{z-z_0}
\\
&=
\underbrace{z^{n-1}+z^{n-2}z_0+z^{n-3}z_0^2+\dots+z^{n-1}
}_{\displaystyle \text{$n$ Summanden}}.
\end{align*}
Lassen wir jetzt $z$ gegen $z_0$ gehen, wird die rechte Seite
zu $nz_0^{n-1}$.
\end{beispiel}

\begin{beispiel}
Die Funktion $z\mapsto f(z)=\bar z=x-iy$ ist nicht differenzierbar.
Wenn $f(z)=\bar z$ differenzierbar w"are, dann m"usste es eine Zahl
$a\in\mathbb C$ geben, so dass 
\[
\bar z-\bar z_0=a(z-z_0)+o(z-z_0)
\]
gilt.
w"ahlen wir $z=z_0+x$ bzw.~$z=z_0+iy$, dann erhalten wir
\[
\begin{aligned}
z-z_0&=x:&
\bar z-\bar z_0&=x
&&\Rightarrow&
\bar z-\bar z_0&=1\cdot x
&&\Rightarrow&
a&=1
\\
z-z_0&=iy:&
\bar z-\bar z_0&=-iy
&&\Rightarrow&
\bar z-\bar z_0&=-1\cdot iy
&&\Rightarrow&
a&=-1
\end{aligned}
\]
Es ist also nicht m"oglich, eine einzige Zahl $a$ zu finden, die als
die Ableitung der Funktion $z\mapsto \bar z$ betrachtet werden k"onnte.
\end{beispiel}

Das letzte Beispiel zeigt, dass
selbst Funktionen, deren Real- und Imagin"arteil beliebig oft stetig
differenzierbare Funktionen sind, nicht komplex differenzierbar
sein m"ussen.
Komplexe Differenzierbarkeit ist eine wesentlich st"arkere Bedingung
an eine Funktion, komplex differenzierbare Funktionen bilden eine
echte Teilmenge aller Funktionen, deren Real- und Imagin"arteil
differenzierbar ist.

%
% Cauchy-Riemann-Differentialgleichungen
%
\subsection{Die Cauchy-Riemann-Differentialgleichungen}
Komplexe Funktionen k"onnen nur differenzierbar sein, wenn sich die vier
partiellen Ableitungen zu einer einzigen komplexen Zahl zusammenfassen
lassen.
Um diese Beziehung zu finden, gehen wir von einer komplexen Funktion
\[
f(x+iy) = u(x,y) + iv(x,y)
\]
aus, und berechnen die Ableitung auf zwei verschiedene Arten, indem
wir sowohl nach $x$ als auch nach $iy$ ableiten:
\begin{align*}
f'(z)&
=
\lim_{x\to 0}\frac{f(z+x)-f(z)}{x}
=
\frac{\partial u}{\partial x}+i\frac{\partial v}{\partial x}
\\
f'(z)&
=
\lim_{y\to 0}\frac{f(z+iy)-f(z)}{iy}
=
\frac1{i}
\frac{\partial u}{\partial y}+\frac{\partial v}{\partial y}
=
\frac{\partial v}{\partial y}
-i
\frac{\partial u}{\partial y}.
\end{align*}
Dies ist nur m"oglich, wenn Real- und Imagin"arteile "ubereinstimmen.
Es folgt also

\begin{satz}
\label{komplex:satz:cauchy-riemann}
Real- und Imagin"arteil $u(x,y)$ und $v(x,y)$ einer
komplex differenzierbaren Funktion $f(z)$ mit $f(x+iy)=u(x,y)+iv(x,y)$
erf"ullen die Cauchy-Riemannschen Differentialgleichungen
\index{Cauchy-Riemann-Differentialgleichungen}
\begin{equation}
\begin{aligned}
\frac{\partial u}{\partial x}
&=
\frac{\partial v}{\partial y},
&
\frac{\partial u}{\partial y}
&=
-
\frac{\partial v}{\partial x}.
\end{aligned}
\label{komplex:dgl:cauchy-riemann}
\end{equation}
\end{satz}

Leitet man die Cauchy-Riemann-Differentialgleichungen nochmals nach
$x$ und $y$ ab, erh"alt man
\begin{equation*}
\begin{aligned}
\frac{\partial^2 u}{\partial x^2}
&=
\frac{\partial^2 v}{\partial x\,\partial y},
&
\frac{\partial^2 u}{\partial x\,\partial y}
&=
-\frac{\partial^2 v}{\partial x^2},
&
\frac{\partial^2 u}{\partial y\,\partial x}
&=
\frac{\partial^2 v}{\partial y^2},
&
\frac{\partial^2 u}{\partial y^2}
&=
-\frac{\partial^2 v}{\partial y\,\partial x}.
\end{aligned}
\end{equation*}
Die erste und die letzte sowie die mittleren zwei k"onnen zu jeweils
einer Differentialgleichung f"ur die Funktionen $u$ und $v$ zusammengefasst
werden, n"amlich
\begin{equation*}
\frac{\partial^2 u}{\partial x^2}
+
\frac{\partial^2 u}{\partial y^2}
=
0
\qquad\text{und}\qquad
\frac{\partial^2 v}{\partial x^2}
+
\frac{\partial^2 v}{\partial y^2}
=
0.
\end{equation*}

\begin{definition}
Der Operator 
\[
\Delta =
\frac{\partial^2}{\partial x^2}
+
\frac{\partial^2}{\partial y^2}
\]
heisst der {\em Laplace-Operator} in zwei Dimensionen.
\index{Laplace-Operator}%
\end{definition}

\begin{definition}
Eine Funktion $h(x,y)$ von zwei Variablen heisst {\em harmonisch}, wenn sie
die Gleichung
\[
\Delta h=0
\]
erf"ullt.
\index{harmonische Funktion}%
\index{harmonisch}%
\end{definition}

\begin{satz}
Real- und Imagin"arteil einer komplexen Funktion sind harmonische Funktionen.
\end{satz}

Die Cauchy-Riemann-Differentialgleichungen schr"anken also einerseits stark
ein, welche Funktionen "uberhaupt als Real- und Imagin"arteil einer
komplex differenzierbaren Funktion in Frage kommen.
Andererseits koppeln sie auch Real- und Imagin"arteil stark zusammen.

\begin{beispiel}
Von einer komplex differenzierbaren Funktion $f(z)$ sei nur der Realteil
$u(x,y)=x^3 -3xy^2$ bekannt.
Man finde alle m"oglichen Funktionen $f(z)$.

Zun"achst kontrollieren wir, ob dies "uberhaupt ein Realteil sein kann,
indem wir nachrechnen, ob $u(x,y)$ harmonisch ist.
\begin{equation*}
\begin{aligned}
\frac{\partial u}{\partial x}
&=
3x^2-3y^2
&&\Rightarrow&
\frac{\partial^2 u}{\partial x^2}
&=
6x
\\
\frac{\partial u}{\partial y}
&=
-6xy
&&\Rightarrow&
\frac{\partial^2 u}{\partial y^2}
&=
-6x
\\
&&&&\Delta u&=\frac{\partial^2u}{\partial x^2}+\frac{\partial^2u}{\partial y^2}=6x-6x=0,
\end{aligned}
\end{equation*}
$u$ ist also harmonisch.

Um die Funktion $f$ zu finden, brauchen wir jetzt noch den Imagin"arteil.
Wir finden ihn mit Hilfe der Cauchy-Riemann-Differentialgleichungen.
Es gilt
\begin{equation}
\begin{aligned}
\frac{\partial v}{\partial x}
&=
-\frac{\partial u}{\partial y}=6xy,
&
\frac{\partial v}{\partial y}
&=
\frac{\partial u}{\partial x}=3x^2-3y^2
\end{aligned}
\label{komplex:crbeispiel}
\end{equation}
Aus der ersten Gleichung erh"alt man durch Integrieren nach $x$ 
\[
v(x,y)=-3x^2y + C(y),
\]
die Integrations-``Konstante'' ist eine Funktion, die aber nur von $y$
abh"angen darf.
Die zweite Cauchy-Riemann-Gleichung verwendet die Ableitung von $v$ nach $y$,
sie ist
\[
\frac{\partial v}{\partial y}=3x^2+C'(y).
\]
Aus der zweiten Gleichung von \eqref{komplex:crbeispiel} liest man
ab, dass
\[
C'(y)=-3y^2
\qquad\Rightarrow\qquad
C(y)=-y^3+k
\]
sein muss.
Damit ist $v$ bis auf eine Konstante bestimmt.
Die zugeh"orige Funktion $f(z)$ ist daher
\[
f(z)=f(x+iy)=x^3-3xy^2+i(3x^2y-y^3)+ik
=x^3 + 3x^2iy + 3x(iy)^2+(iy)^3+ik=z^3+ik.
\]
Wir haben die Funktion $f(z)$ bis auf eine Konstanten $ik$ 
aus ihrem Realteil rekonstruiert.
\end{beispiel}

Die Cauchy-Riemann-Differentialgleichungen besagen auch, dass man nur
die Ableitungen nach $x$ zu berechnen braucht, um die Ableitung $f'(x)$
zu bestimmen.
Die Rechenregeln f"ur die Ableitung lassen sich daher direkt auf
komplexe Funktionen "ubertragen:
\begin{align*}
\frac{d}{dz}z^n
&=
nz^{n-1}
\\
\frac{d}{dz}e^z
&=
e^z
\\
\frac{d}{dz}f(g(z))
&=
f'(g(z)) g'(z)
\\
\frac{d}{dz}\bigl(f(z)g(z)\bigr)
&=
f'(z)g(z)+f(z)g'(z)
\end{align*}

%
% Analytische Funktionen
%
\subsection{Analytische Funktionen}
Als wichtiges Beispiel komplex differenzierbarer Funktionen betrachten
wir die analytischen Funktionen.
\begin{definition}
Eine Funktion $f\colon\mathbb C\to\mathbb C$ heisst analytisch im Punkt
\index{analytisch}%
$z_0$, wenn sie in eine konvergente Potenzreihe
\begin{equation}
f(z)=\sum_{k=0}^\infty a_k(z-z_0)^k
\label{komplex:freihe}
\end{equation}
entwickelt werden kann.
\end{definition}

Da die Ableitungsregeln f"ur komplex differenzierbare Funktionen nicht
anders sind als die Ableitungsregeln f"ur relle Funktionen, muss gelten
\begin{equation}
f'(z)=\sum_{k=1}^\infty ka_k(z-z_0)^{k-1},
\label{komplex:fpreihe}
\end{equation}
und es stellt sich nur die Frage, ob diese Potenzreihe ebenfalls konvergent
ist.
Man kann dies mit der Formel f"ur den Konvergenzradius pr"ufen.
\index{Konvergenzradius}%
F"ur die Potenzreihe \eqref{komplex:freihe} f"ur $f(z)$ liefert sie den
Konvergenzradius
\[
\frac{1}{\varrho} = \limsup_{k\to\infty} \root{k}\of{|a_k|}.
\]
Dieselbe Formel f"ur die Reihe~\eqref{komplex:fpreihe} liefert
f"ur den Konvergenzradius der Reihenentwicklung der Ableitung $f'(z)$
\[
\limsup_{k\to\infty} \root{k}\of{k|a_k|}
=
\underbrace{\lim_{k\to\infty} \root{k}\of{k}}_{\textstyle=1}
\cdot
\underbrace{ \limsup_{k\to\infty} \root{k}\of{|a_k|}}_{\textstyle=1/\varrho}
=
\frac1{\varrho}.
\]
Die Reihe f"ur die Ableitung $f'(z)$ hat also den gleichen Konvergenzradius
wie die Reihe f"ur die Funktion $f(z)$, dies gilt nat"urlich auch f"ur
die h"oheren Ableitungen.

Eine analytische Funktion ist somit beliebig oft komplex differenzierbar.
Aus den Ableitungen kann wie bei reellen Funktionen die Taylor-Reihe
gebildet werden, sie muss mit der Potenzreihe \eqref{komplex:freihe}
"ubereinstimmen.
\index{Taylor-Reihe}%
Analytische Funktionen haben also eine konvergente Taylor-Reihe,
\[
f(z) = \sum_{k=0}^\infty \frac{f^{(k)}(z_0}{k!}(z-z_0)^k.
\]

%
% Wegintegrale und die Cauchy-Formel
%
\subsection{Wegintegrale\label{subsection:wegintegrale}}
Das Finden einer Stammfunktion, die Integration, ist die Grundtechnik,
\index{Stammfunktion}%
mit der man den "Ubergang von lokaler Information in Form von Ableitungen,
zu globaler Information "uber reelle Funktionen vollzieht.
Sie liefert aus der Steigung zwischen zwei Punkten $x_0$ und $x$ den
Funktionswert mittels
\[
f(x)=f(x_0)+\int_{x_0}^xf'(\xi)\,d\xi.
\]
Bei einer reellen Funktion gibt es nur eine Richtung, entlang der man
integrieren k"onnte.

Auch in der komplexen Ebene erwarten wir eine Formel
\[
f(z) = f(z_0) + \int_{z_0}^z f'(\zeta)\,d\zeta.
\]
In der komplexen Ebene gibt es aber beliebig viele Wege, mit denen die
Punkte $z_0$ und $z$ verbunden werden k"onnen. 
Der Wert von $f(z)$ muss also durch Integration entlang eines speziell
gew"ahlten Weges $\gamma$
\[
f(z) = f(z_0) + \int_{\gamma} f'(\zeta)\,d\zeta
\]
bestimmt werden.
Es muss also zun"achst gekl"art werden, wie ein solches Wegintegral
"uberhaupt zu verstehen und zu berechnen ist.
Dann gilt es zu untersuchen, inwieweit diese Konstruktion unabh"angig
von der Wahl des Weges ist.
F"ur komplex differenzierbare Funktionen wird sich eine sehr erfolgreiche
Theorie ergeben.

%
% Wegintegrale
%
\subsubsection{Definition des Wegintegrals}
Ein Weg in der komplexen Ebene ist eine Abbildung 
\index{Abbildung}%
\[
\gamma\colon [a,b]\to\mathbb C: t\mapsto \gamma(t).
\]
Wir verlangen f"ur unsere Zwecke zus"atzlich, dass $\gamma$ differenzierbar
ist.
Dann k"onnen wir f"ur jede beliebige Funktion das Wegintegral definieren.

\begin{definition}
Sei $\gamma\colon[a,b]\to\mathbb C$ ein Weg in $\mathbb C$ und $f(z)$
eine stetige komplexe Funktion, dann heisst
\[
\int_{\gamma} f(z)\,dz = \int_a^bf(\gamma(t)) \gamma'(t)\,dt
\]
das {\em Wegintegral} von $f(z)$ entlang der Kurve $\gamma$.
\index{Wegintegral}
\end{definition}

\begin{beispiel}
Man berechne das Wegintegral der Funktion $f(z)=z^n$ entlang des
Weges 
$\gamma(t)=1+t+it^2$
f"ur $t\in[0,1]$.

Die Definition besagt
\begin{align*}
\int_\gamma f(z)\,dz
&=
\int_0^1 f(\gamma(t))\gamma'(t)\,dt
=
\int_0^1 \gamma(t)^n \gamma'(t)\,dt
=
\int_0^1 \frac{d}{dt}\frac{\gamma(t)^{n+1}}{n+1}\,dt
\\
&=
\biggl[\frac{\gamma(t)^{n+1}}{n+1}\biggr]_0^1
=
\frac{(2+i)^{n+1}}{n+1}-\frac{1^{n+1}}{n+1}
=
\frac{(2+i)^{n+1}-1}{n+1}.
\end{align*}
Man stellt in diesem Beispiel auch fest, dass das Integral offenbar
unabh"angig ist von der Wahl des Weges, es kommt einzig auf die
beiden Endpunkte an:
\[
\int_\gamma z^n \,dz = \frac1{n+1}\bigl(\gamma(1)^{n+1}-\gamma(0)^{n+1}\bigr).
\]
\end{beispiel}

\begin{beispiel}
Wir berechnen als Beispiel das Wegintegral der Funktion $f(z)=1/z$ entlang
eines Halbkreises von $1$ zu $-1$. 
Es gibt zwei verschiedene solche Halbkreise:
\begin{equation*}
\begin{aligned}
\gamma_+(t)&=e^{it},&t&\in[0,\pi]
\\
\gamma_-(t)&=e^{-it},&t&\in[0,\pi]
\end{aligned}
\end{equation*}
Wir finden f"ur die Wegintegrale
\begin{align*}
\int_{\gamma_+}\frac1z\,dz
&=
\int_0^\pi \frac1{e^{it}}ie^{it}\,dt=i\int_0^\pi\,dt=i\pi
\\
\int_{\gamma_-}\frac1z\,dz
&=
-\int_0^\pi \frac1{e^{-it}}ie^{-it}\,dt=-i\int_0^\pi\,dt=-i\pi
\end{align*}
Das Wegintegral zwischen $1$ und $-1$ h"angt also mindestens f"ur diese
spezielle Funktion $f(z)=1/z$ von der Wahl des Weges ab.
\end{beispiel}

Wie Wahl der Parametrisierung der Kurve hat keinen Einfluss auf den 
Wert des Wegintegrals.

\begin{satz}
Seien $\gamma_1(t), t\in[a,b],$ und $\gamma_2(s),s\in[c,d]$
verschiedene Parametrisierungen
\index{Parametrisierung}%
der gleichen Kurve, es gebe also eine Funktion $t(s)$ derart, dass
$\gamma_1(t(s))=\gamma_2(s)$.
Dann ist
\[
\int_{\gamma_1}f(z)\,dz
=
\int_{\gamma_2}f(z)\,dz.
\]
\end{satz}

\begin{proof}[Beweis]
Wir verwenden die Definition des Wegintegrals
\begin{align*}
\int_{\gamma_1} f(z)\,dz
&=
\int_a^b f(\gamma_1(t))\,\gamma_1'(t)\,dt
=
\int_c^d f(\gamma_1(t(s))\,\underbrace{\gamma_1'(t(s)) t'(s)}_{\displaystyle
=\frac{d}{ds}\gamma_1(t(s))}\,ds
\\
&=
\int_c^d f(\gamma_2(s)\,\gamma_2'(s)\,ds
=
\int_{\gamma_2}f(z)\,dz
\end{align*}
Beim zweiten Gleichheitszeichen haben wir die Formel f"ur die
Variablentransformation $t=t(s)$ in einem Integral verwendet.
\index{Variablentransformation}%
\end{proof}

Wir erwarten, dass das Wegintegral "ahnlich wie das Integral reeller
Funktionen eine Art ``Umkehroperation'' zur Ableitung ist.
Wir untersuchen daher den Fall, dass $f(z)$ eine komplexe Stammfunktion $F(z)$
hat, also $f(z)=F'(z)$.
Wir berechnen das Wegintegral entlang des Weges $\gamma$:
\begin{align*}
\int_{\gamma}f(z)\,dz
&=
\int_a^bf(\gamma(t))\,\gamma'(t)\,dt
=
\int_a^bF'(\gamma(t))\,\gamma'(t)\,dt
=
\int_a^b\frac{d}{dt}F(\gamma(t))\,dt
=
F(\gamma(a))-F(\gamma(b))
\end{align*}
Dies ist genau die Formel, die man als den Hauptsatz der Infinitesimalrechnung
kennt.
Trotzdem ist die Situation hier etwas anders.
In der reellen Infinitesimalrechnung war die Existenz einer Stammfunktion
durch das Integral gesichert, man konnte mit
\[
F(x)=\int_a^xf(\xi)\,d\xi
\]
immer eine Stammfunktion angeben.
Im komplexen Fall k"onnen wir nat"urlich auch versuchen, eine Stammfunktion
mit Hilfe von 
\[
F(z)=\int_{\gamma_z} f(\zeta)\,d\zeta
\]
zu definieren.
Dabei muss allerdings $\gamma_z$ ein Weg sein, der im Punkt $z$ endet,
und wir wissen noch nicht einmal, ob die Wahl des Weges eine Rolle
spielt.
Bevor wir also sicher sein k"onnen, dass eine Stammfunktion existiert,
m"ussen wir zeigen, dass das Wegintegral einer komplex differenzierbaren
Funktion zwischen zwei Punkten nicht von der Wahl des Weges abh"angt,
der die beiden Punkte verbindet.
Dazu ist notwendig, geschlossene Wege genauer zu betrachten.

%
% Wegintegrale führen auf analytische Funktionen
%
\subsubsection{Wegintegrale f"uhren auf analytische Funktionen}
\begin{figure}
\centering
\includegraphics{chapters/images/komplex-4.pdf}
\caption{Pfad und Konvergenzradius f"ur den Nachweis, dass Wegintegrale
auf analytische Funktionen f"uhren (Satz~\ref{komplex:integralanalytisch})
\label{komplex:integralanalytischpfad}}
\end{figure}
Mit Wegintegralen kann man aus stetigen Funktionen neue Funktionen
konstruieren.
Die folgende Konstruktion liefert "uberraschenderweise immer
analytische Funktionen.
\begin{satz}
\label{komplex:integralanalytisch}
Sei $\gamma\colon [a,b]\to\mathbb C$ ein Weg in $\mathbb C$, der nicht
durch den Nullpunkt verl"auft, und $g$ eine stetige Funktion
auf $\gamma([a,b])$ (Abbildung~\ref{komplex:integralanalytischpfad}).
Dann ist die Funktion
\[
f(z) = \frac1{2\pi i}\int_\gamma \frac{g(x)}{x-z}\,dx
\]
in einer Umgebung des Nullpunktes analytisch:
\[
f(z) = \sum_{k=0}^\infty c_k z^k,\qquad
\text{mit\quad}
c_k=\frac1{2\pi i}\int_\gamma \frac{g(x)}{x^{k+1}}\,dx.
\]
Der Konvergenzradius $\varrho$ dieser Reihe ist der minimale Abstand der
Kurve $\gamma$ vom Nullpunkt.
\end{satz}

\begin{proof}[Beweis]
Zun"achst schreiben wir
\begin{equation}
\frac{1}{x-z}
=
\frac1x\cdot \frac{1}{1-\displaystyle\frac{z}{x}}
=
\frac1x\cdot \sum_{k=0}^\infty \biggl(\frac{z}{x}\biggr)^k
=
\sum_{k=0}^\infty \frac{z^k}{x^{k+1}}.
\label{komplex:georeihe}
\end{equation}
Damit k"onnen wir jetzt die Funktion $f(z)$ berechnen:
\begin{align*}
f(z)
&=
\frac1{2\pi i} \int_{\gamma} \frac{g(x)}{x-z}\,dx
=
\frac1{2\pi i} \int_{\gamma} \sum_{k=0}^\infty \frac{z^k}{x^{k+1}}g(x)\,dx
=
\sum_{k=0}^\infty
\underbrace{\biggl(\frac1{2\pi i} \int_{\gamma} \frac{g(x)}{x^{k+1}}\,dx\biggr)}_{\textstyle =c_k}
z^k
=
\sum_{k=0}^\infty c_kz^k.
\end{align*}
Wir m"ussen uns noch die Konvergenz dieser Reihen "uberlegen.
Wenn $z<\varrho$ ist, dann ist 
\[
\biggl|\frac{z}{x}\biggr| 
=
\frac{|z|}{|x|}
<1,
\]
so dass die geometrische Reihe \eqref{komplex:georeihe} konvergent ist,
daraus lesen wir ab, dass der Konvergenzradius mindestens $\varrho$
ist.
Gr"osser kann er allerdings auch nicht sein, da f"ur $|z|\ge \varrho$
das Integral nicht mehr definiert sein muss.
Nimmt man n"amlich einen Punkt von $g([a,b])$ f"ur $z$ wird der Integrand
unendlich gross.
\end{proof}

Der Satz~\ref{komplex:integralanalytisch} ist nur f"ur Potenzreihen
im Punkt $0$ formuliert, was im Wesentlichen durch die
Umformung~\eqref{komplex:georeihe} bedingt war.
Man kann dies aber auch als Potenzreihe
\[
\frac1{x-z}
=
\frac1{x-z_0-(z-z_0)}
=
\frac1{x-z_0}\cdot\frac1{1-\displaystyle\frac{z-z_0}{x-z_0}}
=
\frac1{x-z_0}\sum_{k=0}^\infty\biggl(\frac{z-z_0}{x-z_0}\biggr)^k
=
\sum_{k=0}^\infty\frac1{(x-z_0)^{k+1}}(z-z_0)^k
\]
im Punkt $z_0$ ausdr"ucken.
Man bekommt dann die Potenzreihe
\[
f(z) = \sum_{k=1}^\infty c_k(z-z_0)^k,\qquad
\text{mit}\quad
c_k=\frac1{2\pi i}\oint_\gamma\frac{g(x)}{(x-z_0)^{k+1}}\,dx
\]
f"ur das Wegintegral.

\subsubsection{Laurent-Reihen}
\label{sssec:LaurentReihen}
\begin{figure}
\centering
\includegraphics{chapters/images/komplex-3.pdf}
\caption{Pfad zur Herleitung der Laurent-Reihe einer Funktion $f(z)$
mit einer Singularit"at $z_0$.
\label{komplex:laurentpfad}}
\end{figure}%
\index{Laurent-Reihe}%
In Satz~\ref{komplex:integralanalytisch} konnten wir eine Potenzreihe f"ur
solche $z$ konstruieren, deren Betrag kleiner ist als der kleinste Abstand
der Kurve $\gamma$ vom Ursprung.
Dies war notwendig, weil in~\eqref{komplex:georeihe} die geometrische Reihe
nur konvergiert, wenn der Quotient $<1$ ist.
Wenn die Funktion $f(z)$ jedoch eine Singularit"at im Punkt $z_0$ hat, dann
kann es nicht m"oglich sein, die Funktion mit einer Potenzreihe zu
beschreiben.

Wir verwenden daher den speziellen Pfad in Abbildung~\ref{komplex:laurentpfad}.
Er f"uhrt in einem grossen Kreis $\gamma_1$ um den Punkt $z_0$ herum,
dann folgt ein zur $x$-Achse paralleler Abschnitt, der bis zum kleinen
Kreis $\gamma_2$ f"uhrt.
Nach Durchlaufen des kleinen Kreises $\gamma_2$ im Uhrzeigersinn folgt wieder
ein zur $x$-Achse paralleles St"uck zur"uck zum grossen Kreis.
Da die geraden St"ucke zweimal in entgegegengesetzer Richtung durchlaufen
werden, heben sie sich weg.
Ein Wegintegral entlang $\gamma$ zerf"allt daher in eine Differenz
\[
\oint_\gamma\dots\,dz
=
\oint_{\gamma_1}\dots\,dz
-
\oint_{\gamma_2}\dots\,dz
\]
von Wegintegralen entlang $\gamma_1$ und $\gamma_2$.

Der "aussere Pfad $\gamma_1$ gibt wie in Satz~\ref{komplex:integralanalytisch}
Anlass zu einer Potenzreihe in $(z-z_0)$.
Der innere Pfad $\gamma_2$ kann aber nicht so behandelt werden, da $z$ immer
weiter von $z_0$ entfernt als die Punkte auf $\gamma_2$.
Allerdings ist $|x/z| < 1$ f"ur Punkte auf $\gamma_2$, wir m"ussen daher
die geometrische Reihe auf $x/z$ anwenden:
\begin{align*}
\frac{1}{x-z}
&=
\frac{1}{x-z_0-(z-z_0)}
=
\frac{1}{z-z_0}
\cdot
\frac{1}{\displaystyle\frac{x-z_0}{z-z_0}-1}
=
-\sum_{k=0}^\infty \frac{(x-z_0)^k}{(z-z_0)^{k+1}}.
\end{align*}
Das Integral entlang der Kurve $\gamma_2$ kann also als Reihe in $1/(z-z_0)$
entwickelt werden:
\begin{align*}
f_2(z)
&=
\frac{1}{2\pi i}\int_{\gamma_2} \frac{g(x)}{x-z}\,dx
=
\frac{1}{2\pi i}\int_{\gamma_2}\sum_{k=0}^\infty
\frac{(x-z_0)^k}{(z-z_0)^{k+1}}\,dx
\\
&=
\sum_{k=0}^\infty
\biggl(
\underbrace{\frac1{2\pi i}\int_{\gamma_2} (x-z_0)^kg(x)\,dx
}_{\textstyle =d_{k+1}}
\biggr)
\frac1{(z-z_0)^{k+1}}
=\sum_{k=1}^\infty \frac{d_k}{(z-z_0)^k}.
\end{align*}
Zusammen mit der vom Integral entlang $\gamma_1$ herr"uhrenden Reihe finden
wir den Satz
\begin{satz}
\label{komplex:laurentreihe}
Ist $g(z)$ eine entlang der Kurve $\gamma$ wie in
Abbildung~\ref{komplex:laurentpfad} definierte stetige Funktion, dann gilt
\[
f(z)=\frac1{2\pi i}\oint_{\gamma} \frac{f(x)}{x-z}\,dx
=
\sum_{k=0}^{\infty} c_k(z-z_0)^k-\sum_{k=1}^\infty \frac{d_k}{(z-z_0)^k},
\]
wobei die Koeffizienten $c_k$ und $d_k$ gegeben sind durch
\[
\begin{aligned}
c_k&=\frac1{2\pi i}\oint_{\gamma_1} \frac{g(x)}{x-z_0}\,dx
&&
\text{und}
&
d_k&=\frac1{2\pi i}\oint_{\gamma_2} g(x)x^{k-1}\,dx.
\end{aligned}
\]
\end{satz}

\begin{definition}
Eine Reihe der Form
\[
\sum_{k=-\infty}^\infty a_k(z-z_0)^k
\]
heisst {\em Laurent-Reihe }
im Punkt $z_0$.
\end{definition}


%
% Geschlossene Wege
%
\subsubsection{Geschlossene Wege}
\begin{definition}
Ein Weg $\gamma\colon[a,b]\to\mathbb C$ heisst {\em geschlossen}, wenn
$\gamma(a)=\gamma(b)$.
\index{geschlossener Weg}
Das Integral entlang eines geschlossenen Weges h"angt nicht von der
Parametrisierung ab und wird zur Verdeutlichung mit
\[
\int_{\gamma}f(z)\,dz
=
\oint_{\gamma}f(z)\,dz
\]
bezeichnet.
\end{definition}

\begin{beispiel}
Wir berechnen das Integral von $f(z)=z^n$ entlang des Einheitskreises,
den wir mit $\gamma(t)=e^{it},t\in[0,2\pi]$ parametrisieren.
Die Definition liefert:
\begin{align*}
\oint_{\gamma}f(z)\,dz
&=
\int_0^{2\pi}e^{int}ie^{it}\,dt
=
i\int_0^{2\pi}e^{i(n+1)t}\,dt
\end{align*}
F"ur $n=-1$ ist dies das Integral einer konstanten Funktion, also
\[
\oint_{\gamma}\frac1z\,dz=2\pi i.
\]
F"ur $n\ne -1$ kann man eine Stammfunktion von $e^{i(n+1)t}$
verwenden:
\[
\oint_{\gamma}f(z)\,dz
=
i\left[\frac1{i(n+1)}e^{i(n+1)t}\right]_0^{2\pi}
=0,
\]
weil $e^{i(n+1)t}$ periodisch ist mit Periode $2\pi$.
\end{beispiel}
Das Beispiel zeigt, dass ein Wegintegral der Potenzfunktionen,
aller Polynome und schliesslich aller konvergenten Potenzreihen
"uber einen geschlossenen Weg verschwinden.
Es zeigt aber auch, dass das Wegintegral "uber einen geschlossenen
Weg nicht zu verschwinden braucht, wie das Beispiel $f(z)=1/z$ 
zeigt.
Letztere Funktion unterscheidet sich von den Potenzfunktionen allerdings
dadurch, dass sie im Nullpunkt nicht definiert ist.

\begin{satz}
Sei $f(z)$ eine in einem zusammenh"angenden Gebiet $\Omega\subset\mathbb C$
definierte komplexe Funktion, f"ur die das Wegintegral "uber jeden
geschlossenen Weg verschwindet.
Dann hat $f(z)$ eine komplexe Stammfunktion $F(z)$.
\end{satz}

\begin{proof}[Beweis]
Wir w"ahlen einen beliebigen Punkt $z_0\in\Omega$ definieren die
komplexe Stammfunktion mit Hilfe des Wegintegrals
\[
F(z)=\int_{\gamma_z} f(\zeta)\,d\zeta,
\]
wobei $\gamma_z$ ein beliebiger Weg ist, der $z_0$ mit $z$ verbindet.

Wir m"ussen uns davon "uberzeugen, dass die Wahl des Weges keinen Einfluss
auf $F(z)$ hat.
Dazu seien $\gamma_1$ und $\gamma_2$ zwei verschiedene Wege, die
$z_0$ mit $z$ verbinden.
Da die Parametrisierung der Wege keinen Einfluss auf das Wegintegral haben,
nehmen wir an, dass beide Wege auf dem Intervall $[0,1]$ definiert sind.

Jetzt konstruieren wir einen geschlossene Weg $\gamma$ durch die
Definition:
\[
\gamma\colon[0,2]\to\mathbb C:t\mapsto
\begin{cases}
\gamma_1(t)&\qquad 0\le t\le 1\\
\gamma_2(2-t)&\qquad 1\le t\le 2
\end{cases}
\]
Der Weg $\gamma$ besteht aus $\gamma_1$ und dem in umgekehrter Richtung
durchlaufenen Weg $\gamma_2$, denn an der Stelle $t=1$ passen die
beiden Teilwege nahtlos zusammen: $\gamma_1(1)=\gamma_2(1)=\gamma_2(2-1)$.
Wegen $\gamma(2)=\gamma_2(2-2)=\gamma_2(0)=\gamma_1(0)$ ist der
Weg geschlossen.
Nach Voraussetzung ist verschwindet das Wegintegral "uber $\gamma$.
Es folgt
\begin{align*}
0
&=
\int_{\gamma}f(z)\,dz
\\
&=
\int_0^1 f(\gamma_1(t))\gamma_1'(t)\,dt
+ \int_1^2f(\gamma_2(2-t))\frac{d}{dt}\gamma_2(2-t)\,dt
\\
&=
\int_0^1 f(\gamma_1(t))\gamma_1'(t)\,dt
- \int_1^2f(\gamma_2(2-t))\gamma_2'(2-t)\,dt
\\
&=
\int_0^1 f(\gamma_1(t))\gamma_1'(t)\,dt
- \int_0^1f(\gamma_2(s))\gamma_2'(s)\,ds
\\
&=
\int_{\gamma_1}f(z)\,dz - \int_{\gamma_2}f(z)\,dz
\\
\Rightarrow\qquad
\int_{\gamma_2}f(z)\,dz&=\int_{\gamma_1}f(z)\,dz.
\end{align*}
Da die Wahl des Weges keine Rolle spielt, ist $F(z)$ wohldefiniert.
\end{proof}

Die Bedingung des eben bewiesenen Satzes ist nicht wirklich n"utzlich,
sie ist kaum nachpr"ufbar.
Es braucht also zus"atzliche Anstrengungen um gen"ugend viele
Funktionen zu finden, welche die Eigenschaft haben, dass Wegintegrale
"uber geschlossene Wege verschwinden.
Wir zielen dabei auf den folgenden Satz hin:
\begin{satz}[Cauchy]
Ist $f(z)$ eine in einem Gebiet $\Omega\subset\mathbb C$ definierte
komplex differenzierbare Funktion, und ist $\gamma$ ein im Gebiet
$\Omega$ auf einen Punkt zusammenziehbarer geschlossener Weg, dann gilt
\[
\oint_{\gamma}f(z)\,dz=0.
\]
Ist insbesondere $\Omega$ {\em einfach zusammenh"angend}
\index{einfach zusammenhangend@einfach zusammenh\"angend}%
\index{zusammenziehbar}%
(d.~h.~jeder geschlossene Weg l"asst sich in einen Punkt zusammenziehen),
dann verschwindet das Wegintegral von $f(z)$ "uber jeden geschlossenen
Weg in $\Omega$.
\index{einfach zusammenhangend@einfach zusammenh\"angend}
\end{satz}

\begin{proof}[Beweis]
Wir verwenden f"ur den folgenden Beweis den Satz von Green "uber
\index{Green, Satz von}%
Wegintegrale in der Ebene.
Er besagt, dass f"ur einen geschlossenen Weg $\gamma$ der in der Ebene
das Gebiet $D$ berandet, und zwei Funktionen $L(x,y)$ und $M(x,y)$, gilt
\[
\oint_\gamma(L\,dx + M\,dy)
=
\int_D \biggl(\frac{\partial M}{\partial x}
-\frac{\partial L}{\partial y}\biggr)\,dx\,dy.
\]
Wir berechnen jetzt das Integral einer komplex differenzierbaren Funktion
$f(z)$
\begin{align*}
\oint_\gamma f(z)\,dz
&=
\int (u(x,y)+iv(x,y))(\dot x(t)+i\dot y(t))\,dt
\\
&=
\int u(x,y)\dot x(t) -v(x,y)\dot y(t)\,dt
+
i \int u(x,y)\dot y(t)+v(x,y)\dot x(t)\,dt
\\
&=\oint_\gamma(u\,dx - v\,dy) + i\oint_\gamma(v\,dx + u\,dy)
\\
&=
\int_D
\underbrace{-\frac{\partial v}{\partial x}}_{\displaystyle=\frac{\partial u}{\partial y}}
-\frac{\partial u}{\partial y}
\,dx\,dy
+i
\int_D
\underbrace{\frac{\partial u}{\partial x}}_{\displaystyle=\frac{\partial v}{\partial y}}
-\frac{\partial v}{\partial y}\,dx\,dy
=0.
\end{align*}
Dabei haben wir auf der dritten Zeile den Satz von Green angewendet,
und auf der letzten Zeile die Cauchy-Riemann-Differentialgleichungen.
\end{proof}

\subsection{Beispiel: Airy-Differentialgleichung}
\label{komplex:airydgl}
\index{Airy-Differentialgleichung}%
\index{Differentialgleichung!Airy-}
\begin{figure}
\centering
\includegraphics{chapters/images/airy-1.pdf}
\caption{Kurven zur Berechnung der Airy-Funktionen
\label{komplex:airy}}
\end{figure}
Als Beispiel soll die Airy-Differentialgleichung
\[
y''(x)-xy(x)=0
\]
gel"ost werden.
Wir nehmen an, dass die Differentialgleichung mit der Laplace-Transformation
\index{Laplace-Transformation}%
gel"ost werden kann.
Allerdings verwenden wir nicht die Integration entlang der positiven
reellen Achse, wie das bei der traditionellen Laplace-Transformation
"ublich ist, sondern Integration eintlang eines beliebigen Weges in
der komplexen Ebene.
Wir nehmen also an, dass sich $y(x)$ schreiben l"asst als
\[
y(x)=\int_\gamma e^{xz}v(z)\,dz
\]
f"ur eine komplex differenzierbare Funktion $v(z)$ und eine Kurve
$\gamma\colon\mathbb R\to \mathbb C$.
Setzen wir diesen Ansatz in die Airy-Differentialgleichung ein,
erhalten wir
\begin{align*}
\int_\gamma z^2e^{xz}v(z)\,dz-\int_\gamma xe^{xz}v(z)\,dz&=0
\end{align*}
Das zweite Integral kann mit partieller Integration umgeformt werden,
man erh"alt
\begin{align*}
\int_\gamma xe^{xz}v(z)\,dz
&=
\int_{-\infty}^{\infty} xe^{x\gamma(t)}v(\gamma(t))\dot\gamma(t)\,dt
=
\left[e^{xz}v(z)\right]_{\gamma(-\infty)}^{\gamma(\infty)}
-\int_\gamma e^{xz}v'(z)\,dz
\end{align*}
Dieser Ausdruck ist nur dann sinnvoll, wenn der Weg $\gamma$ so gew"ahlt
wird, dass an den Endpunkte des Weges der Integrand beliebig klein wird,
dies wird sp"ater die Wahl des Weges einschr"anken.
Andererseits erhalten wir die Differentialgleichung erster Ordnung
\begin{align*}
\int_\gamma e^{xz}(z^2v(z)+v'(z))\,dz=0
\qquad
\Rightarrow
\qquad
v'(z)+z^2v(z)=0
\end{align*}
f"ur $v(z)$, die mit Separation gel"ost werden kann:
\[
\int \frac{dv}{v}=-\int z^2\,dz
\qquad\Rightarrow\qquad
v(z)=-\frac13z^3.
\]
Setzen wir dies in den Ansatz f"ur $y(x)$ ein, finden wir
\[
y(x)=\int_\gamma e^{xz-\frac13z^3}\,dz.
\]
Jetzt muss die Kurve $\gamma$ so gew"ahlt werden, dass das Integral
wohldefiniert ist. 
Dazu muss der Realteil des Exponenten f"ur jedes beliebige reelle $x$ 
gegen $-\infty$ gehen, wenn $z$ gegen unendlich geht.
In Abbildung~\ref{komplex:airy} sind die Gebiete grau eingezeichnet,
in denen $-\frac13z^3$ negativen Realteil hat.
Zul"assige Wege m"ussen daher ``Enden'' in den grauen Gebieten haben,
d.~h.~$\gamma(t)$ muss in diesen Gebieten gegen Unendlich gehen, wenn
$t\to\pm\infty$.
Damit sind im Wesentlichen die drei Kurven $\gamma_0$, $\gamma_+$ und
$\gamma_-$ aus Abbildung~\ref{komplex:airy} w"ahlbar.

F"ur $x\to\infty$ w"achst $e^{xz}$ im positiven grauen Teilgebiet
"uber alle Grenzen, die Kurven $\gamma_+$ und $\gamma_-$ ergeben daher
eine unbeschr"ankte L"osung.
Nur die Kurve $\gamma_0$ kann ein beschr"ankte L"osung der Airy-Gleichung
geben.
Man nennt
\[
\operatorname{Ai}(x)
=
\frac{1}{2\pi i}\int_{\gamma_0} e^{xz-\frac13z^3}\,dz
\]
die Airy-Funktion.
\index{Airy-Funktion}%
\index{Funktion!Airy-}%
\index{Ai}%
Auf die Wahl der Kurve $\gamma_0$ kommt es nicht an, solange sie in den
beiden grauen Gebieten links der imagin"aren Achse endet.
Deformiert man die Kurve $\gamma_0$ in die imagin"are Achse, erh"alt
man daher die folgende Integraldarstellung der Airy-Funktion:
\begin{align*}
\operatorname{Ai}(x)
&=
\frac1{2\pi i}\int_{-\infty}^{\infty} e^{xit-\frac13(it)^3}i\,dt
\\
&=\frac{1}{2\pi i}\int_{-\infty}^{\infty} ie^{i(xt+\frac13t^3)}\,dt
\\
&=
\frac{1}{2\pi i}\int_{-\infty}^{\infty}
i\cos\biggl(xt+\frac13t^3\biggr)-\sin\biggl(xt+\frac13t^3\biggr)\,dt
\\
&=\frac{1}{\pi}\int_0^{\infty}\cos\biggl(xt+\frac13t^3\biggr)\,dt.
\end{align*}
Dabei haben wir im letzten Schritt verwendet, dass das Integral der
ungeraden Funktion $\sin(xt+\frac13t^3)$ "uber ein symmetrisches
Interval verschwindet.

Weitere Information "uber die Airy-Funktionen sind in \cite{skript:airy}
zusammengefasst.

\subsection{Die Cauchy-Integralformel}
\index{Cauchy-Integralformel}%
Sei jetzt $f(z)$ eine komplex differenzierbare Funktion.
Dann ist auch die Funktion
\[
g(z)=\frac{f(z)}{z-a}
\]
komplex differenzierbar f"ur $z\ne a$.
Insbesondere ist der Wert des Wegintegrals von $g(z)$ entlang
eines geschlossenen Pfades um den Punkt $a$ unabh"angig von der Wahl
des Weges.
Zum Beispiel k"onnten wir das Wegintegral mit Hilfe eines kleinen Kreises
um $a$ mit Radius $r$ mit der Parametrisierung
\[
t\mapsto \gamma(t)=a+re^{it},\quad t\in[0,2\pi]
\]
berechnen.
Die Rechnung ergibt
\begin{align*}
\oint_\gamma \frac{f(z)}{z-a}\,dz
&=
\int_0^{2\pi} \frac{f(a+re^{it})}{re^{it}}ire^{it}\,dt
=
i\int_0^{2\pi} f(a+re^{it})\,dt
\end{align*}
Da $f(z)$ komplex differenzierbar ist, k"onnen wir $f(z)$ approximieren
durch $f(z)=f(a)+f'(a)(z-a)+o(z-a)$, also
\begin{align*}
\oint_{\gamma} \frac{f(z)}{z-a}\,dz
&=
i\int_0^{2\pi}f(a) + f'(a)re^{it}+o(r)\,dt
\\
&=
f(a)i\int_0^{2\pi}\,dt
+ irf'(a)\int_0^{2\pi} e^{it}\,dt + i\int_0^{2\pi}o(r)\,dt
\\
&=
2\pi i f(a) + irf'(a)\underbrace{\left[\frac1{i}e^{it}\right]_0^{2\pi}}_{=0}+o(r)
\\
&=2\pi i f(a)+o(r).
\end{align*}
Da das Wegintegral einer komplex differenzierbaren Funktion aber unabh"angig
vom Weg und damit vom Radius $r$ sein muss, folgt
\[
\oint_\gamma \frac{f(z)}{z-a}\,dz=2\pi i f(a).
\]
Wir haben damit den folgenden Satz bewiesen:

\begin{satz}[Cauchy]
Ist $\gamma$ ein geschlossener Weg in der komplexen Ebene, die ein
Gebiet umrandet, in dem die komplexe Funktion $f(z)$ komplex
differenzierbar ist, dann gilt
\[
f(a)=\frac{1}{2\pi i}\oint_{\gamma}\frac{f(z)}{z-a}\,dz.
\]
Insbesondere sind die Werte einer komplex differenzierbaren Funktion 
im Inneren eines Gebietes durch die Werte auf dem Rand bereits vollst"andig
bestimmt.
\end{satz}

\subsubsection{Ableitungen und Cauchy-Formel}
Sei $f(z)$ eine komplex differenzierbare Funktion, als Definitionsgebiet
nehmen wir der Einfachheit halber einen Kreis vom Radius $r$ um den Nullpunkt,
sein Rand ist die Kurve $\gamma$.
Durch Ableiten der Cachyschen Integralformel finden wir
\begin{align*}
f(z)
&=
\frac1{2\pi i}\oint_{\gamma}\frac{f(\zeta)}{\zeta-z}\,d\zeta
\\
f'(z)
&=
\frac1{2\pi i}\oint_{\gamma}\frac{f(\zeta)}{(\zeta-z)^2}\,d\zeta
\\
f'' (z)
&=
\frac1{2\pi i}\oint_{\gamma}2\frac{f(\zeta)}{(\zeta-z)^3}\,d\zeta
\\
f'''(z)
&=
\frac1{2\pi i}\oint_{\gamma}2\cdot 3\frac{f(\zeta)}{(\zeta-z)^4}\,d\zeta
\\
&\vdots
\\
f^{(k)}(z)
&=
\frac{k!}{2\pi i}\oint_{\gamma}\frac{f(\zeta)}{(\zeta-z)^{k+1}}\,d\zeta
\end{align*}
Es folgt

\begin{satz}
Eine komplex differenzierbare Funktion ist beliebig oft differenzierbar.
\end{satz}

\subsubsection{Komplex differenzierbare Funktionen sind analytisch}
Wir haben fr"uher gesehen, dass Wegintegrale auf analytische Funktionen
f"uhren.
Andererseits zeigt das Cauchy-Integral, dass komplex differenzierbare
Funktionen durch genau die Integrale bestimmt sind, die in den
Reihenentwicklungen in Satz~\ref{komplex:integralanalytisch} auftraten.
Diese Resultate k"onnen wir im folgenden Satz zusammenfassen.

\begin{satz}
Eine komplex differenzierbare Funktion $f(z)$, die in einer Kreisscheibe
vom Radius $r$ um den Punkt $z_0$ definiert ist, ist analytisch.
Ihre Potenzreihenentwicklung
\[
f(z)=\sum_{k=0}^na_k(z-z_0)^k
\]
hat die Koeffizienten
\[
a_k=\frac1{2\pi i}\int_{\gamma}\frac{f(z)}{(z-z_0)^{k+1}}\,dz,\quad
k\ge 0
\]
\end{satz}

\begin{proof}[Beweis]
Da $f$ komplex differenzierbar ist, gilt
\[
f(z)=\frac1{2\pi i}\oint_\gamma \frac{f(\zeta)}{\zeta-z}\,d\zeta.
\]
In Satz~\ref{komplex:integralanalytisch} wurde gezeigt, dass $f(z)$
analytisch ist, und dass die Koeffizienten der Potenzreihe von
der verlangten Form sind.
\end{proof}

F"ur eine komplexe Funktion, die im Punkt $z_0$ eine Singularit"at hat,
also in einer Umgebung von $z_0$ ohne den Punkt $z_0$ definiert ist,
k"onnen wir das Resultat aus Satz~\ref{komplex:laurentreihe} verwenden,
und zum folgenden analogen Resultat gelangen:

\begin{satz}
Eine komplex differenzierbare Funktion $f(z)$, die in einer Kreisscheibe
vom Radius $r$ um den Punkt $z_0$ mit Ausnahme des Punktes $z_0$
definiert ist, kann in eine konvergente Laurent-Reihe
\[
f(z)=\sum_{k=-\infty}^{\infty} c_k(z-z_0)^k
\]
entwickelt werden, deren Koeffizienten durch
\[
c_k = \frac1{2\pi i}\oint_\gamma \frac{f(\zeta)}{(z-z_0)^{k+1}}\,d\zeta,\qquad k\in\mathbb Z
\]
gegeben sind.
\end{satz}

%
% Analytische Fortsetzung
%
\section{Analytische Fortsetzung}
\rhead{Analytische Fortsetzung}
\label{sec:fortsetzung}
Wir haben schon gesehen, dass eine reelle Funktion, die in einem
Punkte eine konvergente
Potenzreihe besitzt, auf nat"urliche Weise auch als komplexe Funktion
betrachtet werden kann, indem man komplexe Argumente in der Potenzreihe
zul"asst.
Die neue komplexe Funktion ist ein einem Kreis um den Punkt
konvergent.
Mit Hilfe der Potenzreihe kann man also immer eine Funktion auf ein
Kreisgebiet ausdehen.
Dieser Abschnitt untersucht die Frage, ob man diese Idee auch auf 
noch gr"ossere Gebiete ausdehnen kann.
\subsection{Analytische Fortsetzung mit Potenzreihen}
\begin{figure}
\centering
\includegraphics{chapters/images/komplex-1.pdf}
\caption{Analytische Fortsetzung einer komplexen Funktion entlang einer
Kurve $\gamma$.
\label{komplex:fortsetzung}}
\end{figure}
Eine komplex differenzierbare Funktion $f(z)$ ist immer darstellbar als
Potenzreihe, und ist daher analytisch.
So kann zum Beispiel die Funktion $1/z$ als Potenzreihe um jeden 
beliebigen Punkt $z_0$ entwickelt werden:
\begin{align}
f(z)
&=
\frac1z
=
\frac1{z_0-(z_0-z)}
=
\frac1{z_0}\cdot
\frac1{1-\displaystyle\frac{z_0-z}{z_0}}
=
\frac1{z_0}\sum_{k=0}^{\infty} \biggl(\frac{z_0-z}{z_0}\biggr)^k
=
\sum_{k=0}^{\infty} \frac{(-1)^k}{z_0^{k+1}} (z-z_0)^k,
\label{komplex:1durchreihe}
\end{align}
Die Koeffizienten dieser Potenzreihe sind
\[
a_k=\frac{(-1)^k}{z_0^{k+1}},
\]
und man kann den Konvergenzradius ausrechnen:
\[
\frac1{\varrho}
=
\limsup_{k\to\infty} \root{k}\of{|a_k|} = \lim_{k\to\infty}\frac1{|z_0|^{\frac{k+1}{k}}}
=
\frac1{|z_0|}.
\]
Der Konvergenzradius ist limitiert durch die Singularit"at bei an der Stelle
$z=0$.

Es gibt also keine einzelne Potenzreihe, die die Funktion $f(z)=\frac1z$ in der
ganzen komplexen Ebene darstellen kann.
W"ahlt man aber einzelne Punkte $z_0$ und $z_1$ derart, dass der Kreis
um $z_0$ mit Radius $|z_0|$ und der Kreis um $z_1$ mit Radius $|z_1|$
"uberlappen, dann werden die beiden Potenzreihen im "Uberlappungsgebiet
die gleichen Werte annehmen.

Man k"onnte allso eine Kurve $\gamma$ in der komplexen Ebene w"ahlen,
entlang der man in jedem Punkt die Funktion $f(z)$ in eine Potenzreihe 
entwickelt.
Liegen zwei Punkte nahe genug auf der Kurve $\gamma$, werden die
Konvergenzkreise der Potenzreihen "uberlappen, und die Potenzreihen
werden im "Uberlappungsgebiet die gleichen Werte liefern.

Selbst wenn man eine Funktion $f(z)$ nur in einem Kreis um den Punkt $z_0$
kennt, zum Beispiel durch eine Potenzreihe im Punkt $z_0$, kann man entlang
einer Kurve, die $z_0$ mit $z_1$ verbindet, in jedem Punkt eine Potenzreihe
finden, die mit der Potenzreihe in den Nachbarpunkten "ubereinstimmt, und
so die Definition der Funktion entlang dieser Kurve auf ein gr"osseres
Gebiet ausweiten, wie in Abbildung~\ref{komplex:fortsetzung} dargestellt.
Man nennt dies die {\em analytische Fortsetzung} der Funktion $f(z)$ 
entlange der Kurve $\gamma$.
\index{analytische Fortsetzung}
\index{Fortsetzung, analytische}

\begin{beispiel}
Wir haben bereits gesehen, dass sich die Funktion $f(z)=1/z$ in jedem
Punkt $z_0$ der komplexen Ebene in die Potenzreihe~\eqref{komplex:1durchreihe}
entwickeln l"asst.
Diese Reihe l"asst sich integrieren
\[
F(z,z_0)
=
\sum_{k=0}^\infty\frac{(-1)^k}{(k+1)z_0^{k+1}}z^{k+1},
\]
diese Reihe ist ebenfalls auf einem Kreis vom Radius $|z_0|$ um den
Punkt $z_0$ konvergent.
Wir vermuten nat"urlich, dass dies eine Darstellung des nat"urlichen
Logarithmus einer komplexen Zahl ist.
Nat"urlich ist das immer nur auf einem Kreisgebiet m"oglich, die Reihe
f"ur $z=1$ ist zum Beispiel im Punkt $z=-1$ nicht konvergent.

Um eine in der ganzen komplexen Ebene definierte Funktion $\log(z)$ zu
konstruieren, m"ussen wir also eine analytische Fortsetzung aufbauen.
Bei der Integration haben wir eine frei w"ahlbare Integrationskonstante
$C(z_0)$, die wir so w"ahlen m"ussen, dass die Reihen im "Uberlappungsgebiet
"ubereinstimmen:
\[
F(z,z_0) + C(z_0) = F(z,z_1)  + C(z_1)
\]
f"ur jedes $z$ im "Uberlappungsgebiet.
Dadurch wird aber nur die Differenz $C(z_1)-C(z_0)$ der Werte festgelegt.
Da wir "Ubereinstimmung mit der "ublichen Definition des Logarithmus
erreichen m"ochten, k"onnen wir $C(1)=0$ festlegen.

\begin{figure}
\centering
\includegraphics{chapters/images/komplex-2.pdf}
\caption{Analytische Fortsetzung f"ur die Funktion $\frac1z$ 
entlang der Pfade $\gamma_+$ und $\gamma_-$
\label{komplex:logfortsetzung}}
\end{figure}
Wir konstruieren jetzt die analytische Forstsetzung entlang der Kurven
$\gamma_+$ und $\gamma_-$ wie in Abbildung~\ref{komplex:logfortsetzung}
dargestellt.
Um die Differenz $C(z_1)-C(z_0)$ zu bestimmen, Werten wir die Funktionen
$F(z,z_0)$ und $F(z,z_1)$ jeweils im rot eingezeichneten Punkt aus.
Die exakte Berechnung ist etwas m"uhsam, da es sich ja nur um ein Beispiel
handelt, k"onnen wir die Reihen auch numerisch ausrechnen, und so die
Differenzen bestimmen:
\begin{align*}
&\text{Startpunkt $z_0=1$:}& C(1)&=0             &       &       \\
&\text{entlang $\gamma_+$:}& C(i)&= i\frac{\pi}2 & C(-1) &=  i\pi\\
&\text{entlang $\gamma_-$:}&C(-i)&=-i\frac{\pi}2 & C(-1) &= -i\pi
\end{align*}
Wir stellen fest, dass die analytische Fortsetzung der Logarthmusfunktion
entlang der Kurve $\gamma_+$ die Potenzreihe
\[
\log_+(z)
=
i\pi +\sum_{k=1}^\infty \frac{(-1)^{k+1}}{k(-1)^k}(z+1)^k
=
i\pi
-
\sum_{k=1}^\infty \frac{(z+1)^k}{k}
\]
ergibt, w"ahrend man entlang der  Kurve $\gamma_-$
\[
\log_-(z)
=
-i\pi +\sum_{k=1}^\infty \frac{(-1)^{k+1}}{k(-1)^k}(z+1)^k
=
-i\pi
-
\sum_{k=1}^\infty \frac{(z+1)^k}{k}
\]
findet.
Die beiden analytischen Fortsetzungen entlang der Kurven $\gamma_+$ und
$\gamma_-$ stimmen auf der negativen reellen Achse nicht "uberein,
sie unterscheiden sich um $2\pi i$:
\[
\log_+(z)-\log_-(z)=2\pi i.
\]
\end{beispiel}

Das Beispiel zeigt, dass es im Allgmeinen eine auf der ganzen komplexen
Ebene definierte komplexe Entsprechung einer reellen Funktion nicht
zu geben braucht.
Dieses Ph"anomen tritt zum Beispiel auch bei der Wurzelfunktion $f(z)=\sqrt{z}$
auf.
Diese Funktion ist im Punkt $z=0$ nicht differenzierbar, man muss diesen
Punkt also aus dem Definitionsbereich ausschliessen.
F"uhrt man man analog zum Beispiel eine analytische Fortsetzung durch,
findet man, dass sich die Werte von $f(z)$ f"ur die beiden Wege $\gamma_+$
und $\gamma_-$ durch das Vorzeichen unterscheiden.

\subsection{Analytische Fortsetzung mit Differentialgleichungen
\label{komplex:analytische-fortsetzung-dgl}}
In Abschnitt~\ref{subsection:wegintegrale} wurde gezeigt, wie Wegintegrale
Stammfunktionen komplexer Funktionen liefern k"onnen.
Im vorangegangenen Abschnitt wurde untersucht, wie eine komplex differenzierbare
Funktion mit Hilfe von analytischer Fortsetzung entlang einer Kurve
ausgedehnt werden kann.

Sei $f(z)$ eine komplex differenzierbare Funktion.
In jedem beliebigen Punkt des Definitionsbereichs k"onnen wir $f(z)$
in eine Potenzreihe entwickeln, und nat"urlich auch termweise integrieren.
Es gibt also in jedem Punkt $z_0$ des Definitionsbereichs eine
Funktion $F_{z_0}(z)$, die $F'_{z_0}(z)=f(z)$ erf"ullt.
Durch analytische Fortsetzung entlang einer Kurve $\gamma$ k"onnen
wir eine komplex differenzierbare Funktion $f(z)$ finden, die in einer
Umgebung der Kurve $F'(z)=f(z)$ erf"ullt.

Sei andererseits $\gamma\colon[a,b]\to\mathbb C$ eine Kurve in $\mathbb C$.
Dann k"onnen wir die Werte der Stammfunktion im Punkt $\gamma(b)$ durch
\[
F(\gamma(b)) = F(\gamma(a))+\int_\gamma f(z)\,dz
\]
berechnen.

\begin{beispiel}
\begin{figure}
\centering
\includegraphics{chapters/images/komplex-5.pdf}
\caption{Analytische Fortsetzung des Logarithmus als L"osung der
Differentialgleichung $y'=\frac1z$.
Bei einem Umlauf um den Nullpunkt nimmt der Wert von $y(z)$ um
$2\pi i$ zu.
\label{komplex:analytische-fortsetzung-log}
}
\end{figure}
Wir bestimmen die Stammfunktion von $f(z)=1/z$.
Entlang der reellen Achse weiss man bereits, dass die Stammfunktion
der nat"urliche Logarithmus ist, also $F(x)=\log x$.
Um diese Stammfunktion auf $\mathbb C$ auszudehnen, verwenden wir einen
kreisf"ormigen Pfad von der reellen Achse bis zum Punkt $z$.
Liegt $z$ in der oberen Halbebene, w"ahlen wir einen Pfad in der
oberen Halbebene, und umgekehrt.
Wir k"onnen die Zahl $z$ in Polarkoordinaten darstellen als $z=re^{i\varphi}$.
Ein Pfad von der reellen Achse kann mit
\[
\gamma\colon [0,1]\to\mathbb C: t\mapsto re^{it\varphi}
\]
parametrisiert werden.
Der Zuwachs der Stammfunktion entlang dieses Pfades ist
\[
F(z)-F(r)
=
\int_\gamma\frac1z\,dz
=
\int_0^1 \frac1{e^{it\varphi}}i\varphi e^{it\varphi}\,dt
=
i\varphi \int_0^1\,dt
=
i\varphi.
\]
Der Wert der Stammfunktion am Anfang der Kurve ist $\log r$, somit
folgt, dass
\[
\log z = \log r + i\varphi
\]
(Abbildung~\ref{komplex:analytische-fortsetzung-log}).
\end{beispiel}

\section{"Ubungsaufgaben}
\rhead{"Ubungsaufgaben}
\uebungsaufgabe{701}
\uebungsaufgabe{702}
\uebungsaufgabe{703}
\uebungsaufgabe{704}
\uebungsaufgabe{705}
\uebungsaufgabe{706}


%
% stabilitaet.tex -- Stabilität der Lösungen von Differentialgleichungen
%
% (c) 2015 Prof Dr Andreas Mueller, Hochschule Rapperswil
%
\chapter{Stabilit"at\label{chapter:stabilitaet}}
\lhead{}
\rhead{Stabilit"at}


%
% chaos.tex -- Grundlagen des "Ubergangs zum Chaos
%
% (c) 2015 Prof Dr Andreas Mueller, Hochschule Rapperswil
%
\chapter{Chaos\label{chapter:chaos}}
\lhead{}
\rhead{Chaos}


%
% stochastisch.tex -- Kapitel ueber stochastische Differentialgleichungen
%
\chapter{Stochastische Differentialgleichungen\label{chapter:stochastisch}}
\lhead{Stochastische Differentialgleichungen}
\rhead{}
In vielen Anwendungen wird die Bewegung eines Systems auch von
zuf"alligen Einfl"ussen bestimmt, die man oft auch Rauschen nennt.
Die Natur des Rauschen bedeutet, dass aufeinanderfolgende inkremente
v"ollig unkorreliert sind, w"ahrend Inkremente einer differenzierbaren
Funktion voneinander abh"angig sind.
Die L"osung einer Differentialgleichung unter Einfluss von Rauschen 
kann daher niemals eine differenzierbare Funktion sein, und sie kann
niemals eine L"osung der Differentialgleichung im bisher verwendeten
Sinn sein.
Um der Idee einen mathematischen Sinn zu geben, der auch erlaubt,
solche Differentialgleichungen zu l"osen und in Anwendungen
einzusetzen, muss daher zuerst gekl"art werden, was Rauschen genau ist.
Anschliessend muss das Konzept einer Differentialgleichung so formuliert
werden, dass es auch f"ur nicht differenzierbare Funktionen und Rauschen
anwendbar ist.

Die Darstellung in diesem Kapitel orientiert sich in vielen Punkten
an dem hervorragenden und leicht lesbaren Buch \cite{skript:evans}.
Eine mathematisch vertieftere Entwicklung ist in \cite{skript:oksendal}
zu finden.

%
% Ein Modell f"ur Rauschen
%
\section{Modell f"ur Rauschen: der Wiener-Prozess\label{section:wiener}}
\rhead{Wiener-Prozess}
Rauschen ist ein Zufallsph"anomen, die Wiederholung eines Experimentes
wird im Allgemeinen einen anderen Verlauf ergeben.
Der Pfad eines Teilchens $W(t)$ in Abh"angikeit ist daher ein Zufallsresultat.
Wir brauchen daher einen Wahrscheinlichkeitsraum $\Omega$ und ein
Wahrscheinlichkeitsmass $P$, und die Wege $W$ sind abh"angig von 
der Durchf"uhrung $\omega\in\Omega$ des Experiments. 
Genau genommen m"ussen wir also sagen, dass f"ur jedes $\omega\in\Omega$
der Weg $W(\omega)$ eine Funktion
\[
W(\omega)\colon\mathbb R \to\mathbb R:t\mapsto W(\omega)(t)
\]
ist.
Wir nennen eine solche Funktion einen {\em stochastischen Prozess}.
\index{stochastischer Prozess}
Im Folgenden werden wir die etwas schwerf"allige Notation etwas
vereinfachen, und das $\omega$ weglassen.

Wir m"ochten die Position eines Teilchens berechnen, dessen Geschwindigkeit
ein solches ``Rauschen'' ist.
Diese Position $W(t)$ ist ein stochastischer Prozess im eben erkl"arten Sinne.
Die Brownsche Bewegung ist ein solcher Prozess, die Position $W(t)$ eines
Teilchens unter dem Einfluss der thermischen Bewegung der Teilchen
in einer Fl"ussigkeit als Funktion der Zeit wird eine nicht
differenzierbare Funktion sein.
Das beste, was wir erwarten k"onnen, ist dass die Positionsunterschiede
\[
W(t+\Delta t)-W(t),
\quad
W(t + 2\Delta t)-W(t+\Delta t),
\quad
W(t + 3\Delta t)-W(t+2\Delta t),\quad\dots
\]
voneinander unabh"angig sind, und nicht beliebig gross sind.
Wir erwarten, dass diese Differenzen, die sich aus vielen kleinen
St"ossen zusammensetzen, normalverteilt sind.
Wir definieren daher

\begin{definition}
Ein stochastischer Prozess $W(t)$ heisst {\em Brownsche Bewegung} oder
{\em Wiener Prozess}, wenn gilt
\begin{compactenum}
\item $W(0)=0$
\item $W(t)-W(s)$ ist normalverteilt mit Erwartungswert $0$ und
Varianz $t-s$, f"ur beliebige $t\ge s\ge 0$.
\item F"ur beliebige Werte $t_i$ mit $0<t_1<t_2<\dots<t_n$, dann sind
die Zufallsvariablen
$W(t_1), W(t_2)-W(t_1),\dots,W(t_n)-W(t_{n-1})$ unabh"angig.
\end{compactenum}
\index{Brownsche Bewegung}
\index{Wiener Prozess}
\end{definition}

A priori ist nicht klar, dass es so einen Prozess "uberhaupt gibt, wir
m"ussen daher zeigen, dass sich eine solche Funktion konstruieren l"asst.
Eine solche Funktion ist aber sicher nicht differenzierbar, denn
der Differenzenquotient "uber ein Interval der L"ange $2\Delta t$
\begin{align*}
\frac{W(t+2\Delta)-W(t)}{2\Delta t}
&=
\frac{W(t+2\Delta t)-W(t+\Delta t)}{2\Delta t}
+
\frac{W(t+\Delta t)-W(t)}{2\Delta t}
\\
&=
\frac12\biggl(
\frac{W(t+2\Delta t)-W(t+\Delta t)}{\Delta t}
+
\frac{W(t+\Delta t)-W(t)}{\Delta t}
\biggr)
\end{align*}
ist Mittelwert aus zwei Differenzenquotienten "uber k"urzere Intervalle,
aber diese beiden Differenzenquotienten sind voneinander unabh"angig.
Ein Grenzwert des Differenzenquotienten kann daher nicht existieren.

\subsection{Eigenschaften des Wiener-Prozesses}
Wir brauchen Rechenregeln, wie man mit Wiener-Prozessen Funktionen
rechnen kann.
Zum Beispiel ist $W(t)$ wegen Eigenschaft~2 normalverteilt mit Erwartungswert
$0$ und Varianz $t$, also gilt
\[
E(W(t))=0,\qquad E(W(t)^2)=t,\qquad \forall\;t>0.
\]
Etwas weniger offensichtlich ist
\begin{hilfssatz}
Wenn $W(t)$ eine Brownsche Bewegung ist, dann ist
\[
E(W(t)W(s)) = t\wedge s = \min\{s,t\}
\]
f"ur beliebige $t,s\ge 0$.
\end{hilfssatz}

\begin{proof}[Beweis]
Nehmen wir an, dass $t\ge s$, dass also $t\wedge s = s$.
Dann k"onnen wir $E(W(t)W(s))$ berechnen
\begin{align*}
E(W(t)W(s))
&=
E((W(t)-W(s)+W(s))W(s))
=
E((W(t)-W(s))W(s))+E(W(s)^2)
\end{align*}
Eigenschaften~1 und~2 zeigen, dass $E(W(s)^2)=s$ ist.
Eigenschaft~3 besagt, dass $W(t)-W(s)$ und $W(s)$ unabh"angig sind,
der Erwartungswert ihres Produktes ist daher das Produkt der Erwartungswerte:
\begin{align*}
E(W(t)W(s))
&=
E((W(t)-W(s))W(s))+E(W(s)^2)
=
\underbrace{E(W(t)-W(s))}_{\textstyle =0} \underbrace{E(W(s))}_{\textstyle =0} + s
\end{align*}
wobei wir erneut Eigenschaft~2 verwendet haben.
\end{proof}
Man k"onnte diese Eigenschaft umschreiben als die Beobachtung,
dass die weitere Entwicklung von $W(t)$ nach der Zeit $s$ bedeutungslos ist.

\begin{figure}
\centering
\includegraphics{chapters/images/stochastisch-1.pdf}
\caption{Wiener-Prozess $W(t)$ in Abh"angigkeit von der Zeit
\label{stochastisch:wiener}}
\end{figure}

\subsection{Konstruktion des Wiener-Prozesses}
Wir m"ussen eine Konstruktion angeben, mit der wir zu einem gegebenen
Interval $[0,T]$ eine Funktion $W(t)$ konstruieren k"onnen, die
die Eigenschaften des Wiener-Prozesses erf"ullt.

Zun"achst verlangen die Eigenschaften~1 und 2 des Wiener-Prozesses,
dass $X(T)$ eine normalverteilte Zufallsvariable ist mit Erwartungswert
$X(0)=0$ und Varianz $T$.
Da $X(t)$ ausserdem stetig sein soll, verwenden wir als erste Iteration
die lineare Funktion:
\[
W_1(t) = X(T)\frac{t}{T}
\]
Die Eigenschaft~3 verlangt, dass auch $X(T/2)$ normalverteilt ist mit
Erwartungswert $0$ und Varianz $T/2$.
Dies kann dadurch erreicht werden, dass wir $W_1$ durch einen 
Polygonzug $W_2$ ersetzen, der bei $t=T/2$ einen zus"atzlichen
Eckpunkt besitzt.
Wir bezeichnen den Unterschied zwischen $W_2$ und $W_1$ an der
Stelle $t=T/2$ mit
\[
Y=W_2(T/2)-W_1(T/2)=W_2(T/2) - W_1(T)/2.
\]
$Y$ muss so gew"ahlt werden, dass $W_2(T/2)$ eine normalverteilte
Zufallsvariable mit Erwartungswert $0$ und Varianz $T/2$ wird,
somit ist $Y$ auch normalverteilt.
Der Erwartungswert von $Y$ ist
\begin{align*}
E(Y)&=E(W_2(T/2) - W_1(T)/2)=E(W_2(T/2))-E(W_1(T))/2=0,
\end{align*}
es ist also nur noch die Varianz $\sigma^2_Y$ w"ahlbar.
Sie muss so gew"ahlt werden, dass $W_2(T/2)$ Varianz $T/2$
bekommt:
\begin{align*}
\operatorname{var}\biggl(W_2\biggl(\frac{T}2\biggr))\biggr))
&=
\operatorname{var}\biggl(W_1\biggl(\frac{T}2\biggr) + Y\biggr)
=
\frac{\operatorname{var}(W_1)}4 + \operatorname{var}Y
=
\frac{T}4 +\sigma_Y^2
\end{align*}
Es folgt $\sigma_Y^2=\frac{T}4$.

Wir m"ussen kontrollieren, ob die Inkremente unabh"angig sind.
Dazu berechnen wir der Erwartungswert des Produktes der
beiden Inkremente
\begin{align*}
W_2(T)-W_2\biggl(\frac{T}2\biggr)
&=
W_1(T)-\biggl(\frac{W_1(T)}2 + Y\biggr)
=
Y-\frac{W_1(T)}2
\\
W_2\biggl(\frac{T}2\biggr)-W_2(0)
&=
Y+
\frac{W_1(T)}2
\\
\biggl(W_2(T)-W_2\biggl(\frac{T}2\biggr)\biggr)
\biggl(W_2\biggl(\frac{T}2\biggr)-W_2(0)\biggr)
&=
\biggl(Y-\frac{W_1(T)}2\biggr)\biggl(Y+\frac{W_1(T)}2\biggr)
=
Y^2-\frac{W_1(T)^2}4
\\
E\biggl(
\biggl(W_2(T)-W_2\biggl(\frac{T}2\biggr)\biggr)
\biggl(W_2\biggl(\frac{T}2\biggr)-W_2(0)\biggr)
\biggr)
&=
E(Y^2)-E\biggl(\frac{W_1(T)^2}4\biggr)
=
\sigma_Y^2 - \frac14\operatorname{var}(W_1(T))
=
0
\end{align*}
Somit sind die Inkremente unabh"angig.

\begin{figure}
\centering
\includegraphics{chapters/images/stochastisch-3.pdf}\\
\includegraphics{chapters/images/stochastisch-4.pdf}\\
\includegraphics{chapters/images/stochastisch-5.pdf}\\
\includegraphics{chapters/images/stochastisch-6.pdf}\\
\includegraphics{chapters/images/stochastisch-7.pdf}
\caption{Schrittweise Konstruktion des Wiener-Prozesses als Grenzewert
der Folge $W_n(t)$, $n=1,\dots,5$
\label{stochastisch:folge1}}
\end{figure}
\begin{figure}
\centering
\includegraphics{chapters/images/stochastisch-8.pdf}\\
\includegraphics{chapters/images/stochastisch-9.pdf}\\
\includegraphics{chapters/images/stochastisch-10.pdf}\\
\includegraphics{chapters/images/stochastisch-11.pdf}\\
\includegraphics{chapters/images/stochastisch-12.pdf}
\caption{Schrittweise Konstruktion des Wiener-Prozesses als Grenzewert
der Folge $W_n(t)$, $n=6,\dots,10$
\label{stochastisch:folge2}}
\end{figure}

Diesen Prozess k"onnen wir fortsetzen: f"ur $W_3$ nehmen wir einen
Polygonzug mit zus"atzlichen Eckpunkten an den Stellen $T/4$ und $3T/4$.
Die Differenz zwischen $W_3$ und $W_2$ an diesen Stellen m"ussen
normalverteilt sein mit Erwartungswert $0$ und Varianz $T/8$.
Auf diese Weise k"onnen wir eine Folge $W_n$ von Prozessen konstruieren,
wie in den Abbildungen~\ref{stochastisch:folge1} und \ref{stochastisch:folge2}
dargestellt.

Dies reicht aber nicht.
F"ur eine vollst"andige Konstruktion muss man noch die folgenden zwei
Dinge zeigen.
\begin{compactenum}
\item
Der Grenzwert existiert tats"achlich.
Weil die einzelnen Zufallsvariablen $Y$ normalverteilt sind, k"onnen
sie beliebig grosse Werte annehmen.
Dies bedeutet auch, dass wir im Allgemeinen nicht davon ausgehen 
k"onnen, dass die Folge $W_n(t)$ konvergiert.
Zwar sind die Werte von $W_m$ an den Stellen $t_{k,n}=kT/2^{n-1}$ f"ur
$k=0,\dots,2^{n-1}$ fest, sobald $m\ge n$.
Zwischen diesen Werten k"onnen aber immer wieder grosse Werte auftreten,
so dass die Folge $W_n$ weder gleichm"assig noch punktweise konvergieren
kann.
Die Frage der Konvergenz muss daher als Konvergenz in einer Art Mittel 
angegangen werden.
\item
Die Forderung, dass die Inkremente $W(t)-W(s)$ und $W(s)-W(r)$ f"ur jedes 
Tripel $t \ge s\ge r$ unabh"angig sein m"ussen.
Die Konstruktion stellt nur sicher, dass dies gilt, wenn die Zeitpunkte
des Tripels Eckpunkte der Polygonkonstruktion sind.
\end{compactenum}

%
%
%
\section{Stochastische Differentialgleichungen\label{section:stochdgl}}
\rhead{Differentialgleichungen}
Wir m"ochten gerne eine Differentialgleichung f"ur den Zustand
$X(t)$ eines Systems l"osen, welches von Rauschen mit beeinfluss wird.
Eine solche Differentialgleichung k"onnten wir schreiben als
\[
\frac{dX(t)}{dt}
=
b(X(t)) + B(X(t))\frac{dW(t)}{dt}
\]
wobei $b$ die Rolle der Funktion $f$ aus
Abschnitt~\ref{section:anfangswertprobleme} spielt.
Die Ableitung von $W$ spielt die Rolle des Rauschens, wir wissen aber
bereits, dass $W$ nicht differenzierbar sein kann, die Gleichung
in dieser Form kann daher gar nicht sinnvoll sein.

Formal kann man die Gleichung mit $dt$ multiplizieren, so dass man
das formale Gleichungssystem
\begin{align*}
dX(t)
&=
b(X(t)) + B(X(t),t)\,dW(t)
\\
X(0)
&=
x_0
\end{align*}
erh"alt, aber auch in diesr Form sind die Ausdr"ucke $dX$ und $dW$ nicht
ohne zus"atzliche Definition sinnvoll.
Am ehesten hat man eine Chance, dieser Gleichung einen Sinn zu geben,
wenn man integriert:
\begin{align*}
X(t)=x_0+\int_0^t b(X(s))\,ds + \int_0^t B(X(s), s)\, dW,\qquad t>0.
\end{align*}
Das erste Integral ist ein gew"ohnliches Integral, denn wir
gehen davon aus, dass die Funktion $X(t)$ stetig ist.
Wenn $B=0$ ist, dann liefert diese Formel genau L"osungen der
urspr"unglichen Differentialgleichung.
Wir brauchen aber immer noch eine Interpretation des zweiten Integrals,
diese werden wir in Abschnitt~\ref{subsection:stochint} geben.

Nehmen wir f"ur den Moment an, dass die L"osung $X(t)$ gefunden werden
kann, und sei $u$ eine beliebig oft differenzierbare Funktion.
Wir erwarten, dass die Ableitung von $Y(t)=u(X(t))$ nach der
Kettenregel
\[
dY = u'\,dX = u'b\,dt + u'\,dW
\]
sein sollte.
Tats"achlich ist das Rauschen so stark, dass die Kr"ummung der Funktion
$u$ bereits eine Rolle spielt.
Bei einer gew"ohnlichen Differentialgleichung sind auf kleine Entfernungen
die zweiten Ableitungen von $u$ vernachl"assigbar.
Die schnellen, vom Rauschen verursachten Abweichungen f"uhren sind aber nicht
klein, so dass die korrekte Kettenregel bei Anwesenheit von Rauschen die
It\^o-sche Kettenregel wird, die stattdessen den Ausdruck
\[
dY =  \biggl(u'b + \frac12u''\biggr)\,dt + u'\,dW
\]
liefert.
Der Term $\frac12u''$ ist in der klassischen Analysis nicht vorhanden.
Wir m"ussen daher auch alle gewohnten Rechenregeln der Analysis "uberpr"ufen.

\subsection{Stochastische Integrale\label{subsection:stochint}}
\index{stochastisches Integral}
Wir m"ochten eine Definition f"ur ein Integral der Form
\[
\int_0^T G\,dW
\]
f"ur zwei stochastische Prozesse $G(t)$ und $W(t)$.

\subsubsection{Das Paley-Wiener-Zygmund Integral}
\index{Paley-Wiener-Zygmund Integral}
F"ur differenzierbare Funktionen $g(t)$ und $w(t)$ ist klar, was 
mit dem Integral gemeint ist:
\[
\int_0^T g(t)\,dw = \int_0^T g(t) w'(t)\,dt.
\]
Wenn $g(0)=g(T)=0$ ist, dann kann man dies mit Hilfe partieller Integration
vereinfachen:
\[
\int_0^T g(t)\,dw
=
\int_0^T g(t) w'(t)\,dt
=
[g(t)w(t)]_0^T
-
\int_0^T g'(t) w(t)\,dt
=
-\int_0^T g'(t) w(t)\,dt
\]
Diese letzte Formel ist aber auch geeignet als Definition des Integrals
f"ur eine brownsche Bewegung $W(t)$ an Stelle von $w(t)$.

\begin{definition}
F"ur eine stetig differenzierbare Funktion $g\colon[0,T]\to \mathbb R$ 
mit $g(0)=g(T)=0$, setze
\[
\int_0^T g\,dW = -\int g'(t) W(t)\,dt.
\]
\end{definition}

\begin{hilfssatz}
Erwartungswert und Varianz der Zufallsvariablen
\[
Z=\int_0^T g\,dW
\]
ist
\begin{compactenum}
\item $E(Z)=0$
\item $\operatorname{var}(Z)=\int_0^Tg(t)^2\,dt$
\end{compactenum}
\end{hilfssatz}

\begin{proof}[Beweis]
F"ur den Erwartungswert finden wir mit Hilfe der Rechenregeln f"ur den
Erwartungswert
\begin{align*}
E(Z)
&=
E\biggl(\int_0^T g\,dW\biggr)
=
E\biggl(
\int_0^T g'(t) W(t)\,dt
\biggr)
=
\int_0^T g'(t) \underbrace{E(W(t))}_{\textstyle =0}\,dt=0
\end{align*}
Die Varianz ist daher der Erwartungswert $E(Z^2)$, die wir ebenfalls unter
Verwendung der Rechenregeln berechnen k"onnen
\begin{align*}
E(Z^2)
&=
E\biggl(\biggl(\int_0^T g\,dW\biggr)^2\biggr)
=
E\biggl(
\biggl(
\int_0^T g'(t)W(t)\,dt
\biggr)^2
\biggr)
\\
&=
E\biggl(
\int_0^T g'(t)W(t)\,dt
\int_0^T g'(s)W(s)\,ds
\biggr)
=
E\biggl(
\int_0^T \int_0^T g'(t)W(t) g'(s)W(s) \,ds \,dt
\biggr)
\\
&=
\int_0^T\int_0^Tg'(t)g'(s) \underbrace{E(W(t)W(s))}_{\textstyle =t\wedge s}\,ds\,dt
\\
&=
\int_0^Tg'(t) \biggl(\int_0^t g'(s) s\,ds + \int_t^T g'(s)t\,ds\biggr)
\,dt
\\
&=
\int_0^T g'(t)\biggl(
\underbrace{[g(s)s]_0^t}_{\textstyle =tg(t)}-\int_0^t g(s)\,ds+t(\underbrace{g(T)}_{\textstyle =0}-g(t))
\biggr)\,dt
\\
&=
\int_0^T g'(t)\biggl(-\int_0^t g(s)\,ds\biggr)\,dt
\\
&=
\biggl[
g(t)\biggl(-\int_0^t\int_0^tg(s)\,ds\biggr)
\biggr]_0^T
+
\int_0^Tg(t)^2\,dt
\end{align*}
Damit ist die Behauptung bewiesen.
\end{proof}
Der Hilfssatz kann dazu verwendet werden, die Definition des Integrals auf
weitere Funktionen auszudehnen.
Eine Folge von Funktionen $g_n$ f"uhrt auf eine Folge von Werten des
Integrals. 
Um daraus eine Erweiterung des Integrals zu konstruieren, m"ussen wir 
definieren, was es heissen soll, dass die Folge $g_n$ konvergiert.
Wir verwenden die Norm
\[
\|f\|_2^2=\int_0^T f(t)^2 \,dt,
\]
um den Abstand zwischen Funktionen zu messen.
Eine Cauchy-Folge von Funktionen in dieser Norm ist dann eine Folge so,
dass f"ur jedes $\varepsilon>0$ ein $N>0$ existiert, so dass aus
$n,m>N$ folgt
\[
\|g_n-g_m\|_2^2=\int_0^T (g_n(t)-g_m(t))^2\,dt<\varepsilon.
\]
In diesem Fall gilt, dass
\[
E\biggl(
\biggl(
\int_0^T g_n\,dW
-
\int_0^T g_m\,dW
\biggr)^2
\biggr)
=
\int_0^T(g_m(t)-g_n(t))^2\,dt<\varepsilon
\]
Die Zufallsvariablen
\[
Z_n = \int_0^T g_n\,dW
\]
bildet daher eine Cauchy-Folge von quadratintegrierbaren Funktionen
in $L^2(\Omega)$, und wir k"onnen deren Grenzwert
\[
\int_0^T g\,dW
=
\lim_{n\to\infty}\int_0^T g_n\,dW
\]
als den verallgemeinerten Wert f"ur das Integral einer beliebigen Funktion
$g\in L^2([0,T])$ definieren.

\subsubsection{Riemann-Summen}
Die Verallgemeinerung des Paley-Wiener-Zygmund-Integral auf
quadratintegrierbare Funktionen ist allerdings nicht ausreichend, um
\[
\int_0^T W\,dW
\]
zu definieren.
Wir k"onnen aber noch weiter zur"uck gehen zur Definition des
Riemann-Integrals, und sie verallgemeinern auf einen stochastischen
Prozess.

\begin{definition}
Eine {\em Unterteilung} $P$ des Intevals $[0,T]$ ist eine endliche Menge
von Teilpunkten
\[
P=\{ 0=t_0<t_1<t_2<\dots<t_m=T\}.
\]
Die {\em Maschenweite} $|P|$ der Unterteilung $P$ ist
\[
|P|=\max_{0\le k\le m-1}|t_{k+1\mathstrut}-t_{k\mathstrut}|.
\]
F"ur ein festes $0\le \lambda\le 1$ setzen wir
\[
\tau_k = (1-\lambda)t_{k\mathstrut}+\lambda t_{k+1\mathstrut}.
\]
\end{definition}
F"ur $\lambda=0$ ist $\tau_k=t_k$, f"ur $\lambda=1$ gilt $\tau_k=t_{k+1}$.
F"ur andere Werte von $\lambda$ ist $\tau_k$ ein Punkt im Inneren des
Intervals $[t_k,t_{k+1}]$.
Wie "ublich kann eine solche Unterteilung dazu verwendet werden, die 
Riemann-Summe als Approximation des Integrals zu definieren:

\begin{definition}
Die Riemann-Summe von $W$ f"ur die Unterteilung $P$ ist
\[
R=R(P,\lambda) = \sum_{k=0}^{m-1} W(\tau_k)(W(t_{k+1})-W(t_k)).
\]
\index{Riemann-Summe}
\end{definition}

\begin{hilfssatz}
\label{stochastisch:quadrvariation}
Sei
\[
P^{(n)}
=
\{0=t_0^{(n)}<t_1^{(n)}<\dots < t_{m_n}^{(n)}=T\}
\]
eine Folge von Unterteilungen des Intervals $[0,T]$ derart,
dass die Maschenweite gegen $0$ strebt, dann gilt
\[
\lim_{n\to\infty}
\sum_{k=0}^{m_n-1} \bigl(W(t_{k+1}^{(n)})-W(t_{k}^{(n)})\bigr)^2
=T
\]
in $L^2(\Omega)$.
\end{hilfssatz}

\begin{proof}[Beweis]
Konvergenz in $L^2(\Omega)$ bedeutet, dass der Erwartungswert der
quadratische Abweichung gegen $0$ streibt.
Setzen wir
\[
Q_n
= 
\sum_{k=0}^{m_n-1} \left(W(t_{k+1}^{(n)})-W(t_{k}^{(n)})\right)^2,
\]
und m"ussen untersuchen, ob $E((Q_n - T)^2)$ gegen $0$ strebt.
Wir beginnen mit der Bemerkung, dass
\begin{align*}
Q_n-T
&=
\sum_{k=0}^{m_n-1}
\left((W(t_{k+1}^{(n)})-W(t_{k}^{(n)}))^2 - (t_{k+1}^{(n)}-t_k^{(n)})\right).
\end{align*}
Die einzelnen Differenzen k"urzen wir ab als
\begin{align*}
Y_k
&=
\frac{W(t_{k+1}^{(n)})-W(t_k^{(n)})}{\sqrt{t_{k+1}^{(n)}-t_{k}^{(n)}}},
\\
\Delta_k
&=
t_{k+1}^{(n)}-t_k^{(n)}.
\end{align*}
F"ur $j\ne k$ sind $Y_j^{(n)}$ und $Y_{k}^{(n)}$ unabh"angig.
Ausserdem ist $Y_k$ standardnormalverteilt, also $E(Y_k)=0$
und $E(Y_k^2)=1$.
Die Summanden in der Summe lassen sich damit kompakter ausdr"ucken:
\begin{align*}
Q_n-T
&=
\sum_{k=0}^{m_n-1} (Y_k^2-1) \Delta_k
\end{align*}
Dies k"onnen wir in $E((Q_n-T)^2)$ einsetzen:
\begin{align*}
E((Q_n-T)^2)
&=
\sum_{k=0}^{m_n-1}
\sum_{j=0}^{m_n-1}
E\biggl(
(Y_k^2-1)\Delta_k
(Y_j^2-1)\Delta_j
\biggr)
\end{align*}
Die Doppelsumme kann zerlegt werden in Terme mit $j=k$ und $j\ne k$.

In den Terme mit $j\ne k$ sind die $Y_k$ und $Y_j$ voneinander unabh"angig,
der Erwartungswert des Produktes kann daher in das Produkt der Erwartungswerte
zerlegt werden:
\begin{align*}
E\biggl(
(Y_k^2-1)\Delta_k
(Y_j^2-1)\Delta_j
\biggr)
&=
E(Y_k^2-1)
E(Y_j^2-1)
\Delta_k
\Delta_j
\\
&=
\bigl(E(Y_k^2) -1\bigr)
\bigl(E(Y_j^2) -1\bigr)
\Delta_k
\Delta_j
=0
\end{align*}
Im letzten Schritt haben wir verwendet, dass $E(Y_k^2)=\operatorname{var}Y_k=1$
ist.

Die Terme mit $j=k$ sind 
\begin{align*}
E((Q_n-T)^2)
&=
\sum_{k=0}^{m_n-1} E\biggl((Y_k^2-1)^2\Delta_k^2\biggr)
\\
&=
\sum_{k=0}^{m_n-1} E\bigl((Y_k^2-1)^2\bigr)\Delta_k^2
\le 
\biggl(\sum_{k=0}^{m_n-1} E\bigl((Y_k^2-1)^2\bigr)\Delta_k\biggr)\, |P^{(n)}|
\end{align*}
Im letzten Schritt haben wir einen Faktor $\Delta_k$ aus der Summe
herausgenommen und durch die Maschenweite $|P^{(n)}|$ abgesch"atzt.

Da $Y_k$ standardnormalverteilt ist, ist der Erwartungswert $E((Y_k^2-1)^2)$
unabh"angig von $k$, wir nennen diesen Wert $C$, der genaue Wert ist
nicht wichtig\footnote{Man kann den Wert nat"urlich wie folgt berechnen:
\begin{align*}
E((Y_k^2-1)^2)
&=
E(Y_k^4-2Y_k^2+1)
=
E(Y_k^4)-2E(Y_k^2)+E(1)
=
6-2+1=5.
\end{align*}}.
Damit l"asst sich die Differenz jetzt unabh"angig von $k$ absch"atzen
\begin{align*}
E((Q_n-T)^2)
&
\le
C\biggl(\sum_{k=1}^{m_n}\Delta_k\biggr)\, |P^{(n)}|
=CT|P^{(n)}|.
\end{align*}
Da die Maschenweite $|P^{(n)}|\to 0$ f"ur $n\to\infty$ folgt, dass
$Q_n\to T$ in $L^2(\Omega)$.
\end{proof}

\begin{hilfssatz} Sei $P^{(n)}$ eine Folge von Unterteilungen des
Intervals $[0,T]$ derart, dass die Maschenweite gegen $0$ strebt und
sei $R_n=R(P^{(n)},\lambda)$ die zugeh"orige Riemann-Summe.
Dann gilt
\[
\lim_{n\to\infty} R_n = \frac{W(T)^2}2 + \biggl(\lambda-\frac12\biggr)T
\]
als Funktion in $L^2(\Omega)$.
\end{hilfssatz}

\begin{proof}[Beweis]
Wir m"ussen die Riemann-Summe
\begin{align*}
R_n
&=
\sum_{k=0}^{m_n-1}
W(\tau_k) (W(t_{k+1}^{(n)})-W(t_k^{(n)}))
\end{align*}
durch Inkremente von Werten von $W(t)$ ausdr"ucken, konkret durch
\[
W(t_{k+1}^{(n)})-W(t_k^{(n)}),\quad
W(t_{k+1}^{(n)})-W(\tau_k^{(n)})
\quad \text{und} \quad
W(\tau_k^{(n)})-W(t_k^{(n)}).
\]
Um herauszufinden, wie dies m"oglich sein k"onnte, k"urzen wir die Werte
von $W$ ab durch
\[
a_k=W(t_k^{(n)}),\quad
b_k=W(\tau_k^{(n)})\quad\text{und}\quad
c_k=W(t_{k+1}^{(n)}),
\]
wobei wir im folgenden zur Vereinfachung der Rechnung die Indizes auch
weglassen.
Jetzt versuchen wir $b(c-a)$ durch andere Differenzen auszudr"ucken.
\begin{align*}
(b-a)^2
&=
b^2-2ab + a^2
\\
(c-a)^2
&=
c^2-2ac+a^2
\\
(c-b)(b-a)
&=
cb-b^2-ac+ab
\end{align*}
Um den Ausdruck $b(c-a)$ zu produzieren, muss der letzte Term verwendet
werden, denn er ist der einzige, der $bc$ enth"alt.
Dann muss aber auch der erste Term verwendet werden, um den Term $b^2$ zum
Verschwinden  zu bringen.
Ebenso muss der zweite Term $\frac12$-mal subtrahiert werden, damit der
Ausdruck $ac$ wegf"allt:
\begin{align*}
(b-a)^2-\frac12(c-a)^2+(c-b)(b-a)
&=
(b^2-2ab + a^2)
-\frac12(c^2-2ac+a^2)
+(cb-b^2-ac+ab)
\\
&=
cb-ab + \frac12a^2
-\frac12c^2
\end{align*}
Die quadratischen Terme sind $\frac12c_k^2=\frac12W(t_{k+1}^{(n)})^2$ und
$\frac12a_k^2=\frac12W(t_{k}^{(n)})^2$, die sich in aufeinanderfolgenden Termen
jeweils wegheben.
Nur der erste und letzte Term bleibt in der Summe bestehen, was wir
aber leicht korrigieren k"onnen.
So finden wir daher:
\begin{align*}
R_n
&=
\underbrace{
\sum_{k=0}^{m_n-1} (b_k-a_k)^2
}_{\textstyle\to\lambda T}
-
\underbrace{
\frac12\sum_{k=0}^{m_n-1} (c_k-a_k)^2
}_{\textstyle\to T}
+\sum_{k=0}^{m_n-1} (c_k-b_k)(b_k-a_k)
- \frac12a_0^2
+ \frac12c_{m_n-1}^2
\end{align*}
Die zweite Summe wurde im Hilfssatz~\ref{stochastisch:quadrvariation}
berechnet, dort wurde gefunden, dass die Summe f"ur $n\to\infty$ gegen $T$
konvergiert.
Mit der gleichen Rechnung wie in \ref{stochastisch:quadrvariation}
kann man finden, dass die zweite Summe gegen $\lambda T$ strebt.

Die Inkremente in der dritten Summe sind unabh"angig voneinander, daher
verschwinden die Erwartunsgewerte gemischter Produkte, es bleiben nur
die Terme
\begin{align*}
E\biggl(\biggl(\sum_{k=0}^{m_n-1}(c_k-b_k)(b_k-a_k)\biggr)^2\biggr)
&=
\sum_{k=0}^{m_n-1}
E\bigl((c_k-b_k)^2\bigr)E\bigl((b_k-a_k)^2\bigr)
\\
&=
\sum_{k=0}^{m_n-1}
E\bigl((W(t_{k+1}^{(n)})-W(\tau_k^{(n)}))^2\bigr)E\bigl((W(\tau_k^{(n)})-W(t_k^{(n)})^2\bigr)
\\
&=
\sum_{k=0}^{m_n-1}
(t_{k+1}^{(n)}-\tau_k^{(n)}) (\tau_k^{(n)}-t_k^{(n)})
\\
&=
\sum_{k=0}^{m_n-1}
(1-\lambda)(t_{k+1}^{(n)}-t_k^{(n)}) \lambda(t_{k+1}^{(n)}-t_k^{(n)})
\\
&\le 
(1-\lambda)\lambda
|P^{(n)}|
\sum_{k=0}^{m_n-1}
(t_{k+1}^{(n)}-t_k^{(n)})
\le (1-\lambda)\lambda T|P^{(n})|\to 0
\end{align*}
f"ur $n\to\infty$.
Damit ist gezeigt, dass die Riemann-Summe in $L^2(\Omega)$ gegen
\[
R_n\to
\frac{W(T)^2}2+\biggl(\lambda-\frac12\biggr)T
\]
konvergiert.
\end{proof}

\subsubsection{Das It\^o-Integral}
Ausgehend von der Riemann-Summe und den f"ur sie hergeleiteten Eigenschaften
wollen wir jetzt ein Integral
\[
\int_0^T G\,dW
\]
f"ur eine breitere Klasse von stochastischen Prozessen $G$ definieren.
Der erste Schritt dazu, ist das Integral f"ur Stufenprozesse zu definieren.

\begin{definition}
\index{Stufenprozess}
Ein stochastischer Prozess $G(t)$ ist eine {\em Stufenprozess}, wenn es eine
Unterteilung $P=\{0=t_0<t_1<\dots<t_m=T\}$ gibt derart, dass
$G(t)=G(t_k)$ f"ur $t_k\le t<t_{k+1}$.
\end{definition}

\begin{definition}
Ist $G$ ein Stufenprozess, dann ist das {\em It\^o-Integral} von $G$ definiert
als
\[
\int_0^TG\,dW = \sum_{k=0}^{m-1}G(t_k)(W(t_{k+1})-W(t_k)).
\]
Das It\^o-Integral ist eine Zufallsvariable.
\end{definition}

Das It\^o-Integral hat die folgenden Eigenschaften
\begin{hilfssatz}
Wenn $G$ und $H$ Stufenprozesse sind und $a,b\in\mathbb R$, dann gilt
\begin{compactenum}
\item Das It\^o-Integral ist linear:
\[
\int_0^T aG+bH\,dW
=
a\int_0^T G\,dW + b \int_0^TH\,dW
\]
\item Der Erwartungswert des It\^o-Integrals ist $0$:
\[
E\biggl(\int_0^T G\,dW\biggr)=0.
\]
\item Die Varianz des It\^o-Integrals ist
\[
E\biggl(\biggl(\int_0^T G\,dW\biggr)^2\biggr)
=
E\biggl(\int_0^T G^2\,dt\biggr)
\]
\end{compactenum}
\end{hilfssatz}

\begin{proof}[Beweis]
F"ur die Linearit"at verwenden wir eine Unterteilung $P$, f"ur die sowohl
$G$ also auch $H$ ein ein Stufenprozess ist.
Dann gilt
\begin{align*}
\int_0^T aG+bH\,dW
&=
\sum_{k=0}^{m-1} (aG(t_k)+bH(t_k))(W(t_{k+1})-W(t_k))
\\
&=
a\sum_{k=0}^{m-1} G(t_k)(W(t_{k+1})-W(t_k))
+
b\sum_{k=0}^{m-1} H(t_k)(W(t_{k+1})-W(t_k))
\\
&=a\int_0^TG\,dW+b\int_0^TH\,dW
\end{align*}
womit die Linearit"at bewiesen ist.

F"ur den Erwartungswert berechnen wir mit Hilfe einer passenden Unterteilung
\begin{align*}
E\biggl(\int_0^T G\,dW\biggr)
&=
E\biggl(\sum_{k=0}^{m-1} G(t_k) (W(t_{k+1}) - W(t_k))\biggr)
\\
&=
\sum_{k=0}^{m-1} E(G(t_k)) \underbrace{E(W(t_{k+1}) - W(t_k))}_{\textstyle =0}=0.
\end{align*}
Damit ist 2.~beweisen.

F"ur die Berechnung der Varianz verwendet wieder die Unabh"angigkeit der
Inkremente:
\begin{align*}
E\biggl(\biggl(\int_0^T G\,dW\biggr)^2\biggr)
&=
E\biggl(\biggl(\sum_{k=0}^{m-1}G(t_k)(W(t_{k+1})-W(t_k))\biggr)^2\biggr)
\\
&=
E\biggl(
\sum_{k=0}^{m-1}G(t_k)(W(t_{k+1})-W(t_k))
\sum_{j=0}^{m-1}G(t_j)(W(t_{j+1})-W(t_j))
\biggr)
\\
&=
\sum_{k=0}^{m-1}
\sum_{j=0}^{m-1}
E(G(t_k) G(t_j)
(W(t_{k+1})-W(t_k))
(W(t_{j+1})-W(t_j))
)
\\
&=
\sum_{k=0}^{m-1}
E(G(t_k)^2) E(W(t_{k+1})-W(t_k))^2)
\\
&=
\sum_{k=0}^{m-1}
E(G(t_k)^2) (t_{k+1}-t_k)
=
E\biggl(
\sum_{k=0}^{m-1}
G(t_k)^2 (t_{k+1}-t_k)
\biggr)
=E\biggl(\int_0^T G(t)^2\,dt\biggr).
\end{align*}
Somit ist auch 3.~bewiesen.
\end{proof}

Die Bedeutung der dritten Eigenschaft besteht darin, dass eine
Folge von Stufen-Prozessen, die $L^2(0,T)$ konvergiert, zu einer
konvergenten Folge von It\^o-Integralen f"uhrt.

\begin{definition}
\index{It\^o-Integral}
Ist $G$ ein beliebiger auf $[0,T]$ quadratintegrierbarer Prozess,
und $G_n$ eine Folge von stochastischen Prozessen, die in $L^2([0,T])$
gegen $G$ konvergiert.
Dann ist des It\^o-Integral von $G$
\[
\int_0^T G\,dW
=
\
\lim_{n\to\infty} \int_0^T G_n\,dW
\]
\end{definition}

Das It\^o-Integral hat die folgenden Eigenschaften:

\begin{satz}
\label{satz:ito-integral}
Seien $G$ und $H$ auf $[0,T]$ quadratintegrierbare stochastische Prozesse
und $a,b\in\mathbb R$. Dann gilt
\begin{compactenum}
\item Das It\^o-Integral ist linear:
\[
\int_0^T aG+bH\,dW
=
a\int_0^TG\,dW
+
b\int_0^TH\,dW
\]
\item Der Erwartungswert des It\^o-Integrals ist
\[
E\biggl(\int_0^T G\,dW\biggr)=0
\]
\item Die Varianz des It\^o-Integrals ist
\[
E\biggl(\biggl(\int_0^TG\,dW\biggr)^2\biggr)
=
E\biggl(\int_0^T G^2\,dt\biggr).
\]
\end{compactenum}
\end{satz}

Mit dem It\^o-Integral k"onnen wir jetzt alle Komponenten einer
stochastischen Differentialgleichung definieren.
Zun"achst k"onnen wir f"ur jeden auf $[0,T]$ quadratintegrierbaren Prozess $G$ 
und f"ur jede integrierbare Funktion $F\in L^1([0,T])$ die Gr"ossen
\begin{align*}
\int_s^r F(t)\,dt&=\int_0^r F(t)\,dt - \int_0^s F(t)\,dt
\\
\int_s^r G\,dW&=\int_0^r G\,dW - \int_0^s G\,dW
\end{align*}
definieren.
\begin{definition}
Wenn $X$ ein stochastischer Prozess ist, der
\[
X(r)=X(s)+\int_s^rF(t)\,dt + \int_s^tG\,dW
\]
erf"ullt, dann sagt man $X$ habe das {\em stochastische Differential}
\[
dX=F\,dt + G\,dW.
\]
\end{definition}

%
%
%
\subsection{Rechenregeln}
Bisher wissen wir nur, dass das It\^o-Integral linear ist, dies reicht
aber nicht f"ur einen Kalk"ul, mit dem wir hoffen k"onnen,
Differentialgleichungen zu l"osen.u
Dazu brauchen wir mindestens noch eine Kettenregel und eine
Produktregel.

\begin{satz}[It\^o's Kettenregel]
\index{It\^o-Kettenregel}
Falls $X$ das stochastische Differential
\[
dX=F\,dt + G\,dW
\]
hat, und falls $u\colon \mathbb R\times [0,T]\to\mathbb R$ eine
stetig differenzierbare Funktion.
Dann hat die Funktion $Y(t)=u(X(t), t)$ das stochastische Differential
\begin{align*}
du(X,t)
&=\frac{\partial u}{\partial t}\,dt + \frac{\partial u}{\partial x}\,dX 
+\frac12\frac{\partial u^2}{\partial x^2}G^2\,dt
\\
&=
\biggl(
\frac{\partial u}{\partial t}+\frac{\partial u}{\partial x}F
+\frac12\frac{\partial^2u}{\partial x^2}G^2
\biggr)\,dt
+
\frac{\partial u}{\partial x}G\,dW.
\end{align*}
\end{satz}
Bis auf den Term in den zweiten Ableitungen sind die Terme genau
diejenigen, die man von der klassischen Kettenregel her erwarten
k"onnte.

\begin{beispiel}
Potenzen eines Wiener-Prozesses.
Sei $u(x)=x^m$ und $X=W$, $F=0$ und $G=1$.
Also ist $Y=W^m$.
Dann hat nach der It\^o-schen Kettenregel das stochastische Differential
\[
d(W^m)
=
\frac12m(m-1)W^{m-2}\,dt + mW^{m-1}\,dW.
\]
Im Spezialfall $m=2$ folgt
\[
dW^2 = 2W\,dW + dt.
\]
\end{beispiel}

\begin{beispiel} Wir betrachten die Funktion
$u(x,t)=e^{\lambda x-\frac12\lambda^2 t}$, und wie vorhin $X=W$, $F=0$
und $G=1$.
Es folgt
\begin{align*}
d\biggl(
e^{\lambda W(t)-\frac12\lambda^2 t}
\biggr)
&=
\biggl(
-\frac12\lambda^2 e^{\lambda W-\frac12\lambda^2 t}
+
\frac12\lambda^2 e^{\lambda W-\frac12\lambda^2 t}
\biggr)\,dt
+
\lambda e^{\lambda W -\frac12\lambda^2 t}\,dW
\\
dY&=\lambda Y\,dW
\end{align*}
also ist der Prozess $Y$ eine L"osung der stochastischen Differentialgleichung
\begin{align*}
dy&=\lambda Y\,dW
\\
Y(0)&=1.
\end{align*}
Dieses Beispiel zeigt, dass der entwickelte Kalk"ul dazu geeignet sein kann,
stochastische Differentialgleichungen zu l"osen.
\end{beispiel}

\begin{satz}[Produktregel von It\^o]
\index{It\^o-Produktregel}
\index{Produktregel von It\^o}
Falls die Prozesse $X_1$ und $X_2$ die stochastischen Differentiale
\begin{align*}
dX_1
&=
F_1\,dt + G_1\,dW
\\
dX_2
&=
F_2\,dt + G_2\,dW
\end{align*}
haben, dann ist
\begin{equation}
d(X_1X_2)
=
X_2\,dX_1 + X_1\,dX_2 + G_1G_2\,dt,
\label{stochastisch:ito-produkt}
\end{equation}
dies ist die {\em It\^o-sche Produktformel}.
\end{satz}

F"ur den Beweis der Kettenregel brauchen wir die Produktregel, wir geben
daher zuerst einen Beweis f"ur die Produktregel an.

\begin{proof}[Beweis der Produktformel]
Die integrierte Form der Produktregel besagt, dass 
\begin{equation}
X_1(r)X_2(r)-X_1(0)X_2(0)
=
\int_0^r X_2\,dX_1 + \int_0^r X_1\,dX_2 + \int_0^r G_1G_2\,dt
\label{stochastisch:ito-produkt-integriert}
\end{equation}
diese m"ussen wir nachrechnen.
Wir nehmen an der Einfachheit halber an, dass $X_1(0)=X_2(0)=0$ ist.

1.~Wir nehmen zus"atzlich an, dass die Funktion $G_i$ und $F_i$ nicht von
der Zeit abh"angen.
Dies bedeutet, dass sich die stochastische Differentialgleichung
$dX_i=F_i\,dt+G_i\,dW$ integrieren l"asst:
\[
X_i(t) = \int_0^t F_i\,d\tau + G_i\int_0^r dW =  F_it+G_iW(t).
\]
Diese Bedingung werden wir im zweiten Schritt wieder aufheben.
Wir berechnen die rechte Seite der integrierten
Produktregel~(\ref{stochastisch:ito-produkt-integriert}):
\begin{align*}
\int_0^r X_2\,dX_1 + X_1\,dX_2 + G_1G_2\,dt
&=
\int_0^rX_1F_2+X_2F_1\,dt + \int_0^r X_1G_2+X_2G_1\,dW + \int_0^r G_1G_2\,dt
\\
&=
\int_0^r(F_1t+G_1W)F_2+(F_2t+G_2W)F_1\,dt
\\
&\qquad
+ \int_0^r (F_1t+G_1W)G_2+(F_2t+G_2W)G_1\,dW + \int_0^r G_1G_2\,dt
\\
&=F_1F_2r^2+(G_1F_2+G_2F_1)\underbrace{\biggl(\int_0^rW\,dt + \int_0^rt\,dW\biggr)}_{\textstyle = rW(r)}
\\
&\qquad
+G_1G_2\cdot \underbrace{2\int_0^r W\,dW}_{\textstyle W(r)^2-r}
+ G_1G_2\underbrace{\int_0^r\,dt}_{\textstyle =r}
\\
&= 
F_1F_2r^2 + (G_1F_2+G_2F_1)rW(r) + G_1G_2W(r)^2
\\
&= 
F_1r\cdot F_2r + F_2r\cdot G_1W(r)+F_1r\cdot G_2W(r) + G_1W(r)\cdot G_2W(r)
\\
&= 
(F_1r + G_1W(r))\cdot(F_2r +  G_2W(r))
=
X_1(r) \cdot X_2(r).
\end{align*}
Dies ist die linke Seite von (\ref{stochastisch:ito-produkt-integriert}), die
damit f"ur diesen Spezialfall bewiesen ist.

2.~Wir ersetzen jetzt die Konstanten $G_i$ und $F_i$ durch Stufenprozesse.
Es gibt eine Unterteilung des Intervals $[0,T]$ so, dass die Prozesse $G_i$
und $F_i$ in jedem Teilinterval konstant sind.
In jedem solchen Teilinterval gilt daher die Ito-Produktformel, und damit
auch f"ur das ganze Interval.

3.~Im allgemeinen Fall k"onnen wir die Prozesse $G_i$ und $F_i$ mit Hilfe
einer Formel von Stufenprozessen $G_i^n$ und $F_i^n$ approximieren.
Wir k"onnen zus"atzlich fordern, dass
\[
E\biggl(\int_0^T |F_i^n-F_i|\,dt\biggr)\to 0
\qquad\text{und}\qquad
E\biggl(\int_0^T (G_i^n-G_i)^2\,dt\biggr)\to 0,
\]
was wir weiter unten brauchen.
Es gibt nach dem zweiten Schritt eine Folge von Prozessen $X_i^n$ mit
\[
X_i^n(t) = X_i(0) + \int_0^t F_i^n\,d\tau + \int_0^t G_i^n\,dW,
\]
f"ur die jeweils die Produktregel gilt.
\begin{align*}
X_i^n(r)X_i^n(r)-X_i^n(s)X_i^n(s)
&=
\int_0^r X_2^n\,dX_1^n+\int_0^r X_1^n\,dX_2^n + \int_0^r G_1^nG_2^n\,dW
\\
&=
\int_0^r X_2^nF_1^n\,dt
+
\int_0^r X_2^nG_1^n\,dW
+
\int_0^r X_1^nF_2^n\,dt
+
\int_0^r X_1^nG_2^n\,dW
+
\int_0^r G_1^nG_2^n\,dW
\\
\intertext{Der Grenz"ubergang $n\to\infty$ f"uhrt auf}
X_i(r)X_i(r)-X_i(s)X_i(s)
&=
\int_0^r X_2F_1\,dt
+
\int_0^r X_2G_1\,dW
+
\int_0^r X_1F_2\,dt
+
\int_0^r X_1G_2\,dW
+
\int_0^r G_1G_2\,dW
\end{align*}
Damit ist die Produktformel bewiesen.
\end{proof}

\begin{proof}[Beweis der Kettenregel]
Der Beweis der Kettenregel verwendet, dass die Funktion $u(x,t)$ durch
Polynome approximiert werden kann.
Wir m"ussen also zun"achst die Kettenregel f"ur Polynome in $x$ beweisen,
und dann mit Hilfe eines Grenz"ubergangs mit approximierenden Polynomen
den allgemeinen Fall gewinnen.

1. Sei $u(x)=x^m$, wir behaupten, dass
\[
d(X^m)=mX^{m-1}\,dX + \frac12m(m-1)X^{m-2}G^2\,dt,
\]
dies ist die It\^o-Produktformel f"ur $u(x)=x^m$.
Wir beweisen diese Formel durch vollst"andige Induktion.
F"ur $m=0$ ist sie trivialerweise korrekt.
Wir nehmen daher an, dass Sie f"ur $m-1$ gilt, dass also
\begin{align*}
d(X^{m-1})
&=
(m-1)X^{m-2}\,dX + \frac12(m-1)(m-2)X^{m-3}G^2\,dt
\\
&=
(m-1)X^{m-2}F\,dt + (m-1)X^{m-2}G\,dW + \frac12(m-1)(m-2)X^{m-3}G^2\,dt.
\end{align*}
Dies wenden jetzt die Produktregel auf $X^m = XX^{m-1}$ an.
F"ur $X_1=X$ ist $G_1=G$, und $X_2=X^{m-1}$ und
\[
G_2
=
(m-1)X^{m-2}G
\]
\begin{align*}
d(X^m)
&=
d(XX^{m-1})
=
X\,d(X^{m-1}) + X^{m-1}\,dX +G_1G_2\,dt
\\
&=
X\biggl( (m-1)X^{m-2}\,dX + \frac12(m-1)(m-2)X^{m-3}G^2\,dt\biggr)
+
X^{m-1}\,dX
+
(m-1)X^{m-2}G^2\,dt
\\
&=
mX^{m-1}\,dX
+\biggl(\frac12(m-2)+1\biggr)(m-1)X^{m-2}G^2\,dt
\\
&=
mX^{m-1}\,dX +\frac12m(m-1)X^{m-2}G^2\,dt,
\end{align*}
damit ist der Induktionsschritt vollzogen.

Wegen der Linearit"at der Ableitung ist die Kettenregel damit
f"ur beliebige Polynome in $x$ bewiesen.

2.~Wir approximieren die Funktion $u(x,t)$ jetzt als Produkt
$u(x,t)=f(x)g(t)$, wobei $f(x)$ und $g(t)$ Polynome sein sollen.
Dann gilt
\begin{align*}
d(u(X,t))
&=
d(f(X)g)
=
f(X)\,dg + g\,df(X)
\\
&=
f(X)g'\,dt + g\biggl(f'(X)\,dX + \frac12 f''(X)G^2\,dt\biggr)
\\
&=
\frac{\partial u}{\partial t}\,dt
+
\frac{\partial u}{\partial x}\,dX
+
\frac12\frac{\partial^2u}{\partial x^2}G^2\,dt.
\end{align*}
Die It\^o-Kettenregel ist damit gezeigt f"ur ein Produkt von Polynomen.

3.~Sei jetzt $u^n$ eine Folge von Polynomen wie im zweiten Schritt,
so dass $u^n$ sowie die Ableitungen gegen $u$ gleichm"assig konvergieren:
\[
u^n\to u,
\qquad
\frac{\partial u^n}{\partial t} \to \frac{\partial u}{\partial t},
\qquad
\frac{\partial u^n}{\partial x} \to \frac{\partial u}{\partial x},
\qquad
\frac{\partial^2 u^n}{\partial x^2} \to \frac{\partial^2 u}{\partial x^2}.
\]
F"ur jedes $u^n$ gilt die It\^o-Kettenregel:
\[
u^n(X(r),r)-u^n(X(0),0)
=
\int_0^r
\frac{\partial u^n}{\partial t}
+
\frac{\partial u^n}{\partial x}F
+
\frac12 \frac{\partial^2 u^n}{\partial x^2}G^2\,dt
+
\int_0^r\frac{\partial u^n}{\partial x}G\,dW,
\]
durch Grenz"ubergang $n\to \infty$ erhalten wir daraus die allgemeine
Form der It\^o-Kettenregel.
\end{proof}

%
% Beispiele 
%
\subsection{L"osungen von stochastischen Differentialgleichungen}
In diesem Abschnitt betrachten wir einige Beispiele von stochatischen
Differentialgleichungen und verwenden die im vorangegangenen Abschnitt
diskutierten Rechenregeln verwendet werden k"onnen, die L"osungen
zu finden.

\subsubsection{Lineare stochastische Differentialgleichung mit $F=0$}
Sei $g$ eine stetige Funktionen, man finde eine L"osung der
Differenzialgleichung
\begin{equation}
\begin{aligned}
dX&=gX\,dW\\
X(0)&=1.
\end{aligned}
\label{stochastisch:beispiel1-dgl}
\end{equation}
Man beachte, dass ohne den stochastischen Einfluss nur noch die Gleichung
$dX=0$ "ubrig bleibt, in diesem Fall ist also $X$ einfach nur eine
Konstante, $X(t)=1$.
Wir k"onnen aber auch nicht erwarten, dass der Erwartungswert der
L"osung durch die L"osung $X(t)=1$ gegeben ist.
Die Gleichung ist zwar linear, aber die Wirkung der stochastischen
Schwankungen ist proportional zum aktuellen Wert von $X$.

W"are $W$ einfach nur eine differenzierbare Funktion, dann w"urde die
Differentialgleichung~(\ref{stochastisch:beispiel1-dgl}) zu der
gew"ohnlichen Differentialgleichung
\begin{align*}
\frac{\dot X}{X} &= g\dot W,
\\
\intertext{die durch direkte Integration gel"ost werden kann:}
\frac{d}{dt}\log X&= g(t) \dot W(t)
\\
\log X(t)&=\int_0^t g(\tau)\dot W(\tau)\,d\tau,
\\
X(t)&=e^{\int_0^t g(\tau)\dot W(\tau)\,d\tau}.
\end{align*}
Daraus k"onnte man ableiten, dass f"ur einen Wiener-Prozess $W$ die L"osung
\[
X(t)=e^{\int_0^t g\,dW}
\]
sein m"usste.
Um dies zu kontrollieren, schreiben wir 
\[
Y(t)=\int_0^t g\,dW
\qquad\Rightarrow\qquad
dY = g\,dW = \underbrace{0}_{\textstyle F}\,dt
+
\underbrace{g}_{\textstyle G}\,dW
\]
f"ur das Integral im Exponenten, und beachten, dass $X(t)=e^{Y(t)}$.
Wir haben die Funktionen $F$ und $G$ wie in der Formulierung der
It\^o-Kettenregel gew"ahlt.

Um zu kontrollieren, ob $X(t)$ tats"achlich die L"osung der
Differentialgleichung~(\ref{stochastisch:beispiel1-dgl}), m"ussen wir
also $e^{Y(t)}$ ableiten, dazu ist die It\^o-Kettenregel zu verwenden,
denn $X(t)=u(Y(t))$ mit $u(x)=e^x$.
Die It\^o-Kettenregel ergibt:
\begin{align*}
dX
&=
\biggl( u'F+\frac12u''G^2 \biggr)\,dt + u' G\,dW
\\
&=
\underbrace{e^Y}_{\textstyle X}\biggl(\frac12g^2\,dt + g\,dW\biggr)
\\
&=
\frac12g^2X\,dt + gX\,dW
\end{align*}
Offensichtlich ist das weit davon entfernt, die L"osung der
Differentialgleichung zu sein.
Insbesondere der erste Term mit $g^2$, der von der It\^o-Kettenregel
beigesteuert wird, ist zu viel.

Um den Term der zweiten Ableitungen wieder los zu werden, versuchen wir
den Ansatz
\begin{equation}
X(t) = e^{-\frac12\int_0^tg^2\,d\tau+\int_0^t g\,dW}.
\label{stochastisch:beispiel1-lsg}
\end{equation}
Um dies nachzupr"ufen schreiben wir wieder 
\[
Y(t)
=
-\frac12\int_0^tg^2\,d\tau+\int_0^t g\,dW
\]
f"ur den Exponenten, dies bedeutet, dass
\[
dY
=
\underbrace{-\frac12g^2}_{\textstyle F}\,dt
+
\underbrace{g}_{\textstyle G}\,dW.
\]
Wir wenden die It\^o-Kettenregel auf die Funktion
$X(t)=u(Y(t))=e^{Y(t)}$, also $u(x)=e^x$, an.
\begin{align*}
dX
&=
\biggl( u'F + \frac12u''G^2\biggr)\,dt + u'G\,dW
\\
&=
\underbrace{\biggl(-\frac12 g^2 + \frac12 g^2\biggr)}_{\textstyle=0}\,dt
+
\underbrace{e^Y}_{\textstyle X}g\,dW
\\
&=gX\,dW.
\end{align*}
Damit haben wir nachgewiesen, dass~(\ref{stochastisch:beispiel1-lsg})
die L"osung der stochastischen
Differentialgleichung~(\ref{stochastisch:beispiel1-dgl}) ist.

Da $g^2\ge 0$ ist, ist das Integral von $g^2$ im Exponenten
immer positiv, somit ist $X(t)$ immer um einen Faktor $\le1$ 
kleiner als die urspr"unglich vermutete L"osung:
\[
X(t)
=
\underbrace{e^{-\frac12 \int_0^tg^2\,d\tau}}_{\textstyle \le 1}\cdot e^{\int_0^tg\,dW}.
\]
Im Spezialfall $g=1$ ist finden wir, dass
\[
X(t)=e^{-\frac12t}e^{W(t)}=e^{W(t)-\frac12t}
\]
die L"osung der Differentialgleichung $dX=X\,dW$ ist.

\subsubsection{Lineare stochastische Differentialgleichung}
Die allgemeinste lineare stochastische Differentialgleichung hat die Form
\begin{equation}
dX=fX\,dt+gX\,dW
\label{stochastisch:beispiel2-dgl}
\end{equation}
mit stetigen Funktionen $f$ und $g$.
Nehmen wir wieder an, dass $W$ eine differenzierbare Funktion ist,
dann m"usste $X$ die Differentialgleichung
\begin{align*}
dX&=fX\,dt + gX\dot W\,\,dt
\\
\intertext{erf"ullen, die wieder durch Integration gel"ost werden kann:}
\frac{\dot X}{X}
&=
f+g\dot W
\\
\Rightarrow\qquad
\log X(t)
&=
e^{\int_0^t f+g\dot W\,d\tau}
=
e^{\int_0^t f\,d\tau+ \int_0^t g\,dW}.
\end{align*}
Wir verzichten darauf nachzurechnen, dass dies nicht die L"osung sein kann,
denn wie vorhin wird ein Term fehlen.
Wir vermuten, dass
\begin{equation}
X(t)
=
e^{\int_0^t f -\frac12g^2\,d\tau + \int_0^t g\,dW}
\label{stochastisch:beispiel2-lsg}
\end{equation}
die L"osung ist.
Wir setzen daher wieder
\[
Y(t)=\int_0^t f-\frac12g^2\,d\tau + \int_0^tg\,dW,
\]
der stochastische Prozess $Y$ erf"ullt die stochastische Differentialgleichung
\[
dY=
\underbrace{\biggl( f-\frac12g^2\biggr)}_{\textstyle F}\,dt
+
\underbrace{g}_{\textstyle G}\,dW.
\]
Wir vermuten, dass $X(t)=u(Y(t))=e^{Y(t)}$ die L"osung der
Differentialgeichung ist.

Wir wenden die It\^o-Kettenregel auf $X(t)$ an und finden
\begin{align*}
dX
&=
\biggl(u'F+\frac12u''G^2\biggr)\,dt + u'G\,dW
\\
&=
e^Y\biggl( f\underbrace{-\frac12g^2+\frac12 g^2}_{\textstyle=0}\biggr)\,dt
+
e^Yg\,dW
\\
&=
Xf\,dt + Xg\,dW.
\end{align*}
Sind $f$ und $g$ konstante Funktionen, also $f(t)=a$ und $g(t)=b$, dann
kann man $Y$ direkt berechnen:
\[
Y(t)=\biggl(a-\frac12b^2\biggr)t + bW(t),
\]
also gilt
\[
X(t)=e^{(a-\frac12b^2)t} e^{bW(t)}.
\]

\subsubsection{Langevin-Gleichung}
Der Wiener-Prozess versucht, die Brownsche Bewegung zu modellieren.
Er kann aber nur funktionieren, wenn sich das Teilchen im Wesentlichen
reibungsfrei bewegen kann.
Ein realistischeres Modells stammt von Langevin, der stochastische 
Prozess $X$ beschreibt die Geschwindigkeit des Teilchens, und erf"ullt
die stochastische Differentialgleichung
\begin{equation}
\begin{aligned}
dX&=-bX\,dt+\sigma\,dW\\
X(0)&=X_0
\end{aligned}
\label{stochastisch:langevin-dgl}
\end{equation}
Der Koeffizient $b$ beschreibt die Reibung, $\sigma$ die Wirkung der
Diffusion.
\index{Langevin-Gleichung}
F"ur $\sigma=0$ bleibt die gew"ohnliche Differentialgleichung
\[
\frac{\dot X}{X}=-b
\qquad\Rightarrow\qquad
\log X(t)=-bt
\qquad\Rightarrow\qquad
X(t)=e^{-bt}X_0.
\]
Der stochastische Term h"angt in dieser Differentialgleichung nicht von $X$
ab, wir k"onnen daher erwarten, dass die Differentialgleichung wenigstens
formal die gleiche L"osung haben wird wie eine gew"ohnliche
inhomogene Differentialgleichung.
Wir vermuten daher, dass
\begin{equation}
X(t)
=
e^{-bt}X_0 + \sigma\int_0^te^{-b(t-\tau)}\,dW
=
e^{-bt}\biggl(X_0 + \sigma\int_0^te^{b\tau}\,dW\biggr)
\label{stochastisch:langevin-lsg}
\end{equation}
die L"osung der Langevin-Gleichung~(\ref{stochastisch:langevin-dgl})
sein wird.
Tats"achlich ist die Ableitung davon
\[
dX
=
-be^{-bt}
\underbrace{\biggl(X_0 + \sigma\int_0^te^{b\tau}\,dW\biggr)}_{\textstyle =X}\,dt
+
e^{-bt}
\sigma e^{bt}dW
=
-bX\,dt +\sigma\,dW,
\]
die Differentialgleichung ist also erf"ullt.

Die L"osung erlaubt, Erwartungswert und Varianz von $X(t)$ zu berechnen.
Dazu verwenden wir die Eigenschaften des Wiener-Prozesses und des
It\^o-Integrals in Satz~\ref{satz:ito-integral}.
F"ur den Erwartungswert erhalten wir.
\begin{align*}
E(X(t))
&=
e^{-bt}E(X_0)
+
\sigma \underbrace{E\biggl(\int_0^t e^{-b(t-\tau)}\,dW\biggr)}_{\textstyle=0}
=
e^{-bt}X_0.
\end{align*}
F"ur den Erwartungswert von $X(t)^2$ berechnen wir
\begin{align*}
E(X(t)^2)
&=
E\biggl(e^{-2bt}X_0^2+2\sigma e^{-bt}X_0\int_0^te^{-b(t-\tau)}\,dW
+\sigma^2\biggl(\int_0^t e^{-b(t-\tau)}\,dW\biggr)^2
\biggr)
\\
&=
e^{-2bt}E(X_0^2)
+
2\sigma e^{-bt}E(X_0)E\biggl( \int_0^te^{-b(t-\tau)}\,dW\biggr)
+
\sigma^2E\biggl(\biggl(\int_0^t e^{-b(t-\tau)}\,dW\biggr)^2\biggr)
\\
&=
e^{-2bt}E(X_0^2)
+
\sigma^2 \int_0^t e^{-2b(t-\tau)}\,d\tau
\\
&=
e^{-2bt}E(X_0^2)
+
\frac{\sigma^2}{2b}(1-e^{-2bt}).
\end{align*}
Die Varianz von $X(t)$ ist daher
\begin{align*}
\operatorname{var}(X(t))
&=
E(X(t)^2)-E(X(t))^2
=
e^{-2bt}E(X_0^2)
+
\frac{\sigma^2}{2b}(1-e^{-2bt}).
-
e^{-2bt}E(X_0)^2
\\
&=
e^{-2bt}\operatorname{var}(X_0)
+
\frac{\sigma^2}{2b}(1-e^{-2bt}).
\end{align*}
Die Varianz von $X(t)$ setzt sich also aus zwei Komponenten zusammen.
Sie ist das gewichtete Mittel der Varianz des Anfangsbedingung
$\operatorname{var}(X_0)$ und der Varianz des Diffusionsterms, n"amlich
$\sigma^2/2b$.
Die Gewichtsfaktoren sind $e^{-2bt}$ und $1-e^{-2bt}$, die sich zu
$1$ addieren.
Zur Zeit $t=0$ ist die Varianz nat"urlich nur die Varianz der Anfangsbedingung.
Mit zunehmender Zeit wird die Varianz der Anfangsbedingung unbedeutend,
und es bleibt nur die Varianz des Diffusionsterms "ubrig.

Der Integralterm in $X(t)$ ist normalverteilt, wie alles was aus dem
Wienerprozess entsteht.
Der Term $e^{-bt}X_0$ wird aber in $X(t)$ immer unbedeutender, somit ist
nach gen"ugend langer Zeit $t$ die Zufallsvariable $X(t)$ normalverteilt
mit Erwartungswert $0$ und Varianz $\sigma^2/2b$.

Kehren wir zur physikalischen Interpretation zur"uck, dann sehen wir,
dass die Geschwindigkeit eines dem Langenvin-Prozesses unterworfenen
Teilchens nach einiger Zeit seine Anfangsgeschwindigkeit ``vergisst'',
die Zeitkonstante daf"ur ist gegeben durch die Reibung $b$.
Die Schwangungen der Geschwindigkeit sind umso gr"osser, je geringer
die Reibung ist.

\subsection{Stochastische Differentialgleichungen 2. Ordnung}
Die meisten Bewegungsgleichungen der Physik sind Differentialgleichungen
zweiter Ordnung.
In diesem Abschnitt wollen wir daher zwei Beispiele untersichen, wie
bekannte Systeme sich unter Einfluss von Rauschen verhalten werden.

\subsubsection{Ornstein-Uhlenbeck-Prozess}
\index{Ornstein-Uhlenbeck-Prozess}
Die Langevin-Gleichung beschreibt die Geschwindigkeit eines Teilchens,
dass der Brownschen Bewegung mit Reibung unterworfen ist.
Wir m"ochten aber auch die Position des Teilchens kennen, wir braucht
also noch eine weitere Zufallsvariable $Y$, welche die Position
beschreibt.
Nat"urlich soll $X$ die Ableitung von $Y$ sein: $X=\dot Y$.
Als klassische Differentialgleichung w"urden wir die Bewegungsgleichung
schreiben als
\[
\ddot Y=-b\dot Y+\sigma\frac{dW}{dt},
\]
wobei der letzte Term nat"urlich nicht wirklich sinnvoll ist.
Um dieser Gleichung Sinn zu geben, schreiben wir sie als
Differentialgleichungssystem erster Ordnung mit den beiden Variablen
$Y_1=Y=\dot X$ und $Y_2=X$ mit den Gleichungen
\begin{align*}
\dot Y_1&=-b Y_1+\sigma \frac{dW}{dt}\\
\dot Y_2&=Y_1.
\end{align*}
Als stochastische Differentialgleichung mit Anfangsbedingungen
wird dies
\begin{equation}
\begin{aligned}
dY_1&=-bY_1\,dt +\sigma\,dW,
&&&
Y_1(0)&=Y_{10}
\\
dY_2&=Y_1\,dt
&&&
Y_2(0)&=Y_{20}
\end{aligned}
\label{stochastisch:ornstein-uhlenbeck-dgl}
\end{equation}
Die erste Gleichung f"ur $Y_1$ ist die Langevin-Gleichung, die wir bereits
in~\ref{stochastisch:langevin-lsg}) gel"ost wurde.
Durch eine weitere Integration kann man auch die Gleichung f"ur $Y_2$
l"osen:
\begin{align*}
Y_1(t)&=e^{-bt}Y_{10}+\int_0^te^{-b(t-\tau)}\,dW\\
Y_2(t)&=Y_{20}+\int_0^t Y_1(\tau)\,d\tau
\end{align*}

\subsubsection{Harmonischer Oszillator}
Der Ornstein-Uhlenbeck-Prozess war insofern einfach zu behandeln, als 
die Geschwindigkeit aus dem Langevin-Prozess einfach noch einmal
integriert werden musste.
Es gab also keine R"uckkopplung zwischen der zweiten Ableitung und
dem Prozess selbst.
Erst eine solche R"uckkopplung kann zu Schwingungen f"uhren.
Wir versuchen daher jetzt einen harmonischen Oszillator zu modellieren,
der zus"atzlich dem Rauschen aus einem Wiener-Prozess unterworfen ist.

Ein harmonischer Oszillator hat die gew"ohnliche Differentialgleichung
zweiter Ordnung
\begin{equation}
\ddot X=-\lambda^2 X-b\dot X, X(0)=X_0, \dot X(0)=X_1.
\label{stochastisch:harmosz-dgl2}
\end{equation}
Da wir zweite Ableitungen in stochastischen Differentialgleichungen
nicht verwenden k"onnen, ersetzen wir~(\ref{stochastisch:harmosz-dgl2})
wieder durch zwei gekoppelte Gleichungen erster Ordnung:
\begin{equation}
\begin{aligned}
\dot Y_1&=Y_2,                &&&Y_1(0)&=X_0, \\
\dot Y_2&=-\lambda^2 Y_1-bY_2,&&&Y_2(0)&=X_1.
\end{aligned}
\label{stochastsich:harmosz-dgl1}
\end{equation}
In dieser Form k"onnen wir jetzt in der Form eines Systems stochastischer
Differentialgleichungen formulieren:
\begin{align*}
dY_1&=Y_2\,dt,\\
dY_2&=(-\lambda^2Y_1-bY_2)\,dt.
\end{align*}
Wenn wir auf der rechten Seite von~(\ref{stochastisch:harmosz-dgl2})
einen Rauschterm, also eine Ableitung eines Wiener-Prozesses hinzuf"ugen
wollen, dann wird daraus das stochastische Differentialgleichungssystem
\begin{equation}
\begin{aligned}
dY_1&=Y_2\,dt,                                &&&Y_1(0)&=X_0,\\
dY_2&=(-\lambda^2 Y_1-bY_2)\,dt + \sigma\,dW, &&&Y_2(0)&=X_1.
\end{aligned}
\end{equation}

Wie bei den gew"ohnlichen Differentialgleichungen k"onnen wir diese
Gleichung einfacher l"osen, wenn wir sie in Matrixform schreiben:
\begin{equation}
d\begin{pmatrix}Y_1\\Y_2\end{pmatrix}
=
\underbrace{
\begin{pmatrix}
         0& 1\\
-\lambda^2&-b
\end{pmatrix}}_{\textstyle=D}
\begin{pmatrix}Y_1\\Y_2\end{pmatrix}\,dt
+
\begin{pmatrix}
\sigma\\0
\end{pmatrix}\,dW.
\end{equation}
Die L"osung wird dann
\begin{equation}
\begin{pmatrix}
Y_1(t)\\Y_2(t)
\end{pmatrix}
=
e^{Dt}\begin{pmatrix}X_0\\X_1\end{pmatrix}
+
\sigma \int_0^t e^{D(t-\tau)}\begin{pmatrix}0\\\sigma\end{pmatrix}\,dW.
\label{stochastisch:harmosz-explsg}
\end{equation}

In Abschnitt~\ref{linear:harmosz} wurde der Spezialfall $b=0$ bereits
behandelt.
Die dort gefundenen Formeln f"ur $e^{Dx}$ erm"oglichen, direkt eine
L"osung anzugeben:
%\begin{beispiel}
%Wir berechnen die L"osung f"ur den Spezialfall $b=0$.
%Die Matrix $D$ ist
%\[
%D=\begin{pmatrix}
%0&1\\-\lambda^2&0
%\end{pmatrix}.
%\]
%Die Matrix $e^{Dt}$ kann durch Diagonalisierung berechnet werden.
%Die Transformationsmatrix
%\[
T
%=
%\begin{pmatrix}
%1&1\\
%i\lambda&-i\lambda
%\end{pmatrix},
%\qquad
%T^{-1}
%=
%\frac12
%\begin{pmatrix}
%1& 1/i\lambda\\
%1&-1/i\lambda
%\end{pmatrix}
%\]
%bringt die Matrix $D$ in Diagonalform:
%\[
%T^{-1}DT
%=
%\begin{pmatrix}
%i\lambda&        0\\
%        &-i\lambda
%\end{pmatrix}.
%\]
%Daraus kann man jetzt die Exponentialfunktion berechnen:
%\begin{align*}
%T^{-1}e^{Dt}T
%&=
%\begin{pmatrix}
%\cos\lambda t+i\sin\lambda t&              0             \\
%              0             &\cos\lambda t-i\sin\lambda t
%\end{pmatrix}
%\\
%\Rightarrow\qquad\qquad
%e^{Dt}
%&=
%\begin{pmatrix}
%                \cos\lambda t&\frac1{\lambda}\sin\lambda t\\
%-\frac1{\lambda}\sin\lambda t&               \cos\lambda t
%\end{pmatrix}
%\end{align*}
%Daraus k"onnen wir jetzt die L"osung mit der
%Formel~(\ref{stochastisch:harmosz-explsg}) ablesen:
\begin{equation}
\begin{pmatrix}
Y_1(t)\\Y_2(t)
\end{pmatrix}
=
\begin{pmatrix}
                \cos\lambda t&\frac1{\lambda}\sin\lambda t\\
-\frac1{\lambda}\sin\lambda t&               \cos\lambda t
\end{pmatrix}
\begin{pmatrix}X_0\\X_1\end{pmatrix}
+
\sigma\int_0^t
\begin{pmatrix}
\frac1{\lambda}\sin\lambda(t-\tau)\\
               \cos\lambda(t-\tau)
\end{pmatrix}\,dW.
\label{stochastisch:harmosz-y2}
\end{equation}
Die erste Komponenten davon ist
\begin{equation}
Y_1(t)
=
X_0\cos\lambda t+\frac{X_1}{\lambda}\sin\lambda t
+
\frac{\sigma}{\lambda}\int_0^t\sin\lambda(t-\tau)\,dW,
\label{stochastisch:harmosz-y}
\end{equation}
die L"osung der Differentialgleichung.

Die L"osungsformel~(\ref{stochastisch:harmosz-y}) zeigt auch, dass
$Y_1(t)$ selbst dann von $0$ verschieden sein kann, wenn $X_0=X_1=0$ ist.
Wir wollen f"ur diesen Spezialfall Erwartungswert und Varianz 
von $Y_1(t)$ berechnen.
Wir verwenden wieder die Resultate von Satz~\ref{satz:ito-integral}.
Der Integralterm verschwindet bei der Berechnung des Erwartungswertes:
\begin{align*}
E(Y_1(t))
&=
E(X_0)\cos\lambda t
+
\frac{E(X_1)}{\lambda}\sin\lambda t
+
\frac{\sigma}{\lambda}E\biggl(\int_0^t\sin\lambda(t-\tau)\,dW\biggr)
\\
&=
E(X_0)\cos\lambda t+\frac{E(X_1)}{\lambda}\sin\lambda t.
\\
\intertext{Den Erwartungswert von $Y_1(t)^2$ k"onnen wir wie folgt berechnen:}
E(Y_1(t)^2)
&=
E(X_0^2)\cos^2\lambda t
+
2E(X_0X_1)\frac1{\lambda}\cos\lambda t\sin\lambda t
+
\frac{E(X_1^2)}{\lambda^2}\sin^2\lambda t
%+
%\frac{\sigma E(X_1)}{\lambda^2}E\biggl(\int_0^t\sin\lambda(t-\tau)\,dW\biggr)
%+
%2E(X_0)\frac{\sigma}{\lambda}\cos\lambda t
%E\biggl(\int_0^t\sin\lambda(t-\tau)\,dW\biggr)
%\\
%&\qquad
+
\frac{\sigma^2}{\lambda^2}E\biggl(
\biggl(\int_0^t\sin\lambda(t-\tau)\,dW\biggr)^2
\biggr)
\\
&=
E(X_0^2)\cos^2\lambda t
+
2E(X_0X_1)\frac1{\lambda}\cos\lambda t\sin\lambda t
+
\frac{E(X_1^2)}{\lambda^2}\sin^2\lambda t
+
\frac{\sigma^2}{\lambda^2}\int_0^t \sin^2\lambda(t-\tau)\,d\tau
\\
&=
E(X_0^2)\cos^2\lambda t
+
2E(X_0X_1)\frac1{\lambda}\cos\lambda t\sin\lambda t
+
\frac{E(X_1^2)}{\lambda^2}\sin^2\lambda t
+
\frac{\sigma^2}{4\lambda^3}
(2\lambda t -\sin 2\lambda t).
\\
\intertext{Die Varianz ist
$\operatorname{var}(Y_1(t))=E(Y_1(t)^2)-E(Y_1(t))^2$:}
\operatorname{var}(Y_1(t))
&=
\operatorname{var}(X_0)\cos^2\lambda t
+2\operatorname{cov}(X_0,X_1)\cos\lambda t\sin\lambda t
+
\frac{\sin^2\lambda t}{\lambda^2}\operatorname{var}(X_1)
+\frac{\sigma^2}{4\lambda^3}(2\lambda t-\sin2\lambda t).
\end{align*}
Wie im Beispiel der Langevin-Gleichung hat die Varianz
also zwei Komponenten.
Die ersten drei Komponenten beschreiben den Einfluss der Anfangsbedingungen.
Der letzte Term beschreibt die zus"atzliche Varianz, die durch
das Rauschen in den Oszillator eingebracht wird.

Der Term zu den Anfangsbedingungen kann in Matrixform geschrieben
werden:
\begin{gather*}
\operatorname{var}(X_0)\cos^2\lambda t
+2\operatorname{cov}(X_0,X_1)\cos\lambda t\sin\lambda t
+
\frac{\sin^2\lambda t}{\lambda^2}\operatorname{var}(X_1)
\qquad
\qquad
\qquad
\qquad
\\
\qquad
\qquad
\qquad
\qquad
=
\begin{pmatrix}
\cos\lambda t&\frac1\lambda\sin\lambda t
\end{pmatrix}
\begin{pmatrix}
\operatorname{var}(X_0)    &\operatorname{cov}(X_0,X_1)\\
\operatorname{cov}(X_0,X_1)&\operatorname{var}(X_1)
\end{pmatrix}
\begin{pmatrix}
\cos\lambda t\\
\frac1\lambda\sin\lambda t
\end{pmatrix}
\end{gather*}
Diese Form kann man auch erhalten, indem man die Kovarianzen mit Hilfe von
\[
\begin{pmatrix} Y_1(t)&Y_2(t) \end{pmatrix}
\begin{pmatrix}
Y_1(t)\\
Y_2(t)
\end{pmatrix}
=
\begin{pmatrix}
Y_1(t)^2&Y_1(t)Y_2(t)\\
Y_1(t)Y_2(t)&Y_2(t)^2
\end{pmatrix}
\]
berechnet, und darauf die Formel~(\ref{stochastisch:harmosz-y2}) anwendet.

%
% Partielle Differentialgleichungen
%
\section{Partielle Differentialgleichungen\label{section:pdgl}}
\rhead{Partielle Differentialgleichungen}
Die It\^osche Kettenregel stellt einen Zusammenhang her zwischen der zweiten
Ableitung von $u$ und den Werten von $u$ am Ende eines Pfades.
In diesem Abschnitt wollen wir den Zusammenhang wie folgt konkretisieren.
Wir werden zun"achst f"ur ein Gebiet $U$ und einen Punkt $x$ die
Zeit $\tau_x$ definieren, zu der eine Brownsche Bewegung das Gebiet zum
ersten Mal verl"asst.
Nat"urlich ist $\tau_x$ eine Zufallsvariable, und damit auch $u(X(\tau_x))$,
wobei der Pfad ist, der zur Zeit $\tau_x$ das Gebiet verl"asst.
Dann werden wir zeigen, dass $u(x)$ der Erwartungswert von $u(X(\tau_x))$
sein.

In der Theorie der quaslinearen partiellen Differentialgleichungen erster
Ordnung wird gezeigt, wie man die Funktionswerte der L"osungsfunktion
dadurch bestimmen kann, dass man vom Punkt $x$ aus die Charakteristik 
konstruiert, und den Wert der Randbedingung an dem Punkt ermittelt,
wo die Charakteristik das Gebiet verl"asst.
Aus diesem Wert l"asst sich dann der Wert der L"osung bestimmen.
Brownsche Bewegungen spielen also f"ur elliptische partielle
Differentialgleichungen eine "ahnliche Rolle wie Charakteristiken f"ur
quasliineare partielle Differentialgleichungen erster Ordnung.

\subsection{Stopzeiten}
Wie fr"uher betrachten wir wieder einen stochastischen Prozess $X(t)$,
der L"osung einer stochastischen Differentialgleichung
\begin{equation}
\begin{aligned}
dX(t)&=b(t,X)\,dt + B(t,X)\,dW
\\
X(0)&=X_0
\end{aligned}
\label{stochastisch:stopzeitdgl}
\end{equation}
sein soll.
\begin{figure}
\centering
\includegraphics{chapters/images/stochastisch-2.pdf}
\caption{Brownsche Bewegung in zwei Dimension und Definition der
Stopzeit $\tau$, zu der der Pfad $X(t)$ das Gebiet verl"asst.
\label{stochastisch:pfad}}
\end{figure}

\begin{definition}
Sei $E$ eine beliebige offene oder abgeschlossene nicht leere Teilmenge
von $\mathbb R^n$.
Dann setzen wir
\[
\tau = \inf\{t\ge 0\;|\;X(t)\in E\},
\]
$\tau$ ist also die fr"uheste Zeit, zu der der Weg $X(t)$ die Menge $E$
erreicht.
\end{definition}

Die charakteristische Funktion
\[
\chi_{[0,\tau]}(t)=\begin{cases}
1\qquad\qquad&t\le \tau\\
0            &t>\tau
\end{cases}
\]
ist nat"urlich auch ein stochastischer Prozess, und damit auch 
$\chi_{[0,\tau]}G$.
Damit gelten die Regeln f"ur das It\^o-Integral aus
Satz~\ref{satz:ito-integral} auch f"ur diesen Prozess, jetzt allerdings
als Integrale mit oberer Grenze $\tau$ statt $T$.

\begin{hilfssatz}
Seien $G$ und $H$ auf $[0,T]$ quadratintegrierbare stochastische Prozesse
und $a,b\in\mathbb R$
\begin{compactenum}
\item
Das It\^o-Integral ist linear:
\begin{align*}
\int_0^\tau aG+bH\,dW
&=
\int_0^T a\chi_{[0,\tau]} G + b \chi_{[0,\tau]} H\,dW
=
a\int_0^T \chi_{[0,\tau]} G\,dW + b \int_0^T\chi_{[0,\tau]} H\,dW
\\
&=
a\int_0^\tau G\,dW + b \int_0^\tau H\,dW
\end{align*}
\item Der Erwartungswert des It\^o-Integrals ist
\[
E\biggl(\int_0^\tau G\,dW\biggr)
=
E\biggl(\int_0^T \chi_{[0,\tau]}G\,dW\biggr)
=
0.
\]
\item Die Varianz des It\^o-Integrals ist
\[
E\biggl(\biggl(\int_0^\tau G\,dW\biggr)^2\biggr)
=
E\biggl(\biggl(\int_0^T \chi_{[0,\tau]}G\,dW\biggr)^2\biggr)
=
E\biggl(\int_0^T\chi_{[0,\tau]}G^2\,dt\biggr)
=
E\biggl(\int_0^\tau G^2\,dt\biggr).
\]
\end{compactenum}
\end{hilfssatz}

Die It\^o-sche Kettenregel funktioniert auch f"ur $\tau$ als obere
Grenze f"ur die stochastischen Integrale.
Wenn also $X$ als stochastischer Prozess eine L"osung der stochastischen
Differentialgleichung (\ref{stochastisch:stopzeitdgl}) ist, dann gilt
nach der der It\^o-schen Kettenregel
\[
du(X,t)
=
\frac{\partial u}{\partial t}\,dt
+
\sum_{i=1}^n\frac{\partial u}{\partial x_i}\,dX_i
+
\frac12\sum_{i,j=1}^n\frac{\partial^2 u}{\partial x_i\partial x_j}
\sum_{k=1}^n b_{ik}b_{jk}\,dt.
\]
In integrierter Form bedeutet dies
\[
u(X(t),t)-u(X(0),0)
=
\int_0^t
\frac{\partial u}{\partial t}
+
\frac12\sum_{k=1}^nb_{ik}b_{jk}
\sum_{i,j=1}^n \frac{\partial^2u}{\partial x_i\,\partial x_j} \,ds
+
\int_0^t \operatorname{grad} u\cdot B\,dW.
\]
Und nat"urlich gelten diese Formeln auch dann, wenn man $t$ durch $\tau$
ersetzt.

Im Folgenden interessiert uns nur der Fall $b=0$, $b_{ik}=\delta_{ik}$
und Funktionen $u$, die nicht von der Zeit abh"angen.
Dann vereinfacht sich die Formel zu
\begin{equation}
u(X(t))-u(X(0))
=
\int_0^t \frac12\sum_{i=1}^n\frac{\partial^2 u}{\partial x_i^2}\,ds
=
\int_0^t \frac12\Delta u\,ds
\label{stochastisch:laplaceinkrement}
\end{equation}
Ausserdem ist in diesem Fall $X$ nichts anderes als eine Brownsche Bewegung,
$X=W$.

\subsection{Brownsche Bewegung und der Laplace-Operator}
Wir wenden die Formel (\ref{stochastisch:laplaceinkrement}) jetzt in
zwei Beispielen an.

\subsubsection{Zeit bis zum Verlassen eines Gebietes}
F"ur das erste Beispiel sei $U$ ein beschr"anktes Gebiet in
$\mathbb R^n$, und $u$ eine L"osung der partiellen Differentialgleichung
\begin{equation}
\begin{aligned}
-\frac12\Delta u&=1&\qquad&\text{in $U$}\\
               u&=0&      &\text{auf $\partial U$}
\end{aligned}
\label{stochastisch:hittingtime}
\end{equation}
Wir m"ochten die L"osung $u$ dazu verwenden, die Zeit $\tau_x$ zu berechnen,
zu der eine im Punkt $x\in U$ beginnende Brownsche Bewegung zum ersten
Mal das Gebiet $U$ verl"asst.
Die Formel (\ref{stochastische:laplaceinkrement}) liefert
\[
u(X(\tau_x))-u(X(0)) = \int_0^{\tau_x} \frac12\Delta u\,ds
\]
Da $u$ eine L"osung von (\ref{stochastisch:hittingtime}) ist, ist der 
Integrand auf der rechten Seite gleich $-1$:
\[
u(X(\tau_x))-u(X(0)) = -\int_0^{\tau_x} \,ds
\]
Da der Prozess $X$ zur Zeit $\tau_x$ den Rand des Gebietes "uberquert,
ist wegen der Randbedingung $u(X(\tau_x))=0$. 
Zusammen erhalten wir
\[
-u(X(0)) = -\tau_x.
\]
Nun interessiert uns aber nicht der Wert der Zufallsvariablen, sondern
nur deren Erwartungswert:
\[
E(\tau_x)=E(u(X(0)))=u(x).
\]
Die L"osung $u$ der Differentialgleichung (\ref{stochastisch:hittingtime})
gibt die erwartete Zeit an, bis eine bei $x$ beginnende Brownsche Bewegung 
das Gebiet $U$ verlassen hat.

\subsubsection{Charakterisierung von harmonischen Funktionen}
Sei wieder $U$ ein beschr"anktes Gebiet mit glattem Rand und
$g$ eine stetige Funktion auf $\partial U$.
Ausserdem sei $u$ eine harmonische Funktion in $U$ mit den Randwerten $g$, 
also eine L"osung der partiellen Differentialgleichung
\begin{equation}
\begin{aligned}
\Delta u&=0&\qquad&\text{in $U$}\\
       u&=g&      &\text{auf $\partial U$}
\end{aligned}
\label{stochastisch:harmonisch}
\end{equation}
Wir wollen (\ref{stochastisch:laplaceinkrement}) verwenden, die Funktion
$u$ zu charakterisieren.
Dazu sei wieder $\tau_x$ die Zeit, zu der eine im Punkt $x\in U$ beginnende
Brownsche Bewegung das Gebiet $U$ verl"asst.
Aus (\ref{stochastisch:laplaceinkrement}) folgt:
\[
u(X(\tau_x))-u(X(0))
=
\int_0^{\tau_x} \frac12\underbrace{\Delta u}_{\textstyle=0}\,ds=0.
\]
Zur Zeit $\tau_x$ ist $X(\tau_x)$ ein Randpunkt des Gebiets, der mit
der Randbedingung bestimmt werden kann, es folgt:
\[
u(X(0)) = g(X(\tau_x)).
\]
Wieder interessiert uns der einzelne Wert nicht, sondern der Erwartungswert:
\[
u(x)=E(g(X(\tau_x))).
\]
Den Wert im Punkt $x$ einer harmonischen Funktion mit Randwerten $g$
kann man wie folgt finden: man l"asst eine Brownsche Bewegung von $x$ 
aus laufen, bis sie den Rand "uberquert, und nimmt den Mittelwert
der derart erreichten Randwerte.

Aus dieser Charakterisierung der harmonischen Funktionen kann man
auch deren Mittelwerteigenschaft ableiten. 
Da die Brownsche Bewegung isotrop ist, muss sich im Zentrum
eines kugelf"ormigen Gebietes immer der gleiche Wert f"ur $u$
ergeben, selbst wenn man eine beliebige Drehung auf die Randwerte 
anwendet.
Also muss der Wert von $u(x)$ der Mittelwert der Werte
auf einer Kugel um den Punkt $x$ sein.

%
%
%
\section{Kalman-Filter\label{section:kalman}}
\rhead{Kalman-Filter}






\begin{appendices}
\chapter{Newton-Verfahren}
\lhead{Newton-Verfahren}
\rhead{}

\section{Nullstellen von Funktionen}
\rhead{Nullstellen von Funktionen}
\begin{figure}
\centering
\includegraphics{chapters/images/randwert-2.pdf}
\caption{Bestimmung der Nullstelle einer Funktion $f(x)$ mit dem
Newton-Verfahren.
Die Approximation $x_{n+1}$ wird gefunden als Schnittpunkt der Tangente
im Punkt $(x_n,f(x_n))$ (mit Steigung $f'(x_n)$) mit der $x$-Achse.
\label{newton:graphik}}
\end{figure}
Das Ziel dieses Anhangs ist, die folgende Aufgabe numerisch zu l"osen:
\begin{aufgabe}
Gegen ist eine differenzierbar Funktion
$f\colon\mathbb R\to\mathbb R:x\mapsto f(x)$
und eine Zahl $y$ im Wertebereich von $f$.
Finde $\hat{x}\in\mathbb R$ so, dass $f(\hat{x})=y$.
\end{aufgabe}
Im allgemeinen kann man nicht davon ausgehen, dass sich eine L"osung der
Gleichung $f(x)=y$ in geschlossener Form finden l"asst.
Nur einige wenige Klassen von Gleichungen haben L"osungsformeln dieser Art.
Wir beschr"anken uns daher auf das Problem, eine Approximation f"ur die
L"osung zu bestimmen.

Indem wir statt der Funktion $f(x)$ die Funktion $x\mapsto g(x)=f(x)-y$
betrachten, k"onnen wir die gesuchte Zahl $x$ auch als L"osung der
Gleichung $g(x)=0$ finden:
\begin{equation}
f(x)=y
\qquad\qquad
\Rightarrow
\qquad\qquad
g(x)=f(x)-y = 0.
\label{newton:reduktion}
\end{equation}
Es gen"ugt also, ein L"osungsverfahren zu entwickeln f"ur die Aufgabe
\begin{aufgabe}
Gegen ist eine differenzierbar Funktion
$f\colon\mathbb R\to\mathbb R:x\mapsto f(x)$,
finde $\hat{x}\in\mathbb R$ so, dass $f(\hat{x})=0$.
\end{aufgabe}
Da wir nur eine numerische L"osung brauchen, versuchen wir sie dadurch
zu finden, dass wir eine Anfangssch"atzung $x_0$ wiederholt korrigieren,
bis der Fehler klein genug ist.
Es soll also eine Folge $x_0,x_1,x_2,\dots$ konstruiert werden, welche
gegen die L"osung $\hat{x}$ konvergiert.
Der Differenzenquotient ist eine Approximation f"ur die Steigung
$f'(x_n)$,
\begin{equation}
\frac{
f(x_{\mathstrut n+1})-f(x_{\mathstrut n})
}{
x_{\mathstrut n+1}-x_{\mathstrut n}
}
\simeq = f'(x_n).
\label{newton:pre}
\end{equation}
Wir m"ochten gerne, dass $f(x_{n+1})=0$ ist, und k"onnen (\ref{newton:pre})
unter dieser Annahme nach $x_{n+1}$ aufl"osen:
\begin{align*}
-f(x_n)
&\simeq
f'(x_n)\,(x_{\mathstrut n+1}-x_{\mathstrut n})
\\
x_{\mathstrut n}-\frac{f(x_n)}{f'(x_n)}
&\simeq x_{\mathstrut n+1}
\end{align*}
Damit haben wir ein L"osungsverfahren gefunden:
\begin{satz}[Newton]
Ist $f$ eine differenzierbare Funktion, deren Ableitung bei der Nullstelle
$\hat{x}$ nicht verschwindet, also $f(\hat{x})\ne 0$, und $x_0$ eine erste
Approximation f"ur $\hat{x}$, dann konvergiert die Folge
definiert durch die Rekursionsformel
\[
x_{n+1}=x_n-\frac{f(x_n)}{f'(x_n)},
\]
dann konvergiert $x_n$ gegen $\hat{x}$, falls $x_0$ nahe genug bei
$\hat{x}$ liegt.
\end{satz}

\begin{beispiel}
Man finde die Wurzel der Zahl $y$, d.~h.~man muss die Nullstellen
der Funktion $f(x)=x^2-y$ finden.
Das Newton-Verfahren ben"otigt die Ableitung von $f$, sie ist
$f'(x)=2x$, und konstruiert daraus die Folge
\begin{equation}
x_{n+1} = x_n - \frac{f(x_n)}{f'(x_n)}=x_n-\frac{x_n^2-y}{2x_n}
=
\frac{2x_n^2-x_n^2+y}{2x_n}
=
\frac12\biggl(x_n + \frac{y}{x_n}\biggr).
\label{newton:mittel}
\end{equation}
Die Quaratwurzel von $y$ erf"ullt nat"urlich
\[
\sqrt{y} = \frac12\biggl( \sqrt{y}+\frac{y}{\sqrt{y}}\biggr).
\]
Mit $x_n$ ist auch $y/x_n$ eine Approximation von $\sqrt{y}$.
Die neue Approximation $x_{n+1}$ ist das arithmetische Mittel der
beiden Approximationen $x_n$ und $y/x_n$ von $\sqrt{y}$.
Die Konvergenz dieser Folge ist sehr schnell, wie Tabelle~\ref{newton:sqrt2}
zeigt.
\begin{table}
\centering
\begin{tabular}{|>{$}r<{$}|>{$}r<{$}|}
\hline
n&x_n\\
\hline
0 &  2.00000000000000\\
1 &  1.50000000000000\\
2 &  1.\underline{41}666666666667\\
3 &  1.\underline{41421}568627451\\
4 &  1.\underline{41421356237}469\\
5 &  1.\underline{41421356237309}\\
6 &  1.\underline{41421356237309}\\
\hline
\end{tabular}
\caption{Approxmationen von $\sqrt{2}$ mit Hilfe des Newton-Algorithmus,
korrekte Stellen unterstrichen.
Die Anzahl korrekter Stellen verdoppelt sich in jedem Schritt.
\label{newton:sqrt2}}
\end{table}
In jedem Schritt verdoppelt sich die Anzahl korrekter Stellen.
Dies ist eine allgemeine Eigenschaft des Newton-Algorithmus, wie
in Abschnitt~\ref{section:newton:konvergenz} erkl"art wird.
\end{beispiel}

\section{Konvergenzgeschwindigkeit\label{section:newton:konvergenz}}

\section{L"osung von Vektorgleichungen\label{section:newton:vektor}}
\rhead{L"osung von Vektorgleichungen}
Wir m"ochten das Verfahren nun erweitern, so dass wir nicht nur eine
einzige Gleichung $f(x)=y$ nach $x$ aufl"osen k"onnen, wir m"ochten dazu 
f"ur ein Gleichungssystem von nichtlinearen Gleichungen
\begin{align*}
f_1(x_1,\dots,x_n)&=y_1\\
f_2(x_1,\dots,x_n)&=y_2\\
&\;\;\vdots\\
f_m(x_1,\dots,x_n)&=y_m
\end{align*}
ebenfalls in der Lage sein.
Wie bei einer einzigen Gleichung k"onnen wir das Problem reduzieren
auf das Finden von gleichzeitigen Nullstellen der Funktionen $g_i$ mit
\begin{align*}
g_1(x_1,\dots,x_n)&=f_1(x_1,\dots,x_n)-y_1=0\\
g_2(x_1,\dots,x_n)&=f_2(x_1,\dots,x_n)-y_2=0\\
&\;\;\vdots\\
g_m(x_1,\dots,x_n)&=f_m(x_1,\dots,x_n)-y_m=0.
\end{align*}
Wir k"onnen dies auch in Vektor-Form schreiben,
wir betrachten $x$ als Vektor der $x_1,\dots,x_n$









\input{chapters/komplexezahlen.tex}
\end{appendices}
\vfill
\pagebreak
\ifodd\value{page}\else\null\clearpage\fi
\lhead{Literatur}
\rhead{}
\printbibliography[heading=subbibliography]
\label{skript:literatur}
\end{refsection}

\part{Anwendungen und Weiterf"uhrende Themen}
\lhead{Anwendungen}
%
% uebersicht.tex -- Uebersicht ueber die Seminar-Arbeiten
%
% (c) 2015 Prof Dr Andreas Mueller, Hochschule Rapperswil
%
\chapter*{"Ubersicht}
\lhead{"Ubersicht}
\rhead{}
\label{skript:uebersicht}
Im zweiten Teil kommen die Teilnehmer des Seminars selbst zu Wort.
Sie zeigen Anwendungsbeispiele f"ur die im ersten
Teil entwickelte Theorie der gew"ohnlichen Differentialgleichungen.



\def\chapterauthor#1{{\large #1}\bigskip\bigskip}
% Artikel
\chapter{Wo steckt die zweite L"osung?\label{chapter:komplex}}
\lhead{Bessel-Funktionen zweiter Art}
\begin{refsection}
\chapterauthor{Stefan Kull und Roy Seitz}

\printbibliography[heading=subbibliography]
\end{refsection}

\chapter{Wo steckt die zweite L"osung?\label{chapter:komplex}}
\lhead{Bessel-Funktionen zweiter Art}
\begin{refsection}
\chapterauthor{Stefan Kull und Roy Seitz}

\printbibliography[heading=subbibliography]
\end{refsection}

\chapter{Wo steckt die zweite L"osung?\label{chapter:komplex}}
\lhead{Bessel-Funktionen zweiter Art}
\begin{refsection}
\chapterauthor{Stefan Kull und Roy Seitz}

\printbibliography[heading=subbibliography]
\end{refsection}

\chapter{Wo steckt die zweite L"osung?\label{chapter:komplex}}
\lhead{Bessel-Funktionen zweiter Art}
\begin{refsection}
\chapterauthor{Stefan Kull und Roy Seitz}

\printbibliography[heading=subbibliography]
\end{refsection}

\chapter{Wo steckt die zweite L"osung?\label{chapter:komplex}}
\lhead{Bessel-Funktionen zweiter Art}
\begin{refsection}
\chapterauthor{Stefan Kull und Roy Seitz}

\printbibliography[heading=subbibliography]
\end{refsection}

\chapter{Wo steckt die zweite L"osung?\label{chapter:komplex}}
\lhead{Bessel-Funktionen zweiter Art}
\begin{refsection}
\chapterauthor{Stefan Kull und Roy Seitz}

\printbibliography[heading=subbibliography]
\end{refsection}

\chapter{Wo steckt die zweite L"osung?\label{chapter:komplex}}
\lhead{Bessel-Funktionen zweiter Art}
\begin{refsection}
\chapterauthor{Stefan Kull und Roy Seitz}

\printbibliography[heading=subbibliography]
\end{refsection}

\chapter{Wo steckt die zweite L"osung?\label{chapter:komplex}}
\lhead{Bessel-Funktionen zweiter Art}
\begin{refsection}
\chapterauthor{Stefan Kull und Roy Seitz}

\printbibliography[heading=subbibliography]
\end{refsection}

\vfill
\pagebreak
\ifodd\value{page}\else\null\clearpage\fi
\lhead{Index}
\rhead{}
%
% skript.tex -- Skript ueber Differentialgleichungen
%
% (c) 2014 Prof. Dr. Andreas Mueller, HSR
%
\documentclass{book}
\usepackage{etex}
\usepackage{geometry}
\geometry{papersize={170mm,240mm},total={140mm,200mm},top=21mm,bindingoffset=10mm}
\usepackage[english,ngerman]{babel}
\usepackage{times}
\usepackage{amsmath,amscd}
\usepackage{amssymb}
\usepackage{amsfonts}
\usepackage{amsthm}
\usepackage{graphicx}
\usepackage{fancyhdr}
\usepackage{textcomp}
\usepackage[all]{xy}
\usepackage{txfonts}
\usepackage{alltt} 
\usepackage{verbatim}
\usepackage{paralist}
\usepackage{makeidx}
\usepackage{array}
\usepackage[colorlinks=true]{hyperref}
\usepackage{tikz}
\usepackage{pgfplots}
\usepackage{pgfplotstable}
\usepackage{pdftexcmds}
%\usepackage{pgfmath}
\usepackage{placeins}
\usepackage{subfigure}
\usepackage[autostyle=false,english=american]{csquotes}
\usepackage{float}
\usepackage{enumitem}
\usepackage{wasysym}
\usepackage{environ}
\usepackage{pifont}
\usepackage{feynmp}
\usepackage{appendix}
\usetikzlibrary{calc,intersections,through,backgrounds,graphs,positioning,shapes,arrows,fit}
\usetikzlibrary{patterns,decorations.pathreplacing}
\usetikzlibrary{decorations.pathreplacing}
\usetikzlibrary{external}
\usepackage[europeanvoltages,
            europeancurrents,
            europeanresistors,   % rectangular shape
            americaninductors,   % "4-bumbs" shape
            europeanports,       % rectangular logic ports
            siunitx,             % #1<#2>
            emptydiodes,
            noarrowmos,
            smartlabels]         % lables are rotated in a smart way
           {circuitikz}          %
\usepackage{siunitx}
\usepackage{tabularx}
\usetikzlibrary{arrows}

\usepackage{algpseudocode}
\usepackage{algorithm}

\usepackage{listings}
\lstdefinestyle{Matlab}{
  numbers=left,
  belowcaptionskip=1\baselineskip,
  breaklines=true,
  frame=L,
  xleftmargin=\parindent,
  language=Matlab,
  showstringspaces=false,
  basicstyle=\footnotesize\ttfamily,
  keywordstyle=\bfseries\color{green!40!black},
  commentstyle=\itshape\color{purple!40!black},
  identifierstyle=\color{blue},
  stringstyle=\color{orange},
  numberstyle=\ttfamily\tiny
}
\lstdefinelanguage{Maxima}{
  keywords={addrow,addcol,zeromatrix,ident,augcoefmatrix,ratsubst,diff,ev,tex,%
    with_stdout,nouns,express,depends,load,submatrix,div,grad,curl,matrix,%
    invert,lambda,facsum,expand,false,then,if,else,subst,%
    rootscontract,solve,part,assume,sqrt,integrate,abs,inf,exp,sin,cos,sinh,cosh},
  sensitive=true,
  comment=[n][\itshape]{/*}{*/}
}
\lstdefinestyle{Maxima}{
  numbers=left,
  belowcaptionskip=1\baselineskip,
  breaklines=true,
  frame=L,
  xleftmargin=\parindent,
  language=Maxima,
  showstringspaces=false,
  basicstyle=\footnotesize\ttfamily,
  keywordstyle=\bfseries\color{green!40!black},
  commentstyle=\itshape\color{purple!40!black},
  identifierstyle=\color{blue},
  stringstyle=\color{orange},
  numberstyle=\ttfamily\tiny
}
\usepackage{caption}
\usepackage[mode=buildnew]{standalone}
\usepackage[backend=bibtex]{biblatex}
\addbibresource{references.bib}
% Bibresources für jeden einzelnen Artikel
%\addbibresource{thema/main.bib}
\AtEndDocument{\clearpage\ifodd\value{page}\else\null\clearpage\fi}
\makeindex
%\pgfplotsset{compat=1.12}
\setlength{\headheight}{15pt} % fix headheight warning
\DeclareGraphicsRule{*}{mps}{*}{}
\begin{document}
\pagestyle{fancy}
\frontmatter
\newcommand\HRule{\noindent\rule{\linewidth}{1.5pt}}
\begin{titlepage}
\vspace*{\stretch{1}}
\HRule
\vspace*{5pt}
\begin{flushright}
{
\LARGE
Mathematisches Seminar\\
\vspace*{20pt}
\Huge
Differentialgleichungen%
}
\vspace*{5pt}
\end{flushright}
\HRule
\begin{flushright}
\vspace{60pt}
\Large
Leitung: Andreas M"uller\\
\vspace{40pt}
\Large
Reto~Christen,
Kevin~Cina,
Andri~Hartmann,
Pascal~Horat %,
Matthias~Kn"opfel,
Stefan Kull,
Daniela~Meier,
Max~Obrist %,
Hansruedi~Patzen,
Benjamin~R"aber,
Simon~Schaefer %,
Tibor~Schneider,
Tobias~Schuler,
Roy~Seitz,
Martin~Stypinski
\end{flushright}
\vspace*{\stretch{2}}
\begin{center}
Hochschule f"ur Technik, Rapperswil, 2016
\end{center}
\end{titlepage}
\hypersetup{
    linktoc=all,
    linkcolor=blue
}
\newcounter{beispiel}
\newenvironment{beispiele}{
\bgroup\smallskip\parindent0pt\bf Beispiele\egroup

\begin{list}{\arabic{beispiel}.}
  {\usecounter{beispiel}
  \setlength{\labelsep}{5mm}
  \setlength{\rightmargin}{0pt}
}}{\end{list}}
\newcounter{uebungsaufgabe}
% environment fuer uebungsaufgaben
\newenvironment{uebungsaufgaben}{
\begin{list}{\arabic{uebungsaufgabe}.}
  {\usecounter{uebungsaufgabe}
  \setlength{\labelwidth}{2cm}
  \setlength{\leftmargin}{0pt}
  \setlength{\labelsep}{5mm}
  \setlength{\rightmargin}{0pt}
  \setlength{\itemindent}{0pt}
}}{\end{list}\vfill\pagebreak}
\newenvironment{teilaufgaben}{
\begin{enumerate}
\renewcommand{\labelenumi}{\alph{enumi})}
}{\end{enumerate}}
% Loesung
\def\swallow#1{
%nothing
}
\NewEnviron{loesung}[1][L"osung]{%
\begin{proof}[#1]%
\renewcommand{\qedsymbol}{$\bigcirc$}
\BODY
\end{proof}
}
\NewEnviron{bewertung}{%
\begin{proof}[Bewertung]%
\renewcommand{\qedsymbol}{}
\BODY
\end{proof}
}
\NewEnviron{diskussion}{%
\begin{proof}[Diskussion]%
\renewcommand{\qedsymbol}{}
\BODY
\end{proof}
}
\NewEnviron{hinweis}{%
\begin{proof}[Hinweis]%
\renewcommand{\qedsymbol}{}
\BODY
\end{proof}
}
\def\keineloesungen{%
\RenewEnviron{loesung}{\relax}
\RenewEnviron{bewertung}{\relax}
\RenewEnviron{diskussion}{\relax}
}
\newenvironment{beispiel}{%
\begin{proof}[Beispiel]%
\renewcommand{\qedsymbol}{$\bigcirc$}
}{\end{proof}}

\input{linsys.tex}
\allowdisplaybreaks

\lhead{Inhaltsverzeichnis}
\rhead{}
\tableofcontents
\newtheorem{satz}{Satz}[chapter]
\newtheorem{hilfssatz}[satz]{Hilfssatz}
\newtheorem{definition}[satz]{Definition}
\newtheorem{annahme}[satz]{Annahme}
\renewcommand{\floatpagefraction}{0.75}
\mainmatter
%
% vorwort.tex -- Vorwort zum Buch zum Seminar
%
% (c) 2015 Prof Dr Andreas Mueller, Hochschule Rapperswil
%
\chapter*{Vorwort}
\lhead{Vorwort}
\rhead{}
Dieses Buch entstand im Rahmen des Mathematischen Seminars
im Fr"uhjahrssemester 2016 an der Hochschule f"ur Technik Rapperswil.
Die Teilnehmer, Studierende der Abteilungen f"ur Elektrotechnik,
Informatik und Bauingenieurwesen der
HSR, erarbeiteten nach einer Einf"uhrung in das Themengebiet jeweils
einzelne Aspekte des Gebietes in Form einer Seminararbeit, "uber
deren Resultate sie auch in einem Vortrag informierten. 

Im Fr"uhjahr 2016 war das Thema des Seminars ``Differentialgleichungen''.
Die Einf"uhrung bestand aus einigen Vorlesungsstunden, deren
Inhalt im ersten Teil dieses Skripts zusammengefasst ist.
Es ging darum, die zum Teil aus dem Analysis-Unterricht bekannte
Theorie der Differentialgleichungen zu vertiefen, mit anderen Gebieten
wie zum Beispiel der komplexen Analysis zu verkn"upfen und sie
auf die Analyse einiger relevanter Praxisprobleme anzuwenden.
Dabei ging es nicht um die analytische L"osung von Differentialgleichungen,
die meisten Differentialgleichungen lassen sich ohnehin nicht in
geschlossener Form l"osen.
Einzelne Differentialgleichungen wurden untersucht, weil sie Anlass
zu einer wichtigen Familie von Funktionen geben, zum Beispiel die
Bessel- und Airy-Funktionen.
In anderen Beispielen ging es um die Schwierigkeiten, die bei einer
numerischen L"osung zu meistern sind.
Besonders anspruchsvoll sind jedoch "Uberlegungen zum Verhalten der
L"osung f"ur lange Zeiten, zum Beispiel Stabilit"at, das Auftreten
von Schwingungen bei der Hopf-Bifurkation oder der "Ubergang zum
Chaos.

Im zweiten Teil dieses Skripts kommen dann die Teilnehmer selbst zu Wort.
Ihre Arbeiten wurden jeweils als einzelne
Kapitel mit meist nur typographischen "Anderungen "ubernommen.
Diese weiterf"uhrenden Kapitel sind sehr verschiedenartig.
Eine "Ubersicht und Einf"uhrung befindet sich in der Einleitung
zum zweiten Teil auf Seite~\pageref{skript:uebersicht}.

In einigen Arbeiten wurde auch Code zur Demonstration der 
besprochenen Methoden und Resultate geschrieben, soweit
m"oglich und sinnvoll wurde dieser Code im Github-Repository
dieses Kurses\footnote{\url{https://github.com/AndreasFMueller/SeminarDGL.git}}
abgelegt, in anderen F"allen verweisen die Artikel selbst auf
das zugeh"orige Code-Repository.

Im genannten Repository findet sich auch der Source-Code dieses
Skriptes, es wird hier unter einer Creative Commons Lizenz
zur Verf"ugung gestellt.


\part{Grundlagen}
%\keineloesungen
\begin{refsection}
\section{Einleitung}
Wellen umgeben uns st"andig, selbst wenn wir es uns dem nicht direkt bewusst 
sind. Sei es nun in Form von Lichtwellen, Schallwellen, Wasserwellen und vielen 
mehr.

Die Wasserwellen welche ans Ufer schlagen, sind wohl ein Beispiel, bei dem sich 
jeder etwas darunter vorstellen kann. Dieses an sich sch"one Naturschauspiel 
kann aber auch destruktiv sein, so kann eine pl"otzliche Absenkung am 
Meeresgrund beispielsweise einen Tsunami ausl"osen.

In diesem Kapitel wird anfangs die Titelgleichung, welche die Ausbreitung einer 
Welle in einem parabolischen Kanal beschreibt, genauer analysiert. Gegen Ende 
soll aber auch noch eine allgemeinere Kanalform untersucht werden.

Nat"urlich sind diese Gleichungen nur eine Ann"aherung und entsprechen nur 
bedingt der in der Realit"at vorkommenen Wellenausbreitung. So kann sich eine 
Wasserwelle zum Beispiel "uberschlagen, was hier so nicht abgebildet wird. 
Trotzdem kann anhand von diesen Modellrechnungen versucht werden, die 
Ausbreitung einer Welle nachzuvollziehen.

%\input{kapitel.tex}
%
% grundlagen.tex -- Grundlagen ueber Differentialgleichungen
%
% (c) 2015 Prof Dr Andreas Mueller, Hochschule Rapperswil
%
\chapter{Grundlagen der Theorie der gew"ohnlichen Differentialgleichungen
\label{chapter:grundlagen}}
\lhead{}
\rhead{Grundlagen}
\section{Differentialgleichungen\label{section:differentialgleichungen}}
Eine gew"ohnliche Differentialgleichung f"ur eine reellwertige
Funktion $y(x)$ stellt einen Zusammenhang her zwischen der Funktion
und ihren Ableitungen.
Wir schreiben die Ableitungen als $y'$, $y''$, $y'''$ und $y^{(n)}$
f"ur die $n$-te Ableitung.
Wir lassen oft das Argument der Funktion weg.
Beispiele von Differentialgleichungen sind
\begin{align*}
y'&=-Ny
&&\text{Ordnung: $1$}
\\
y''&=-\omega^2 y
&&\text{Ordnung: $2$}
\\
x^2y''+xy'+(x^2-n^2)y&=0
&&\text{Ordnung: $2$}
\end{align*}
Die Abh"angigkeit kann in expliziter Form als
\begin{equation}
y^{(n)}=f(x,y,y',\dots,y^{(n-1)})
\label{grundlagen:explizit}
\end{equation}
oder in impliziter Form
\[
F(x,y,y',\dots,y^{(n)})=0
\]
gegeben sein.
Die Ordnung einer Differentialgleichung ist die h"ochste vorkommende
Ableitung.

Insbesondere in Anwendungen in der Physik ist die Zeit die
unabh"angige Variable.
Die abh"angige Variable ist dann zum Beispiel die Ortskoordinate
$x(t)$ und wir bezeichnen ihre Ableitungen mit $\dot{x}(t)$ f"ur
die Geschwindigkeit, $\ddot{x}(t)$ f"ur die Beschleunigung.
Dieses Beispiel suggeriert auch, dass die abh"angige Variable 
ein Vektor sein kann, den man als den Ortsvektor eines Teilchens
interpretieren kann.
Die Funktion $f(t,x,\dots,x^{(n-1)})$ ist dann auch vektorwertig, und
alle Argumente ausser dem ersten von $f$ sind vektorwertig.

Eine Differentialgleichung $n$-ter Ordnung f"ur eine skalare Funktion
kann in eine Vektor-Differentialgleichung erster Ordnung f"ur eine
$n$-dimensionale vektorwertige Funktion umgewandelt werden.
Ist $y(x)$ die gesuchte Funktion in der
Differentialgleichung~(\ref{grundlagen:explizit}), dann kann man
den Vektor
\[
u(x)=\begin{pmatrix}
y(x)\\y'(x)\\\vdots\\y^{(n-1)}(x)
\end{pmatrix}
\in\mathbb R^n
\]
bilden.
Er erf"ullt die Differentialgleichung
\begin{equation}
\frac{d}{dx}\begin{pmatrix}
y\\y'\\\vdots\\y^{(n-1)}
\end{pmatrix}
=
\begin{pmatrix}
y'\\y''\\\vdots\\y^{(n)}
\end{pmatrix}
=
\begin{pmatrix}
y'\\y''\\\vdots\\f(x,y,y',\dots,y^{(n-1)}.
\end{pmatrix}
\label{grundlagen:vektordgl}
\end{equation}
Der Vektor auf der rechten Seite h"ang nur von $x$, der Funktion $y$
und ihren Ableitungen bis zur $n-1$-ten Ordnung ab, also von $u$, man
kann (\ref{grundlagen:vektordgl}) daher als
\begin{equation}
\frac{d}{dx}u=\tilde{f}(x,u)
\end{equation}
schreiben.

\section{Anfangswertprobleme\label{section:anfangswertprobleme}}

\section{Randwertprobleme\label{section:randwertprobleme}}

\section{Analytische L"osungsverfahren\label{section:analytischeverfahren}}
\subsection{Separation der Variablen}
Differentialgleichungen erster Ordnung lassen sich oft durch sogenannte
Trennung der Variablen auf die Berechnung von Integralen reduzieren.
Dank der Schreibweise der Ableitung als Differentialquotient wird
dieser L"osungsweg sehr suggestiv.
Wir betrachten als Beispiel die Differentialgleichung
\[
y'=-Ny.
\]
Schreibt man die Ableitung als Differentialquotient, wird daraus die
Gleichung
\[
\frac{dy}{dx}=-Ny.
\]
Durch Division durch $y$ und formale Multiplikation mit $dx$ wird daraus
die formale Gleichung
\begin{equation}
\frac{dy}{y}=-N\,dx.
\label{grundlagen:separiert}
\end{equation}
In dieser Gleichung kommt die Variable $y$ nur auf der linken, die Variable
$x$ nur auf der rechten Seite vor.
Man sagt, die Variablen seien {\em separiert}.
\index{separiert}
\index{Variablen, Separation der}
Man beachte, dass die Gleichung (\ref{grundlagen:separiert}) nur eine
formale Bedeutung haben kann, die Symbole $dy$ und $dx$ sind ja keine Zahlen,
mit denen man algebraische Operationen durchf"uhren k"onnte.
Mit etwas Vorsicht angewandt f"uhrt dieser Kalk"ul aber nicht auf
Widerspr"uche.

Wir integrieren jetzt beide Seiten von (\ref{grundlagen:separiert}), und
erhalten 
\[
\int\frac1y\,dy=-N\int\,dx
\]
Beide Integrale lassen sich in geschlossener Form auswerten:
\[
\log|y|=-Nx+C.
\]
Aufgel"ost nach $y$ ergibt sich
\[
y=\pm e^{C}e^{-Nx},
\]
wobei die beiden Vorzeichen $\pm$ das Betragszeichen in der Stammfunktion
von $\frac1y$ reflektieren.
Man kann den Faktor $\pm e^{C}$ in eine neue Konstante $a$ zusammenfassen,
und erh"alt somit als L"osung der urspr"unglichen Differentialgleichung
die Familie
\[
y(x)=ae^{-Nx}
\]
von Funktionen.
Der Paramter $a$ muss mit Hilfe der Anfangsbedingung festgelegt werden.

\subsection{Lineare Differentialgleichungen}
Eine Differentialgleichung der Form
\begin{equation}
a_n(x)y^{(n)}+a_{n-1}(x)y^{(n-1)}+\dots+a_2(x)y''+a_1(x)y'+a_0(x)=f(x)
\label{grundlagen:linearedgl}
\end{equation}
heisst {\em lineare Differentialgleichung}.
\index{lineare Differentialgleichung}
Ist $f(x)=0$, nennt man die Differentialgleichung {\em homogen}, die
Funktion $f(x)$ wird auch die {\em Inhomogenit"at} genannt.
\index{homogen Differentialgleichung}
\index{Inhomogenitat@Inhomogenit\"at}
Die L"osungsmenge einer homogenen linearen Differentialgleichung
bildet einen Vektorraum: jede Linearkombination von L"osungen
ist wieder eine L"osung.
Seien zum Beispiel $y_1(x)$ und $y_2(x)$ L"osungen der Differentialgleichung
(\ref{grundlagen:linearedgl}).
Wir m"ochten zeigen, dass
$y(x)=\alpha y_1(x)+\beta y_2(x)$ eine L"osung ist.
Die Ableitungen selbst sind linear:
\begin{align*}
y^{(k)}&=\alpha y_1^{(k)}(x)+\beta y_2^{(k)}(x).
\end{align*}
Setzt man dies in die Differentialgleichung ein, erh"alt man
\begin{align*}
a_ny^{(n)}+\dots+a_1y'+a_0y
&=
a_n\alpha y_1^{(n)}+a_n\beta y_2^{(n)}+\dots+a_1\alpha y_1'+a_1\beta y_2'
+ a_0\alpha y_1+a_0\beta y_2
\\
&=
\alpha(\underbrace{a_ny_1^{(n)}+\dots+a_1y_1'+a_0y_1}_{=0})
+
\beta(\underbrace{a_ny_2^{(n)}+\dots+a_1y_2'+a_0y_2}_{=0})=0,
\end{align*}
die Linearkombination $y$ erf"ullt also die homogene Differentialgleichung
ebenfalls.


\subsection{Variation der Konstanten}
\subsection{Laplace-Transformation}

%
% numerik.tex -- numerische Lösung von gewöhnlichen Differentialgleichungen
%
% (c) 2015 Prof Dr Andreas Mueller, Hochschule Rapperswil
%
\chapter{Numerische L"osung\label{chapter:numerik}}
\lhead{}
\rhead{Numerische L"osung}
\index{Numerische Loesung@Numerische L\"osung}
Im Kapitel~\ref{chapter:grundlagen} waren wir in der Lage, f"ur einige
einfache Differentialgleichungen eine L"osung in geschlossener Form
zu finden.
Zum Beispiel konnten wir lineare Differentialgleichungen mit Hilfe
der Exponentialfunktion l"osen.
Dieses Bild tr"ugt allerdings.
Die meisten Differentialgleichungen k"onnen nicht in geschlossener
Form gel"ost werden.
Wir k"onnen daher nicht erwarten, dass wir die L"osungen beliebiger
Differentialgleichungen einfach dadurch verstehen, dass wir
L"osungsfunktionen diskutieren.
Stattdessen bleiben uns nur die folgenden zwei M"oglichkeiten:
\begin{enumerate}
\item
Wir l"osen die Differentialgleichung mit Hilfe eines Computers,
und studieren den Verlauf der L"osungsfunktionen oder die Abh"angigkeit
von Parameter oder Anfangsbedingungen durch Vergleich verschiedener
numerisch gefundener L"osungen.
\item
Wir entwickeln Methoden, mit denen sich Aussagen "uber den Verlauf der
L"osungskurven studieren lassen, ohne dass man sie berechnet haben muss.
Nat"urlich kann man nicht erwarten, dass eine solche Methode genaue
Aussagen dar"uber erlaubt, wann eine L"osungskurve wo genau durchgehen
wird.
Es werden nur qualitative Aussagen m"oglich sein, zum Beispiel ob
Gleichgewichtsl"osungen stabil sind, ob es periodische L"osungen gibt
und ob L"osungskurven zu den periodischen L"osungen konvergieren.
\end{enumerate}
In diesem Kapitel entwickeln wir Methoden, Differentialgleichungen 
numerisch zu l"osen.

\section{Grundprinzip}
\lhead{Grundprinzip}
\begin{figure}
\centering
\includegraphics{chapters/images/numerik-2.pdf}
\caption{Lineare Approximation von $y(x+\Delta x)$ durch Information,
die am Punkt $x$ verf"ugbar ist.
\label{numerik:lineareapproximation}}
\end{figure}
Wir versuchen die Differentialgleichung
\begin{equation}
y'=-\alpha y,\qquad y(0)=y_0
\label{numerik:expdgl}
\end{equation}
numerisch zu l"osen. 
Dazu unterteilen wir die $x$-Achse in diskrete Abschnitte der L"ange $h$,
und bezeichnen die Teilpunkte mit $x_k=kh$.
Das Ziel ist jetzt, $y(x_k)$ n"aherungsweise zu berechnen.
Wir schreiben $y_k$ f"ur die N"aherungswerte von $y(x_k)$.
Die Ableitung liefert eine lineare Approximation f"ur $y(x)$,
n"amlich
\[
y(x+\Delta x)\simeq y(x) + y'(x)\cdot\Delta x
\]
(Abbildung~\ref{numerik:lineareapproximation}).
F"ur die Punkte $x_k$ bedeutet das
\[
y(x_{k+1})\simeq y(x_{k})+y'(x_k)h.
\]
Die Differentialgleichung liefert Werte f"ur $y'(x_k)$ aus $x_k$ und $y(x_k)$,
damit k"onnen wir aus dieser Approximation ein allgemeines
N"aherungsverfahren f"ur die L"osung einer Differentialgleichung
konstruieren.

\begin{satz}[Euler-Verfahren]
\index{Euler-Verfahren}
Die Differentialgleichung
\begin{equation}
y'=f(x,y),\qquad y(0)=y_0
\label{numerik:eulerdgl}
\end{equation}
und die Schrittweite $h$ definieren eine Folge 
\[
y_{\mathstrut k}=y_{k-1} + h\cdot f(x_{k-1}, y_{k-1}),\quad k>0,
\]
mit $x_k=kh$,
die eine N"aherung f"ur die Funktionswerte $y(x_k)$ der L"osung $y(x)$
der Differentialgleichung~(\ref{numerik:eulerdgl}) ist.
\end{satz}

Dieses Verfahren ist nicht besonders gut, wie wir im Folgenden zeigen
wollen.
Die Diskussion soll uns aber zeigen, worauf bei der Weiterentwicklung
des Verfahrens geachtet werden muss.

Im vorliegenden Beispiel liefert die
Differentialgleichung~(\ref{numerik:expdgl})
den Wert $y'(x_k)=-\alpha y(x_k)$ f"ur die Ableitung,
woraus wir die Rekursionsformel
\[
y_{k+1}=y_k - \alpha y_k \dot h.
\]
gewinnen.
Die Rekursionsgleichung kann in diesem Fall exakt gel"ost werden,
und wir finden
\begin{equation}
y(x_{k+1}) = y(x_k)-\alpha y(x_k) h=(1-\alpha h) y(x_k)=\dots
=(1-\alpha h)^{k+1}y_0
\label{numerik:rekursion}
\end{equation}
f"ur die N"aherung $y_k$ der Funktionswerte $y(x_k)$.
%Angewendet auf eine beliebige Differentialgleichung, ist dieses
%einfache numerische Verfahren bekannt als das {\em Euler-Verfahren}.
%Es ist nicht besonders genau, aber soll in diesem Abschnitt dazu
%dienen, die Anforderungen an ein gutes numerisches Verfahren
%zu illustrieren.



Wir m"ochten $y(x)$ f"ur einen ganz bestimmten $x$-Wert berechnen.
Dazu unterteilen wir das Intervall $[0,x]$ in $n$ Teilschritte der
Breite $x/n$, und wenden die Formel~(\ref{numerik:rekursion}) an:
\[
y(x)=y(x_n)=(1-\alpha h)^n y_0=\biggl(1+\frac{-\alpha x}{n}\biggr)^n y_0.
\]
F"ur eine grosse Zahl von Teilschritten erhalten wir so tats"achlich die
korrekte L"osung:
\[
\lim_{n\to\infty}y_0\biggl(1+\frac{-\alpha x}n\biggr)^n=y_0 e^{-\alpha x}.
\]
\begin{figure}
\centering
\includegraphics{chapters/images/numerik-1.pdf}
\caption{Approximationen der L"osung der Differentialgleichung $y'=-\alpha y$
mit verschiedener Anzahl Schritte (rot) n"ahern sich f"ur wachsendes
$n$ der exakten L"osung (blau).
\label{numerik:approximation}}
\end{figure}%
Abbildung~\ref{numerik:approximation} zeigt, wie die
durch~(\ref{numerik:rekursion}) gegebenen Approximationen mit zunehmendem
$n$ der exakten L"osung $y(x)=e^{-\alpha x}$ n"aher kommen.

Wir k"onnen auch den Fehler des numerischen Verfahrens berechnen.
Bei der Schrittweite $h$ ist der Fehler von $y_k$ die Differenz
\[
y(x_k)-y_k
=
y_0e^{-\alpha kh}-y_0(1-\alpha h)^k
=
y_0((e^{-\alpha h})^k - (1-\alpha h)^k)
=
y_0e^{-\alpha hk}\biggl(
1-\biggl(\frac{1-\alpha h}{e^{-\alpha h}}\biggr)^k
\biggr).
\]
Man beachte, dass der Z"ahler $1-\alpha h$ die Approximation
$y_1$ ist, als eine Approximation von $e^{-\alpha h}$, dem Nenner.
Schreiben wir
\[
q=\frac{1-\alpha h}{e^{-\alpha h}},
\]
f"ur den Quotienten zwischen der Approximation und dem korrekten Wert,
dann ist sicher immer $q<1$.
Den Fehler k"onnen wir jetzt schreiben
\[
y(x_k)-y_k = y_0e^{-\alpha hk}(1-q^k) = y(x_k)(1-q^k).
\]
Der relative Fehler des Verfahrens ist also
\[
\frac{y(x_k)-y_k}{y(x_k)}=(1-q^k).
\]
\begin{figure}
\centering
\includegraphics{chapters/images/numerik-3.pdf}
\caption{Relativer Fehler des Euler-Verfahrens f"ur die Differentialgleichung
(\ref{numerik:expdgl}) in Abh"angigkeit von der Anzahl $k$ der Schritte.
\label{numerik:relfehler}}
\end{figure}%
Ganz unabh"angig von der Schrittweite $h$ wird der relative Fehler
des Verfahrens immer gegen 1 streben, der Fehler wird also von der
gleichen Gr"ossenordnung wie die berechneten Resultate.

Die Abbildung~\ref{numerik:relfehler} zeigt, dass zu Beginn des Verfahrens
der relative Fehler ungef"ahr linear mit der Anzahl der Schritt zunimmt.
Um eine angemessene Genauigkeit "uber einen gr"osseren Bereich
zu erreichen, muss das Euler-Verfahren also sehr viel kleinere Schritte
und eine entsprechend gr"ossere Anzahl von Schritten ausf"uhren,
die entsprechend viel Rechenzeit ben"otigen.

Ein praktisch n"utzliches Verfahren muss also anstreben, mit einer
sehr viel kleineren Anzahl von Schritten eine viel gr"ossere Genaugikeit
der Approximation zu erreichen.

\section{Fehler-Entwicklung numerischer L"osungen}
\lhead{Fehler-Entwicklung}
Wir betrachten wieder die Differentialgleichung~(\ref{numerik:eulerdgl})
und versuchen, den Fehler eines N"aherungsverfahrens zu bestimmen,
welches Schritte der Gr"osse $h$ durchf"uhrt, um den Wert $y(x)$
zu approximieren.

Das Euler-Verfahren verwendet Schritte der Form
\[
y_{k+1}=y_{k\mathstrut} + hf(x_{k\mathstrut},y_{k\mathstrut}).
\]
In jedem einzelnen Schritt entsteht ein Fehler, dessen Gr"osse wir
aus der Taylor-Entwicklung
\[
y(x+\Delta x)=
y(x) + y'(x)\cdot \Delta x + R(x) \Delta x^2
\]
absch"atzen k"onnen.
Die Funktion $R(x)$ ist beschr"ankt und beschreibt den verbleibenden
Fehler.
Um $y(x)$ zu approximieren, m"ussen $n=x/h$ Schritte der Schrittweite
$h$ durchgef"uhrt werden, von denen jeder einen Fehler
von der Gr"ossenordnung $R(x)h^2$ hat.
Der Gesamtfehler ist daher von der Gr"ossenordnung
\[
y(x)-y_n=O\biggl(R(x)h^2\frac{x}h\biggr)=O(h),
\]
er ist also von erster Ordnung in $h$.
Um eine zus"atzliche Stelle Genauigkeit zu erhalten, muss man also zehnmal
so viele Schritte von zehnmal kleinerer Gr"osse durchf"uhren,
wodurch auch wieder Rundungsfehler eingef"uhrt werden.

K"onnte man den Fehler des Einzelschrittes wesentlich verkleinern, w"urde
auch die Abh"angigkeit des Fehlers des Verfahrens vorteilhafter.
W"are der Fehler des Einzelschrittes $O(h^k)$ statt $O(h^2)$, dann
w"are der Gesamtfehler des Verfahrens nur noch $O(h^{k-1})$.
F"ur $k=3$ bedeutet dies, dass eine Halbierung der Schrittweite
zwar doppelt so viele Schritte braucht, aber auch, dass in jedem
Schritt nur ein Achtel des Fehlers auftritt.
Der Gesamtfehler ist also nur ein Viertel.
Mit zehnmal mehr Arbeit kann man also nicht nur eine Stelle an
Genauigkeit gewinnen, sondern gleich deren zwei.

Man nennt ein Verfahren, bei dem der Gesamt-Fehler von der Gr"ossenordnung
$O(h^k)$ ist, von einem Verfahren $k$-ter Ordnung.
Das Euler-Verfahren ist also ein Verfahren erster Ordnung oder ein
lineares Verfahren.
In der Praxis werden Verfahren bis zu vierter und f"unfter Ordnung
verwendet, so dass eine zehnmal kleinere Schrittweite zu gleich
vier Stellen Genauigkeitsgewinn f"uhren.
Das Ziel der kommenden Abschnitte muss daher sein, einfach
berechnebare Approximationen der Funktion mit m"oglichst geringen
Einzelschrittfehlern zu finden.

\section{Einschritt-Verfahren\label{section:numerik:einschritt}}
\lhead{Einschritt-Verfahren}
Die relativ geringe Genauigkeit des Eulerschrittes beruht darauf,
dass die zu Beginn des Schrittes berechnete Ableitung $f(x_k,y_k)$
nur f"ur das linke Ende des Intervalls $[x_k, x_k+h]$ zutrifft,
weiter rechts im Intervall wird die Abweichung immer gr"osser.
Eine m"ogliche L"osung des Problems k"onnte darin bestehen, statt
nur einer linearen N"aherung zus"atzliche Glieder der Taylorreihe
\begin{equation}
y(x+\Delta x)
=
y(x)
+
y'(x)\cdot \Delta x
+
\frac12 y''(x)\cdot \Delta x^2
+
\frac16 y'''(x)\cdot \Delta x^3
+
o(\Delta x^3)
\label{numerik:taylor}
\end{equation}
zu verwenden.
In (\ref{numerik:taylor}) werden h"ohere Ableitungen von $y(x)$ ben"otigt,
w"ahrend die Differentialgleichung nur die erste Ableitung liefert.
Die h"oheren Ableitungen wurden aber bereits im
Abschnitt~\ref{grundlagen:hoehere-ableitungen} berechnet.

Wir untersuchen, wie sich das Verfahren f"ur die Beispiel-Gleichung
(\ref{numerik:expdgl}) anwenden l"asst.
Dort gilt
\begin{equation*}
\begin{aligned}
y'(x)&=f(x,y)=-\alpha y
\\
\Rightarrow\qquad
\frac{\partial f}{\partial x}&=0&\frac{\partial f}{\partial y}&=-\alpha
\end{aligned}
\end{equation*}
Alle zweiten Ableitungen verschwinden.
Die Gleichungen werden damit einfach:
\begin{align*}
y''(x)&=-\alpha f(x,y)=\alpha^2 y
\\
y'''(x)&=\alpha^2f(x,y)=-\alpha^3 y.
\end{align*}
Statt der linearen Approximation sollte daher die kubische Approximation
\begin{equation}
y_{k+1}
=
y_{k}-\alpha h y_k +\frac12\alpha^2 h^2 y_k -\frac16 \alpha^3h^3 y_k
=
y_{k}\underbrace{\biggl(1-\alpha h +\frac12\alpha^2h^2 -\frac16 \alpha^3h^3\biggr)}_{\simeq e^{-\alpha h}}
\label{numerik:kubisch}
\end{equation}
verwendet werden.
Dass man hier mit einer gr"osseren Genauigkeit rechnen darf ist schon daran
erkennbar, dass der Klammerausdruck auf der rechten Seite eine viel
bessere Approximation von $e^{-\alpha x}$ ist also der Faktor
$(1-\alpha h)$ im Euler-Verfahren.
Genauer erwarten wir, dass wir hier ein kubisches Verfahren konstruiert haben.

\begin{table}
\centering
\begin{tabular}{|r|c|r|r|r|}
\hline
$i$&$x$&$e^{-\alpha x}$&Euler&kubisch\\
\hline
 1 & 0.1 & 0.95122942 & 0.\underline{95}000000 & 0.\underline{951229}17 \\
 2 & 0.2 & 0.90483742 & 0.\underline{90}250000 & 0.\underline{904836}93 \\
 3 & 0.3 & 0.86070798 & 0.\underline{85}737500 & 0.\underline{860707}28 \\
 4 & 0.4 & 0.81873075 & 0.\underline{81}450625 & 0.\underline{818729}87 \\
 5 & 0.5 & 0.77880078 & 0.\underline{77}378094 & 0.\underline{778799}73 \\
 6 & 0.6 & 0.74081822 & 0.\underline{73}509189 & 0.\underline{74081}702 \\
 7 & 0.7 & 0.70468809 & 0.\underline{6}9833730 & 0.\underline{70468}675 \\
 8 & 0.8 & 0.67032005 & 0.\underline{6}6342043 & 0.\underline{67031}859 \\
 9 & 0.9 & 0.63762815 & 0.\underline{6}3024941 & 0.\underline{63762}660 \\
10 & 1.0 & 0.60653066 & 0.\underline{5}9873694 & 0.\underline{60652}902 \\
\hline
\end{tabular}
\caption{N"aherungswerte f"ur die L"osung $e^{-\alpha x}$ der
Beispieldifferentialgleichung (\ref{numerik:expdgl}) nach dem Euler-Verfahren
und nach dem kubischen Verfahren (\ref{numerik:kubisch}) mit einer
Schrittweite von 0.1. Unterstrichen ist jeweils die Stellen, die nach
Rundung auf die angegebene Anzahl stellen mit dem exakten Wert "ubereinstimmt.
\label{numerik:euler-kubisch}}
\end{table}%
In Tabelle~\ref{numerik:euler-kubisch} werden die Resultate des
kubischen Verfahrens denen des Euler-Verfahrens gegen"ubergestellt.
Im ersten Schritt ist der Fehler des Euler-Verfahrens kleiner als $10^{-2}$,
was einer Einheit in der zweiten Nachkommastelle entspricht.
Der Fehler des kubischen Verfahrens ist kleiner als $10^{-6}$, eine
Einheit in der sechsten Nachkommastelle, ungef"ahr die von einem
kubischen Verfahren zu erwartende Verbesserung.
Nach zehn Rechenschritten liefert das Euler-Verfahren dank Rundung
gerade noch eine korrekte Stelle, w"ahrend das kubische Verfahren immer noch
gerundet f"unf korrekte Stellen gibt.

Es wurde bereits darauf hingewiesen, dass die Terme f"ur die Ableitungen
sehr kompliziert werden.
noch viel gravierender ist allerdings, dass auch die partiellen Ableitungen
von $f$ nach $x$ und $y$ bekannt sein m"ussen.
Es ist zwar im Prinzip m"oglich, diese zu berechnen, der Rechenaufwand 
daf"ur kann aber so erheblich sein, dass er den Genauigkeitsgewinn
leicht wieder zunichte machen kann.
Praktisch n"utzliche Verfahren m"ussen daher danach streben,
die h"oheren Ableitungen von $y(x)$ ausschliesslich aus Funktionswerten
von $f(x,y)$ zu berechnen.

Wir m"ochten aber weiterhin nur $y_{k+1}$ ausschliesslich aus $x_k$ und $y_k$
berechnen, also in einem einzelnen Schritt der Form
\[
y_{k+1}=y_k + h\, F(x_k, y_k, h).
\]
Die Funktion $F(x,y,h)$ heisst die {\em Inkrement-Funktion}
\index{Inkrement-Funktion}
des Verfahrens.
F"ur das Euler-Verfahren ist $F(x,y,h)=f(x,y)$.
Es soll also eine Inkrement-Funktion gefunden werden, bei der $y(x+\Delta x)$
durch $y(x) + \Delta x\cdot F(x,y,\Delta x)$ bis auf Terme h"oherer
Ordnung approximiert werden kann.

\subsection{Quadratische Verfahren}
Ein quadratisches Verfahren verwendet eine Inkrement-Funktion $F(x,y,h)$,
welche
\[
y(x+h)=y(x)+hF(x,y,h)+O(h^3)
\]
erf"ullt.
Aus den einleitenden Bemerkungen von~\ref{section:numerik:einschritt}
folgt, dass dieses Ziel m"oglicherweise dadurch erreicht werden kann,
dass man Werte von $f$ f"ur verschiedene $x$ geeignet miteinander
kombiniert.
Ein denkbarer Ansatz daf"ur ist
\[
F(x,y,h)=af(x,y) + bf(x+\alpha h, y +\beta hf(x,y)),
\]
oder anders ausgedr"uckt: Man f"uhrt zuerst etwas "ahnliches wie einen
Eulerschritt durch, um zum Punkt $(x+\alpha h,y+\beta hf(x,y))$ zu
gelangen.
Dort berechnet man den Wert von $f$, und bildet dann einen geeigneten
Mittelwert davon  mit $f(x,y)$.
Durch geeignete Wahl von $a$, $b$, $\alpha$ und $\beta$ sollte es m"oglich
sein, dass die Inkrement-Funktion einen Fehler h"ochstens dritter Ordnung
hat, womit wir dann ein Integrationsverfahren zweiter Ordnung gewonnen
h"atten.

Wir m"ussen jetzt die Parameter $a$, $b$, $\alpha$ und $\beta$ bestimmen.
Da wir mit dem "ubereinstimmen der ersten zwei Ableitungen
nur zwei Bedingungen haben, k"onnen wir nicht erwarten, dass wir
eine eindeutige L"osung finden werden.
Vielmehr werden einzelne Parameter frei w"ahlbar sein, es wird eine
ganze Familie von quadratischen L"osungsverfahren entstehen, parametrisiert
durch eine der Variablen $a$, $b$, $\alpha$ und $\beta$.

Wir berechnen nun $F(x,y,h)$ bis zur zweiten Ordnung, damit wird 
$y(x+h)$ bis zur dritten Ordnung ausdr"ucken k"onnen.
\begin{align*}
f(x+\alpha h, y + \beta h f(x,y))
&=
f(x,y)+\alpha h\frac{\partial f(x,y)}{\partial x}
+ \beta h \frac{\partial f(x,y)}{\partial y} + O(h^2)
\end{align*}
\begin{align}
F(x,y,h)
&=
af(x,y) + bf(x+\alpha h, y + \beta h f(x,y))
\notag
\\
&=
(a+b)f(x,y) + \biggl(\alpha b\frac{\partial f(x,y)}{\partial x}
+ \beta b\frac{\partial f(x,y)}{\partial y} f(x,y))\biggr)h+O(h^2)
\label{numerik:inkrementF}
\end{align}
Damit dies bis zur zweiten Ordnung mit dem Inkrement zwischen $x$ und $x+h$
"ubereinstimmt, muss~(\ref{numerik:inkrementF}) mit der Taylorreihe
von $y(x)$ "ubereinstimmen, also mit
\begin{equation}
\frac{y(x+h)-y(x)}{h}=y'(x) + \frac12y''(x)h + O(h^2)
=f(x,y) + \frac12\frac{\partial f(x,y)}{\partial x}
+\frac12\frac{\partial f(x,y)}{\partial y}f(x,y) + O(h^2),
\label{numerik:ytaylor}
\end{equation}
wobei wir f"ur $y''(x)$ die Gleichung (\ref{grundlagen:2abl}) verwendet haben.
Durch Koeffizientenvergleich finden wir die Bedingungen
\[
\begin{aligned}
a+b&=1,&
\alpha b&=\frac12,&
\beta b&=\frac12.
\end{aligned}
\]
Einzig $b$ kommt in allen drei Gleichungen vor, und bestimmt den Wert der
jeweiligen anderen Variablen:
\[
\begin{aligned}
a&=1-b,&\alpha&= \beta=\frac{1}{2b}.
\end{aligned}
\]
Jeder Wert von $b$ zwischen $0$ und $1$ liefert ein Verfahren mit quadratischer
Genauigkeit.

Der Parameterwert $b=1$ f"uhrt auf $\alpha=\beta=1$ und $a=0$, die
Rekursionsformel ist in diesem Falle
\begin{equation}
y_{k+1}=y_{k}+hf\biggl(x_k+\frac{h}2,y_k+\frac{h}2 f(x_k,y_k)\biggr).
\label{numerik:improved-euler}
\end{equation}
Das Verfahren f"uhrt also erst einen halben Eulerschritt zum Punkt
$(x_k+\frac12h,y_k+\frac{h}2f(x_k,y_k))$ durch, berechnet dort mit Hilfe
von $f$ die Steigung, die dann f"ur einen Eulerschritt der L"ange $h$
verwendet wird.u
Daher heisst dieses Verfahren auch das {\em verbesserte Euler-Verfahren}.
\index{Euler-Verfahren!verbessertes}

Verwendet man $b=\frac12$, folgt zun"achst $a=\frac12$ und $\alpha=\beta=1$.
Daraus erh"alt man die Rekursionsformel
\begin{equation}
y_{k+1}=y_k+\frac{h}2\biggl(
f(x_k,y_k) + f(x_k+h, y_k + hf(x_k,y_k))
\biggr)
\label{numerik:simplified-runge-kutta}
\end{equation}
In diesem Verfahren f"uhrt man also zuerst einen Eulerschritt der L"ange
$h$ durch, mit dem man zum Punkt $(x_k+h, y_k+hf(x_k,y_k))$ gelangt.
Dort berechnet mit mit Hilfe von $f$ die Steigung.
Das arithmetische Mittel dieser Steigung mit der im Euler-Verfahren
verwendeten Steigung $f(x_k,y_k)$ im Punkt $x_k$ wird dann als
Steigung f"ur einen Eulerschritt verwendet.
Statt eines einzigen Steigungswertes werden hier also zwei Steigungswerte
von den Enden des Intervalls $[x_k,x_k+1]$ gemittelt.
Wegen der "Ahnlichkeit dieses Vorgehens mit dem sp"ater zu besprechenden
Runge-Kutte-Verfahren heisst diese Verfahren auch das {\em
vereinfachte Runge-Kutta-Verfahren}.
\index{Runge-Kutta-Verfahren!vereinfachtes}

\subsection{Runge-Kutta-Verfahren\label{subsection:numerik:runge-kutta}}
\index{Runge-Kutta-Verfahren}
Das {\em Runge-Kutta-Verfahren} erweitert die Inkrement-Funktion derart,
dass der Einzelschritt bis zur f"unften Ordnung mit der Taylorreihe von
$y(x)$ "ubereinstimmt.
So entsteht ein Verfahren vierter Ordnung, es stellt einen guten Kompromiss
zwischen Genauigkeit und Rechenaufwand dar.

Da vier Ableitungen korrekt dargestellt werden m"ussen, ist zu erwarten,
dass vier verschiedene Werte von $f$ an verschiedenen Punkten $(x,y)$
ausgewertet und geeignet miteinander kombiniert werden m"ussen.
Genauer: Man bestimmt zuerst die Werte
\begin{align*}
k_1&=f(x_k,y_k)\\
k_2&=f\biggl(x_k+\frac{h}2,y_k+\frac{h}2k_1\biggr)\\
k_3&=f\biggl(x_k+\frac{h}2,y_k+\frac{h}2k_2\biggr)\\
k_4&=f(x_k+h, y_k+hk_3)
\end{align*}
und setzt diese dann zusammen, um den n"achsten Wert $y_{k+1}$
zu berechnen:
\begin{equation}
y_{k+1} = y_k + h\frac{1}6(k_1 + 2k_2 + 2k_3 + k_4).
\label{numerik:runge-kutta-rekursion}
\end{equation}
Man kann die Formeln wie folgt interpretieren.
Zuerst wird ein halber Eulerschritt mit der Steigung $k_1=f(x_k,y_k)$,
durchgef"uhrt, und und am Zielpunkt die Steigung $k_2$ ermittelt.
Mit dieser Steigung wird dann erneut ein halber Schritt von $(x_k,y_k)$
aus durchgef"uhrt, und am Zielpunkt erneut die Steigung $k_3$ ermittelt.
Damit f"uhrt man einen ganzen Schritt aus, an dessen Zielpunkt man die
Steigung $k_4$ findet.
Diese vier Steigungen werden jetzt gewichtet gemittelt, wobei
$k_2$ und $k_3$ doppeltes Gewicht erhalten, und mit dieser
Steigung wird ein ganzer Schritt vorgenommen.

Die Formeln f"ur die $k_i$ sowie (\ref{numerik:runge-kutta-rekursion})
k"onnen ganz "ahnlich wie das verbesserte Euler-Verfahren bzw.~das
vereinfachte Runge-Kutta-Verfahren begr"undet werden.
Der Aufwand daf"ur ist aber betr"achtlich, so dass wir auf die
detaillierte Darstellung dieser Herleitung verzichten wollen.

\begin{table}
\centering
\begin{tabular}{|r|c|r|r|r|r|r|}
\hline
$i$& $x$ & $y(x)=e^{-\alpha x}$&Euler&verbessert&vereinfacht&Runge-Kutta\\
\hline
 0 & 0.0 & 1.00000000 & 1.000 & 1.00000000 & 1.00000000 & 1.0000000000 \\
 1 & 0.1 & 0.95122942 & 0.\underline{95}0 & 0.\underline{9512}5000 & 0.\underline{9512}5000 & 0.\underline{95122942}71 \\
 2 & 0.2 & 0.90483742 & 0.\underline{90}2 & 0.\underline{9048}7656 & 0.\underline{9048}7656 & 0.\underline{9048374}229 \\
 3 & 0.3 & 0.86070798 & 0.\underline{85}7 & 0.\underline{8607}6383 & 0.\underline{8607}6383 & 0.\underline{8607079}834 \\
 4 & 0.4 & 0.81873075 & 0.\underline{81}4 & 0.\underline{8188}0159 & 0.\underline{8188}0159 & 0.\underline{8187307}620 \\
 5 & 0.5 & 0.77880078 & 0.\underline{77}3 & 0.\underline{7788}8502 & 0.\underline{7788}8502 & 0.\underline{7788007}936 \\
 6 & 0.6 & 0.74081822 & 0.\underline{73}5 & 0.\underline{7409}1437 & 0.\underline{7409}1437 & 0.\underline{7408182}327 \\
 7 & 0.7 & 0.70468809 & 0.\underline{69}8 & 0.\underline{704}79480 & 0.\underline{704}79480 & 0.\underline{7046881}031 \\
 8 & 0.8 & 0.67032005 & 0.\underline{6}63 & 0.\underline{670}43605 & 0.\underline{670}43605 & 0.\underline{6703200}606 \\
 9 & 0.9 & 0.63762815 & 0.\underline{6}30 & 0.\underline{637}75229 & 0.\underline{637}75229 & 0.\underline{6376281}672 \\
10 & 1.0 & 0.60653066 & 0.\underline{5}98 & 0.\underline{606}66187 & 0.\underline{606}66187 & 0.\underline{6065306}762 \\
\hline
\end{tabular}
\caption{Vergleich der Genauigkeit der verbesserten numerischen Verfahren.
Unterstrichen jeweils die nach Rundung korrekten Stellen der L"osung.
\label{numerik:genauigkeit}}
\end{table}


\begin{table}
\centering
\begin{tabular}{|l|l|c|r|>{$}r<{$}|}
\hline
Verfahren                           &$h$  &Schritte&$y_n$&\text{Fehler}\\
\hline
Euler-Verfahren                     &0.025&  40    & 0.\underline{60}462232 &  0.00190834 \\
verbessertes Euler-Verfahren        &0.05 &  20    & 0.\underline{6065}6285 & -0.00003219 \\
vereinfachtes Runge-Kutta-Verfahren &0.05 &  20    & 0.\underline{6065}6285 & -0.00003219 \\
Runge-Kutta-Verfahren               &0.1  &  10    & 0.\underline{6065306}7 & -0.00000001 \\
\hline
\end{tabular}
\caption{Vergleich der verschiedenen Verfahren bei gleichbleibendem 
Rechenaufwand.
Die Schrittweite wurde jeweils so angepasst, dass in allen Verfahren bis
zum Wert $x=1$ die gleiche Anzahl von Auswertungen der Funktion $f$
notwendig wurde.
\label{numerik:vergleich-aufwand}}
\end{table}

Die Tabelle~\ref{numerik:genauigkeit} demonstriert die "uberragende
Genauigkeit des Runge-Kutta-Verfahrens.
Trotz der relativ grossen Schrittweite von $h=0.1$ erreicht das
Verfahren nach zehn Schritten eine Genauigkeit von sieben signifikanten
Stellen.
Da in jedem Schritt die Funktion $f$ viermal ausgewertet werden muss,
ist der Rechenaufwand mit dem Runge-Kutta-Verfahren viermal gr"osser
als im Euler-Verfahren, letzteres kann aber mit nur einer signifikanten
Stelle kaum als brauchbar bezeichnet werden.
Passt man in jedem Verfahren die Schrittweite so an, dass f"ur die
Berechnung der N"aherung f"ur $y(1)$ immer gleich viele Auswertungen
der Funktion $f(x,y)$ n"otig sind, ergeben sich die Resultate in
Tabelle~\ref{numerik:vergleich-aufwand}.
Bei gleichem Rechenaufwand ist das Runge-Kutta-Verfahren um viele
Gr"ossenordungen pr"aziser.
Es gibt daher eigentlich keinen praktischen Grund, "uberhaupt je etwas
anderes als das Runge-Kutta-Verfahren zu verwenden.


\section{Mehrschritt-Verfahren}
\lhead{Mehrschritt-Verfahren}
In den Einschritt-Verfahren wurde wiederholt die Funktion $f$ ausgewertet,
um die Inkrement-Funktion f"ur einen einzigen Schritt zu bestimmen.
Das Ziel dabei war, $y(x+h)$ in "Ubereinstimmung mit der Taylorreihe
bis zu m"oglichst hoher Ordnung zu bestimmen.
Im Runge-Kutta-Verfahren wurden dabei halbe Eulerschritte durchgef"uhrt,
man hat also eigentlich die Aufl"osung nochmals halbiert, um die
Inkrement-Funktion zu ermitteln.
Diese Zwischenwerte geben dem Verfahren die Information "uber die
h"oheren Ableitungen der Funktionen.

Sobald einige Werte der L"osung berechnet sind, l"asst sich die Kr"ummung
der L"osungskurve auch aus diesen Werten ablesen.
Es sollte daher auch m"oglich sein, aus mehreren bereits
ermittelten Werten $y_{n\mathstrut},y_{n+1},\dots,y_{n+s-1}$
den n"achsten Wert $y_{n+s\mathstrut}$ mit der verlangten Genauigkeit
zu berechnen.
Der Vorteil eines solchen Vorgehens ist, dass f"ur jeden Schritt nur 
eine einzige Auswertung der Funktion $f$ n"otig ist,
nicht mehrere wie bei den besprochenen Einschritt-Verfahren.

Als Beispiel versuchen wir daher ein Verfahren aufzubauen, welches
$y_{n+2}$ aus den bereits berechneten Werten $y_{n\mathstrut}$ und
$y_{n+1}$ berechnet.
Wir nehmen dabei an, dass $y_{n\mathstrut}$ und $y_{n+1}$ exakt
sind.
Der neue Datenpunkt soll mit Hilfe eines Ausdrucks der Form
\begin{equation}
y_{n+2}=y_{n+1} + h(af(x_{n+1},y_{n+1}) + b f(x_{n\mathstrut},y_{n\mathstrut}))
\label{numerik:zweischrittansatz}
\end{equation}
gefunden werden.
Die N"aherung kann wieder mit Hilfe der Ableitungen alleine
durch Werte bei $x_{n+1}$ ausgedr"uckt werden:
\begin{align*}
y_{n+2}
&=
y_{n+1}+h(af(x_{n+1}, y_{n+1}) + bf(x_{n+1}-h, y_{n\mathstrut}))
\\
&=
y_{n+1}+haf(x_{n+1}, y_{n+1}) + hbf(x_{n+1}-h, y_{n+1} - h f(x_{n+1},y_{n+1}) + O(h^2))
\\
&=
y_{n+1}+haf(x_{n+1}, y_{n+1}) + hb
\biggl(
f(x_{n+1},y_{n+1})
-h \frac{\partial f(x_{n+1},y_{n+1})}{\partial x}
\\
&\qquad
-h
\frac{\partial f(x_{n+1},y_{n+1})}{\partial y}
f(x_{n+1},y_{n+1})
+ 
\frac{\partial f(x_{n+1},y_{n+1})}{\partial y}
O(h^2)
\biggr)
\\
&=
y_{n+1}
+ (a+b)hf(x_{n+1},y_{n+1})
- bh^2\biggl(
\frac{\partial f(x_{n+1},y_{n+1})}{\partial x}
+
\frac{\partial f(x_{n+1},y_{n+1})}{\partial y}
f(x_{n+1},y_{n+1})
+O(h^3)
\biggr)
\end{align*}
Sie muss bis zur zweiten Ordnung mit der Taylorreihe "ubereinstimmen:
\begin{align*}
y(x_{n+2})
&=
y_{n+1} + hy'(x_{n+1}) + \frac12h^2 y''(x_{n+1})+O(h^3)
\\
&=
y_{n+1}+hf(x_{n+1},y_{n+1})+\frac12h^2\biggl(
\frac{\partial f(x_{n+1},y_{n+1})}{\partial x}
+
\frac{\partial f(x_{n+1},y_{n+1})}{\partial y}
f(x_{n+1},y_{n+1})
\biggr)
\end{align*}
Vergleicht man Koeffizienten, findet man
\[
\begin{aligned}
a+b&=1&-b&=\frac12&&\Rightarrow&a=\frac32
\end{aligned}
\]
Aus der Formel (\ref{numerik:zweischrittansatz}) wird somit die
Iterationsformel
\begin{equation}
y_{n+2}=y_{n+1}+h\biggl(\frac32f(x_{n+1},y_{n+1})
- \frac12 f(x_{n\mathstrut},y_{n\mathstrut})\biggr)
\end{equation}
Diese Rekursionsformel definiert ein quadratisches Verfahren, das
{\em Adams-Bashforth-Verfahren} mit $s=2$.
\index{Adams-Bashforth-Verfahren}

Das Verfahren kann "ahnlich wie das Runge-Kutta-Verfahren auf h"ohere
Ordnung erweitert werden.
Man findet nach einiger Rechnung
\begin{align*}
s&=1\colon&
y_{n+1}
&=
y_n+hf(x_n,y_n)
\\
s&=2\colon&
y_{n+2}
&=
y_{n+1}+h\biggl(\frac32f(x_{n+1},y_{n+1})-\frac12f(x_n,y_n)\biggr)
\\
s&=3\colon&
y_{n+3}
&=
y_{n+2}+h\biggl(\frac{23}{12}f(x_{n+2},y_{n+2})-\frac43f(x_{n+1},y_{n+1})+\frac{5}{12}f(x_n,y_n)\biggr)
\\
s&=4\colon&
y_{n+4}
&=
y_{n+3}+h\biggl(\frac{55}{24}f(x_{n+3},y_{n+3})
	-\frac{59}{24}f(x_{n+2},y_{n+2})
	+\frac{37}{24}f(x_{n+1},y_{n+1})
	-\frac{3}{8}f(x_n,y_n)
\biggr)
\end{align*}
Es ist also m"oglich, ausgehend von dieser Idee Verfahren beliebig hoher
Ordnung zu produzieren.

\begin{table}
\centering
\begin{tabular}{|r|c|r|r|r|r|}
\hline
$i$& $x$ & $y(x)=e^{-\alpha x}$&Euler&Adams-Bashforth&Runge-Kutta\\
\hline
 0 & 0.0 & 1.00000000 & 1.00000000 & 1.00000000 & 1.0000000000 \\
 1 & 0.1 & 0.95122942 & 0.\underline{95}000000 & 0.\underline{9512}8178 & 0.\underline{95122942}71 \\
 2 & 0.2 & 0.90483742 & 0.\underline{90}250000 & 0.\underline{904}93564 & 0.\underline{90483742}29 \\
 3 & 0.3 & 0.86070798 & 0.\underline{85}737500 & 0.\underline{860}84752 & 0.\underline{86070798}34 \\
 4 & 0.4 & 0.81873075 & 0.\underline{81}450625 & 0.\underline{818}90734 & 0.\underline{81873076}20 \\
 5 & 0.5 & 0.77880078 & 0.\underline{77}378094 & 0.\underline{779}01048 & 0.\underline{77880079}36 \\
 6 & 0.6 & 0.74081822 & 0.\underline{73}509189 & 0.\underline{741}05738 & 0.\underline{74081823}27 \\
 7 & 0.7 & 0.70468809 & 0.\underline{69}833730 & 0.\underline{704}95334 & 0.\underline{7046881}031 \\
 8 & 0.8 & 0.67032005 & 0.\underline{6}6342043 & 0.\underline{670}60827 & 0.\underline{6703200}606 \\
 9 & 0.9 & 0.63762815 & 0.\underline{63}024941 & 0.\underline{637}93648 & 0.\underline{6376281}672 \\
10 & 1.0 & 0.60653066 & 0.\underline{59}873694 & 0.\underline{606}85645 & 0.\underline{6065306}762 \\
\hline
\end{tabular}
\caption{Vergleich der Genauigkeit der Verfahren von Euler,
Adams-Bashforth und Runge-Kutta.
Als Startwerte f"ur das Adams-Bashforth-Verfahren wurden die
Werte $y(-h)=e^{-\alpha h}$ und $y(0)=1$ verwendet, um keine zus"atzlichen
Fehler aus der Durchf"uhrung des ersten Schrittes hinzuzuf"ugen.
\label{numerik:genauigkeit-adams-bashforth}}
\end{table}

In der Tabelle~\ref{numerik:genauigkeit-adams-bashforth} wird
das Adams-Bashforth-Verfahren verglichen mit dem lineare Euler-Verfahren 
und dem Verfahren vierter Ordnung von Runge-Kutta.
Die Verbesserung der Genauigkeit des Adams-Bashforth-Verfahrens
gegen"uber dem Euler-Ver\-fah\-ren ist konsistent damit, dass
das Adams-Bashforth-Verfahren ein quadratisches Verfahren ist.

Nachteilig an den Mehrschritt-Verfahren ist die Notwendigkeit,
gen"ugend viele Werte $y_{n},\dots,y_{n+s-1}$ mit ausreichend
hoher Genauigkeit zu bestimmen, bevor das Mehrschritt-Verfahren
seine Schritte der Ordnung $s$ beginnen kann.
Solange diese Werte nicht zur Verf"ugung stehen, kann ein Mehrschritt-Verfahren
nur Schritte niedrigerer Ordnung als $s$ durchf"uhren.

Bei einem Einschritt-Verfahren kann in jedem Schritt die Schrittweite $h$
ver"andert werden, zum Beispiel f"ur Bereiche von $x$-Werten, in denen
die Steigung von $y(x)$ sehr rasch "andert.

F"ur die Beispiel-Differentialgleichung (\ref{numerik:expdgl}) k"onnen
wir das Adams-Bashforth-Verfahren zweiter Ordnung ($s=2$) vollst"andig
analysieren.
Die Rekursionsformel wird zu
\[
y_{n+2}=y_{n+1}+h\biggl(\frac32 (-\alpha y_{n+1})-\frac12(-\alpha y_n)\biggr)
=
\biggl(1-\frac32\alpha h\biggr)
y_{n+1}
+\frac{\alpha h}{2}
y_{n\mathstrut}
\]
Dies ist eine Differenzengleichung mit konstanten Koeffizienten, man kann
sie mit Hilfe eines Potenzansatzes l"osen. 
Wir nehmen also an, dass $y_n=\lambda^n$, und setzen dies in die
Rekursionsformel ein.
Ausserdem k"urzen wir $\alpha h/2=\delta$  ab.
Wir erhalten
\[
\lambda^{n+2}-(1-3\delta)\lambda^{n+1}-\delta\lambda^n=0.
\]
Nach Division durch $\lambda^n$ erhalten wir die quadratische Gleichung
\[
\lambda^2-(1-3\delta )\lambda-\delta=0
\]
f"ur $\lambda$ mit den L"osungen
\[
\lambda_\pm
=
\frac12(1-3\delta) \pm \frac12\sqrt{(1-3\delta)^2+4\delta}.
\]
Da $\delta$ klein ist, wird $\lambda_-$ ebenfalls klein sein,
w"ahrend $\lambda_+$ n"aher bei $1$ sein wird.
Der dominante Einfluss auf die L"osung r"uhrt also von $\lambda_+$ her.
Um diesen Unterschied genauer zu verstehen, verwenden wir eine
lineare Approximation der Wurzel auf der rechten Seite von $\lambda_\pm$:
\begin{align*}
\sqrt{1+x}
&=
1+\frac{x}{2}-\frac{x^2}{4}+\frac{3x^3}{8}-\dots
\\
\sqrt{x}
&=
\sqrt{x_0+x-x_0}
=
\sqrt{x_0}\sqrt{1+\frac{x-x_0}{x_0}}
=
\sqrt{x_0}\biggl(1+\frac12\frac{x-x_0}{x_0}-\frac14\frac{(x-x_0)^2}{x_0^2}+\dots\biggr)
\\
&=
\sqrt{x_0}+\frac12\frac{x-x_0}{\sqrt{x_0}}-\frac14\frac{(x-x_0)^2}{\sqrt{x_0}^3}+\dots
\end{align*}
Wir verwenden diese Approximation mit $x_0=(1-3\delta)^2$ und $x-x_0=-4\delta$
\begin{align*}
\sqrt{(1-3\delta)^2+4\delta}
&=
(1-3\delta)\biggl(1+\frac12\frac{4\delta}{(1-3\delta)^2}
-\frac14\frac{16\delta^2}{(1-3\delta)^4}+\dots\biggr)
\\
&=(1-3\delta)+\frac12\frac{4\delta}{1-3\delta}
-\frac14\frac{16\delta^2}{(1-3\delta)^3}+\dots
\\
&=1-3\delta+2\delta(1+3\delta)-4\delta^2+O(\delta^3)
\\
&=1-3\delta+2\delta + 2\delta^2+O(\delta^3)
\\
&=1-3\delta+2\delta + \frac12(2\delta)^2+O(\delta^3)
\end{align*}
Damit k"onnen wir jetzt $\lambda_+$ bis zur zweiten Ordnung berechnen:
\begin{align*}
\lambda_+
&=
\frac12\biggl((1-3\delta)+ (1-3\delta)+2\delta+\frac12(2\delta)^2\biggr)
+O(\delta^3)
\\
&=
1-2\delta+\frac12(2\delta)^2+O(\delta^3)
\\
&=e^{-2\delta}+O(\delta^3).
\end{align*}
Die exakte L"osung erf"ullt $y_{n+1}=e^{-2\delta}y_n$, der Faktor
$\lambda_+$ stimmt bis auf Terme mindestens dritter Ordnung mit 
$e^{-2\delta}$ "uberein.
Damit ist erneut best"atigt, dass wir es mit einem quadratischen Verfahren zu
tun haben.

Wir k"onnen auch $\lambda_-$ berechnen, und erhalten
\[
\lambda_-=-\delta-2\delta^2+O(\delta^3).
\]
Da $\delta$ klein ist, ist eine Komponente der L"osung bereits nach
drei Schritten kleiner als $O(\delta^3)$, und spielt daher im Vergleich
zu den von $\lambda_+$ herr"uhrenden L"osungen in dritter Ordnung keine
Rolle.

\section{Software}
Die im letzten Abschnitt entwickelten numerischen Verfahren zur L"osung
einer Differentialgleichung kommen ausschliesslich mit Auswertungen der
Funktion $f$ aus, die Ableitungen der Funktion $f$ m"ussen nicht bekannt
sein.
Es sollte also ein Leichtes sein, eine Softwarebibliothek zur
Verf"ugung zu stellen, mit der eine beliebige gew"ohnliche
Differentialgleichung gel"ost werden kann.
Als Input braucht es nur die Funktion $f$ und die Anfangsbedingungen.

Als Beispiel wollen wir in diesem Abschnitt die Differentialgleichung
\[
y''+y=\sin \frac{x}{10},\qquad y(0)=y'(0)=0
\]
in verschiedenen Programmierumgebungen l"osen.
Als erstes bringen wir die Differentialgleichung wieder in die Standardform
einer Vektordifferentialgleichung erste Ordnung:
\begin{equation}
\frac{d}{dt}Y
=
\frac{d}{dx}\begin{pmatrix}y_1\\y_2\end{pmatrix}
=
\begin{pmatrix}
y_2\\
-y_1+\sin\frac{x}{10}
\end{pmatrix}
=
f(x,Y)
\label{numerik:beispieldgl}
\end{equation}
Ein numerisches Verfahren braucht also als Input eine Anfangsbedingung
sowie die Funktion $f$.
Ausserdem muss es M"oglichkeiten bereitstellen, wie man den Gang der
Rechnung beeinflussen kann, z.~B.~um die $x$-Werte anzugeben, f"ur die
die $Y(x)$ bestimmt werden sollen, oder um Genauigkeitsziele zu erreichen.

\subsection{Octave}
In Octave steht eine einzige Funktion \texttt{lsode} zur Verf"ugung, welche
auf zuverl"assige Art Differentialgleichungen l"ost.
Der Anwender muss eine Implementation der Funktion $f$ zur Verf"ugung
stellen, allerdings werden die Argument in der umgekehrten Reihenfolge
zu der erwarte, die wir in diesem Skript bisher verwendet haben.
F"ur die Beispieldifferentialgleichung (\ref{numerik:beispieldgl})
kann man sie zum Beispiel so definieren:
\verbatiminput{chapters/examples/octave-dgl-f.m}

Beim Aufruf der Funktion \texttt{lsode} muss man den {\em Namen}
der Funktion, die Anfangsbedingung, sowie einen Vektoren mit $x$-Werten,
f"ur die man die L"osung ausgegeben haben m"ochte, als Argumente
"ubergeben.
Der erste Wert im $x$-Vektor muss der $x$-Wert f"ur die Anfangsbedingung
sein, in unserem Fall also $0$.
Um die Werte von $y(x)$ f"ur ganzzahlige Werte von $x$ zu erhalten,
muss man also die Befehle
\verbatiminput{chapters/examples/octave-dgl-sol.m}
ausf"uhren.
Als R"uckgabewert erh"alt man eine Matrix, die in jeder Zeile die
Werte von $y(x)$ und $y'(x)$ zum entsprechenden Wert von $x$
aus dem \texttt{x}-Argument enth"alt.
Die Resultate sind zusammen mit den Werten der exakten
L"osung~(\ref{grundlagen:numerik-beispiel-loesung}) in der dritten Spalte 
in der Tabelle~\ref{numerik:octave-resultate} zusammengestellt.
Es ist gut erkennbar, wie der Fehler anf"anglich langsam ansteigt,
dann aber unter Kontrolle bleibt.
Die Dokumentation der Funktion \texttt{lsode} beschreibt, wie man mit
Hilfe von Optionen ihr Verhalten und insbesondere die Gr"osse der
Fehler weiter beeinflussen kann.
\begin{table}
\centering
\begin{tabular}{|>{$}r<{$}|>{$}r<{$}|>{$}r<{$}|>{$}r<{$}|}
\hline
    x&  y_{\text{numerisch}}(x)&y_{\text{exakt}}(x) & \text{Fehler}\\
\hline
    0&  0.00000000&  0.00000000&  0.00000000\\
    1&  0.00158525&  0.00158528&  0.00000003\\
    2&  0.01090682&  0.01090678&  0.00000003\\
    3&  0.02858723&  0.02858716&  0.00000007\\
    4&  0.04756207&  0.04756212&  0.00000004\\
    5&  0.05957416&  0.05957437&  0.00000020\\
    6&  0.06276426&  0.06276444&  0.00000018\\
    7&  0.06337942&  0.06337932&  0.00000010\\
    8&  0.07002849&  0.07002811&  0.00000037\\
    9&  0.08576626&  0.08576594&  0.00000032\\
   10&  0.10528405&  0.10528416&  0.00000010\\
  100&  0.84661503&  0.84661930&  0.00000427\\
 1000& -0.55228836& -0.55234514&  0.00005678\\
 2000&  0.90392063&  0.90373523&  0.00018540\\
 3000& -0.99018339& -0.99032256&  0.00013917\\
 4000&  0.75185982&  0.75202340&  0.00016358\\
 5000& -0.25298074& -0.25252044&  0.00046030\\
 6000& -0.30093757& -0.30056348&  0.00037408\\
 7000&  0.76905079&  0.76889872&  0.00015207\\
 8000& -1.00327790& -1.00396748&  0.00068958\\
 9000&  0.88860437&  0.88793099&  0.00067338\\
10000& -0.50337856& -0.50335983&  0.00001873\\
\hline
\end{tabular}
\caption{Exakte und numerische L"osung der Beispieldifferentialgleichung
berechnet mit der Funktion \texttt{lsode} von Octave.
\label{numerik:octave-resultate}}
\end{table}

\subsection{GNU Scientific Library}
W"ahrend Octave dem Benutzer die Wahl eines geeigneten Verfahrens abnimmt
und ihm "uberhaupt wenig Kontrolle "uber den Gang der Rechnung gibt,
kann ein Programmierer durch den Einsatz der GNU Scientific Library (GSL) die
volle Kontrolle "uber alle Aspekte der Iteration erhalten.
Der Preis ist eine wesentlich h"ohere Komplexit"at.
Ziel dieses Abschnitts ist, ein einfaches Beispielprogramm zu
zeigen, welches als Basis eigener Programme dienen kann.
Es verwendet eine Runge-Kutta-Verfahren achter Ordnung.

Die Funktionen zum L"osen von gew"ohnlichen Differentialgleichungen
der GSL haben alle das Pr"afix \texttt{gsl\_odeiv2\_}. 
Zun"achst braucht es nat"urlich wieder eine Implementation der
Funktion $f$. 
Die GSL "ubergibt zwei Arrays, im einen findet die Funktion die aktuellen
$Y$-Werte, im anderen soll sie die Werte der Ableitung zur"uckgeben.
F"ur die Beispiel-Differentialgleichung (\ref{numerik:beispieldgl})
sieht der Code wie folgt aus:
\verbatiminput{chapters/examples/dgl-f.c}
Der Parameter \texttt{params} dient dazu, der Funktion zus"atzliche
Parameter zu "ubergeben.
In unserem Fall ist das nur die Zahl $\omega$.
Da \texttt{params} ein \texttt{void}-Pointer ist, kann eine beliebige
Struktur zur Parameter"ubergabe verwendet werden.

Die Differentialgleichung wird beschrieben durch eine Struktur vom Typ
\texttt{gsl\_odeiv2\_system}, welche ausser Zeigern auf die Funktion
und die Parameter-Struktur auch noch die Dimension der Vektoren enth"alt.
Es kann auch noch ein Funktionszeiger f"ur eine eine Funktion "ubergeben
werden, die die Jacobi-Matrix berechnet, in unserem Beispiel wird dies
jedoch nicht ben"otigt.

Die eigentliche wird von einer ``driver''-Funktion durchgef"uhrt.
Diese sorgt im wesentlichen f"ur die Wahl der Schrittweite, verwaltet
Datenstrukturen, und ruft die Funktionen auf, die die einzelnen Schritte
durchf"uhren.
Die Treiber-Funktion f"uhrt die einzelnen Schritte (im Sinne der
in Abschnitt~\ref{section:numerik:einschritt} besprochenen
Einschritt-Verfahren) mit
Hilfe der Schritt-Funktionen durch, von denen die Bibliothek eine
ganze Reihe bereitstellt.
Die Funktion \texttt{gsl\_odeiv2\_step\_rk4} ist das klassische
Runge-Kutta-Verfahren vierter Ordnung, welches in
Abschnitt~\ref{subsection:numerik:runge-kutta}
beschrieben wurde.
Im Beispielverfahren verwenden wir \texttt{gsl\_odeiv2\_step\_rk8pd},
das Runge-Kutta Prince-Dormand Verfahren achter Ordnung.
F"ur Aufgaben allgemeiner Art ebenfalls sehr gut geeignet ist das
Runge-Kutta-Fehlberg-Verfahren f"unfter Ordnung mit dem Namen
\texttt{gsl\_odeiv2\_step\_rkf45}.
Diese Datenstrukturen werden mit dem Code
\verbatiminput{chapters/examples/dgl-init.c}
initialisiert.
Durch Austausch des zweiten Arguments der Driver-Allozierungs-Funktion
kann man leicht das Verfahren wechseln und so Zeitaufwand und Genauigkeit
f"ur verschiedene L"osungsverfahren vergleichen.

Um die Rechnung durchzuf"uhren, muss jetzt die Driver-Funktion so oft
angewendet werden, wie man Punkt der L"osungskurve ausgeben will.
Dazu dient die Funktion \texttt{gsl\_odeiv2\_driver\_apply}. 
An Argument braucht sie den eben initialisierten Driver, den aktuellen
$x$-Wert, den $x_{\text{next}}$-Wert, f"ur den der n"achste Punkt
ausgegeben werden soll, sowie einen Vektor, in dem der aktuelle Anfangswert
f"ur $Y(x)$ "ubergeben und $Y(x_{\text{next}})$ zur"uckgegeben wird.
$x$ wird als Referenz "ubergeben, wenn die Funktion zur"uckkehrt,
findet man dort den neuen aktuellen Wert von $x$, also im Erfolgsfall
$x_{\text{next}}$.
In unserem Fall brauchen wir $X(x)$ f"ur ganzzahlige $x$, die folgende
Schleife bewerkstelligt dies:
\verbatiminput{chapters/examples/dgl-loop.c}

\begin{table}
\centering
\begin{tabular}{|>{$}r<{$}|>{$}r<{$}|>{$}r<{$}|>{$}r<{$}|}
\hline
    x&  y_{\text{numerisch}}(x)&y_{\text{exakt}}(x) & \text{Fehler}\\
\hline
    1&   0.01584477&   0.01584477&  -0.00000000\\
    2&   0.10882786&   0.10882787&  -0.00000000\\
    3&   0.28425071&   0.28425071&  -0.00000001\\
    4&   0.46979656&   0.46979656&  -0.00000000\\
    5&   0.58112927&   0.58112926&   0.00000001\\
    6&   0.59856974&   0.59856972&   0.00000002\\
    7&   0.58436266&   0.58436265&   0.00000001\\
    8&   0.62466692&   0.62466694&  -0.00000001\\
    9&   0.74961115&   0.74961117&  -0.00000003\\
   10&   0.90492231&   0.90492232&  -0.00000001\\
   20&   0.82626555&   0.82626556&  -0.00000000\\
   30&   0.24234669&   0.24234664&   0.00000005\\
   40&  -0.83971103&  -0.83971092&  -0.00000011\\
   50&  -0.94210772&  -0.94210787&   0.00000015\\
   60&  -0.25144906&  -0.25144893&  -0.00000013\\
   70&   0.58545210&   0.58545205&   0.00000005\\
   80&   1.09974465&   1.09974456&   0.00000009\\
   90&   0.32597838&   0.32597860&  -0.00000023\\
  100&  -0.49836792&  -0.49836823&   0.00000031\\
  200&   1.01037929&   1.01037877&   0.00000052\\
  300&  -0.89702589&  -0.89702630&   0.00000041\\
  400&   0.83859090&   0.83859101&  -0.00000010\\
  500&  -0.21777633&  -0.21777543&  -0.00000090\\
  600&  -0.31235410&  -0.31235239&  -0.00000171\\
  700&   0.72675906&   0.72676124&  -0.00000219\\
  800&  -1.09422992&  -1.09422790&  -0.00000202\\
  900&   0.80223761&   0.80223872&  -0.00000111\\
 1000&  -0.59500324&  -0.59500363&   0.00000040\\
 2000&  -0.97606658&  -0.97606187&  -0.00000471\\
 3000&  -1.03200392&  -1.03199479&  -0.00000914\\
 4000&  -0.79047800&  -0.79047372&  -0.00000428\\
 5000&  -0.37269297&  -0.37270218&   0.00000921\\
 6000&   0.08785158&   0.08783273&   0.00001886\\
 7000&   0.49827647&   0.49826463&   0.00001184\\
 8000&   0.80219750&   0.80220742&  -0.00000992\\
 9000&   0.94568799&   0.94571577&  -0.00002778\\
10000&   0.86607968&   0.86610200&  -0.00002232\\
\hline
\end{tabular}
\caption{L"osungen der Beispieldifferentialgleichung (\ref{numerik:beispieldgl})
mit Hilfe der GNU Scientific Library (GSL).
\label{numerik:gsl-resultate}}
\end{table}

Man kann die Funktion $f$ im Programm nat"urlich auch mit einem Z"ahler
ausstatten und damit herausfinden, wie viele Aufrufe der Funktion
f"ur die numerische L"osung ben"otigt werden.
Es stellt sich heraus, dass f"ur das erste Intervall von $0$ bis $1$
die Funktion $f$ 131 mal aufgerufen wird, hier versucht die Bibliothek
die optimale Schrittweite $h$ zu bestimmen.
In allen folgenden Intervallen der L"ange $1$ von $n$ bis $n+1$ werden nur
noch jeweils 13 Aufrufe der Funktion ben"otigt.
Verwendet man stattdessen das Runge-Kutta-Fehlberg-Verfahren,
werden pro Intervall 18 Auswertungen der Funktion $f$ ben"otigt,
und die Genauigkeit sinkt auf zwei Stellen nach dem Komma.

\section{Randwertprobleme\label{section:numerik:randwertprobleme}}
Die bisher beschriebenen Verfahren gehen von einer Anfangsbedingung
aus, und berechnen die dadurch eindeutig festgelegte L"osungskurve.
Randwertproblem, beschrieben in Abschnitt~\ref{section:randwertprobleme},
verkn"upfen dagegen Werte von einzelnen Komponenten von $Y$ an den
R"andern eines Intervalls.

Wir betrachten zwei prototypische Randwertprobleme, die auch gleich
zwei v"ollig verschiedene L"osungsverfahren motivieren.

\newtheorem{aufgabe}{Aufgabe}[chapter]
\begin{aufgabe}
\label{numerik:aufgabe-ball}
Mit einem nur der Schwerkraft unterworfenen Ball, der im Ursprung des
Koordinatensystems geworfen wird, soll ein Ziel im Punkt $P$ getroffen
werden.
In welcher Richtung und mit welcher Anfangsgeschwindigkeit muss er geworfen
werden?
\end{aufgabe}

Um das Problem einfach zu halten, modellieren wir diese Aufgabe wie
folgt.
Der Ball der Masse $m$ bewegt sich in der $x$-$y$-Ebene, wobei die
Schwerkraft in negativer $y$-Richtung zeigt.
Das Newtonsche Gesetz liefert die Differentialgleichung zweiter Ordnung
\begin{equation}
m\frac{d^2}{dt^2}\begin{pmatrix}x\\y\end{pmatrix}
=
\begin{pmatrix}
0\\-mg
\end{pmatrix}
\label{numerik:ball-dgl}
\end{equation}
Die Masse $m$ kann herausgek"urzt werden.
Gesucht ist eine L"osung so, dass die Bahn durch die Punkte $(0,0)$
und $P=(p,0)$ geht.

Genau genommen ist dies nicht ein Randwertproblem wie in 
Abschnitt~\ref{section:randwertprobleme}, denn es wird nicht verlangt,
dass der Ball zu einer bestimmten Zeit $t$ beim Punkt $P$ eintrifft.
Die Differentialgleichung bedeutet aber, dass die Horizontalgeschwindigkeit
des Balls konstant ist (die horizontale Beschleunigung ist immer $0$).
Ist $v_x$ die Horizontalgeschwindigkeit, dann erreicht der Ball zur
Zeit $t_1=p/v_x$ die $x$-Koordinate des Ziels.
Gesucht ist also die anf"angliche Vertikalgeschwindigkeit, die man
dem Ball geben muss, dass zur Zeit $p/v_x$ die $y$-Komponente
der L"osung den Wert $0$ hat.
In dieser Form liegt ein Randwertproblem wie in
Abschnitt~\ref{section:randwertprobleme} vor.

Die L"osungen der Differentialgleichung~\ref{numerik:ball-dgl} sind aus
dem Physik-Unterricht bekannt:
es gilt
\begin{equation}
\begin{pmatrix}x(t)\\y(t)\end{pmatrix}
=
\begin{pmatrix}v_xt\\ v_yt-\frac12gt^2\end{pmatrix}
\end{equation}
Damit l"asst sich auch das Randwertproblem l"osen.
F"ur $t=v_x/p$ muss $y(t)=0$ sein, also
\begin{align}
y(t)=y\biggl(\frac{p}{v_x}\biggr)
=v_y\frac{p}{v_x}-\frac12g\biggl(\frac{p}{v_x}\biggr)^2&=0
\notag
\\
\Rightarrow\qquad
v_y
&=
\frac{v_x}p\frac12g\frac{p^2}{v_x^2}=\frac{gp}{2v_x}.
\label{numerik:ball-bedingung}
\end{align}
Offenbar gibt es zu jedem $v_x$ einen passenden Wert von $v_y$,
mit dem das Ziel getroffen wird.

Die Differentialgleichung (\ref{numerik:ball-dgl}) ist nicht in einer
Form, die der numerischen L"osung zug"anglich ist.
Wir schreiben Sie daher als Differentialgleichung erster Ordnung 
f"ur vierdimensionale Vektoren:
\begin{align}
\frac{d}{dt}Y
=
\frac{d}{dt}\begin{pmatrix}x\\y\\\dot x\\\dot y\end{pmatrix}
&=
\begin{pmatrix}\dot x\\\dot y\\ 0\\ -g\end{pmatrix}.
\label{numerik:ball-dgl-1}
\end{align}
Gesucht ist eine L"osung, die die Randbedingungen
\begin{equation}
Y(0)
=
\begin{pmatrix}0\\0\\v_x\\\color{red}v_y\end{pmatrix},
\qquad
Y\biggl(\frac{p}{v_x}\biggr)
=
\begin{pmatrix}p\\0\\\color{red}?\\\color{red}?\end{pmatrix}
\label{numerik:ball-dgl-2}
\end{equation}
erf"ullt.
Darin stehen die roten Eintr"age f"ur Werte, die nicht vorgegeben sind.
Aus der Symmetrie des Problems kann man nat"urlich auch die Endgeschwindigkeit
ablesen.
Zu bestimmen ist also $v_y$ so, dass die L"osungskurve durch den Punkt
$(p,0)$ geht.

Wird statt der Horizontalkomponenten der Anfangsgeschwindigkeit die
gesamte Anfangsgeschwindigkeit $v_0$ vorgegeben, dann muss der
Winkel gefunden werden, unter dem der Ball geworfen werden muss,
um das Ziel zu trefen.
Bei der Elevation $\alpha$ sind die Komponenten der Anfangsgeschwindigkeit
$v_x=v_0\cos\alpha$ und $v_y=v_0\sin\alpha$. 
Setzt man dies in die Bedingung~(\ref{numerik:ball-bedingung}) ein,
findet man
\begin{align*}
v_0 \sin \alpha &=\frac{gp}{2v_0\cos\alpha}
\\
2\sin\alpha\cos\alpha&=\frac{gp}{v_0^2}
\\
\sin2\alpha&=\frac{gp}{v_0^2}
\\
\alpha&= \frac12 \arcsin\frac{gp}{v_0^2}
\end{align*}
Im Nenner rechts steht im wesentlichen die kinetische Energie,
je mehr kinetische Energie der Ball zu Beginn hat, desto kleiner
ist der Winkel, man trifft das Ziel mit einer sehr flachen Bahn.
Kleine Winkel reichen auch f"ur geringe Schwerkraft ($g$ klein)
und kurze Distanzen ($p$ klein).
Die maximale Distanz wird erreicht, wenn das Argument des Arcussinus
den Wert $1$ erreicht, gr"osser darf $p$ nicht werden, weil es sonst
keine L"osung mehr f"ur $\alpha$ gibt.
Die Maximaldistanz ist daher
\[
p_{\text{max}} = \frac{v_0^2}{g}.
\]

\begin{aufgabe}
\label{numerik:aufgabe-seil}
Ein Seil ist zwischen zwei Punkten aufgeh"angt, welche Form nimmt es
allein unter der Wirkung seines Eigengewichtes an?
\end{aufgabe}

\subsection{Schiess-Verfahren\label{numerik:schiess-verfahren}}
Wenn man experimentell versucht, ein Ziel zu treffen, dann wird man
in wiederholten Versuchen die Richtung anpassen, so dass man dem Ziel
immer n"aher kommt.
Der $y$-Wert zur Zeit $p/v_x$ h"angt von der Vertikalgeschwindigkeit ab,
wir bezeichnen ihn mit $h(v_y)$.
Man ver"andert also $v_y$, bis die Gleichung $h(v_y)=0$ erf"ullt ist.
Um das Randwertproblem zu l"osen, muss man also die Gleichung $h(v_y)=0$
numerisch l"osen.

Man kann dies zum Beispiel dadurch machen, dass man nach zwei Werten von $v_y$
sucht, so dass die zum einen geh"orige Bahn unter dem Punkt $P$ durchgeht,
w"ahrend der Ball im anderen Fall dar"uber hinwegfliegt.
Durch wiederholte Halbierung des Intervalls kann man dann den korrekten
Wert f"ur $v_y$ immer genauer eingrenzen%
\footnote{%
Tats"achlich wird dieses Verfahren in der Artillerie verwendet.
Der Schiesskommandant beobachtet die einschlagenden Granaten und kommandiert
"Anderungen der Anfangs-Elevation an die Gesch"utzbatterien.
Dabei sucht er Einschl"age, die aus seiner Perspektive vor bzw.~hinter
dem Ziel liegen, und halbiert dann das Intervall, bis die Einschl"age dem
Ziel gen"ugend nahe kommen.}.
Der Nachteil dieses Verfahrens ist, dass mit jedem Schritt die Genauigkeit
nur um in Bit ansteigt, es sind also sehr viele Iterationen notwendig.

Schnellere Konvergenz kann mit dem Newton-Verfahren erreicht werden,
welches in Anhang~\ref{chapter:newton} beschrieben wird.
F"ur die Anwendung des Newton-Verfahrens auf das Randwert-Problem
ist die Bestimmung der Steigung der Funktion n"otig, die die Abweichung
der Kurve von der Randbedingung am rechten Rand angibt.
Wir m"ussen also berechnen, wie schnell sich $y(p/v_x)$ "andert,
wenn $v_y$ ver"andert wird.
Dies ist die Ableitung
\[
h'(v_y)= \frac{\partial y}{\partial v_y},
\]
ein Eintrag der Jacobi-Matrix.
In Abschnitt~\ref{grundlagen:XXX} wurde gezeigt, wie man auch f"ur
die Jacobi-Matrix eine Differentialgleichung aufstellen kann, die
man nat"urlich ebenfalls mit den fr"uher beschriebenen numerischen
Bibliotheken l"osen kann.

Das Randwertproblem kann daher mit folgendem Algorithmus numerisch gel"ost
werden.
\begin{enumerate}
\item Beginne mit einer Sch"atzung f"ur $v_y$
\item Finde numerisch die L"osung des Anfangswertproblems mit $v_y$
als anf"angliche Vertikalgeschwindigkeit.
Berechnet dabei auch die Jacobi-Matrix
\item Lese die $h(v_y)$ aus der L"osung zur Zeit $p/v_x$ ab, und $h'(v_y)$
aus der Jacobi-Matrix und verwendet den Newton-Algorithmus
(\ref{numerik:ball-newton}), um eine verbesserte Sch"atzung von $v_y$ 
zu bekommen.
\item Wiederhole Schritte 2 und 3 bis die Randbedingung f"ur $t=p/v_x$
gen"ugend genau erf"ullt ist.
\item Die L"osung des Anfangswertproblems mit diesem $v_y$ ist die
L"osung des gestellten Randwertproblems.
\end{enumerate}

\begin{beispiel}
\begin{figure}
\centering
\includegraphics{chapters/images/randwert-1.pdf}
\caption{L"osungen des Anfangswertproblems~(\ref{numerik:ball-dgl-1}) und
(\ref{numerik:ball-dgl-2}).
Das Newton-Verfahren korrigiert $v_y$ derart, dass $h(v_y)=0$ wird.
So wird die L"osung des Randwertproblems (rot) gefunden.
\label{numerik:randwert-bild}}
\end{figure}
Wir f"uhren den eben skizzierten Algorithmus f"ur das Ball-Problem durch.
Um die Jacobi-Matrix zu berechnen, m"ussen wir die Ableitung von $f$ berechnen:
\begin{equation}
\frac{\partial f(x,y)}{\partial y}
=
\begin{pmatrix}
0& 0& 1& 0\\
0& 0& 0& 1\\
0& 0& 0& 0\\
0& 0& 0& 0
\end{pmatrix}.
\end{equation}
Da die rechte Seite nicht von $y$ abh"angt, k"onnen wir die Gleichung f"ur
die Jacobi-Matrix ganz unabh"angig von $y$ l"osen.
Da $F$ so einfach ist, kann man das Matrizenprodukt direkt ausrechnen, 
so wird die Differentialgleichung f"ur $J$
\begin{equation}
\begin{pmatrix}
J'_{11}&J'_{12}&J'_{13}&J'_{14}\\
J'_{21}&J'_{22}&J'_{23}&J'_{24}\\
J'_{31}&J'_{32}&J'_{33}&J'_{34}\\
J'_{41}&J'_{42}&J'_{43}&J'_{44}
\end{pmatrix}
=
\begin{pmatrix}
0& 0& 1& 0\\
0& 0& 0& 1\\
0& 0& 0& 0\\
0& 0& 0& 0
\end{pmatrix}
\begin{pmatrix}
J_{11}&J_{12}&J_{13}&J_{14}\\
J_{21}&J_{22}&J_{23}&J_{24}\\
J_{31}&J_{32}&J_{33}&J_{34}\\
J_{41}&J_{42}&J_{43}&J_{44}
\end{pmatrix}
=
\begin{pmatrix}
J_{31}&J_{32}&J_{33}&J_{34}\\
J_{41}&J_{42}&J_{43}&J_{44}\\
     0&     0&     0&     0\\
     0&     0&     0&     0
\end{pmatrix}
\end{equation}
\begin{table}
\centering
\begin{tabular}{|>{$}r<{$}|>{$}r<{$}|>{$}r<{$}|>{$}r<{$}|>{$}r<{$}|>{$}r<{$}|>{$}r<{$}|>{$}r<{$}|}
\hline
n&    v_y&    t& x(t)&      y(x)&\displaystyle\frac{\partial^{\mathstrut}y}{\partial v_y}&v_{y,\text{new}}&\Delta\\
\hline
0& 7.0000&  2.5& 20.0&-13.156250&  2.5& 12.26250000& -5.2625000000\\
1&12.2625&  2.5& 20.0& -0.000004&  2.5& 12.26250145& -0.0000014458\\
2&12.2625&  2.5& 20.0&  0.000000&  2.5& 12.26250143&  0.0000000204\\
\hline
\end{tabular}
\caption{Newton-Algorithmus f"ur das Ball-Problem, Resultate der numerischen
Rechnung.
$v_y$ wird in drei Schritten mit einer Genauigkeit von mehr als 10 Stellen
gefunden.
\label{numerik:newton-resultate}}
\end{table}%
Daraus kann man ablesen, dass die Elemente $J_{3j}$ und $J_{4j}$ sich
nicht "andern, sie bleiben also konstant.
Aber auch in den ersten zwei Zeilen k"onnen sich nur die Elemente $J_{13}$
und $J_{24}$ "andern, die Differentialgleichungen f"ur diese Elemente
sind
\begin{align*}
J'_{13}&=1\\
J'_{24}&=1
\end{align*}
oder in Matrixform:
\begin{equation}
J(x) = \begin{pmatrix}
1&0&x&0\\
0&1&0&x\\
0&0&1&0\\
0&0&0&1
\end{pmatrix}
\end{equation}
Die L"osung $J_{13}(x)=x$ und $J_{24}=x$.
Damit haben wir die n"otige Information, um den Newton-Algorithmus
durchzuf"uhren.
In Tabelle~\ref{numerik:newton-resultate}
sind die Resultate der numerischen Rechnung zusammengestellt.
Es zeigt sich, dass der korrekte Wert f"ur $v_y$ in drei Iterationen
mit 10 Stellen Genauigkeit gefunden werden kann.
Damit ist das Randwertproblem numerisch gel"ost.
\end{beispiel}

%
% potenzreihen.tex -- Lösung von Differentialgleichungen mit Potenzreihen
%
% (c) 2015 Prof Dr Andreas Mueller, Hochschule Rapperswil
%
\chapter{Potenzreihen-Methode\label{chapter:potenzreihen}}
\lhead{}
\rhead{Potenzreihen-Methode}
\section{Analytische L"osungen}
\section{Trigonometrische Funktionen}

%
% linear.tex -- L"osung linearer Differentialgleichungen
%
% (c) 2015 Prof Dr Andreas Mueller, Hochschule Rapperswil
%
\chapter{Lineare Differentialgleichungen\label{chapter:linear}}
\lhead{}
\rhead{Lineare Differentialgleichungen}
Es lohnt sich, lineare Differentialgleichungen unter Zuhilfenahme
der linearen Algebra etwas genauer zu untersuchen.
\section{Definition}
\section{L"osungsmenge}
\section{Normalformen}
\subsection{Diagonalisierung}
\subsection{Jordan-Normalform}


%
% geometry.tex -- L"osung linearer Differentialgleichungen
%
% (c) 2016 Prof Dr Andreas Mueller, Hochschule Rapperswil
%
\chapter{Geometrische Eigenschaften\label{chapter:geometrie}}
\rhead{}
\lhead{Geometrische Eigenschaften}
Die Geometrie schr"ankt die m"oglichen Bahnen eines
Differentialgleichungssystems bereits wesentlich ein.
In einem eindimensionalen autonomen System sind keine
Schwingungen m"oglich, in einem zweidimensionalen
System gibt es keine chaotischen Bewegungen.
Ziel dieses Kapitels ist, in diese geometrische
Denkweise einzuf"uhren.
Das Buch \cite{skript:hirsch} f"uhrt diesen Ansatz weiter bis zu
einer Einf"uhrung in chaotische Bewegung.

\section{Autonome Systeme}
\rhead{Autonome Systeme}
\begin{figure}
\centering
\includegraphics{chapters/images/geometrie-13.pdf}
\caption{Entwicklung des Systems~(\ref{geometrie:harvest-equation})
mit $a=5$ und $h=0.8$
\label{geometrie:harvest-graph}}
\end{figure}%
Ein eindimensionales System k"onnen wir schreiben als
\[
y'=f(x,y).
\]
Die L"osungskurven dieses Systems k"onnen wir in einem $x$-$y$-Diagramm
als Graphen darstellen.
Zum Beispiel k"onnen wir die Entwicklung des Systems
\begin{equation}
y' = ay(1-y)-h(1+\sin 2\pi x)
\label{geometrie:harvest-equation}
\end{equation}
wie in Abbildung~\ref{geometrie:harvest-graph} darstellen.
Dieses nicht-autonome System hat zwei Grenzzyklen: 
Anfangswerte oberhalb der roten Kurve f"uhren zu L"osungskurven, die
sich f"ur zunehmendes $x$ der blauen Kurve ann"ahern.
Anfangswerte unterhalb der blauen Kurve f"uhren zu L"osungskurven, die
sich f"ur abnehmendes $x$ der roten Kurve n"ahern.

Startwerte in der N"ahe der roten Kurve sind nicht stabil,
die Entwicklung f"ur zunehmende $x$ f"uhrt den Wert immer weiter von
der roten Kurve weg.
Lag der Startwert unterhalb der roten Kurve, wird die L"osung gegen
$-\infty$ divergieren.
Alle anderen L"osungen konvergieren gegen die blaue Kurve.

Offenbar ist es ziemlich schwierig, das
System~(\ref{geometrie:harvest-equation}) "uber gr"ossere $x$-Intervalle
zu verstehen. 
Zum Beispiel h"angt das Verhalten davon ab, ob der Startwert zwischen den
farbigen Kurven liegt oder ausserhalb, und diese h"angt vom Start-$x$ ab.

In Kapitel~\ref{chapter:grundlagen} wurde beschrieben, wie jedes
Differentialgleichungssystem, m"oglicherweise nach Erweiterung um eine
explizite Zeitkoordinate, zu einem autonomen System gemacht und damit
durch ein Vektorfeld ersetzt werden kann,
und wie die L"osungen als Bahnen eines Teilchens in einem zeitlich
unver"anderlichen Vektorfelde verstanden werden k"onnen.
Daraus ergibt sich auch, dass das Verhalten der L"osungen einer
Differentialgleichungen studiert werden kann, ohne dass man
die Abh"angigkeit von $x$ im Detail kennt.
Die Bahnen zerlegen den Raum und k"onnen sich nicht schneiden,
so schr"ankt die Geometrie des Raumes die M"oglichkeiten f"ur das
Verhalten der L"osung "uber lange Zeiten ein.
Besonders ausgepr"agt sind diese Einschr"ankungen im zweidimensionalen
Raum.
Eine geschlossene Kurve in der Ebene unterteilt diese in zwei Bereiche,
keine L"osung kann vom einen Bereich in den anderen f"uhren.

Ein Parameter in der Differentialgleichung modifiziert das Vektorfeld,
und kann damit Eigenschaften von L"osungen "uber lange Zeiten ver"andern.
Zum Beispiel k"onnen periodische Bahnen sich aufl"osen und zu Spiralbahnen
werden.

Wir gehen in diesem Kapitel immer von einem autonomen System
\[
\frac{d}{dx}y(x)=f(y),
\]
wobei das Vektorfeld $f(y)$ nicht von $x$ abh"angt.
Uns interessieren die geometrische Eigenschaften der L"osungskurven, 
soweit sie sich direkt aus den Eigenschaften des Vektorfeldes ableiten
lassen.
Insbesondere interessieren uns also spezielle Punkte des Vektorfeldes,
zum Beispiel Nullstellen.


%
% Spezielle Punkte und Bahnen
%
\section{Spezielle Punkte und Bahnen}
\rhead{Spezielle Punkte und Bahnen}
Das Verhalten der L"osungskurven "uber lange Zeiten wird wesentlich
beeinflusst von Punkten, in denen die Bewegung auf der Bahn
zum Stillstand kommt oder sogar die Richtung umkehrt, und von Punkten,
in die ein Punkt periodisch zur"uckkehrt.

%
% Kritische Punkte
%
\subsection{Kritische Punkte}
Wir betrachten zun"achst einen Punkt $y_0$ in dem $f(y_0)\ne 0$.
Dann gilt $f(y)\ne 0$ auch f"ur Punkte $y$, die gen"ugend nahe an 
$y_0$ sind.
Eine L"osungskurve durch $y_0$ hat im Punkt $y_0$ die Tangentenrichtung
$f(y_0)$.
In einer gen"ugend kleinen Umgebung von $y_0$ werden sich verschiedene
L"osungskurven nicht schneiden.
Ein solcher Schnittpunkt w"are ein Anfangsbedingung f"ur zwei
verschiedene L"osungen, aber der Eindeutigkeitssatz f"ur die
L"osung einer Differentialgleichung sagt, dass es zu jeder 
Anfangsbedingung nur eine L"osung geben kann.
Dies bedeutet, dass in einer Umgebung des Punktes $y_0$ nichts
Spannendes passiert, die Bahnen sehen ungef"ahr aus wie in
Abbildung~\ref{geometrie:parallelebahnen}.
\begin{figure}
\centering
\includegraphics{chapters/images/geometrie-12.pdf}
\caption{Verlauf der Bahnen eines autonomen Differentialgleichungssystems
in der N"ahe eines Punktes $y_0$ mit $f(y_0)\ne 0$.
\label{geometrie:parallelebahnen}}
\end{figure}

Von besonderem Interesse sind daher Punkte, in denen das Vektorfeld
verschwindet:

\begin{definition}
$y$ heisst {\em kritischer Punkt} des Vektorfeldes $f$, wenn $f(y)=0$.
\end{definition}

Dass in einem kritischen Punkt das Vektorfeld verschwindet bedeutet nicht,
dass die Bahn nicht dar"uber hinweg fortgesetzt werden kann,
wie das folgende Beispiel zeigt.

\begin{beispiel}
Das eindimensionale Gleichungssystem
\[
y'=\sqrt{|y|}
\]
hat einen kritischen Punkt bei $y=0$.
Die Funktion
\begin{equation}
y(x)=\frac14x^2\operatorname{sign}(x)
\label{geometrie:1dimkritloesung}
\end{equation}
ist eine L"osungsfunktion, denn
\begin{align*}
y'(x)&=\frac12|x|\\
\sqrt{\left|\frac14x^2\operatorname{sign}(x)\right|}
&=
\sqrt{\frac14x^2}
=
\frac12|x|=y'(x).
\end{align*}
Trotzdem erreicht $y(x)$ jeden beliebigen Punkt der $y$-Achse, denn 
\[
x=2\sqrt{|y|}\operatorname{sign}(y)
\]
ist die Umkehrfunktion von $y(x)$. 
Die L"osungskurve bewegt sich also "uber den kritischen Punkt hinweg.
\begin{figure}
\centering
\includegraphics{chapters/images/geometrie-1.pdf}
\caption{Bahnen das Differentialgleichungssystems $y'=\sqrt{|y|}$
k"onnen den kritischen Punkt bei $y=0$ traversieren.
Die L"osung~(\ref{geometrie:1dimkritloesung}) ist etwas fetter rot
eingezeichnet.
L"osungen k"onnen aber auch beliebig lange im Punkt $y$ verweilen,
wie die blaue Kurve illustriert.
\label{geometrie:1dimkrit}}
\end{figure}
Abbildung~\ref{geometrie:1dimkrit} zeigt die Bahnen.
Die L"osung~(\ref{geometrie:1dimkritloesung}) ist nicht die einzige.
Andere L"osungen bestehen jeweils aus einem Ast der Kurve mit positiven
$y$ und einem mit negativen $y$, dazwischen kann die L"osung beliebig lange
im kritischen Punkt verweilen.
F"ur den Punkt $y=0$ ist also der Eindeutigkeitssatz verletzt, zur
Anfangsbedingung $y=0$ gibt es beliebig viele L"osungen.
\end{beispiel}

%
% Linearisierung
%
\subsection{Linearisierung}
In einem kritischen Punkt $y_0$ verschwinden alle Komponenten von $f(y_0)$.
Falls $f$ stetig differenzierbar ist, k"onnen wird $f$ in einer Umgebung
des kritischen Punktes in erster Ordnung mit Hilfe der Ableitung
approximieren:
\[
f(y)=\frac{\partial f}{\partial y}(y-y_0) + o(|y-y_0|)
\]
Die partielle Ableitung bezeichnet im mehrdimensionalen Fall
die Jacobi-Matrix, sie ist die $n\times n$-Matrix bestehend aus allen
partiellen Ableitungen der Komponenten von $f$ nach den Variablen $y_i$.
In einer Umgebung eines kritischen Punktes ist die Gestalt der Bahnen also
im wesentlichen durch die Jacobi-Matrix bestimmt.

\begin{beispiel}
Wir betrachten als Beispiel die eindimensionale ($n=1$) Differentialgleichung
\[
y'=f(y)=y,
\]
die in $y_0=0$ einen kritischen Punkt hat.
Die Jacobi-Matrix ist konstant: $f'(y)=1$.
L"osungskurven mit Anfangsbedingungen $>0$ werden daher anwachsen,
w"ahrend L"osungskurven mit Anfangsbedingungen $<0$ abnehmen werden.
Die L"osungskurven werden daher vom kritischen Punkt ``abgestossen''.

W"ahlen wir stattdessen das System
\[
y'=f(y)=-y,
\]
dann ist die Jacobi-Matrix $f'(y)=-1$, und wir k"onnen analog schliessen,
dass L"osungskurven immer zum kritischen Punkt hinstreben.
\end{beispiel}

\begin{beispiel}
Das Beispiel im letzten Abschnitt verwendet
\[
y'=f(y)=\sqrt{|y|}
\]
ist im kritischen Punkt $y=0$ nicht stetig differenzierbar, daher ist
eine lineare Approximation nicht m"oglich.
Das fr"uher beobachtete Verhalten, dass sich L"osungskurven auf der
einen Seite vom kritischen Punkt entfernen, auf der anderen aber ann"ahern,
kann nur auftreten, wenn auch $f'(0)=0$ gilt, so dass wir $f$ in zweiter
Ordnung als
\[
y'=f(y)=\frac12 f''(0)y^2.
\]
Diese Differentialgleichung hat die Funktion
\[
y(x)=-\frac{2}{f''(0)x+C}
\]
als L"osung.
Je nach Wert von $C$ bekommen wir eine L"osung, f"ur $x>0$ anwachsen
oder abfallen, wenigstens f"ur ein kleines Intervall.
Die L"osung n"ahert sich dem kritischen Punkt, oder entfernt sich, je
nachdem auf welcher Seite die L"osungskurve beginnt.
\end{beispiel}

%
% Eindimensionale Systeme
%
\section{Eindimensionale Systeme}
\rhead{Eindimensionale Systeme}
Wir untersuchen in diesem Abschnitt das Verhalten eindimensionaler autonomer
Systeme, und konzentrieren uns dabei auf Eigenschaften, die allein
auf Grund der geringen Dimension erschliessen lassen.
Ein solches System hat die Form
\[
y'=f(y),\quad y\in\mathbb R
\]
und kann sofort mit Hilfe von Separation der Variablen gel"ost werden:
\[
\int\frac{dy}{f(y)} = t+C,
\]
was nat"urlich nur funktioniert, wenn $f$ konstantes Vorzeichen hat.
Daraus folgt, dass ein Startwert $y_0$, f"ur den $f(y_0)>0$ ist,
zu einer monoton wachsenden L"osung f"uhrt, w"ahrend $f(y_0)<0$
bedeutet, dass die L"osung monoton f"allt.
Erf"ullt ein Startwert $f(y_0)=0$, dann ist $y(x)=y_0$ eine
L"osung der Differentialgleichung.
Die kritischen Punkte von $f$ strukturieren also wie erwartet die
L"osungsmenge.
Wir suchen eine "ubersichtliche Visualisierung dieser Beobachtung
und wollen verstehen, wie sich die Struktur in Abh"angigkeit
von einem Parameter in der Differentialgleichung ver"andern kann.

%
% Mögliche Phasendiagramme in einer Dimension
%
\subsection{Phasenportraits}
Seien jetzt $y_1,y_2,\dots$ Nullstellen der Funktion $f(y)$.
Dann sind die Funktion $y(x)=y_i$ L"osungen der Differentialgleichung
$y'=f(y)$.
Die Bewegung zwischen den Nullstellen wird vollst"andig dominiert durch
das Vorzeichen von $f(y)$ zwischen den Nullstellen.
Wir k"onnen dies in einem sogenannten Phasen-Portrait visualisieren.
Abbildung~\ref{geometrie:phasenportrait} zeigt die L"osungen der
Differentialgleichung
\begin{equation}
y'=f(y)=y^3-y
\label{geometrie:cube}
\end{equation}
Die Nullstellen dieser Funktion sind $-1$, $0$ und $1$.
\begin{figure}
\centering
\includegraphics{chapters/images/geometrie-14.pdf}
\caption{Fluss der Differentialgleichung~(\ref{geometrie:cube}) und
Phasenportrait
\label{geometrie:phasenportrait}}
\end{figure}
Die exakte Abh"angigkeit von $x$ ist oft nicht wichtig, entscheidend
ist nur die Tatsache, dass Punkte im blauen Gebiet sich mit zunehmendem
$x$ nach unten bewegen, w"ahrend Punkte im roten Gebiet sich nach
oben bewegen.
Diese Information wird auch durch das Phasenportrait am linken
Rand der Abbildung~\ref{geometrie:phasenportrait} dargestellt.
Der Vortail eines Phasenportraits gegen"uber der Darstellung
der L"osung ist, dass sie mit einer Dimension auskommt.

%
% Mögliche Änderungen (Bifurkationen) der Phasendiagramme
%
\subsection{Bifurkationen\label{geometrie:subsection:bifurkationen}}
Betrachten wir jetzt eine Differentialgleichung, die ausserdem von einem
Parameter $b$ abh"angt:
\[
y'=f(y,b).
\]
Das Phasenportrait wird sich ver"andern, wenn $b$ variert, wir
m"ochten die verschiedenen dabei m"oglichen "Uberg"ange verstehen
k"onnen.
Es interessiert vor allem kleine "Anderungen in der Umgebung eines
Wertes $b_0$.
Dabei kommt es vor allem darauf an, dass mit Nullstellen passiert, und
welches Vorzeichen $f$ links und rechts von der Nullstelle hat.
Wir gehen daher davon aus, dass $f$ f"ur $b_0$ an der Stelle $y=0$
hat, dass also $a_0(b_0)=0$ ist.

Wenn $a_1(b_0)\ne0$ ist, also wenn $f'(0, b_0)\ne 0$ ist, dann 
gilt dies auch in einer Umgebung von $b_0$, und daher wird die
Nullstelle in erster N"aherung nur verschoben:
\[
y_0(b) \simeq -\frac{f(0,b)}{f'(0,b)},
\]
Das Phasenportrait erlebt also keine grundds"atzlichen "Anderung.

\subsubsection{Sattel-Knoten-Bifurkation}
Wir betrachten jetzt den Fall $f(0,b_0)=0$ und $f'(0,b_0)=0$, die Funktion
$f$ ist f"ur $b=b_0$ quadratisch, mit einer Nullstelle im Scheitel.
Variert $b$, wird zwar der Graph von $y\mapsto f(y,b)$ seine
Gestalt "andern, aber er wird weiterhin "ahnlich wie eine Parabel
aussehen.
Etwas interessantes passiert nur, wenn der Scheitel von die horizontale
Achse "uberquert, wenn also beim Durchgang von $b$ durch den Wert $b_0$
die Zahl der Nullstellen von $0$ auf $2$ steigt oder umgekehrt.
Dies bedeutet, dass 
\[
f''(y_0,b_0)\ne 0
\qquad
\text{und}
\qquad
\frac{\partial f}{\partial b}(y_0, b_0)\ne 0.
\]
Modellhaft k"onnen wir dies durch das System
\[
f(y,b)=y^2+b
\]
wiedergeben.
Es hat f"ur $b<0$ die Nullstellen $\pm\sqrt{-b}$, f"ur $b>0$ jedoch
keine Nullstellen.
Das zugeh"orige Phasenportrait ist in Abbildung~\ref{geometrie:saddle-node}
dargestellt.
Bei dieser Art von Bifurkation entstehen neue kritische Punkte
paarweise, davon ist jeweils einer stabil, der andere instabil.
Diese Art der Bifurkation heisst {\em Sattel-Knoten-}
oder {\em Saddle-Node-Bifurkation}.
\index{Sattel-Knoten-Bifurkation}
\index{Saddle-Node-Bifurkation}

\begin{figure}
\centering
\includegraphics{chapters/images/bifurkation-1.pdf}
\caption{Phasendiagramm f"ur die Sattel-Knoten-Bifurkation in
Abh"angigkeit vom Parameter $b$.
\label{geometrie:saddle-node}}
\end{figure}

\subsubsection{Heugabel-Bifurkation}
\begin{figure}
\centering
\includegraphics{chapters/images/bifurkation-2.pdf}
\caption{Phasendiagramm f"ur die Heugabel-Bifurkation in Abh"angigkeit vom
Parameter $b$.
\label{geometrie:pitchfork}}
\end{figure}
Eine andere Art von Bifurkatione zeigt das Modell
\[
f(y,b)=y^3-by.
\]
F"ur $b<0$ hat die Funktion $y\mapsto f(y,b)$ nur die eine
reelle Nullstelle $y=0$.
F"ur $b>0$ dagegen liegen die drei verschiedenen Nullstellen
$-\sqrt{b}$, $0$  und $\sqrt{b}$.
Das zugeh"orige Phasendiagramm ist in Abbildung~\ref{geometrie:pitchfork}
dargestellt, diese Art von Bifurkation heisst {\em Heugabel-}
oder {\em Pitchformk-Bifurkation}.
\index{Heugabel-Bifurkation}
\index{Pitchfork-Bifurkation}
Bei dieser Art von Bifurkation werden aus einer Nullstelle deren drei.
War die Nullstelle instabil (wie in Abbildung~\ref{geometrie:pitchfork}),
entstehen zwei neue instabile Nullstellen, die bisherige Nullstellen 
wird stabil.
War die eine Nullstelle stabil, entstehen zwei stabile Nullstellen, die
eine Nullstelle wird instabil.

\subsubsection{Transkritische Bifurkation}
\begin{figure}
\centering
\includegraphics[width=\hsize]{chapters/images/bifurkation-3.pdf}
\caption{Phasendigramm f"ur die transkritische Bifurkation in Abh"angigkeit
vom Parameter $b$.
\label{geometrie:transkritisch}}
\end{figure}
Das Standard-Modell f"ur die sogenannten {\em transkritische Bifurkation} ist 
\index{transkritische Bifurkation}
\[
f(y,b)=y^2+yb = y(y+b)
\]
(Abbildung~\ref{geometrie:transkritisch}).
F"ur $b=0$ hat dieses System einen kritischen Punkt bei $y=0$.
Eine L"osungskurve, die bei negativem $y$ beginnt, wird immer n"aher
an den Punkt $y=0$ herankommen.
Liegt der Anfangswert jedoch im Gebiet $y>0$, wird sich die L"osungskurve
immer weiter von $y=0$ entfernen.
Der Fixpunkt bei $y=0$ ist also weder stabil noch instabil.

F"ur $b\ne 0$ entsteht ein weiterer kritischer Punkt bei $y=-b$.
Da f"ur $y\to\pm\infty$ die Funktion $y\mapsto f(y,b)$ immer positiv
ist, ist der rechte der beiden Fixpunkte immer stabil, der
linke immer instabil.


%
% Zweidimensionale Systeme
%
\section{Zweidimensionale Systeme}
\rhead{Zweidimensionale Systeme}
In diesem Abschnitt betrachten wir die geometrischen Einschr"ankungen, denen
zweidimensionale autonome Systeme unterliegen.
Wieder sind Punkte von besonderem Interesse, in denen $f$ verschwindet.
Bei eindimensionalen Systemen teilen solche Punkte den $y$-Bereich in
verschiedene Intervall, in denen die Bewegung nur in jeweils
eine Richtung erfolgen kann, damit ist es leicht zu entscheiden,
ob ein solcher kritischer Punkt stabil oder instabil ist.
In zwei Dimensionen legen die Punkte alleine den Charakter noch nicht
fest, insbesondere ist es ja auch m"oglich, dass eine Bahn einen solche
Punkt umschliesst.

\subsection{Nullklinen}
Kritische Punkte von $f$ sind solche, in denen beide Komponenten
des Vektors $f(y)$ verschwinden.
Die Gleichungen
\[
f_1(y)=0
\qquad\text{und}\qquad
f_2(y)=0
\]
beschreiben zwei Kurvenscharen in der Ebene, die sogenannten 
{\em Nullklinen}.
\index{Nullkline}
Die Schnittpunkte von Nullklinen sind die kritischen Punkte.

Da auf einer Nullkline jeweils eine der Komponenten von $f(y)$ verschwindet,
schneiden L"osungskurven Nullklinen immer entweder horizontal oder
vertikal.
Ausserdem trennen die Nullklinen Gebiete unterschiedlicher Bewegungsrichtung
entlang einer Achse. 
Die Nullkline $f_1(y)=0$ trennt Gebiete, in denen sich eine L"osungskurve
nach rechts (zu gr"osseren $y_1$ hin) bewegt ($f_1(y)>0$) von Gebieten,
in denen sich die L"osungskurve nach links ($f_1(y)<0$) bewegt.
Desgleichen trennt die Nullkline $f_2(y)=0$ Gebiete mit Bewegung ``nach oben''
von Gebieten mit Bewegung ``nach unten''.
Allein aus dieser Information kann man sich bereits ein recht gutes
qualitives Bild "uber den Verlauf der L"osungen verschaffen, wie wir
im Folgenden an zwei Beispiel illustrieren wollen.

\begin{beispiel}
\begin{figure}
\centering
\includegraphics{chapters/images/nullklinen-1.pdf}
\caption{Nullklinen der Differentialgleichung~(\ref{geometrie:nullklinen-dgl1}),
die $y_1$-Nullklinen in rot, die $y_2$-Nullklinen in blau.
Kritische Punkte sind die Schnittpunkte verschiedenfarbiger Nullklinen.
\label{geometrie:nullklinen1}}
\end{figure}
\begin{figure}
\centering
\includegraphics{chapters/images/nullklinen-2.pdf}
\caption{Vektorfeld der Differentialgleichung~(\ref{geometrie:nullklinen-dgl1}),
es best"atigt die Resultate der qualitiativen Diskussion aus
Abbildung~\ref{geometrie:nullklinen1}.
\label{geometrie:nullklinen-fluss}}
\end{figure}
\begin{figure}
\centering
\includegraphics{chapters/images/nullklinen-3.pdf}
\caption{Vektorfeld der Differentialgleichung~(\ref{geometrie:nullklinen-dgl1})
in einer Umgebung des instabilen kritischen Punktes $(\frac12,\frac12)$.
\label{geometrie:nullklinen-instabil}}
\end{figure}
\begin{figure}
\centering
\includegraphics{chapters/images/nullklinen-4.pdf}
\caption{Vektorfeld der Differentialgleichung~(\ref{geometrie:nullklinen-dgl1})
in einer Umgebung des stabilen kritischen Punktes $(2,0)$.
\label{geometrie:nullklinen-stabil}}
\end{figure}
Wir betrachten das nichtlineare System 
\begin{equation}
\frac{d}{dx} \begin{pmatrix}y_1\\y_2\end{pmatrix}
=
\begin{pmatrix}
2y_1\biggl(1-\displaystyle\frac{y_1}2\biggr)-3y_1y_2\\
y_2(1-y_2)-y_1y_2
\end{pmatrix}.
\label{geometrie:nullklinen-dgl1}
\end{equation}
Die direkte L"osung ist ziemlich aussichtslos, wir versuchen daher mit
Hilfe der Nullklinen ein qualitatives Bild
(Abbildung~\ref{geometrie:nullklinen1}) zu erhalten.

Die $y_2$-Nullkline hat die Gleichung
\[
0=y_2(1-y_2)-y_1y_2=y_2(1-y_2-y_1)
\qquad\Rightarrow\qquad
y_2=0
\quad\text{oder}\quad
y_1+y_2=1.
\]
L"osungskurven schneiden also die beiden Geraden $y_2=0$ (die $y_1$-Achse)
und $y_1+y_2=1$ horizontal.
Da die $y_1$-Achse bereits horizontal ist, bedeutet dies, dass sich die
L"osungskurven der $y_1$-Achse anschmiegen.

Die $y_1$-Nullkline hat die Gleichung
\[
0=2y_1\biggl(1-\frac{y_1}2\biggr)-3y_1y_2=y_1(2-y_1-3y_2),
\qquad\Rightarrow\qquad
y_1=0
\quad\text{oder}\quad
y_1+3y_2=2.
\]
Wir schliessen wieder, dass die L"osungskurven die $y_2$-Achse und
die Gerade $y_1+3y_2=3$ vertikal schneiden.
Da die $y_2$-Achse schon vertikal ist, m"ussen sich auch dort
die L"osungskurven anschmiegen.

Wir k"onnen aus den Nullklinen auch die kritischen Punkte ableiten.
Da sind zun"achst die Punkte auf den Achsen, also zum Beispiel
der Schnittpunkt der $y_2$-Nullkline $y_2=0$ mit der $y_1$-Nullkline
$y_1+3y_2=2$, also $(2,0)$, oder der Schnittpunkt der
$y_1$-Nullkline $y_1=0$ mit der $y_2$-Nullkline $y_1+y_2=0$, also $(0,1)$.
Ausserdem ist nat"urlich $(0,0)$ ein kritischer Punkt.
ein vierter kritischer Punkt entsteht als Schnittpunkt
der $y_1$-Nullkline $y_1+3y_2=2$ mit der $y_2$-Nullkline $y_1+y_2=1$,
also als L"osung des linearen Gleichungssystems
\[
\begin{linsys}{2}
y_1&+&3y_2&=&2\phantom{.}\\
y_1&+& y_2&=&1,
\end{linsys}
\]
es hat die L"osung $(\frac12,\frac12)$.
Die kritischen Punkte sind also
\begin{equation}
(0,0),\quad
(2,0),\quad
(0,1)\quad\text{und}\quad
\biggl(\frac12,\frac12\biggr).
\label{geometrie:nullklinen-krit}
\end{equation}

An den aus den Nullklinen l"asst sich auch die Bewegungsrichtung der
L"osungen im Bezug auf die kritischen Punkte ablesen.
\label{geometrie:nullklinen-stabilitaet}
Aus dem Gebiet oben rechts und unten links in der N"ahe des Ursprungs
bewegen sich die L"osungen zun"achst auf den kritischen Punkt
bei $(\frac12,\frac12)$ zu, weichen dann aber ab in Richtung auf die
kritischen Punkte $(0,1)$ und $(2,0)$ zu.
Die kritischen Punkte $(0,0)$ und $(\frac12,\frac12)$ sind also
instabil, w"ahrend $(2,0)$ und $(0,1)$ stabil sind.
Dies wird auch von dem genaueren Vektorfeld in
Abbildung~\ref{geometrie:nullklinen-fluss} best"atigt.
In Abschnitt~\ref{geometrie:umgebung-kritisch} wird gezeigt, dass man 
die Bewegung in der Umgebung eines kritischen Punktes mit Hilfe der
Eigenwerte der Ableitungen klassifizieren kann.
\end{beispiel}


\begin{beispiel}
Das {\em Fitzhugh-Nagumo-Modell} wird verwendet, um das Verhalten eines Neurons
zu simulieren.
\index{Fitzhugh-Nagumo-Modell}
\begin{figure}
\centering
\includegraphics{chapters/images/nullklinen-5.pdf}
\caption{Nullklinen des Fitzhugh-Nagumo-Modells bei nur einem kritischen Punkt,
$a=-0.8$, $b=0.9$.
\label{geometrie:nullklinen-fh-1}}
\end{figure}
\begin{figure}
\centering
\includegraphics{chapters/images/nullklinen-6.pdf}
\caption{Nullklinen des Fitzhugh-Nagumo-Modells mit drei kritischen Punkten,
$b=4$, $a=0$.
\label{geometrie:nullklinen-fh-2}}
\end{figure}
Es verwendet das Differentialgleichungssystem
\begin{equation}
\begin{aligned}
    \dot v&= v-\frac13v^3-w\\
\tau\dot w&= v-a-bw.
\end{aligned}
\label{geometrie:fitzhugh-dgl}
\end{equation}
Wir versuchen, uns wieder mit Hilfe der Nullklinen ein Bild von den
L"osungskurven zu verschaffen.
Die $v$-Nullkline hat die Gleichung
\[
0=v-\frac13v^3-w
\qquad\Rightarrow\qquad
w=v-\frac13v^3 = v(1-{\textstyle\frac1{\sqrt{3}}}v)(1+{\textstyle\frac1{\sqrt{3}}}v)
\]
Diese kubische Parabel hat im Nullpunkt die Steigung $1$.
Die $w$-Nullkline ist die Gerade
\[
0=v-a-bw
\qquad\Rightarrow\qquad
v=bw+a
\qquad\Rightarrow\qquad
w = \frac{v-a}{b}.
\]
Wenn die Steigung $1/b$ dieser Geraden gr"osser als $1$ ist, dann schneidet
die Gerade die kubische Parabel nur in einem Punkt, es gibt dann nur
einen kritischen Punkt.
Ist die Steigung $1/b<1$, gibt es f"ur nicht zu grosses $|a|$ drei
Schnittpunkte.

In Abbildung~\ref{geometrie:nullklinen-fh-1} sind die Nullklinen des
Fitzhugh-Nagumo-Modells mit nur einem kritischen Punkt dargestellt.
Die L"osungskurven bewegen sich in Spiralen im Gegenurzeigersinn
um den Fixpunkt herum.
In diesem Fall erlauben die Nullklinen keine abschliessende Beurteilung
der Stabilit"at des Fixpunktes.
Die Untersuchung der Eigenwerte der Jacobi-Matrix, die im nächsten
Abschnitt erkl"art wird, erm"oglicht die Entscheidung, und wird in
der Fortsetzung dieses Beispiels auf Seite~\pageref{geometrie:fh-fortsetzung}
durchgef"uhrt.

In Abbildung~\ref{geometrie:nullklinen-fh-2} sind die Nullklinen des
Fitzhugh-Nagumo-Modells dargstellt mit drei kritschen Punkten.
Man kann sofort ablesen, dass $(0,0)$ ein instabiler kritischer Punkt ist.
Die L"osungskurven, die von diesem Punkt abgestossen werden, n"ahern sich
einem der beiden anderen kritischen Punkte und bewegen sich
in einer Spirale um diesen Punkt herum.
Da sich die L"osungskurven nicht schneiden d"urfen kann man folgern,
dass die beiden anderen kritschen Punkte stabil sein m"ussen,
die L"osungskurven werden sich ihnen in Spiralbahnen n"ahern.
\label{geometrie:fh-diskussion}
\end{beispiel}

%
% Bewegung in der Umgebung eines kritischen Punktes
%
\subsection{Bewegung in der Umgebung eines kritischen Punktes
\label{geometrie:umgebung-kritisch}}
Wir gehen jetzt davon aus, dass 
\[
y'=f(y)
\]
einen kritischen Punkt hat, wir k"onnen die Koordinaten immer so w"ahlen,
dass der kritische Punkt im Nullpunkt des Koordinatensystems liegt.
Die Bewegung in unmittelbarer Umgebung des Nullpunktes kann dann approximiert
werden durch die Bewegung des linearisierten Systems
\[
y'=\frac{\partial f(0)}{\partial y}y
\]
Die m"oglichen Bewegungsformen in der Umgebung des kritischen Punktes
sind also bestimmt durch die Jacobi-Matrix.
Jede beliebige $2\times 2$-Matrix kann auch tats"achlich als Jacobi-Matrix
vorkommen, denn das System
$
y'=Ay
$
hat die Matrix $A$ als Jacobi-Matrix.

Wir interessieren uns im Moment nur f"ur eine qualitative Beschreibung
der L"osungen, wir k"onnen also immer eine Koordinatentransformation
vornehmen, um die Situation zu vereinfachen.
Die gesuchte qualitative Klassifizierung von zweidimensionalen
Differentialgleichungssystemen l"auft also auf eine Klassifizierung
von reellen $2\times 2$-Matrizen bis auf Koordinatentransformation
hinaus.

Eine solche Klassifikation kann auf der Basis von Eigenwerten und
Eigenvektoren erfolgen.
Dazu ben"otigen wir eine "Ubersicht "uber die Eigenwerte einer
Matrix
\[
A=\begin{pmatrix}a&b\\c&d\end{pmatrix},
\]
die wir als Nullstellen des charakteristischen Polynoms bestimmen
k"onnen.
Das charakteristische Polynom ist
\[
\chi_A(\lambda)
=
\det(A-\lambda E)
=
\left|\,\begin{matrix}a-\lambda&b\\c&d-\lambda\end{matrix}\,\right|
=
(a-\lambda)(d-\lambda)-bc
=
\lambda^2-(a+b)\lambda + ad-bc,
\]
es ist bestimmt durch die Spur und die Determinante der Matrix
\[
\begin{aligned}
\det A&=ad -bc,
&
\operatorname{Spur}A&=a+d
&
&\Rightarrow
&
\chi_A(\lambda)&=\lambda^2-\lambda \operatorname{Spur}A+\det A
\end{aligned}
\]
Die L"osungsformel f"ur die quadratische Gleichung liefert die
Eigenwerte
\[
\lambda_{1,2}
=
\frac{\operatorname{Spur}A}2\pm\sqrt{\Delta}
\qquad
\qquad
\text{mit}\quad
\Delta = \biggl(\frac{\operatorname{Spur}A}2\biggr)^2 - \det A
\]
Falls die Diskriminanten $\Delta > 0$ ist, sind die beiden Eigenwerte
verschieden, und folglich gibt es zwei verschiedene Eigenvektoren,
die Matrix $A$ kann diagonalisiert werden mit Diagonalelementen
$\lambda_{1,2}$.  Das Differentialgleichungssystem zerf"allt dann
in zwei unabh"angige eindimensionale Systeme
\begin{align*}
y_1'&= \lambda_1 y_1\\
y_2'&= \lambda_2 y_2,
\end{align*}
die auch durch
\begin{align*}
y_1(x)&=y_{10} e^{\lambda_1 x}\\
y_2(x)&=y_{20} e^{\lambda_2 x}
\end{align*}
sofort gel"ost werden k"onnen.
F"ur die Diskussion der Form der L"osungskurven brauchen wir aber die
Abh"angigkeit der beiden Koordinaten untereinander, nicht von $x$.
Wir stellen daher $y_2$ als Funktion von $y_1$ dar:
\[
y_1=y_{10} e^{\lambda_1 x}
\qquad\Rightarrow\qquad
x=\frac1{\lambda_1}\log\frac{y_1}{y_{10}}
\qquad\Rightarrow\qquad
y_2
=
y_{20} e^{\frac{\lambda_2}{\lambda_1}\log\frac{y_1}{y_{10}}}
=
y_{20}\biggl(\frac{y_1}{y_{10}}\biggr)^{\frac{\lambda_2}{\lambda_1}}
=
Cy^{\frac{\lambda_2}{\lambda_1}}
\]
\begin{figure}
\centering
\begin{tabular}{ccc}
\includegraphics{chapters/images/geometrie-2.pdf}&%
\includegraphics{chapters/images/geometrie-3.pdf}&%
\includegraphics{chapters/images/geometrie-4.pdf}\\
$\displaystyle \frac{\lambda_2}{\lambda_1}>1$&%
$\displaystyle \frac{\lambda_2}{\lambda_1}=1$&%
$\displaystyle \frac{\lambda_2}{\lambda_1}<1$
\end{tabular}
\caption{L"osungskurven des linearisierten Systems f"ur
$\frac{\lambda_2}{\lambda_1}>0$.
\label{geometrie:posportraits}}
\end{figure}

\begin{figure}
\centering
\begin{tabular}{ccc}
\includegraphics{chapters/images/geometrie-5.pdf}&%
\includegraphics{chapters/images/geometrie-6.pdf}&%
\includegraphics{chapters/images/geometrie-7.pdf}\\
$\displaystyle \frac{\lambda_2}{\lambda_1}>-1$&%
$\displaystyle \frac{\lambda_2}{\lambda_1}=-1$&%
$\displaystyle \frac{\lambda_2}{\lambda_1}<-1$
\end{tabular}
\caption{L"osungskurven des linearisierten Systems f"ur
$\frac{\lambda_2}{\lambda_1}<0$.
\label{geometrie:negportraits}}
\end{figure}
Die Gestalt der L"osungskurven sind also im Wesentlichen durch den
Quotienten $\lambda_2/\lambda_1$ bestimmt.
Die Abbildung~\ref{geometrie:posportraits} zeigt die L"osungskurven
f"ur positive Werte des Quotienten, w"ahrend
Abbildung~\ref{geometrie:negportraits} die L"osungskurven f"ur
negative Werte des Quotienten zeigt.
F"ur positive Werte von $\lambda_2/\lambda_1$ bewegen sich die 
Punkte entweder immer auf den kritischen Punkt zu, oder entfernen
sich.
F"ur negative Werte von $\lambda_2/\lambda_1$ n"ahert sich ein
Punkt in der N"ahe der einen Achse zun"achst immer mehr dem kritischen
Punkt, um sich dann in Richtung der anderen Achse zu entfernen.

Die Matrix $A$ muss jedoch nicht diagonalisierbar sein, wenn
$\lambda_1=\lambda_2$.
Wenn die Matrix nur einen Eigenvektor hat, dann kann die Matrix
durch eine geeignete Koordinatentransformation in die Form
\[
\begin{pmatrix}
\lambda&      1\\
      0&\lambda
\end{pmatrix}
\]
bringen.
Die Differentialgleichungen lauten in diese Fall
\begin{align*}
y_1'&=\lambda y_1 + y_2\\
y_2'&=\lambda y_2
\end{align*}
Die zweite Gleichung kann wie vorher gel"ost werden, die L"osung f"ur $y_2$
ist
\[
y_2(x)=y_{20}e^{\lambda x},
\]
dies k"onnen wir in die erste Gleichung einsetzen, sie lautet jetzt
\[
y_1' = \lambda y_1 + y_{20}e^{\lambda x},
\]
dies ist wieder eine lineare Differentialgleichung, diesmal jedoch
eine inhomogene. 
Die L"osung der homogenen Gleichung ist $Ce^{\lambda x}$, die L"osung
der inhomogenen Gleichung kann durch Variation der Konstanten gefunden
werden, also $y_1(x)=C(x)e^{\lambda x}$.
Setzen wir dies in die Differentialgleichung 
\[
y_1'(x)
=
C'(x)e^{\lambda x}+C(x)\lambda e^{\lambda x}
=
\lambda y_1(x) + C'(x)e^{\lambda x}
=
\lambda y_1(x) + y_{20}e^{\lambda x}
\]
ein.
Diese Gleichung kann nur erf"ullt sein, wenn
\[
C'(x)=y_{20}
\qquad\Rightarrow\qquad
C(x)=y_{20}x+y_{10},
\]
die L"osung der Gleichung ist also
\[
y(x)=\begin{pmatrix}
y_{20}x+y_{10}\\
y_{20}
\end{pmatrix}e^{\lambda x}.
\]
\begin{figure}
\centering
\includegraphics{chapters/images/geometrie-8.pdf}
\caption{L"osungskurven f"ur den Fall nicht diagonalisierbarer Jacobi-Matrix
mit zwei gleichen Eigenwerten.
\label{geometrie:jnf-kurven}}
\end{figure}%
In Abbildung~\ref{geometrie:jnf-kurven} sind die L"osungskurven dargestellt.
F"ur $\lambda >0$ streben die L"osungskurven gegen den Nullpunkt, aber auf
eine Art, sie im Grenzfall die $y_1$-Achse ber"uhren.

Falls die Diskriminante $\Delta$ negativ ist, gibt es keine reellen
Eigenwerte, also auch keine reellen Eigenvektoren.
Wir k"onnen aber trotzdem eine Basis finden, in der die Geometrie
dar Bahnkurven leichter verst"andlich ist.
Dazu schreiben wir 
\[
\alpha = \frac{\operatorname{Spur}A}2=\frac{a+d}2
\qquad
\text{und}
\qquad
\beta = \sqrt{-\Delta},
\]
und betrachten die beiden Vektoren
\[
w_1 = \begin{pmatrix}0\\\beta\end{pmatrix}.
\qquad\text{und}\qquad
w_2 = \begin{pmatrix}b\\\alpha-a\end{pmatrix}
\]
Wir berechnen die Wirkung der Matrix $A$ auf diesen beiden Vektoren,
und zerlegen jeweils das Resultat wieder in $w_1$ und $w_2$:
\begin{align*}
Aw_1
&=
\begin{pmatrix}a&b\\c&d\end{pmatrix}
\begin{pmatrix}0\\\beta\end{pmatrix}
=
\begin{pmatrix}
\beta b\\
\beta d
\end{pmatrix}
=
v
\begin{pmatrix}0\\\beta\end{pmatrix}
+
\beta
\begin{pmatrix}b\\\alpha-a\end{pmatrix}
\\
Aw_2
&=
\begin{pmatrix}a&b\\c&d\end{pmatrix}
\begin{pmatrix}b\\\alpha-a\end{pmatrix}
=
\begin{pmatrix}
ab-ab+b\alpha\\
bc-ad+\alpha d
\end{pmatrix}
=
\alpha
\begin{pmatrix}b\\\alpha-a\end{pmatrix}
+u
\begin{pmatrix}0\\\beta\end{pmatrix}
\end{align*}
Wir m"ussen nur noch die Konstanten $u$ und $v$ bestimmen:
\begin{align*}
\beta d
&=
v\beta
+
\alpha\beta
-
\beta a
&&\Rightarrow&
v
&=
\alpha a+d-\alpha=2\alpha-\alpha=\alpha
\\
-\det A +\alpha d
&=
\alpha^2- \alpha a+u\beta
&&\Rightarrow&
u\beta&=-\det A +\alpha(a+d)-\alpha^2
=\alpha^2-\det A
=\Delta=-\beta^2
\end{align*}
Aus der zweiten Gleichung folgt $ u=-\beta$.
Damit haben wir die Wirkung der Matrix $A$ auf den Vektoren $w_1$ und $w_2$
bestimmt, und wir k"onnen daraus die Matrix von $A$ in der Basis
$\{w_1,w_2\}$ ablesen, wir bezeichnen sie mit $A'$
\[
\begin{aligned}
Aw_1&=\alpha w_1 + \beta w_2\\
Aw_2&=-\beta w_1 + \alpha w_2
\end{aligned}
\qquad\Rightarrow\qquad
A'=\begin{pmatrix}
\alpha&\beta\\
-\beta&\alpha
\end{pmatrix}
\]
Dies ist die gesuchte Form der Matrix, in der sich die L"osungskurven
leichter beschreiben lassen.
Eine L"osung daf"ur l"asst sich angeben, wenn man ber"ucksichtigt, dass
$A'$ einer Drehmatrix "ahnelt.
Wir vermuten daher, dass die L"osungskurve im wesentlichen den kritischen
Punkt umkreist, m"oglicherweise mit einer "Anderung des Abstandes
zum kritischen Punkt, und schreiben daher
\begin{equation}
y(x)
=
\begin{pmatrix}
r_0e^{ux}\cos(vx+\delta_0)\\
r_0e^{ux}\sin(vx+\delta_0)
\end{pmatrix}
=
r_0e^{ux}
\begin{pmatrix}
\cos(vx+\delta_0)\\
\sin(vx+\delta_0)
\end{pmatrix}
,
\label{geometrie:rotsol}
\end{equation}
wobei wird $r_0$ und $\delta_0$ so w"ahlen, dass
\[
y_{10}=r_0\cos\delta_0
\qquad\text{und}\qquad
y_{20}=r_0\sin\delta_0.
\]
Setzen wir jetzt den Ansatz~(\ref{geometrie:rotsol}) in die
Differentialgleichung ein, erhalten wir
\begin{align*}
y'(x)
&=
\begin{pmatrix}
r_0ue^{ux}\cos(vx+\delta_0)-r_0e^{ux}v\sin(vx+\delta_0)\\
r_0ue^{ux}\sin(vx+\delta_0)+r_0e^{ux}v\cos(vx+\delta_0)
\end{pmatrix}
\\
&=
\begin{pmatrix}
 \alpha&\beta\\
-\beta &\alpha
\end{pmatrix}
\begin{pmatrix}
r_0e^{ux}\cos(vx+\delta_0)\\
r_0e^{ux}\sin(vx+\delta_0)
\end{pmatrix}
=
\begin{pmatrix}
r_0e^{ux}\alpha\cos(vx+\delta_0)+r_0e^{ux}\beta\sin(vx +\delta_0)\\
-r_0e^{ux}\beta\cos(vx+\delta_0)+r_0e^{ux}\alpha\sin(vx+\delta_0)
\end{pmatrix}
\end{align*}
Diese Gleichung ist genau dann korrekt, wenn 
\[
u=\alpha
\qquad\text{und}\qquad
v=-\beta.
\]
Die Zahlen $\alpha$ und $\beta$ charakterisieren also wieder die L"osung.
F"ur $\alpha < 0$ n"ahern sich die L"osungen dem kritischen Punkt, f"ur
$\alpha>0$ entfernen sie sich.
Die Zahl $\beta$ ist die Winkelgeschwindigkeit, mit der die L"osung
um den kritischen Punkt rotiert.
Die L"osungskurven sind daher Spiralen um den kritischen Punkt, sie
sind in Abbildung~\ref{geometrie:rotkurv} dargestellt.
\begin{figure}
\centering
\begin{tabular}{ccc}
\includegraphics{chapters/images/geometrie-9.pdf}&%
\includegraphics{chapters/images/geometrie-11.pdf}&%
\includegraphics{chapters/images/geometrie-10.pdf}\\
$\alpha > 0$&$\alpha = 0$&$\alpha < 0$
\end{tabular}
\caption{L"osungskurven des linearisierten Systems im Falle $\Delta < 0$
sind Spiralen um den kritischen Punkt
\label{geometrie:rotkurv}}
\end{figure}

\begin{beispiel}
Wir kehren nochmals zum Beispiel~(\ref{geometrie:nullklinen-dgl1})
von Seite~\pageref{geometrie:nullklinen-dgl1} zur"uck.
Die kritischen Punkte wurden in (\ref{geometrie:nullklinen-krit}) bereits
zusammengestellt.
Die Ableitung von $f$, also die Jacobi-Matrix, ist
\[
\frac{\partial f}{\partial y}
=
\frac{\partial}{\partial y}
\begin{pmatrix}
2y_1-y_1^2-3y_1y_2\\
y_2-y_2^2-y_1y_2
\end{pmatrix}
=
\begin{pmatrix}
2-2y_1-3y_2 & -3y_1\\
-y_2        &1-2y_2-y_1
\end{pmatrix}
\]
In den vier kritischen Punkten finden wir die folgenden Matrizen und
Eigenwerte
\begin{align*}
&(0,0)
	&\frac{\partial f}{\partial y}&=\begin{pmatrix}2&0\\0&1\end{pmatrix}
		&&\lambda_1=2,\;\lambda_2=1\\
&(2,0)
	&\frac{\partial f}{\partial y}&=\begin{pmatrix}-2&-6\\0&-1\end{pmatrix}
		&&\lambda_1=-2,\;\lambda_2=-1\\
&(0,1)
	&\frac{\partial f}{\partial y}&=\begin{pmatrix}-1&0\\-1&-1\end{pmatrix}
		&&\lambda_1=-1,\;\lambda_2=-1\\
&\textstyle(\frac12,\frac12)
	&\frac{\partial f}{\partial y}&=\begin{pmatrix}-\frac12&-\frac32\\-\frac12&-\frac12\end{pmatrix}
		&&\lambda_1=\frac{\sqrt{3}-1}2,\;\lambda_2=-\frac{\sqrt{3}+1}2
\end{align*}
Nur bei den Punktein $(2,0)$ und $(0,1)$ sind beide Eigenwerte negativ,
nur diese beiden Punkte sind stabil, wie bereits die  Diskussion der
Nullklinen auf Seite~\pageref{geometrie:nullklinen-stabilitaet}
gezeigt hat.
\end{beispiel}

%
% Komplexe Eigenwerte
%
\subsection{Komplexe Eigenwerte}
Die Darstellung im vorangegangenen Abschnitt war darum bem"uht,
komplexe Zahlen zu vermeiden.
Die Darstellung im Falle $\Delta<0$ wurde dadurch unn"otig verkompliziert,
in diesem Abschnitt soll gezeigt werden, wie die Formeln f"ur die Vektoren
$w_1$ und $w_2$ mit Hilfe komplexer Zahlen hergeleitet werden k"onnen.

Zun"achst halten wir fest, dass im Falle $\Delta<0$ zwei konjugiert
komplexe Eigenwerte
$
\lambda= \alpha + i\beta
$
und
$
\overline{\lambda}= \alpha - i\beta
$
existieren.
Nehmen wir an, dass $v$ ein Eigenvektor zum Eigenwert $\lambda$ ist,
dann ist $\overline{v}$, dessen Komponenten die konjugiert komplexen
Komponenten von $v$ sind, ein Eigenvektor zum Eigenwert $\overline{\lambda}$.
Grund daf"ur ist die Tatsache, dass die Matrix $A$ nur reelle Matrixelemente
hat, also gilt
\[
A\overline{v}
=
\overline{Av}
=
\overline{\lambda v}=\overline{\lambda}\overline{v}.
\]
Der Vektor $v$ ist nat"urlich nicht geeignet f"ur eine reelle Beschreibung
der L"osungskurven des linearisierten Systems.
Wir konstruieren daher die Vektoren
\begin{equation}
\begin{aligned}
w_1&=\frac{i}2(v-\overline v)
&&\qquad
&
v&=-iw_1+w_2
\\
w_2&=\frac12(v+\overline v)
&&\qquad
&
\overline{v}&=iw_1+w_2
\end{aligned}
\label{geometrie:wv}
\end{equation}
und untersuchen, wie die Matrix $A$ darauf wirkt:
\begin{align*}
Aw_1
&=
\frac{i}2(Av-A\overline v)
=
\frac{i}2(\lambda v-\overline{\lambda}\overline{v})
\\
Aw_2
&=
\frac12(Av+A\overline{v})
=
\frac12(\lambda v+\overline{\lambda}\overline{v}).
\end{align*}
Setzen wir die Darstellungen von $v$ und $\overline{v}$ durch $w_i$ aus 
(\ref{geometrie:wv}) ein, und erhalten:
\begin{align*}
\frac{i}2(\lambda v-\overline{\lambda}\overline{v})
&=
\frac{i}2(\lambda(-iw_1+w_2) -\overline{\lambda}(iw_1+w_2))
=
\frac{1}2(\lambda+\overline{\lambda}) w_1
+
\frac{i}2(\lambda-\overline{\lambda}) w_2
=\alpha w_1-\beta w_2
\\
\frac12(\lambda v+\overline{\lambda}\overline{v})
&=
\frac12(\lambda(-iw_1+w_2)+\overline{\lambda}(iw_1+w_2))
=
-\frac{i}2(\lambda-\overline{\lambda}) w_1
+
\frac12(\lambda+\overline{\lambda}) w_2.
=\beta w_1+\alpha w_2
\end{align*}
Verwendet man also $\{w_1,w_2\}$ als Basis, dann bekommt die Matrix die
Form
\begin{equation}
A'=\begin{pmatrix}
\alpha&-\beta\\
\beta &\alpha
\end{pmatrix}
\label{geometrie:drehmatrix}
\end{equation}
Auf Grund der Konstruktion haben die Vektoren $w_1$ und $w_2$ reelle
Komponenten, $w_1$ ist der Realteil des Vektors $v$, $w_2$ ist
der Imagin"arteil.
Damit haben wir ein Rezept, wie wir eine Basis von reellen Vektoren
konstruieren k"onnen, in denen das System die
Form~(\ref{geometrie:drehmatrix}) hat.

Die Komponenten eines Eigenvektors $v$ erf"ullen die Gleichung
\[
(a-\lambda)v_1 + bv_2=0
\]
eine L"osung daf"ur ist
\[
v=\begin{pmatrix}
-b\\
a-\alpha-i\beta
\end{pmatrix},
\]
dessen Real- und Imagin"arteile
\[
\begin{pmatrix}
-b\\a-\alpha
\end{pmatrix}
\qquad\text{und}\qquad
\begin{pmatrix}
0\\\beta
\end{pmatrix},
\]
dies sind die Vektoren, die im vorangegangenen Abschnitt aus dem "Armel
gesch"uttelt worden waren, um die Matrix des Systems in die
Form~(\ref{geometrie:drehmatrix}) zu bringen.

\begin{beispiel}
\label{geometrie:fh-fortsetzung}
Wir wenden obige Analyse auf das
Fitzhugh-Nagumo-Modell~(\ref{geometrie:fitzhugh-dgl}) von
Seite~\pageref{geometrie:fitzhugh-dgl} an.
Um die Diskussion etwas zu vereinfachen, untersuchen wir nur den Fall
$\tau = 1$.
Wir m"ussen die Jacobi-Matrix in einem kritischen Punkt bestimmen, sie ist
\begin{equation}
J=
\begin{pmatrix}
1-v^2 &  -1 \\
  1   &  -b
\end{pmatrix}.
\end{equation}
Das charakteristische Polynom ist
\begin{align*}
\det(J-\lambda E)
&=
\left|
\begin{matrix}
1-v^2-\lambda&-1\\
1&-b-\lambda
\end{matrix}
\right|
\\
&=
(1-v^2-\lambda)(-b-\lambda)+1
\\
&=
(\lambda+v^2-1)(\lambda+b)+1
\\
&=
\lambda^2 + (\underbrace{v^2 + b - 1}_{\textstyle p})\lambda
+ \underbrace{b(v^2 - 1)+1}_{\textstyle q}.
\end{align*}
Es hat die Nullstellen
\begin{equation}
\lambda_{1,2}
=
-\frac{p}{2}\pm\sqrt{\frac{p^2}4-q}
=
-\frac{v^2+b-1}2\pm\sqrt{\frac{(v^2+b-1)^2}4-b(v^2-1)-1}.
\label{geometrie:fn-eigenwerte}
\end{equation}
Der erste Term in der Wurzel ist das Quadrat des Terms vor der Wurzel.
Ohne $q$ in der Wurzel g"abe es einen Eigenwert mit dem gleichen Vorzeichen
wie $p$, der andere ist $0$.
Die Vorzeichen von $p$ und $q$ bestimmen also weitgehen, ob ein 
kritischer Punkt stabil oder instabil ist.

\begin{enumerate}
\item
Wenn $q<0$ ist, dann ist die Wurzel gr"osser als $p/2$, die beiden
Eigenwerte haben verschiedenes Vorzeichen, und der kritische Punkt
ist instabil unabh"angig vom Vorzeichen von $p$. 
Dieser Fall tritt ein, wenn 
\[
b(v^2 -1) + 1 < 0
\qquad\Rightarrow\qquad
1-\frac1b > v^2.
\]
F"ur $b<1$ wird die linke Seite negativ, dann kann dieser Fall also gar nicht
eintreten.
\item
Ist $q>0$ und von gen"ugend grossem absolutem Betrag,
dann wird der Radikand negativ, beide Eigenwerte
sind komplex und der kritische Punkt ist genau dann stabil, wenn $p>0$ ist.
Ist $q>0$, aber nicht von gen"ugend grossem absolutem Betrag, dann
bleibt der Radikand positiv.
In diesem Fall haben beide Eigenwerte das gleiche Vorzeichen wie $p$,
auch in diesem Fall hat man also Stabilit"at genau dann, wenn $p>0$ ist.
\end{enumerate}
Wir k"onnen die Resultate der obigen Diskussion in der folgenden
Entscheidungstabelle zusammenfassen:
\begin{center}
\begin{tabular}{|>{$}c<{$}|>{$}c<{$}|l|}
\hline
q=b(v^2-1)+1 & p= v^2 + b -1 &           \\
\hline
     <0      &               &  instabil \\
     >0      &       <0      &  instabil \\
     >0      &       >0      &  stabil   \\
\hline
\end{tabular}
\end{center}

F"ur $b>1$ und $a=0$ ist $(0,0)$ ein kritischer Punkt, und es gilt $q=-b+1<0$
und $p=b-1>0$, wir sind daher im Fall~1, der Ursprung ist instabil.
Es ist aber auch $p=v^2+b-1>v^2\ge 0$, der Fall $p<0$ kann also 
gar nicht eintreten, ein solcher kritischer Punkt muss also immer
stabil sein.
Dies deckt sich mit den Resultaten der Diskussion von
Seite~\pageref{geometrie:fh-diskussion}.
\end{beispiel}

%
% Hopf-Bifurkation
%
\subsection{Hopf-Bifurkation}
\begin{figure}
\centering
\includegraphics{chapters/images/hopf-1.pdf}
\caption{Fluss f"ur $b<0$, der kritische Punkt ist stabil,
Bahnkurven konvergieren gegen $0$.
\label{geometrie:hopf1}}
\end{figure}%
\begin{figure}
\centering
\includegraphics{chapters/images/hopf-2.pdf}
\caption{Fluss f"ur $b=0$, der kritische Punkt ist immer noch stabil,
die Bahnkurven n"ahern sich jedoch nicht mehr exponentiell schnell
dem Nullpunkt.
\label{geometrie:hopf2}}
\end{figure}%
\begin{figure}
\centering
\includegraphics{chapters/images/hopf-3.pdf}
\caption{Fluss f"ur $b>0$, der Nullpunkt ist nicht mehr stabil, daf"ur
ist der Kreis mit Radius $\sqrt{b}$ eine stabile periodische Bahn (blau),
gegen die alle Bahnkurven exponentiell schnell konvergieren.
\label{geometrie:hopf3}}
\end{figure}%
\begin{figure}
\centering
\begin{tabular}{ccc}
\includegraphics{chapters/images/hopf-4.pdf}&%
\includegraphics{chapters/images/hopf-5.pdf}&%
\includegraphics{chapters/images/hopf-6.pdf}%
\end{tabular}
\caption{Vorzeichen von $\dot r$ in Abh"angigkeit von $b$.
Punkte mit $\dot r <0$ sind blau gef"arbt, Punkte mit $\dot r >0$ rot.
\label{geometrie:hopfvorzeichen}}
\end{figure}%
Die in Abschnitt~\ref{geometrie:subsection:bifurkationen} untersuchten
Bifurkationen eindimensionaler Differentialgleichungen k"onnen in
analoger Form auch bei zweidimensionalen Differentialgleichungen auftreten.
Sie sind jedoch immer eindimensionale Bifurkationen, die entlang der
durch die Eigenvektoren der Linearisierung gegebenen Richtungen
auftreten.

Die h"ohere Dimensionszahl erlaubt aber auch eine Bifurkation, bei der
ein stabiler Fixpunkt in stabil wird und einen stabilen Zyklus ``abwirft''
(Abbildungen~\ref{geometrie:hopf1}, \ref{geometrie:hopf2} and
\ref{geometrie:hopf3}).
Sie heisst die Hopf-Bifurkation.
\index{Hopf-Bifurkation}
Wir betrachten dazu das System 
\begin{equation}
\begin{aligned}
\dot r      &= r(b-r^2)\\
\dot \varphi&= -1
\end{aligned}
\label{geometrie:hopfsystem}
\end{equation}
in Polarkoordinaten.
Offenbar ist $r=0$ ein Fixpunkt.
F"ur $b>0$ gibt es ausserdem eine periodische Bahn mit $r=\sqrt{b}$
(Abbildung~\ref{geometrie:hopf3}).
Wir wollen die Stabilit"at des Fixpunktes sowie der periodischen Bahn
untersuchen.
Die Abbildung~\ref{geometrie:hopfvorzeichen} fasst die f"ur das Bahnverhalten
entscheidenden Vorzeichen in den drei F"allen $b<0$, $b=0$ und $b>0$
zusammen.

In Polarkoordinaten beschreibt die Gleichung f"ur $r$ eine
Heugabel-Bifurkation.
Der kritische Punkt $r$ ist f"ur $b<0$ stabil, er wird f"ur $b>0$
instabil, daf"ur entstehen zwei neue stabile kritische Punkte
$r=\pm\sqrt{b}$.

Unsere bisherige Theorie zur Beurteilung von Fixpunkten ging von
kartesischen Koordinaten aus, wir f"uhren daher die Analyse auch noch
in kartesischen Koordinaten durch.
Die Umrechnungsformeln von Polarkoordinaten in kartesische Koordinaten
und ihre Ableitungen
\[
\begin{aligned}
x&=r\cos\varphi&&\qquad&\dot x&=\dot r\cos\varphi-r\sin\varphi\cdot\dot\varphi\\
y&=r\sin\varphi&&\qquad&\dot y&=\dot r\sin\varphi+r\cos\varphi\cdot\dot\varphi
\end{aligned}
\]
erlauben uns,
das System~(\ref{geometrie:hopfsystem}) in kartesische Koordinaten
umzurechnen:
\begin{equation}
\begin{aligned}
\dot x&=r(b-r^2)\frac{x}{r}+y=(b-x^2-y^2)x+y\\
\dot y&=r(b-r^2)\frac{y}{r}-x=(b-x^2-y^2)y-x.
\end{aligned}
\label{geometrie:hopf-kartesisch}
\end{equation}
Zur Beurteilung der Stabilit"at des Nullpunktes berechnen wir die
Jacobi-Matrix
\[
J(x,y)=
\begin{pmatrix}
b-3x^2-y^2&1-2xy\\
-1-2xy&b-x^2-3y^2
\end{pmatrix}
\quad\Rightarrow\quad
J(0,0)=\begin{pmatrix}
b&1\\-1&b
\end{pmatrix}.
\]
Die Matrix $J(0,0)$ hat das charakteristische Polynom
\[
(b-\lambda)^2+1=0
\]
mit den Nullstellen
\[
\lambda=b\mp i.
\]
Stabilit"at wird durch das Vorzeichen des Realteils der Eigenwerte
bestimmt, wir lesen daher ab, dass der kritische Punkt $0$ stabil
ist f"ur $b<0$ und instabil f"ur $b>0$.

\section{"Ubungsaufgaben}
\rhead{"Ubungsaufgaben}
\uebungsaufgabe{601}
\uebungsaufgabe{602}
\uebungsaufgabe{603}


%
% komplex.tex -- Komplexe Differentialgleichungen
%
% (c) 2015 Prof Dr Andreas Mueller, Hochschule Rapperswil
%
\chapter{Komplexe Differentialgleichungen\label{chapter:komplexeanalysis}}
\rhead{}
\lhead{Komplexe Differentialgleichungen}
Die bisher betrachteten Differentialgleichungen waren immer f"ur
$x\in\mathbb R$ definiert.
Bei der L"osung mit Hilfe von Potenzreihen haben wir L"osungsfunktionen
gefunden, die man auch f"ur komplexe $x$-Werte auswerten kann.
Definiert man die Ableitung einer Funktionen einer komplexen Variablen
$z$ rein formal als
\[
\frac{d}{dz}z^n= nz^{n-1},
\]
dann kann man auch Potenzreihen in der Variablen $z$ formal differenzieren,
indem man jeden Term der Potenzreihe ableitet.
Und die mit der Potenzreihen-Methode gefunden L"osungen erf"ullen dann
auch die urspr"ungliche Differentialgleichung.
Dies ist aber eine rein formale "Uberlegung, da die Ableitung nach einer
komplexen Variablen noch gar nicht definiert ist.

%
% Komplex differenzierbare Funktion
%
\section{Komplex differenzierbare Funktionen}
\rhead{Komplex differenzierbare Funktionen}
Wir betrachten in diesem Kapitel komplexwertige Funktionen,
\index{komplexwertige Funktion}%
die ein einem Teilgebiet der komplexen Ebene definiert sind.
Ein {\em Gebiet} ist eine offene Teilmenge $\Omega\subset \mathbb C$.
\index{Gebiet}%
{\em Offen} heisst, dass mit jedem Punkt $z_0\in\Omega$ eine Umgebung
\index{offen}%
\index{Umgebung}%
\[
U=\{z\in\mathbb Z\,|\,|z-z_0|<\varepsilon\}
\]
ebenfalls in $\Omega$ enthalten ist, also $U\subset \Omega$ f"ur gen"ugen
kleines $\varepsilon$.
Sei also $f(z)$ eine in $\Omega\subset\mathbb C$ definierte
Funktion $f\colon\Omega\to\mathbb C$. 

Eine komplexwertige Funktion $f(z)$ kann betrachtet werden als zwei
reellwertige Funktionen von zwei Variablen $x$ und $y$:
\[
f(z)=\operatorname{Re}f(x+iy) + i \operatorname{Im}f(x+iy)
\]
Schreibt man
$\operatorname{Re}f(x+iy)=u(x,y)$
und
$\operatorname{In}f(x+iy)=v(x,y)$,
dann ist die komplexe Funktion vollst"andig durch reelle Funktionen
beschrieben.
Und nat"urlich wissen wir auch, was unter den Ableitungen der Funktionen 
$u(x,y)$ und $v(x,y)$ zu verstehen ist.
Der Funktion $f(z)$ entspricht eine Abbildung $\mathbb R^2\to\mathbb R^2$
\index{Abbildung}%
\[
(x,y)\mapsto\begin{pmatrix}u(x,y)\\v(x,y)\end{pmatrix}.
\]
Die Ableitung einer solchen Funktion im Punkt $(x_0,y_0)$
ist eine lineare Abbildung von Vektoren, die in linearer N"aherung
\index{lineare Naherung@lineare N\"aherung}
\index{Naherung@N\"aherung, lineare}
den Funktionswert bei $f(z_0 + \Delta z)$ 
\[
\begin{pmatrix}
u(x+\Delta x, y +\Delta y)\\
v(x+\Delta x, y +\Delta y)
\end{pmatrix}
=
\begin{pmatrix}
\frac{\partial u}{\partial x}&\frac{\partial u}{\partial y}\\
\frac{\partial v}{\partial x}&\frac{\partial v}{\partial y}
\end{pmatrix}
\begin{pmatrix} \Delta x\\\Delta y \end{pmatrix}
+o(\Delta x, \Delta y).
\]
In dieser Sicht einer komplexen Funktion gibt es keine einzelne Zahl, die
die Funktion einer Ableitung "ubernehmen k"onnte, die Ableitung
ist eine $2\times 2$-Matrix.

%
% Definition der komplexen Ableitungen
%
\subsection{Komplexe Ableitung}
Die Ableitung einer Funktion einer reellen Variablen wird mit Hilfe des
Grenzwertes
\[
f'(x_0)=\lim_{x\to x_0}\frac{f(x)-f(x_0)}{x-x_0}
\]
definiert, oder als diejenige Zahl $f'(x_0)\in\mathbb R$ mit der Eigenschaft,
dass
\begin{equation}
f(x)=f(x_0)+f'(x_0)(x-x_0) + o(x-x_0)
\label{komplex:abldef}
\end{equation}
gilt.
Der Term $x-x_0$ und die Gleichung \eqref{komplex:abldef} sind aber auch
f"ur komplexe Argument sinnvoll, wir definieren daher

\begin{definition}
Die komplexe Funktion $f(z)$ heisst im Punkt $z_0$ komplex differenzierbar
und hat die komplexe Ableitung $f'(z_0)\in\mathbb C$, wenn 
\index{komplex differenzierbar}%
\index{komplexe Ableitung}%
\index{Ableitung!komplexe}%
\begin{equation}
f(z)=f(z_0) + f'(z_0)(z-z_0) +o(z-z_0)
\label{komplex:defkomplabl}
\end{equation}
gilt.
\end{definition}

\begin{beispiel}
Die Funktion $z\mapsto f(z)=z^n$ ist "uberall komplex differenzierbar
und hat die Ableitung $nz^{n-1}$.
Um dies nachzupr"ufen, m"ussen wir die Bedingung~\eqref{komplex:defkomplabl}
verifizieren.
Aus einer wohlbekannten Faktorisierung von $z^n - z_0^n$ k"onnen wir den
Differenzenquotienten finden:
\begin{align*}
\frac{f(z)-f(z_0)}{z-z_0}
&=
\frac{z^n-z_0^n}{z-z_0}
=
\frac{(z-z_0)(z^{n-1}+z^{n-2}z_0+z^{n-3}z_0^2+\dots+z^{n-1})}{z-z_0}
\\
&=
\underbrace{z^{n-1}+z^{n-2}z_0+z^{n-3}z_0^2+\dots+z^{n-1}
}_{\displaystyle \text{$n$ Summanden}}.
\end{align*}
Lassen wir jetzt $z$ gegen $z_0$ gehen, wird die rechte Seite
zu $nz_0^{n-1}$.
\end{beispiel}

\begin{beispiel}
Die Funktion $z\mapsto f(z)=\bar z=x-iy$ ist nicht differenzierbar.
Wenn $f(z)=\bar z$ differenzierbar w"are, dann m"usste es eine Zahl
$a\in\mathbb C$ geben, so dass 
\[
\bar z-\bar z_0=a(z-z_0)+o(z-z_0)
\]
gilt.
w"ahlen wir $z=z_0+x$ bzw.~$z=z_0+iy$, dann erhalten wir
\[
\begin{aligned}
z-z_0&=x:&
\bar z-\bar z_0&=x
&&\Rightarrow&
\bar z-\bar z_0&=1\cdot x
&&\Rightarrow&
a&=1
\\
z-z_0&=iy:&
\bar z-\bar z_0&=-iy
&&\Rightarrow&
\bar z-\bar z_0&=-1\cdot iy
&&\Rightarrow&
a&=-1
\end{aligned}
\]
Es ist also nicht m"oglich, eine einzige Zahl $a$ zu finden, die als
die Ableitung der Funktion $z\mapsto \bar z$ betrachtet werden k"onnte.
\end{beispiel}

Das letzte Beispiel zeigt, dass
selbst Funktionen, deren Real- und Imagin"arteil beliebig oft stetig
differenzierbare Funktionen sind, nicht komplex differenzierbar
sein m"ussen.
Komplexe Differenzierbarkeit ist eine wesentlich st"arkere Bedingung
an eine Funktion, komplex differenzierbare Funktionen bilden eine
echte Teilmenge aller Funktionen, deren Real- und Imagin"arteil
differenzierbar ist.

%
% Cauchy-Riemann-Differentialgleichungen
%
\subsection{Die Cauchy-Riemann-Differentialgleichungen}
Komplexe Funktionen k"onnen nur differenzierbar sein, wenn sich die vier
partiellen Ableitungen zu einer einzigen komplexen Zahl zusammenfassen
lassen.
Um diese Beziehung zu finden, gehen wir von einer komplexen Funktion
\[
f(x+iy) = u(x,y) + iv(x,y)
\]
aus, und berechnen die Ableitung auf zwei verschiedene Arten, indem
wir sowohl nach $x$ als auch nach $iy$ ableiten:
\begin{align*}
f'(z)&
=
\lim_{x\to 0}\frac{f(z+x)-f(z)}{x}
=
\frac{\partial u}{\partial x}+i\frac{\partial v}{\partial x}
\\
f'(z)&
=
\lim_{y\to 0}\frac{f(z+iy)-f(z)}{iy}
=
\frac1{i}
\frac{\partial u}{\partial y}+\frac{\partial v}{\partial y}
=
\frac{\partial v}{\partial y}
-i
\frac{\partial u}{\partial y}.
\end{align*}
Dies ist nur m"oglich, wenn Real- und Imagin"arteile "ubereinstimmen.
Es folgt also

\begin{satz}
\label{komplex:satz:cauchy-riemann}
Real- und Imagin"arteil $u(x,y)$ und $v(x,y)$ einer
komplex differenzierbaren Funktion $f(z)$ mit $f(x+iy)=u(x,y)+iv(x,y)$
erf"ullen die Cauchy-Riemannschen Differentialgleichungen
\index{Cauchy-Riemann-Differentialgleichungen}
\begin{equation}
\begin{aligned}
\frac{\partial u}{\partial x}
&=
\frac{\partial v}{\partial y},
&
\frac{\partial u}{\partial y}
&=
-
\frac{\partial v}{\partial x}.
\end{aligned}
\label{komplex:dgl:cauchy-riemann}
\end{equation}
\end{satz}

Leitet man die Cauchy-Riemann-Differentialgleichungen nochmals nach
$x$ und $y$ ab, erh"alt man
\begin{equation*}
\begin{aligned}
\frac{\partial^2 u}{\partial x^2}
&=
\frac{\partial^2 v}{\partial x\,\partial y},
&
\frac{\partial^2 u}{\partial x\,\partial y}
&=
-\frac{\partial^2 v}{\partial x^2},
&
\frac{\partial^2 u}{\partial y\,\partial x}
&=
\frac{\partial^2 v}{\partial y^2},
&
\frac{\partial^2 u}{\partial y^2}
&=
-\frac{\partial^2 v}{\partial y\,\partial x}.
\end{aligned}
\end{equation*}
Die erste und die letzte sowie die mittleren zwei k"onnen zu jeweils
einer Differentialgleichung f"ur die Funktionen $u$ und $v$ zusammengefasst
werden, n"amlich
\begin{equation*}
\frac{\partial^2 u}{\partial x^2}
+
\frac{\partial^2 u}{\partial y^2}
=
0
\qquad\text{und}\qquad
\frac{\partial^2 v}{\partial x^2}
+
\frac{\partial^2 v}{\partial y^2}
=
0.
\end{equation*}

\begin{definition}
Der Operator 
\[
\Delta =
\frac{\partial^2}{\partial x^2}
+
\frac{\partial^2}{\partial y^2}
\]
heisst der {\em Laplace-Operator} in zwei Dimensionen.
\index{Laplace-Operator}%
\end{definition}

\begin{definition}
Eine Funktion $h(x,y)$ von zwei Variablen heisst {\em harmonisch}, wenn sie
die Gleichung
\[
\Delta h=0
\]
erf"ullt.
\index{harmonische Funktion}%
\index{harmonisch}%
\end{definition}

\begin{satz}
Real- und Imagin"arteil einer komplexen Funktion sind harmonische Funktionen.
\end{satz}

Die Cauchy-Riemann-Differentialgleichungen schr"anken also einerseits stark
ein, welche Funktionen "uberhaupt als Real- und Imagin"arteil einer
komplex differenzierbaren Funktion in Frage kommen.
Andererseits koppeln sie auch Real- und Imagin"arteil stark zusammen.

\begin{beispiel}
Von einer komplex differenzierbaren Funktion $f(z)$ sei nur der Realteil
$u(x,y)=x^3 -3xy^2$ bekannt.
Man finde alle m"oglichen Funktionen $f(z)$.

Zun"achst kontrollieren wir, ob dies "uberhaupt ein Realteil sein kann,
indem wir nachrechnen, ob $u(x,y)$ harmonisch ist.
\begin{equation*}
\begin{aligned}
\frac{\partial u}{\partial x}
&=
3x^2-3y^2
&&\Rightarrow&
\frac{\partial^2 u}{\partial x^2}
&=
6x
\\
\frac{\partial u}{\partial y}
&=
-6xy
&&\Rightarrow&
\frac{\partial^2 u}{\partial y^2}
&=
-6x
\\
&&&&\Delta u&=\frac{\partial^2u}{\partial x^2}+\frac{\partial^2u}{\partial y^2}=6x-6x=0,
\end{aligned}
\end{equation*}
$u$ ist also harmonisch.

Um die Funktion $f$ zu finden, brauchen wir jetzt noch den Imagin"arteil.
Wir finden ihn mit Hilfe der Cauchy-Riemann-Differentialgleichungen.
Es gilt
\begin{equation}
\begin{aligned}
\frac{\partial v}{\partial x}
&=
-\frac{\partial u}{\partial y}=6xy,
&
\frac{\partial v}{\partial y}
&=
\frac{\partial u}{\partial x}=3x^2-3y^2
\end{aligned}
\label{komplex:crbeispiel}
\end{equation}
Aus der ersten Gleichung erh"alt man durch Integrieren nach $x$ 
\[
v(x,y)=-3x^2y + C(y),
\]
die Integrations-``Konstante'' ist eine Funktion, die aber nur von $y$
abh"angen darf.
Die zweite Cauchy-Riemann-Gleichung verwendet die Ableitung von $v$ nach $y$,
sie ist
\[
\frac{\partial v}{\partial y}=3x^2+C'(y).
\]
Aus der zweiten Gleichung von \eqref{komplex:crbeispiel} liest man
ab, dass
\[
C'(y)=-3y^2
\qquad\Rightarrow\qquad
C(y)=-y^3+k
\]
sein muss.
Damit ist $v$ bis auf eine Konstante bestimmt.
Die zugeh"orige Funktion $f(z)$ ist daher
\[
f(z)=f(x+iy)=x^3-3xy^2+i(3x^2y-y^3)+ik
=x^3 + 3x^2iy + 3x(iy)^2+(iy)^3+ik=z^3+ik.
\]
Wir haben die Funktion $f(z)$ bis auf eine Konstanten $ik$ 
aus ihrem Realteil rekonstruiert.
\end{beispiel}

Die Cauchy-Riemann-Differentialgleichungen besagen auch, dass man nur
die Ableitungen nach $x$ zu berechnen braucht, um die Ableitung $f'(x)$
zu bestimmen.
Die Rechenregeln f"ur die Ableitung lassen sich daher direkt auf
komplexe Funktionen "ubertragen:
\begin{align*}
\frac{d}{dz}z^n
&=
nz^{n-1}
\\
\frac{d}{dz}e^z
&=
e^z
\\
\frac{d}{dz}f(g(z))
&=
f'(g(z)) g'(z)
\\
\frac{d}{dz}\bigl(f(z)g(z)\bigr)
&=
f'(z)g(z)+f(z)g'(z)
\end{align*}

%
% Analytische Funktionen
%
\subsection{Analytische Funktionen}
Als wichtiges Beispiel komplex differenzierbarer Funktionen betrachten
wir die analytischen Funktionen.
\begin{definition}
Eine Funktion $f\colon\mathbb C\to\mathbb C$ heisst analytisch im Punkt
\index{analytisch}%
$z_0$, wenn sie in eine konvergente Potenzreihe
\begin{equation}
f(z)=\sum_{k=0}^\infty a_k(z-z_0)^k
\label{komplex:freihe}
\end{equation}
entwickelt werden kann.
\end{definition}

Da die Ableitungsregeln f"ur komplex differenzierbare Funktionen nicht
anders sind als die Ableitungsregeln f"ur relle Funktionen, muss gelten
\begin{equation}
f'(z)=\sum_{k=1}^\infty ka_k(z-z_0)^{k-1},
\label{komplex:fpreihe}
\end{equation}
und es stellt sich nur die Frage, ob diese Potenzreihe ebenfalls konvergent
ist.
Man kann dies mit der Formel f"ur den Konvergenzradius pr"ufen.
\index{Konvergenzradius}%
F"ur die Potenzreihe \eqref{komplex:freihe} f"ur $f(z)$ liefert sie den
Konvergenzradius
\[
\frac{1}{\varrho} = \limsup_{k\to\infty} \root{k}\of{|a_k|}.
\]
Dieselbe Formel f"ur die Reihe~\eqref{komplex:fpreihe} liefert
f"ur den Konvergenzradius der Reihenentwicklung der Ableitung $f'(z)$
\[
\limsup_{k\to\infty} \root{k}\of{k|a_k|}
=
\underbrace{\lim_{k\to\infty} \root{k}\of{k}}_{\textstyle=1}
\cdot
\underbrace{ \limsup_{k\to\infty} \root{k}\of{|a_k|}}_{\textstyle=1/\varrho}
=
\frac1{\varrho}.
\]
Die Reihe f"ur die Ableitung $f'(z)$ hat also den gleichen Konvergenzradius
wie die Reihe f"ur die Funktion $f(z)$, dies gilt nat"urlich auch f"ur
die h"oheren Ableitungen.

Eine analytische Funktion ist somit beliebig oft komplex differenzierbar.
Aus den Ableitungen kann wie bei reellen Funktionen die Taylor-Reihe
gebildet werden, sie muss mit der Potenzreihe \eqref{komplex:freihe}
"ubereinstimmen.
\index{Taylor-Reihe}%
Analytische Funktionen haben also eine konvergente Taylor-Reihe,
\[
f(z) = \sum_{k=0}^\infty \frac{f^{(k)}(z_0}{k!}(z-z_0)^k.
\]

%
% Wegintegrale und die Cauchy-Formel
%
\subsection{Wegintegrale\label{subsection:wegintegrale}}
Das Finden einer Stammfunktion, die Integration, ist die Grundtechnik,
\index{Stammfunktion}%
mit der man den "Ubergang von lokaler Information in Form von Ableitungen,
zu globaler Information "uber reelle Funktionen vollzieht.
Sie liefert aus der Steigung zwischen zwei Punkten $x_0$ und $x$ den
Funktionswert mittels
\[
f(x)=f(x_0)+\int_{x_0}^xf'(\xi)\,d\xi.
\]
Bei einer reellen Funktion gibt es nur eine Richtung, entlang der man
integrieren k"onnte.

Auch in der komplexen Ebene erwarten wir eine Formel
\[
f(z) = f(z_0) + \int_{z_0}^z f'(\zeta)\,d\zeta.
\]
In der komplexen Ebene gibt es aber beliebig viele Wege, mit denen die
Punkte $z_0$ und $z$ verbunden werden k"onnen. 
Der Wert von $f(z)$ muss also durch Integration entlang eines speziell
gew"ahlten Weges $\gamma$
\[
f(z) = f(z_0) + \int_{\gamma} f'(\zeta)\,d\zeta
\]
bestimmt werden.
Es muss also zun"achst gekl"art werden, wie ein solches Wegintegral
"uberhaupt zu verstehen und zu berechnen ist.
Dann gilt es zu untersuchen, inwieweit diese Konstruktion unabh"angig
von der Wahl des Weges ist.
F"ur komplex differenzierbare Funktionen wird sich eine sehr erfolgreiche
Theorie ergeben.

%
% Wegintegrale
%
\subsubsection{Definition des Wegintegrals}
Ein Weg in der komplexen Ebene ist eine Abbildung 
\index{Abbildung}%
\[
\gamma\colon [a,b]\to\mathbb C: t\mapsto \gamma(t).
\]
Wir verlangen f"ur unsere Zwecke zus"atzlich, dass $\gamma$ differenzierbar
ist.
Dann k"onnen wir f"ur jede beliebige Funktion das Wegintegral definieren.

\begin{definition}
Sei $\gamma\colon[a,b]\to\mathbb C$ ein Weg in $\mathbb C$ und $f(z)$
eine stetige komplexe Funktion, dann heisst
\[
\int_{\gamma} f(z)\,dz = \int_a^bf(\gamma(t)) \gamma'(t)\,dt
\]
das {\em Wegintegral} von $f(z)$ entlang der Kurve $\gamma$.
\index{Wegintegral}
\end{definition}

\begin{beispiel}
Man berechne das Wegintegral der Funktion $f(z)=z^n$ entlang des
Weges 
$\gamma(t)=1+t+it^2$
f"ur $t\in[0,1]$.

Die Definition besagt
\begin{align*}
\int_\gamma f(z)\,dz
&=
\int_0^1 f(\gamma(t))\gamma'(t)\,dt
=
\int_0^1 \gamma(t)^n \gamma'(t)\,dt
=
\int_0^1 \frac{d}{dt}\frac{\gamma(t)^{n+1}}{n+1}\,dt
\\
&=
\biggl[\frac{\gamma(t)^{n+1}}{n+1}\biggr]_0^1
=
\frac{(2+i)^{n+1}}{n+1}-\frac{1^{n+1}}{n+1}
=
\frac{(2+i)^{n+1}-1}{n+1}.
\end{align*}
Man stellt in diesem Beispiel auch fest, dass das Integral offenbar
unabh"angig ist von der Wahl des Weges, es kommt einzig auf die
beiden Endpunkte an:
\[
\int_\gamma z^n \,dz = \frac1{n+1}\bigl(\gamma(1)^{n+1}-\gamma(0)^{n+1}\bigr).
\]
\end{beispiel}

\begin{beispiel}
Wir berechnen als Beispiel das Wegintegral der Funktion $f(z)=1/z$ entlang
eines Halbkreises von $1$ zu $-1$. 
Es gibt zwei verschiedene solche Halbkreise:
\begin{equation*}
\begin{aligned}
\gamma_+(t)&=e^{it},&t&\in[0,\pi]
\\
\gamma_-(t)&=e^{-it},&t&\in[0,\pi]
\end{aligned}
\end{equation*}
Wir finden f"ur die Wegintegrale
\begin{align*}
\int_{\gamma_+}\frac1z\,dz
&=
\int_0^\pi \frac1{e^{it}}ie^{it}\,dt=i\int_0^\pi\,dt=i\pi
\\
\int_{\gamma_-}\frac1z\,dz
&=
-\int_0^\pi \frac1{e^{-it}}ie^{-it}\,dt=-i\int_0^\pi\,dt=-i\pi
\end{align*}
Das Wegintegral zwischen $1$ und $-1$ h"angt also mindestens f"ur diese
spezielle Funktion $f(z)=1/z$ von der Wahl des Weges ab.
\end{beispiel}

Wie Wahl der Parametrisierung der Kurve hat keinen Einfluss auf den 
Wert des Wegintegrals.

\begin{satz}
Seien $\gamma_1(t), t\in[a,b],$ und $\gamma_2(s),s\in[c,d]$
verschiedene Parametrisierungen
\index{Parametrisierung}%
der gleichen Kurve, es gebe also eine Funktion $t(s)$ derart, dass
$\gamma_1(t(s))=\gamma_2(s)$.
Dann ist
\[
\int_{\gamma_1}f(z)\,dz
=
\int_{\gamma_2}f(z)\,dz.
\]
\end{satz}

\begin{proof}[Beweis]
Wir verwenden die Definition des Wegintegrals
\begin{align*}
\int_{\gamma_1} f(z)\,dz
&=
\int_a^b f(\gamma_1(t))\,\gamma_1'(t)\,dt
=
\int_c^d f(\gamma_1(t(s))\,\underbrace{\gamma_1'(t(s)) t'(s)}_{\displaystyle
=\frac{d}{ds}\gamma_1(t(s))}\,ds
\\
&=
\int_c^d f(\gamma_2(s)\,\gamma_2'(s)\,ds
=
\int_{\gamma_2}f(z)\,dz
\end{align*}
Beim zweiten Gleichheitszeichen haben wir die Formel f"ur die
Variablentransformation $t=t(s)$ in einem Integral verwendet.
\index{Variablentransformation}%
\end{proof}

Wir erwarten, dass das Wegintegral "ahnlich wie das Integral reeller
Funktionen eine Art ``Umkehroperation'' zur Ableitung ist.
Wir untersuchen daher den Fall, dass $f(z)$ eine komplexe Stammfunktion $F(z)$
hat, also $f(z)=F'(z)$.
Wir berechnen das Wegintegral entlang des Weges $\gamma$:
\begin{align*}
\int_{\gamma}f(z)\,dz
&=
\int_a^bf(\gamma(t))\,\gamma'(t)\,dt
=
\int_a^bF'(\gamma(t))\,\gamma'(t)\,dt
=
\int_a^b\frac{d}{dt}F(\gamma(t))\,dt
=
F(\gamma(a))-F(\gamma(b))
\end{align*}
Dies ist genau die Formel, die man als den Hauptsatz der Infinitesimalrechnung
kennt.
Trotzdem ist die Situation hier etwas anders.
In der reellen Infinitesimalrechnung war die Existenz einer Stammfunktion
durch das Integral gesichert, man konnte mit
\[
F(x)=\int_a^xf(\xi)\,d\xi
\]
immer eine Stammfunktion angeben.
Im komplexen Fall k"onnen wir nat"urlich auch versuchen, eine Stammfunktion
mit Hilfe von 
\[
F(z)=\int_{\gamma_z} f(\zeta)\,d\zeta
\]
zu definieren.
Dabei muss allerdings $\gamma_z$ ein Weg sein, der im Punkt $z$ endet,
und wir wissen noch nicht einmal, ob die Wahl des Weges eine Rolle
spielt.
Bevor wir also sicher sein k"onnen, dass eine Stammfunktion existiert,
m"ussen wir zeigen, dass das Wegintegral einer komplex differenzierbaren
Funktion zwischen zwei Punkten nicht von der Wahl des Weges abh"angt,
der die beiden Punkte verbindet.
Dazu ist notwendig, geschlossene Wege genauer zu betrachten.

%
% Wegintegrale führen auf analytische Funktionen
%
\subsubsection{Wegintegrale f"uhren auf analytische Funktionen}
\begin{figure}
\centering
\includegraphics{chapters/images/komplex-4.pdf}
\caption{Pfad und Konvergenzradius f"ur den Nachweis, dass Wegintegrale
auf analytische Funktionen f"uhren (Satz~\ref{komplex:integralanalytisch})
\label{komplex:integralanalytischpfad}}
\end{figure}
Mit Wegintegralen kann man aus stetigen Funktionen neue Funktionen
konstruieren.
Die folgende Konstruktion liefert "uberraschenderweise immer
analytische Funktionen.
\begin{satz}
\label{komplex:integralanalytisch}
Sei $\gamma\colon [a,b]\to\mathbb C$ ein Weg in $\mathbb C$, der nicht
durch den Nullpunkt verl"auft, und $g$ eine stetige Funktion
auf $\gamma([a,b])$ (Abbildung~\ref{komplex:integralanalytischpfad}).
Dann ist die Funktion
\[
f(z) = \frac1{2\pi i}\int_\gamma \frac{g(x)}{x-z}\,dx
\]
in einer Umgebung des Nullpunktes analytisch:
\[
f(z) = \sum_{k=0}^\infty c_k z^k,\qquad
\text{mit\quad}
c_k=\frac1{2\pi i}\int_\gamma \frac{g(x)}{x^{k+1}}\,dx.
\]
Der Konvergenzradius $\varrho$ dieser Reihe ist der minimale Abstand der
Kurve $\gamma$ vom Nullpunkt.
\end{satz}

\begin{proof}[Beweis]
Zun"achst schreiben wir
\begin{equation}
\frac{1}{x-z}
=
\frac1x\cdot \frac{1}{1-\displaystyle\frac{z}{x}}
=
\frac1x\cdot \sum_{k=0}^\infty \biggl(\frac{z}{x}\biggr)^k
=
\sum_{k=0}^\infty \frac{z^k}{x^{k+1}}.
\label{komplex:georeihe}
\end{equation}
Damit k"onnen wir jetzt die Funktion $f(z)$ berechnen:
\begin{align*}
f(z)
&=
\frac1{2\pi i} \int_{\gamma} \frac{g(x)}{x-z}\,dx
=
\frac1{2\pi i} \int_{\gamma} \sum_{k=0}^\infty \frac{z^k}{x^{k+1}}g(x)\,dx
=
\sum_{k=0}^\infty
\underbrace{\biggl(\frac1{2\pi i} \int_{\gamma} \frac{g(x)}{x^{k+1}}\,dx\biggr)}_{\textstyle =c_k}
z^k
=
\sum_{k=0}^\infty c_kz^k.
\end{align*}
Wir m"ussen uns noch die Konvergenz dieser Reihen "uberlegen.
Wenn $z<\varrho$ ist, dann ist 
\[
\biggl|\frac{z}{x}\biggr| 
=
\frac{|z|}{|x|}
<1,
\]
so dass die geometrische Reihe \eqref{komplex:georeihe} konvergent ist,
daraus lesen wir ab, dass der Konvergenzradius mindestens $\varrho$
ist.
Gr"osser kann er allerdings auch nicht sein, da f"ur $|z|\ge \varrho$
das Integral nicht mehr definiert sein muss.
Nimmt man n"amlich einen Punkt von $g([a,b])$ f"ur $z$ wird der Integrand
unendlich gross.
\end{proof}

Der Satz~\ref{komplex:integralanalytisch} ist nur f"ur Potenzreihen
im Punkt $0$ formuliert, was im Wesentlichen durch die
Umformung~\eqref{komplex:georeihe} bedingt war.
Man kann dies aber auch als Potenzreihe
\[
\frac1{x-z}
=
\frac1{x-z_0-(z-z_0)}
=
\frac1{x-z_0}\cdot\frac1{1-\displaystyle\frac{z-z_0}{x-z_0}}
=
\frac1{x-z_0}\sum_{k=0}^\infty\biggl(\frac{z-z_0}{x-z_0}\biggr)^k
=
\sum_{k=0}^\infty\frac1{(x-z_0)^{k+1}}(z-z_0)^k
\]
im Punkt $z_0$ ausdr"ucken.
Man bekommt dann die Potenzreihe
\[
f(z) = \sum_{k=1}^\infty c_k(z-z_0)^k,\qquad
\text{mit}\quad
c_k=\frac1{2\pi i}\oint_\gamma\frac{g(x)}{(x-z_0)^{k+1}}\,dx
\]
f"ur das Wegintegral.

\subsubsection{Laurent-Reihen}
\label{sssec:LaurentReihen}
\begin{figure}
\centering
\includegraphics{chapters/images/komplex-3.pdf}
\caption{Pfad zur Herleitung der Laurent-Reihe einer Funktion $f(z)$
mit einer Singularit"at $z_0$.
\label{komplex:laurentpfad}}
\end{figure}%
\index{Laurent-Reihe}%
In Satz~\ref{komplex:integralanalytisch} konnten wir eine Potenzreihe f"ur
solche $z$ konstruieren, deren Betrag kleiner ist als der kleinste Abstand
der Kurve $\gamma$ vom Ursprung.
Dies war notwendig, weil in~\eqref{komplex:georeihe} die geometrische Reihe
nur konvergiert, wenn der Quotient $<1$ ist.
Wenn die Funktion $f(z)$ jedoch eine Singularit"at im Punkt $z_0$ hat, dann
kann es nicht m"oglich sein, die Funktion mit einer Potenzreihe zu
beschreiben.

Wir verwenden daher den speziellen Pfad in Abbildung~\ref{komplex:laurentpfad}.
Er f"uhrt in einem grossen Kreis $\gamma_1$ um den Punkt $z_0$ herum,
dann folgt ein zur $x$-Achse paralleler Abschnitt, der bis zum kleinen
Kreis $\gamma_2$ f"uhrt.
Nach Durchlaufen des kleinen Kreises $\gamma_2$ im Uhrzeigersinn folgt wieder
ein zur $x$-Achse paralleles St"uck zur"uck zum grossen Kreis.
Da die geraden St"ucke zweimal in entgegegengesetzer Richtung durchlaufen
werden, heben sie sich weg.
Ein Wegintegral entlang $\gamma$ zerf"allt daher in eine Differenz
\[
\oint_\gamma\dots\,dz
=
\oint_{\gamma_1}\dots\,dz
-
\oint_{\gamma_2}\dots\,dz
\]
von Wegintegralen entlang $\gamma_1$ und $\gamma_2$.

Der "aussere Pfad $\gamma_1$ gibt wie in Satz~\ref{komplex:integralanalytisch}
Anlass zu einer Potenzreihe in $(z-z_0)$.
Der innere Pfad $\gamma_2$ kann aber nicht so behandelt werden, da $z$ immer
weiter von $z_0$ entfernt als die Punkte auf $\gamma_2$.
Allerdings ist $|x/z| < 1$ f"ur Punkte auf $\gamma_2$, wir m"ussen daher
die geometrische Reihe auf $x/z$ anwenden:
\begin{align*}
\frac{1}{x-z}
&=
\frac{1}{x-z_0-(z-z_0)}
=
\frac{1}{z-z_0}
\cdot
\frac{1}{\displaystyle\frac{x-z_0}{z-z_0}-1}
=
-\sum_{k=0}^\infty \frac{(x-z_0)^k}{(z-z_0)^{k+1}}.
\end{align*}
Das Integral entlang der Kurve $\gamma_2$ kann also als Reihe in $1/(z-z_0)$
entwickelt werden:
\begin{align*}
f_2(z)
&=
\frac{1}{2\pi i}\int_{\gamma_2} \frac{g(x)}{x-z}\,dx
=
\frac{1}{2\pi i}\int_{\gamma_2}\sum_{k=0}^\infty
\frac{(x-z_0)^k}{(z-z_0)^{k+1}}\,dx
\\
&=
\sum_{k=0}^\infty
\biggl(
\underbrace{\frac1{2\pi i}\int_{\gamma_2} (x-z_0)^kg(x)\,dx
}_{\textstyle =d_{k+1}}
\biggr)
\frac1{(z-z_0)^{k+1}}
=\sum_{k=1}^\infty \frac{d_k}{(z-z_0)^k}.
\end{align*}
Zusammen mit der vom Integral entlang $\gamma_1$ herr"uhrenden Reihe finden
wir den Satz
\begin{satz}
\label{komplex:laurentreihe}
Ist $g(z)$ eine entlang der Kurve $\gamma$ wie in
Abbildung~\ref{komplex:laurentpfad} definierte stetige Funktion, dann gilt
\[
f(z)=\frac1{2\pi i}\oint_{\gamma} \frac{f(x)}{x-z}\,dx
=
\sum_{k=0}^{\infty} c_k(z-z_0)^k-\sum_{k=1}^\infty \frac{d_k}{(z-z_0)^k},
\]
wobei die Koeffizienten $c_k$ und $d_k$ gegeben sind durch
\[
\begin{aligned}
c_k&=\frac1{2\pi i}\oint_{\gamma_1} \frac{g(x)}{x-z_0}\,dx
&&
\text{und}
&
d_k&=\frac1{2\pi i}\oint_{\gamma_2} g(x)x^{k-1}\,dx.
\end{aligned}
\]
\end{satz}

\begin{definition}
Eine Reihe der Form
\[
\sum_{k=-\infty}^\infty a_k(z-z_0)^k
\]
heisst {\em Laurent-Reihe }
im Punkt $z_0$.
\end{definition}


%
% Geschlossene Wege
%
\subsubsection{Geschlossene Wege}
\begin{definition}
Ein Weg $\gamma\colon[a,b]\to\mathbb C$ heisst {\em geschlossen}, wenn
$\gamma(a)=\gamma(b)$.
\index{geschlossener Weg}
Das Integral entlang eines geschlossenen Weges h"angt nicht von der
Parametrisierung ab und wird zur Verdeutlichung mit
\[
\int_{\gamma}f(z)\,dz
=
\oint_{\gamma}f(z)\,dz
\]
bezeichnet.
\end{definition}

\begin{beispiel}
Wir berechnen das Integral von $f(z)=z^n$ entlang des Einheitskreises,
den wir mit $\gamma(t)=e^{it},t\in[0,2\pi]$ parametrisieren.
Die Definition liefert:
\begin{align*}
\oint_{\gamma}f(z)\,dz
&=
\int_0^{2\pi}e^{int}ie^{it}\,dt
=
i\int_0^{2\pi}e^{i(n+1)t}\,dt
\end{align*}
F"ur $n=-1$ ist dies das Integral einer konstanten Funktion, also
\[
\oint_{\gamma}\frac1z\,dz=2\pi i.
\]
F"ur $n\ne -1$ kann man eine Stammfunktion von $e^{i(n+1)t}$
verwenden:
\[
\oint_{\gamma}f(z)\,dz
=
i\left[\frac1{i(n+1)}e^{i(n+1)t}\right]_0^{2\pi}
=0,
\]
weil $e^{i(n+1)t}$ periodisch ist mit Periode $2\pi$.
\end{beispiel}
Das Beispiel zeigt, dass ein Wegintegral der Potenzfunktionen,
aller Polynome und schliesslich aller konvergenten Potenzreihen
"uber einen geschlossenen Weg verschwinden.
Es zeigt aber auch, dass das Wegintegral "uber einen geschlossenen
Weg nicht zu verschwinden braucht, wie das Beispiel $f(z)=1/z$ 
zeigt.
Letztere Funktion unterscheidet sich von den Potenzfunktionen allerdings
dadurch, dass sie im Nullpunkt nicht definiert ist.

\begin{satz}
Sei $f(z)$ eine in einem zusammenh"angenden Gebiet $\Omega\subset\mathbb C$
definierte komplexe Funktion, f"ur die das Wegintegral "uber jeden
geschlossenen Weg verschwindet.
Dann hat $f(z)$ eine komplexe Stammfunktion $F(z)$.
\end{satz}

\begin{proof}[Beweis]
Wir w"ahlen einen beliebigen Punkt $z_0\in\Omega$ definieren die
komplexe Stammfunktion mit Hilfe des Wegintegrals
\[
F(z)=\int_{\gamma_z} f(\zeta)\,d\zeta,
\]
wobei $\gamma_z$ ein beliebiger Weg ist, der $z_0$ mit $z$ verbindet.

Wir m"ussen uns davon "uberzeugen, dass die Wahl des Weges keinen Einfluss
auf $F(z)$ hat.
Dazu seien $\gamma_1$ und $\gamma_2$ zwei verschiedene Wege, die
$z_0$ mit $z$ verbinden.
Da die Parametrisierung der Wege keinen Einfluss auf das Wegintegral haben,
nehmen wir an, dass beide Wege auf dem Intervall $[0,1]$ definiert sind.

Jetzt konstruieren wir einen geschlossene Weg $\gamma$ durch die
Definition:
\[
\gamma\colon[0,2]\to\mathbb C:t\mapsto
\begin{cases}
\gamma_1(t)&\qquad 0\le t\le 1\\
\gamma_2(2-t)&\qquad 1\le t\le 2
\end{cases}
\]
Der Weg $\gamma$ besteht aus $\gamma_1$ und dem in umgekehrter Richtung
durchlaufenen Weg $\gamma_2$, denn an der Stelle $t=1$ passen die
beiden Teilwege nahtlos zusammen: $\gamma_1(1)=\gamma_2(1)=\gamma_2(2-1)$.
Wegen $\gamma(2)=\gamma_2(2-2)=\gamma_2(0)=\gamma_1(0)$ ist der
Weg geschlossen.
Nach Voraussetzung ist verschwindet das Wegintegral "uber $\gamma$.
Es folgt
\begin{align*}
0
&=
\int_{\gamma}f(z)\,dz
\\
&=
\int_0^1 f(\gamma_1(t))\gamma_1'(t)\,dt
+ \int_1^2f(\gamma_2(2-t))\frac{d}{dt}\gamma_2(2-t)\,dt
\\
&=
\int_0^1 f(\gamma_1(t))\gamma_1'(t)\,dt
- \int_1^2f(\gamma_2(2-t))\gamma_2'(2-t)\,dt
\\
&=
\int_0^1 f(\gamma_1(t))\gamma_1'(t)\,dt
- \int_0^1f(\gamma_2(s))\gamma_2'(s)\,ds
\\
&=
\int_{\gamma_1}f(z)\,dz - \int_{\gamma_2}f(z)\,dz
\\
\Rightarrow\qquad
\int_{\gamma_2}f(z)\,dz&=\int_{\gamma_1}f(z)\,dz.
\end{align*}
Da die Wahl des Weges keine Rolle spielt, ist $F(z)$ wohldefiniert.
\end{proof}

Die Bedingung des eben bewiesenen Satzes ist nicht wirklich n"utzlich,
sie ist kaum nachpr"ufbar.
Es braucht also zus"atzliche Anstrengungen um gen"ugend viele
Funktionen zu finden, welche die Eigenschaft haben, dass Wegintegrale
"uber geschlossene Wege verschwinden.
Wir zielen dabei auf den folgenden Satz hin:
\begin{satz}[Cauchy]
Ist $f(z)$ eine in einem Gebiet $\Omega\subset\mathbb C$ definierte
komplex differenzierbare Funktion, und ist $\gamma$ ein im Gebiet
$\Omega$ auf einen Punkt zusammenziehbarer geschlossener Weg, dann gilt
\[
\oint_{\gamma}f(z)\,dz=0.
\]
Ist insbesondere $\Omega$ {\em einfach zusammenh"angend}
\index{einfach zusammenhangend@einfach zusammenh\"angend}%
\index{zusammenziehbar}%
(d.~h.~jeder geschlossene Weg l"asst sich in einen Punkt zusammenziehen),
dann verschwindet das Wegintegral von $f(z)$ "uber jeden geschlossenen
Weg in $\Omega$.
\index{einfach zusammenhangend@einfach zusammenh\"angend}
\end{satz}

\begin{proof}[Beweis]
Wir verwenden f"ur den folgenden Beweis den Satz von Green "uber
\index{Green, Satz von}%
Wegintegrale in der Ebene.
Er besagt, dass f"ur einen geschlossenen Weg $\gamma$ der in der Ebene
das Gebiet $D$ berandet, und zwei Funktionen $L(x,y)$ und $M(x,y)$, gilt
\[
\oint_\gamma(L\,dx + M\,dy)
=
\int_D \biggl(\frac{\partial M}{\partial x}
-\frac{\partial L}{\partial y}\biggr)\,dx\,dy.
\]
Wir berechnen jetzt das Integral einer komplex differenzierbaren Funktion
$f(z)$
\begin{align*}
\oint_\gamma f(z)\,dz
&=
\int (u(x,y)+iv(x,y))(\dot x(t)+i\dot y(t))\,dt
\\
&=
\int u(x,y)\dot x(t) -v(x,y)\dot y(t)\,dt
+
i \int u(x,y)\dot y(t)+v(x,y)\dot x(t)\,dt
\\
&=\oint_\gamma(u\,dx - v\,dy) + i\oint_\gamma(v\,dx + u\,dy)
\\
&=
\int_D
\underbrace{-\frac{\partial v}{\partial x}}_{\displaystyle=\frac{\partial u}{\partial y}}
-\frac{\partial u}{\partial y}
\,dx\,dy
+i
\int_D
\underbrace{\frac{\partial u}{\partial x}}_{\displaystyle=\frac{\partial v}{\partial y}}
-\frac{\partial v}{\partial y}\,dx\,dy
=0.
\end{align*}
Dabei haben wir auf der dritten Zeile den Satz von Green angewendet,
und auf der letzten Zeile die Cauchy-Riemann-Differentialgleichungen.
\end{proof}

\subsection{Beispiel: Airy-Differentialgleichung}
\label{komplex:airydgl}
\index{Airy-Differentialgleichung}%
\index{Differentialgleichung!Airy-}
\begin{figure}
\centering
\includegraphics{chapters/images/airy-1.pdf}
\caption{Kurven zur Berechnung der Airy-Funktionen
\label{komplex:airy}}
\end{figure}
Als Beispiel soll die Airy-Differentialgleichung
\[
y''(x)-xy(x)=0
\]
gel"ost werden.
Wir nehmen an, dass die Differentialgleichung mit der Laplace-Transformation
\index{Laplace-Transformation}%
gel"ost werden kann.
Allerdings verwenden wir nicht die Integration entlang der positiven
reellen Achse, wie das bei der traditionellen Laplace-Transformation
"ublich ist, sondern Integration eintlang eines beliebigen Weges in
der komplexen Ebene.
Wir nehmen also an, dass sich $y(x)$ schreiben l"asst als
\[
y(x)=\int_\gamma e^{xz}v(z)\,dz
\]
f"ur eine komplex differenzierbare Funktion $v(z)$ und eine Kurve
$\gamma\colon\mathbb R\to \mathbb C$.
Setzen wir diesen Ansatz in die Airy-Differentialgleichung ein,
erhalten wir
\begin{align*}
\int_\gamma z^2e^{xz}v(z)\,dz-\int_\gamma xe^{xz}v(z)\,dz&=0
\end{align*}
Das zweite Integral kann mit partieller Integration umgeformt werden,
man erh"alt
\begin{align*}
\int_\gamma xe^{xz}v(z)\,dz
&=
\int_{-\infty}^{\infty} xe^{x\gamma(t)}v(\gamma(t))\dot\gamma(t)\,dt
=
\left[e^{xz}v(z)\right]_{\gamma(-\infty)}^{\gamma(\infty)}
-\int_\gamma e^{xz}v'(z)\,dz
\end{align*}
Dieser Ausdruck ist nur dann sinnvoll, wenn der Weg $\gamma$ so gew"ahlt
wird, dass an den Endpunkte des Weges der Integrand beliebig klein wird,
dies wird sp"ater die Wahl des Weges einschr"anken.
Andererseits erhalten wir die Differentialgleichung erster Ordnung
\begin{align*}
\int_\gamma e^{xz}(z^2v(z)+v'(z))\,dz=0
\qquad
\Rightarrow
\qquad
v'(z)+z^2v(z)=0
\end{align*}
f"ur $v(z)$, die mit Separation gel"ost werden kann:
\[
\int \frac{dv}{v}=-\int z^2\,dz
\qquad\Rightarrow\qquad
v(z)=-\frac13z^3.
\]
Setzen wir dies in den Ansatz f"ur $y(x)$ ein, finden wir
\[
y(x)=\int_\gamma e^{xz-\frac13z^3}\,dz.
\]
Jetzt muss die Kurve $\gamma$ so gew"ahlt werden, dass das Integral
wohldefiniert ist. 
Dazu muss der Realteil des Exponenten f"ur jedes beliebige reelle $x$ 
gegen $-\infty$ gehen, wenn $z$ gegen unendlich geht.
In Abbildung~\ref{komplex:airy} sind die Gebiete grau eingezeichnet,
in denen $-\frac13z^3$ negativen Realteil hat.
Zul"assige Wege m"ussen daher ``Enden'' in den grauen Gebieten haben,
d.~h.~$\gamma(t)$ muss in diesen Gebieten gegen Unendlich gehen, wenn
$t\to\pm\infty$.
Damit sind im Wesentlichen die drei Kurven $\gamma_0$, $\gamma_+$ und
$\gamma_-$ aus Abbildung~\ref{komplex:airy} w"ahlbar.

F"ur $x\to\infty$ w"achst $e^{xz}$ im positiven grauen Teilgebiet
"uber alle Grenzen, die Kurven $\gamma_+$ und $\gamma_-$ ergeben daher
eine unbeschr"ankte L"osung.
Nur die Kurve $\gamma_0$ kann ein beschr"ankte L"osung der Airy-Gleichung
geben.
Man nennt
\[
\operatorname{Ai}(x)
=
\frac{1}{2\pi i}\int_{\gamma_0} e^{xz-\frac13z^3}\,dz
\]
die Airy-Funktion.
\index{Airy-Funktion}%
\index{Funktion!Airy-}%
\index{Ai}%
Auf die Wahl der Kurve $\gamma_0$ kommt es nicht an, solange sie in den
beiden grauen Gebieten links der imagin"aren Achse endet.
Deformiert man die Kurve $\gamma_0$ in die imagin"are Achse, erh"alt
man daher die folgende Integraldarstellung der Airy-Funktion:
\begin{align*}
\operatorname{Ai}(x)
&=
\frac1{2\pi i}\int_{-\infty}^{\infty} e^{xit-\frac13(it)^3}i\,dt
\\
&=\frac{1}{2\pi i}\int_{-\infty}^{\infty} ie^{i(xt+\frac13t^3)}\,dt
\\
&=
\frac{1}{2\pi i}\int_{-\infty}^{\infty}
i\cos\biggl(xt+\frac13t^3\biggr)-\sin\biggl(xt+\frac13t^3\biggr)\,dt
\\
&=\frac{1}{\pi}\int_0^{\infty}\cos\biggl(xt+\frac13t^3\biggr)\,dt.
\end{align*}
Dabei haben wir im letzten Schritt verwendet, dass das Integral der
ungeraden Funktion $\sin(xt+\frac13t^3)$ "uber ein symmetrisches
Interval verschwindet.

Weitere Information "uber die Airy-Funktionen sind in \cite{skript:airy}
zusammengefasst.

\subsection{Die Cauchy-Integralformel}
\index{Cauchy-Integralformel}%
Sei jetzt $f(z)$ eine komplex differenzierbare Funktion.
Dann ist auch die Funktion
\[
g(z)=\frac{f(z)}{z-a}
\]
komplex differenzierbar f"ur $z\ne a$.
Insbesondere ist der Wert des Wegintegrals von $g(z)$ entlang
eines geschlossenen Pfades um den Punkt $a$ unabh"angig von der Wahl
des Weges.
Zum Beispiel k"onnten wir das Wegintegral mit Hilfe eines kleinen Kreises
um $a$ mit Radius $r$ mit der Parametrisierung
\[
t\mapsto \gamma(t)=a+re^{it},\quad t\in[0,2\pi]
\]
berechnen.
Die Rechnung ergibt
\begin{align*}
\oint_\gamma \frac{f(z)}{z-a}\,dz
&=
\int_0^{2\pi} \frac{f(a+re^{it})}{re^{it}}ire^{it}\,dt
=
i\int_0^{2\pi} f(a+re^{it})\,dt
\end{align*}
Da $f(z)$ komplex differenzierbar ist, k"onnen wir $f(z)$ approximieren
durch $f(z)=f(a)+f'(a)(z-a)+o(z-a)$, also
\begin{align*}
\oint_{\gamma} \frac{f(z)}{z-a}\,dz
&=
i\int_0^{2\pi}f(a) + f'(a)re^{it}+o(r)\,dt
\\
&=
f(a)i\int_0^{2\pi}\,dt
+ irf'(a)\int_0^{2\pi} e^{it}\,dt + i\int_0^{2\pi}o(r)\,dt
\\
&=
2\pi i f(a) + irf'(a)\underbrace{\left[\frac1{i}e^{it}\right]_0^{2\pi}}_{=0}+o(r)
\\
&=2\pi i f(a)+o(r).
\end{align*}
Da das Wegintegral einer komplex differenzierbaren Funktion aber unabh"angig
vom Weg und damit vom Radius $r$ sein muss, folgt
\[
\oint_\gamma \frac{f(z)}{z-a}\,dz=2\pi i f(a).
\]
Wir haben damit den folgenden Satz bewiesen:

\begin{satz}[Cauchy]
Ist $\gamma$ ein geschlossener Weg in der komplexen Ebene, die ein
Gebiet umrandet, in dem die komplexe Funktion $f(z)$ komplex
differenzierbar ist, dann gilt
\[
f(a)=\frac{1}{2\pi i}\oint_{\gamma}\frac{f(z)}{z-a}\,dz.
\]
Insbesondere sind die Werte einer komplex differenzierbaren Funktion 
im Inneren eines Gebietes durch die Werte auf dem Rand bereits vollst"andig
bestimmt.
\end{satz}

\subsubsection{Ableitungen und Cauchy-Formel}
Sei $f(z)$ eine komplex differenzierbare Funktion, als Definitionsgebiet
nehmen wir der Einfachheit halber einen Kreis vom Radius $r$ um den Nullpunkt,
sein Rand ist die Kurve $\gamma$.
Durch Ableiten der Cachyschen Integralformel finden wir
\begin{align*}
f(z)
&=
\frac1{2\pi i}\oint_{\gamma}\frac{f(\zeta)}{\zeta-z}\,d\zeta
\\
f'(z)
&=
\frac1{2\pi i}\oint_{\gamma}\frac{f(\zeta)}{(\zeta-z)^2}\,d\zeta
\\
f'' (z)
&=
\frac1{2\pi i}\oint_{\gamma}2\frac{f(\zeta)}{(\zeta-z)^3}\,d\zeta
\\
f'''(z)
&=
\frac1{2\pi i}\oint_{\gamma}2\cdot 3\frac{f(\zeta)}{(\zeta-z)^4}\,d\zeta
\\
&\vdots
\\
f^{(k)}(z)
&=
\frac{k!}{2\pi i}\oint_{\gamma}\frac{f(\zeta)}{(\zeta-z)^{k+1}}\,d\zeta
\end{align*}
Es folgt

\begin{satz}
Eine komplex differenzierbare Funktion ist beliebig oft differenzierbar.
\end{satz}

\subsubsection{Komplex differenzierbare Funktionen sind analytisch}
Wir haben fr"uher gesehen, dass Wegintegrale auf analytische Funktionen
f"uhren.
Andererseits zeigt das Cauchy-Integral, dass komplex differenzierbare
Funktionen durch genau die Integrale bestimmt sind, die in den
Reihenentwicklungen in Satz~\ref{komplex:integralanalytisch} auftraten.
Diese Resultate k"onnen wir im folgenden Satz zusammenfassen.

\begin{satz}
Eine komplex differenzierbare Funktion $f(z)$, die in einer Kreisscheibe
vom Radius $r$ um den Punkt $z_0$ definiert ist, ist analytisch.
Ihre Potenzreihenentwicklung
\[
f(z)=\sum_{k=0}^na_k(z-z_0)^k
\]
hat die Koeffizienten
\[
a_k=\frac1{2\pi i}\int_{\gamma}\frac{f(z)}{(z-z_0)^{k+1}}\,dz,\quad
k\ge 0
\]
\end{satz}

\begin{proof}[Beweis]
Da $f$ komplex differenzierbar ist, gilt
\[
f(z)=\frac1{2\pi i}\oint_\gamma \frac{f(\zeta)}{\zeta-z}\,d\zeta.
\]
In Satz~\ref{komplex:integralanalytisch} wurde gezeigt, dass $f(z)$
analytisch ist, und dass die Koeffizienten der Potenzreihe von
der verlangten Form sind.
\end{proof}

F"ur eine komplexe Funktion, die im Punkt $z_0$ eine Singularit"at hat,
also in einer Umgebung von $z_0$ ohne den Punkt $z_0$ definiert ist,
k"onnen wir das Resultat aus Satz~\ref{komplex:laurentreihe} verwenden,
und zum folgenden analogen Resultat gelangen:

\begin{satz}
Eine komplex differenzierbare Funktion $f(z)$, die in einer Kreisscheibe
vom Radius $r$ um den Punkt $z_0$ mit Ausnahme des Punktes $z_0$
definiert ist, kann in eine konvergente Laurent-Reihe
\[
f(z)=\sum_{k=-\infty}^{\infty} c_k(z-z_0)^k
\]
entwickelt werden, deren Koeffizienten durch
\[
c_k = \frac1{2\pi i}\oint_\gamma \frac{f(\zeta)}{(z-z_0)^{k+1}}\,d\zeta,\qquad k\in\mathbb Z
\]
gegeben sind.
\end{satz}

%
% Analytische Fortsetzung
%
\section{Analytische Fortsetzung}
\rhead{Analytische Fortsetzung}
\label{sec:fortsetzung}
Wir haben schon gesehen, dass eine reelle Funktion, die in einem
Punkte eine konvergente
Potenzreihe besitzt, auf nat"urliche Weise auch als komplexe Funktion
betrachtet werden kann, indem man komplexe Argumente in der Potenzreihe
zul"asst.
Die neue komplexe Funktion ist ein einem Kreis um den Punkt
konvergent.
Mit Hilfe der Potenzreihe kann man also immer eine Funktion auf ein
Kreisgebiet ausdehen.
Dieser Abschnitt untersucht die Frage, ob man diese Idee auch auf 
noch gr"ossere Gebiete ausdehnen kann.
\subsection{Analytische Fortsetzung mit Potenzreihen}
\begin{figure}
\centering
\includegraphics{chapters/images/komplex-1.pdf}
\caption{Analytische Fortsetzung einer komplexen Funktion entlang einer
Kurve $\gamma$.
\label{komplex:fortsetzung}}
\end{figure}
Eine komplex differenzierbare Funktion $f(z)$ ist immer darstellbar als
Potenzreihe, und ist daher analytisch.
So kann zum Beispiel die Funktion $1/z$ als Potenzreihe um jeden 
beliebigen Punkt $z_0$ entwickelt werden:
\begin{align}
f(z)
&=
\frac1z
=
\frac1{z_0-(z_0-z)}
=
\frac1{z_0}\cdot
\frac1{1-\displaystyle\frac{z_0-z}{z_0}}
=
\frac1{z_0}\sum_{k=0}^{\infty} \biggl(\frac{z_0-z}{z_0}\biggr)^k
=
\sum_{k=0}^{\infty} \frac{(-1)^k}{z_0^{k+1}} (z-z_0)^k,
\label{komplex:1durchreihe}
\end{align}
Die Koeffizienten dieser Potenzreihe sind
\[
a_k=\frac{(-1)^k}{z_0^{k+1}},
\]
und man kann den Konvergenzradius ausrechnen:
\[
\frac1{\varrho}
=
\limsup_{k\to\infty} \root{k}\of{|a_k|} = \lim_{k\to\infty}\frac1{|z_0|^{\frac{k+1}{k}}}
=
\frac1{|z_0|}.
\]
Der Konvergenzradius ist limitiert durch die Singularit"at bei an der Stelle
$z=0$.

Es gibt also keine einzelne Potenzreihe, die die Funktion $f(z)=\frac1z$ in der
ganzen komplexen Ebene darstellen kann.
W"ahlt man aber einzelne Punkte $z_0$ und $z_1$ derart, dass der Kreis
um $z_0$ mit Radius $|z_0|$ und der Kreis um $z_1$ mit Radius $|z_1|$
"uberlappen, dann werden die beiden Potenzreihen im "Uberlappungsgebiet
die gleichen Werte annehmen.

Man k"onnte allso eine Kurve $\gamma$ in der komplexen Ebene w"ahlen,
entlang der man in jedem Punkt die Funktion $f(z)$ in eine Potenzreihe 
entwickelt.
Liegen zwei Punkte nahe genug auf der Kurve $\gamma$, werden die
Konvergenzkreise der Potenzreihen "uberlappen, und die Potenzreihen
werden im "Uberlappungsgebiet die gleichen Werte liefern.

Selbst wenn man eine Funktion $f(z)$ nur in einem Kreis um den Punkt $z_0$
kennt, zum Beispiel durch eine Potenzreihe im Punkt $z_0$, kann man entlang
einer Kurve, die $z_0$ mit $z_1$ verbindet, in jedem Punkt eine Potenzreihe
finden, die mit der Potenzreihe in den Nachbarpunkten "ubereinstimmt, und
so die Definition der Funktion entlang dieser Kurve auf ein gr"osseres
Gebiet ausweiten, wie in Abbildung~\ref{komplex:fortsetzung} dargestellt.
Man nennt dies die {\em analytische Fortsetzung} der Funktion $f(z)$ 
entlange der Kurve $\gamma$.
\index{analytische Fortsetzung}
\index{Fortsetzung, analytische}

\begin{beispiel}
Wir haben bereits gesehen, dass sich die Funktion $f(z)=1/z$ in jedem
Punkt $z_0$ der komplexen Ebene in die Potenzreihe~\eqref{komplex:1durchreihe}
entwickeln l"asst.
Diese Reihe l"asst sich integrieren
\[
F(z,z_0)
=
\sum_{k=0}^\infty\frac{(-1)^k}{(k+1)z_0^{k+1}}z^{k+1},
\]
diese Reihe ist ebenfalls auf einem Kreis vom Radius $|z_0|$ um den
Punkt $z_0$ konvergent.
Wir vermuten nat"urlich, dass dies eine Darstellung des nat"urlichen
Logarithmus einer komplexen Zahl ist.
Nat"urlich ist das immer nur auf einem Kreisgebiet m"oglich, die Reihe
f"ur $z=1$ ist zum Beispiel im Punkt $z=-1$ nicht konvergent.

Um eine in der ganzen komplexen Ebene definierte Funktion $\log(z)$ zu
konstruieren, m"ussen wir also eine analytische Fortsetzung aufbauen.
Bei der Integration haben wir eine frei w"ahlbare Integrationskonstante
$C(z_0)$, die wir so w"ahlen m"ussen, dass die Reihen im "Uberlappungsgebiet
"ubereinstimmen:
\[
F(z,z_0) + C(z_0) = F(z,z_1)  + C(z_1)
\]
f"ur jedes $z$ im "Uberlappungsgebiet.
Dadurch wird aber nur die Differenz $C(z_1)-C(z_0)$ der Werte festgelegt.
Da wir "Ubereinstimmung mit der "ublichen Definition des Logarithmus
erreichen m"ochten, k"onnen wir $C(1)=0$ festlegen.

\begin{figure}
\centering
\includegraphics{chapters/images/komplex-2.pdf}
\caption{Analytische Fortsetzung f"ur die Funktion $\frac1z$ 
entlang der Pfade $\gamma_+$ und $\gamma_-$
\label{komplex:logfortsetzung}}
\end{figure}
Wir konstruieren jetzt die analytische Forstsetzung entlang der Kurven
$\gamma_+$ und $\gamma_-$ wie in Abbildung~\ref{komplex:logfortsetzung}
dargestellt.
Um die Differenz $C(z_1)-C(z_0)$ zu bestimmen, Werten wir die Funktionen
$F(z,z_0)$ und $F(z,z_1)$ jeweils im rot eingezeichneten Punkt aus.
Die exakte Berechnung ist etwas m"uhsam, da es sich ja nur um ein Beispiel
handelt, k"onnen wir die Reihen auch numerisch ausrechnen, und so die
Differenzen bestimmen:
\begin{align*}
&\text{Startpunkt $z_0=1$:}& C(1)&=0             &       &       \\
&\text{entlang $\gamma_+$:}& C(i)&= i\frac{\pi}2 & C(-1) &=  i\pi\\
&\text{entlang $\gamma_-$:}&C(-i)&=-i\frac{\pi}2 & C(-1) &= -i\pi
\end{align*}
Wir stellen fest, dass die analytische Fortsetzung der Logarthmusfunktion
entlang der Kurve $\gamma_+$ die Potenzreihe
\[
\log_+(z)
=
i\pi +\sum_{k=1}^\infty \frac{(-1)^{k+1}}{k(-1)^k}(z+1)^k
=
i\pi
-
\sum_{k=1}^\infty \frac{(z+1)^k}{k}
\]
ergibt, w"ahrend man entlang der  Kurve $\gamma_-$
\[
\log_-(z)
=
-i\pi +\sum_{k=1}^\infty \frac{(-1)^{k+1}}{k(-1)^k}(z+1)^k
=
-i\pi
-
\sum_{k=1}^\infty \frac{(z+1)^k}{k}
\]
findet.
Die beiden analytischen Fortsetzungen entlang der Kurven $\gamma_+$ und
$\gamma_-$ stimmen auf der negativen reellen Achse nicht "uberein,
sie unterscheiden sich um $2\pi i$:
\[
\log_+(z)-\log_-(z)=2\pi i.
\]
\end{beispiel}

Das Beispiel zeigt, dass es im Allgmeinen eine auf der ganzen komplexen
Ebene definierte komplexe Entsprechung einer reellen Funktion nicht
zu geben braucht.
Dieses Ph"anomen tritt zum Beispiel auch bei der Wurzelfunktion $f(z)=\sqrt{z}$
auf.
Diese Funktion ist im Punkt $z=0$ nicht differenzierbar, man muss diesen
Punkt also aus dem Definitionsbereich ausschliessen.
F"uhrt man man analog zum Beispiel eine analytische Fortsetzung durch,
findet man, dass sich die Werte von $f(z)$ f"ur die beiden Wege $\gamma_+$
und $\gamma_-$ durch das Vorzeichen unterscheiden.

\subsection{Analytische Fortsetzung mit Differentialgleichungen
\label{komplex:analytische-fortsetzung-dgl}}
In Abschnitt~\ref{subsection:wegintegrale} wurde gezeigt, wie Wegintegrale
Stammfunktionen komplexer Funktionen liefern k"onnen.
Im vorangegangenen Abschnitt wurde untersucht, wie eine komplex differenzierbare
Funktion mit Hilfe von analytischer Fortsetzung entlang einer Kurve
ausgedehnt werden kann.

Sei $f(z)$ eine komplex differenzierbare Funktion.
In jedem beliebigen Punkt des Definitionsbereichs k"onnen wir $f(z)$
in eine Potenzreihe entwickeln, und nat"urlich auch termweise integrieren.
Es gibt also in jedem Punkt $z_0$ des Definitionsbereichs eine
Funktion $F_{z_0}(z)$, die $F'_{z_0}(z)=f(z)$ erf"ullt.
Durch analytische Fortsetzung entlang einer Kurve $\gamma$ k"onnen
wir eine komplex differenzierbare Funktion $f(z)$ finden, die in einer
Umgebung der Kurve $F'(z)=f(z)$ erf"ullt.

Sei andererseits $\gamma\colon[a,b]\to\mathbb C$ eine Kurve in $\mathbb C$.
Dann k"onnen wir die Werte der Stammfunktion im Punkt $\gamma(b)$ durch
\[
F(\gamma(b)) = F(\gamma(a))+\int_\gamma f(z)\,dz
\]
berechnen.

\begin{beispiel}
\begin{figure}
\centering
\includegraphics{chapters/images/komplex-5.pdf}
\caption{Analytische Fortsetzung des Logarithmus als L"osung der
Differentialgleichung $y'=\frac1z$.
Bei einem Umlauf um den Nullpunkt nimmt der Wert von $y(z)$ um
$2\pi i$ zu.
\label{komplex:analytische-fortsetzung-log}
}
\end{figure}
Wir bestimmen die Stammfunktion von $f(z)=1/z$.
Entlang der reellen Achse weiss man bereits, dass die Stammfunktion
der nat"urliche Logarithmus ist, also $F(x)=\log x$.
Um diese Stammfunktion auf $\mathbb C$ auszudehnen, verwenden wir einen
kreisf"ormigen Pfad von der reellen Achse bis zum Punkt $z$.
Liegt $z$ in der oberen Halbebene, w"ahlen wir einen Pfad in der
oberen Halbebene, und umgekehrt.
Wir k"onnen die Zahl $z$ in Polarkoordinaten darstellen als $z=re^{i\varphi}$.
Ein Pfad von der reellen Achse kann mit
\[
\gamma\colon [0,1]\to\mathbb C: t\mapsto re^{it\varphi}
\]
parametrisiert werden.
Der Zuwachs der Stammfunktion entlang dieses Pfades ist
\[
F(z)-F(r)
=
\int_\gamma\frac1z\,dz
=
\int_0^1 \frac1{e^{it\varphi}}i\varphi e^{it\varphi}\,dt
=
i\varphi \int_0^1\,dt
=
i\varphi.
\]
Der Wert der Stammfunktion am Anfang der Kurve ist $\log r$, somit
folgt, dass
\[
\log z = \log r + i\varphi
\]
(Abbildung~\ref{komplex:analytische-fortsetzung-log}).
\end{beispiel}

\section{"Ubungsaufgaben}
\rhead{"Ubungsaufgaben}
\uebungsaufgabe{701}
\uebungsaufgabe{702}
\uebungsaufgabe{703}
\uebungsaufgabe{704}
\uebungsaufgabe{705}
\uebungsaufgabe{706}


%
% stabilitaet.tex -- Stabilität der Lösungen von Differentialgleichungen
%
% (c) 2015 Prof Dr Andreas Mueller, Hochschule Rapperswil
%
\chapter{Stabilit"at\label{chapter:stabilitaet}}
\lhead{}
\rhead{Stabilit"at}


%
% chaos.tex -- Grundlagen des "Ubergangs zum Chaos
%
% (c) 2015 Prof Dr Andreas Mueller, Hochschule Rapperswil
%
\chapter{Chaos\label{chapter:chaos}}
\lhead{}
\rhead{Chaos}


%
% stochastisch.tex -- Kapitel ueber stochastische Differentialgleichungen
%
\chapter{Stochastische Differentialgleichungen\label{chapter:stochastisch}}
\lhead{Stochastische Differentialgleichungen}
\rhead{}
In vielen Anwendungen wird die Bewegung eines Systems auch von
zuf"alligen Einfl"ussen bestimmt, die man oft auch Rauschen nennt.
Die Natur des Rauschen bedeutet, dass aufeinanderfolgende inkremente
v"ollig unkorreliert sind, w"ahrend Inkremente einer differenzierbaren
Funktion voneinander abh"angig sind.
Die L"osung einer Differentialgleichung unter Einfluss von Rauschen 
kann daher niemals eine differenzierbare Funktion sein, und sie kann
niemals eine L"osung der Differentialgleichung im bisher verwendeten
Sinn sein.
Um der Idee einen mathematischen Sinn zu geben, der auch erlaubt,
solche Differentialgleichungen zu l"osen und in Anwendungen
einzusetzen, muss daher zuerst gekl"art werden, was Rauschen genau ist.
Anschliessend muss das Konzept einer Differentialgleichung so formuliert
werden, dass es auch f"ur nicht differenzierbare Funktionen und Rauschen
anwendbar ist.

Die Darstellung in diesem Kapitel orientiert sich in vielen Punkten
an dem hervorragenden und leicht lesbaren Buch \cite{skript:evans}.
Eine mathematisch vertieftere Entwicklung ist in \cite{skript:oksendal}
zu finden.

%
% Ein Modell f"ur Rauschen
%
\section{Modell f"ur Rauschen: der Wiener-Prozess\label{section:wiener}}
\rhead{Wiener-Prozess}
Rauschen ist ein Zufallsph"anomen, die Wiederholung eines Experimentes
wird im Allgemeinen einen anderen Verlauf ergeben.
Der Pfad eines Teilchens $W(t)$ in Abh"angikeit ist daher ein Zufallsresultat.
Wir brauchen daher einen Wahrscheinlichkeitsraum $\Omega$ und ein
Wahrscheinlichkeitsmass $P$, und die Wege $W$ sind abh"angig von 
der Durchf"uhrung $\omega\in\Omega$ des Experiments. 
Genau genommen m"ussen wir also sagen, dass f"ur jedes $\omega\in\Omega$
der Weg $W(\omega)$ eine Funktion
\[
W(\omega)\colon\mathbb R \to\mathbb R:t\mapsto W(\omega)(t)
\]
ist.
Wir nennen eine solche Funktion einen {\em stochastischen Prozess}.
\index{stochastischer Prozess}
Im Folgenden werden wir die etwas schwerf"allige Notation etwas
vereinfachen, und das $\omega$ weglassen.

Wir m"ochten die Position eines Teilchens berechnen, dessen Geschwindigkeit
ein solches ``Rauschen'' ist.
Diese Position $W(t)$ ist ein stochastischer Prozess im eben erkl"arten Sinne.
Die Brownsche Bewegung ist ein solcher Prozess, die Position $W(t)$ eines
Teilchens unter dem Einfluss der thermischen Bewegung der Teilchen
in einer Fl"ussigkeit als Funktion der Zeit wird eine nicht
differenzierbare Funktion sein.
Das beste, was wir erwarten k"onnen, ist dass die Positionsunterschiede
\[
W(t+\Delta t)-W(t),
\quad
W(t + 2\Delta t)-W(t+\Delta t),
\quad
W(t + 3\Delta t)-W(t+2\Delta t),\quad\dots
\]
voneinander unabh"angig sind, und nicht beliebig gross sind.
Wir erwarten, dass diese Differenzen, die sich aus vielen kleinen
St"ossen zusammensetzen, normalverteilt sind.
Wir definieren daher

\begin{definition}
Ein stochastischer Prozess $W(t)$ heisst {\em Brownsche Bewegung} oder
{\em Wiener Prozess}, wenn gilt
\begin{compactenum}
\item $W(0)=0$
\item $W(t)-W(s)$ ist normalverteilt mit Erwartungswert $0$ und
Varianz $t-s$, f"ur beliebige $t\ge s\ge 0$.
\item F"ur beliebige Werte $t_i$ mit $0<t_1<t_2<\dots<t_n$, dann sind
die Zufallsvariablen
$W(t_1), W(t_2)-W(t_1),\dots,W(t_n)-W(t_{n-1})$ unabh"angig.
\end{compactenum}
\index{Brownsche Bewegung}
\index{Wiener Prozess}
\end{definition}

A priori ist nicht klar, dass es so einen Prozess "uberhaupt gibt, wir
m"ussen daher zeigen, dass sich eine solche Funktion konstruieren l"asst.
Eine solche Funktion ist aber sicher nicht differenzierbar, denn
der Differenzenquotient "uber ein Interval der L"ange $2\Delta t$
\begin{align*}
\frac{W(t+2\Delta)-W(t)}{2\Delta t}
&=
\frac{W(t+2\Delta t)-W(t+\Delta t)}{2\Delta t}
+
\frac{W(t+\Delta t)-W(t)}{2\Delta t}
\\
&=
\frac12\biggl(
\frac{W(t+2\Delta t)-W(t+\Delta t)}{\Delta t}
+
\frac{W(t+\Delta t)-W(t)}{\Delta t}
\biggr)
\end{align*}
ist Mittelwert aus zwei Differenzenquotienten "uber k"urzere Intervalle,
aber diese beiden Differenzenquotienten sind voneinander unabh"angig.
Ein Grenzwert des Differenzenquotienten kann daher nicht existieren.

\subsection{Eigenschaften des Wiener-Prozesses}
Wir brauchen Rechenregeln, wie man mit Wiener-Prozessen Funktionen
rechnen kann.
Zum Beispiel ist $W(t)$ wegen Eigenschaft~2 normalverteilt mit Erwartungswert
$0$ und Varianz $t$, also gilt
\[
E(W(t))=0,\qquad E(W(t)^2)=t,\qquad \forall\;t>0.
\]
Etwas weniger offensichtlich ist
\begin{hilfssatz}
Wenn $W(t)$ eine Brownsche Bewegung ist, dann ist
\[
E(W(t)W(s)) = t\wedge s = \min\{s,t\}
\]
f"ur beliebige $t,s\ge 0$.
\end{hilfssatz}

\begin{proof}[Beweis]
Nehmen wir an, dass $t\ge s$, dass also $t\wedge s = s$.
Dann k"onnen wir $E(W(t)W(s))$ berechnen
\begin{align*}
E(W(t)W(s))
&=
E((W(t)-W(s)+W(s))W(s))
=
E((W(t)-W(s))W(s))+E(W(s)^2)
\end{align*}
Eigenschaften~1 und~2 zeigen, dass $E(W(s)^2)=s$ ist.
Eigenschaft~3 besagt, dass $W(t)-W(s)$ und $W(s)$ unabh"angig sind,
der Erwartungswert ihres Produktes ist daher das Produkt der Erwartungswerte:
\begin{align*}
E(W(t)W(s))
&=
E((W(t)-W(s))W(s))+E(W(s)^2)
=
\underbrace{E(W(t)-W(s))}_{\textstyle =0} \underbrace{E(W(s))}_{\textstyle =0} + s
\end{align*}
wobei wir erneut Eigenschaft~2 verwendet haben.
\end{proof}
Man k"onnte diese Eigenschaft umschreiben als die Beobachtung,
dass die weitere Entwicklung von $W(t)$ nach der Zeit $s$ bedeutungslos ist.

\begin{figure}
\centering
\includegraphics{chapters/images/stochastisch-1.pdf}
\caption{Wiener-Prozess $W(t)$ in Abh"angigkeit von der Zeit
\label{stochastisch:wiener}}
\end{figure}

\subsection{Konstruktion des Wiener-Prozesses}
Wir m"ussen eine Konstruktion angeben, mit der wir zu einem gegebenen
Interval $[0,T]$ eine Funktion $W(t)$ konstruieren k"onnen, die
die Eigenschaften des Wiener-Prozesses erf"ullt.

Zun"achst verlangen die Eigenschaften~1 und 2 des Wiener-Prozesses,
dass $X(T)$ eine normalverteilte Zufallsvariable ist mit Erwartungswert
$X(0)=0$ und Varianz $T$.
Da $X(t)$ ausserdem stetig sein soll, verwenden wir als erste Iteration
die lineare Funktion:
\[
W_1(t) = X(T)\frac{t}{T}
\]
Die Eigenschaft~3 verlangt, dass auch $X(T/2)$ normalverteilt ist mit
Erwartungswert $0$ und Varianz $T/2$.
Dies kann dadurch erreicht werden, dass wir $W_1$ durch einen 
Polygonzug $W_2$ ersetzen, der bei $t=T/2$ einen zus"atzlichen
Eckpunkt besitzt.
Wir bezeichnen den Unterschied zwischen $W_2$ und $W_1$ an der
Stelle $t=T/2$ mit
\[
Y=W_2(T/2)-W_1(T/2)=W_2(T/2) - W_1(T)/2.
\]
$Y$ muss so gew"ahlt werden, dass $W_2(T/2)$ eine normalverteilte
Zufallsvariable mit Erwartungswert $0$ und Varianz $T/2$ wird,
somit ist $Y$ auch normalverteilt.
Der Erwartungswert von $Y$ ist
\begin{align*}
E(Y)&=E(W_2(T/2) - W_1(T)/2)=E(W_2(T/2))-E(W_1(T))/2=0,
\end{align*}
es ist also nur noch die Varianz $\sigma^2_Y$ w"ahlbar.
Sie muss so gew"ahlt werden, dass $W_2(T/2)$ Varianz $T/2$
bekommt:
\begin{align*}
\operatorname{var}\biggl(W_2\biggl(\frac{T}2\biggr))\biggr))
&=
\operatorname{var}\biggl(W_1\biggl(\frac{T}2\biggr) + Y\biggr)
=
\frac{\operatorname{var}(W_1)}4 + \operatorname{var}Y
=
\frac{T}4 +\sigma_Y^2
\end{align*}
Es folgt $\sigma_Y^2=\frac{T}4$.

Wir m"ussen kontrollieren, ob die Inkremente unabh"angig sind.
Dazu berechnen wir der Erwartungswert des Produktes der
beiden Inkremente
\begin{align*}
W_2(T)-W_2\biggl(\frac{T}2\biggr)
&=
W_1(T)-\biggl(\frac{W_1(T)}2 + Y\biggr)
=
Y-\frac{W_1(T)}2
\\
W_2\biggl(\frac{T}2\biggr)-W_2(0)
&=
Y+
\frac{W_1(T)}2
\\
\biggl(W_2(T)-W_2\biggl(\frac{T}2\biggr)\biggr)
\biggl(W_2\biggl(\frac{T}2\biggr)-W_2(0)\biggr)
&=
\biggl(Y-\frac{W_1(T)}2\biggr)\biggl(Y+\frac{W_1(T)}2\biggr)
=
Y^2-\frac{W_1(T)^2}4
\\
E\biggl(
\biggl(W_2(T)-W_2\biggl(\frac{T}2\biggr)\biggr)
\biggl(W_2\biggl(\frac{T}2\biggr)-W_2(0)\biggr)
\biggr)
&=
E(Y^2)-E\biggl(\frac{W_1(T)^2}4\biggr)
=
\sigma_Y^2 - \frac14\operatorname{var}(W_1(T))
=
0
\end{align*}
Somit sind die Inkremente unabh"angig.

\begin{figure}
\centering
\includegraphics{chapters/images/stochastisch-3.pdf}\\
\includegraphics{chapters/images/stochastisch-4.pdf}\\
\includegraphics{chapters/images/stochastisch-5.pdf}\\
\includegraphics{chapters/images/stochastisch-6.pdf}\\
\includegraphics{chapters/images/stochastisch-7.pdf}
\caption{Schrittweise Konstruktion des Wiener-Prozesses als Grenzewert
der Folge $W_n(t)$, $n=1,\dots,5$
\label{stochastisch:folge1}}
\end{figure}
\begin{figure}
\centering
\includegraphics{chapters/images/stochastisch-8.pdf}\\
\includegraphics{chapters/images/stochastisch-9.pdf}\\
\includegraphics{chapters/images/stochastisch-10.pdf}\\
\includegraphics{chapters/images/stochastisch-11.pdf}\\
\includegraphics{chapters/images/stochastisch-12.pdf}
\caption{Schrittweise Konstruktion des Wiener-Prozesses als Grenzewert
der Folge $W_n(t)$, $n=6,\dots,10$
\label{stochastisch:folge2}}
\end{figure}

Diesen Prozess k"onnen wir fortsetzen: f"ur $W_3$ nehmen wir einen
Polygonzug mit zus"atzlichen Eckpunkten an den Stellen $T/4$ und $3T/4$.
Die Differenz zwischen $W_3$ und $W_2$ an diesen Stellen m"ussen
normalverteilt sein mit Erwartungswert $0$ und Varianz $T/8$.
Auf diese Weise k"onnen wir eine Folge $W_n$ von Prozessen konstruieren,
wie in den Abbildungen~\ref{stochastisch:folge1} und \ref{stochastisch:folge2}
dargestellt.

Dies reicht aber nicht.
F"ur eine vollst"andige Konstruktion muss man noch die folgenden zwei
Dinge zeigen.
\begin{compactenum}
\item
Der Grenzwert existiert tats"achlich.
Weil die einzelnen Zufallsvariablen $Y$ normalverteilt sind, k"onnen
sie beliebig grosse Werte annehmen.
Dies bedeutet auch, dass wir im Allgemeinen nicht davon ausgehen 
k"onnen, dass die Folge $W_n(t)$ konvergiert.
Zwar sind die Werte von $W_m$ an den Stellen $t_{k,n}=kT/2^{n-1}$ f"ur
$k=0,\dots,2^{n-1}$ fest, sobald $m\ge n$.
Zwischen diesen Werten k"onnen aber immer wieder grosse Werte auftreten,
so dass die Folge $W_n$ weder gleichm"assig noch punktweise konvergieren
kann.
Die Frage der Konvergenz muss daher als Konvergenz in einer Art Mittel 
angegangen werden.
\item
Die Forderung, dass die Inkremente $W(t)-W(s)$ und $W(s)-W(r)$ f"ur jedes 
Tripel $t \ge s\ge r$ unabh"angig sein m"ussen.
Die Konstruktion stellt nur sicher, dass dies gilt, wenn die Zeitpunkte
des Tripels Eckpunkte der Polygonkonstruktion sind.
\end{compactenum}

%
%
%
\section{Stochastische Differentialgleichungen\label{section:stochdgl}}
\rhead{Differentialgleichungen}
Wir m"ochten gerne eine Differentialgleichung f"ur den Zustand
$X(t)$ eines Systems l"osen, welches von Rauschen mit beeinfluss wird.
Eine solche Differentialgleichung k"onnten wir schreiben als
\[
\frac{dX(t)}{dt}
=
b(X(t)) + B(X(t))\frac{dW(t)}{dt}
\]
wobei $b$ die Rolle der Funktion $f$ aus
Abschnitt~\ref{section:anfangswertprobleme} spielt.
Die Ableitung von $W$ spielt die Rolle des Rauschens, wir wissen aber
bereits, dass $W$ nicht differenzierbar sein kann, die Gleichung
in dieser Form kann daher gar nicht sinnvoll sein.

Formal kann man die Gleichung mit $dt$ multiplizieren, so dass man
das formale Gleichungssystem
\begin{align*}
dX(t)
&=
b(X(t)) + B(X(t),t)\,dW(t)
\\
X(0)
&=
x_0
\end{align*}
erh"alt, aber auch in diesr Form sind die Ausdr"ucke $dX$ und $dW$ nicht
ohne zus"atzliche Definition sinnvoll.
Am ehesten hat man eine Chance, dieser Gleichung einen Sinn zu geben,
wenn man integriert:
\begin{align*}
X(t)=x_0+\int_0^t b(X(s))\,ds + \int_0^t B(X(s), s)\, dW,\qquad t>0.
\end{align*}
Das erste Integral ist ein gew"ohnliches Integral, denn wir
gehen davon aus, dass die Funktion $X(t)$ stetig ist.
Wenn $B=0$ ist, dann liefert diese Formel genau L"osungen der
urspr"unglichen Differentialgleichung.
Wir brauchen aber immer noch eine Interpretation des zweiten Integrals,
diese werden wir in Abschnitt~\ref{subsection:stochint} geben.

Nehmen wir f"ur den Moment an, dass die L"osung $X(t)$ gefunden werden
kann, und sei $u$ eine beliebig oft differenzierbare Funktion.
Wir erwarten, dass die Ableitung von $Y(t)=u(X(t))$ nach der
Kettenregel
\[
dY = u'\,dX = u'b\,dt + u'\,dW
\]
sein sollte.
Tats"achlich ist das Rauschen so stark, dass die Kr"ummung der Funktion
$u$ bereits eine Rolle spielt.
Bei einer gew"ohnlichen Differentialgleichung sind auf kleine Entfernungen
die zweiten Ableitungen von $u$ vernachl"assigbar.
Die schnellen, vom Rauschen verursachten Abweichungen f"uhren sind aber nicht
klein, so dass die korrekte Kettenregel bei Anwesenheit von Rauschen die
It\^o-sche Kettenregel wird, die stattdessen den Ausdruck
\[
dY =  \biggl(u'b + \frac12u''\biggr)\,dt + u'\,dW
\]
liefert.
Der Term $\frac12u''$ ist in der klassischen Analysis nicht vorhanden.
Wir m"ussen daher auch alle gewohnten Rechenregeln der Analysis "uberpr"ufen.

\subsection{Stochastische Integrale\label{subsection:stochint}}
\index{stochastisches Integral}
Wir m"ochten eine Definition f"ur ein Integral der Form
\[
\int_0^T G\,dW
\]
f"ur zwei stochastische Prozesse $G(t)$ und $W(t)$.

\subsubsection{Das Paley-Wiener-Zygmund Integral}
\index{Paley-Wiener-Zygmund Integral}
F"ur differenzierbare Funktionen $g(t)$ und $w(t)$ ist klar, was 
mit dem Integral gemeint ist:
\[
\int_0^T g(t)\,dw = \int_0^T g(t) w'(t)\,dt.
\]
Wenn $g(0)=g(T)=0$ ist, dann kann man dies mit Hilfe partieller Integration
vereinfachen:
\[
\int_0^T g(t)\,dw
=
\int_0^T g(t) w'(t)\,dt
=
[g(t)w(t)]_0^T
-
\int_0^T g'(t) w(t)\,dt
=
-\int_0^T g'(t) w(t)\,dt
\]
Diese letzte Formel ist aber auch geeignet als Definition des Integrals
f"ur eine brownsche Bewegung $W(t)$ an Stelle von $w(t)$.

\begin{definition}
F"ur eine stetig differenzierbare Funktion $g\colon[0,T]\to \mathbb R$ 
mit $g(0)=g(T)=0$, setze
\[
\int_0^T g\,dW = -\int g'(t) W(t)\,dt.
\]
\end{definition}

\begin{hilfssatz}
Erwartungswert und Varianz der Zufallsvariablen
\[
Z=\int_0^T g\,dW
\]
ist
\begin{compactenum}
\item $E(Z)=0$
\item $\operatorname{var}(Z)=\int_0^Tg(t)^2\,dt$
\end{compactenum}
\end{hilfssatz}

\begin{proof}[Beweis]
F"ur den Erwartungswert finden wir mit Hilfe der Rechenregeln f"ur den
Erwartungswert
\begin{align*}
E(Z)
&=
E\biggl(\int_0^T g\,dW\biggr)
=
E\biggl(
\int_0^T g'(t) W(t)\,dt
\biggr)
=
\int_0^T g'(t) \underbrace{E(W(t))}_{\textstyle =0}\,dt=0
\end{align*}
Die Varianz ist daher der Erwartungswert $E(Z^2)$, die wir ebenfalls unter
Verwendung der Rechenregeln berechnen k"onnen
\begin{align*}
E(Z^2)
&=
E\biggl(\biggl(\int_0^T g\,dW\biggr)^2\biggr)
=
E\biggl(
\biggl(
\int_0^T g'(t)W(t)\,dt
\biggr)^2
\biggr)
\\
&=
E\biggl(
\int_0^T g'(t)W(t)\,dt
\int_0^T g'(s)W(s)\,ds
\biggr)
=
E\biggl(
\int_0^T \int_0^T g'(t)W(t) g'(s)W(s) \,ds \,dt
\biggr)
\\
&=
\int_0^T\int_0^Tg'(t)g'(s) \underbrace{E(W(t)W(s))}_{\textstyle =t\wedge s}\,ds\,dt
\\
&=
\int_0^Tg'(t) \biggl(\int_0^t g'(s) s\,ds + \int_t^T g'(s)t\,ds\biggr)
\,dt
\\
&=
\int_0^T g'(t)\biggl(
\underbrace{[g(s)s]_0^t}_{\textstyle =tg(t)}-\int_0^t g(s)\,ds+t(\underbrace{g(T)}_{\textstyle =0}-g(t))
\biggr)\,dt
\\
&=
\int_0^T g'(t)\biggl(-\int_0^t g(s)\,ds\biggr)\,dt
\\
&=
\biggl[
g(t)\biggl(-\int_0^t\int_0^tg(s)\,ds\biggr)
\biggr]_0^T
+
\int_0^Tg(t)^2\,dt
\end{align*}
Damit ist die Behauptung bewiesen.
\end{proof}
Der Hilfssatz kann dazu verwendet werden, die Definition des Integrals auf
weitere Funktionen auszudehnen.
Eine Folge von Funktionen $g_n$ f"uhrt auf eine Folge von Werten des
Integrals. 
Um daraus eine Erweiterung des Integrals zu konstruieren, m"ussen wir 
definieren, was es heissen soll, dass die Folge $g_n$ konvergiert.
Wir verwenden die Norm
\[
\|f\|_2^2=\int_0^T f(t)^2 \,dt,
\]
um den Abstand zwischen Funktionen zu messen.
Eine Cauchy-Folge von Funktionen in dieser Norm ist dann eine Folge so,
dass f"ur jedes $\varepsilon>0$ ein $N>0$ existiert, so dass aus
$n,m>N$ folgt
\[
\|g_n-g_m\|_2^2=\int_0^T (g_n(t)-g_m(t))^2\,dt<\varepsilon.
\]
In diesem Fall gilt, dass
\[
E\biggl(
\biggl(
\int_0^T g_n\,dW
-
\int_0^T g_m\,dW
\biggr)^2
\biggr)
=
\int_0^T(g_m(t)-g_n(t))^2\,dt<\varepsilon
\]
Die Zufallsvariablen
\[
Z_n = \int_0^T g_n\,dW
\]
bildet daher eine Cauchy-Folge von quadratintegrierbaren Funktionen
in $L^2(\Omega)$, und wir k"onnen deren Grenzwert
\[
\int_0^T g\,dW
=
\lim_{n\to\infty}\int_0^T g_n\,dW
\]
als den verallgemeinerten Wert f"ur das Integral einer beliebigen Funktion
$g\in L^2([0,T])$ definieren.

\subsubsection{Riemann-Summen}
Die Verallgemeinerung des Paley-Wiener-Zygmund-Integral auf
quadratintegrierbare Funktionen ist allerdings nicht ausreichend, um
\[
\int_0^T W\,dW
\]
zu definieren.
Wir k"onnen aber noch weiter zur"uck gehen zur Definition des
Riemann-Integrals, und sie verallgemeinern auf einen stochastischen
Prozess.

\begin{definition}
Eine {\em Unterteilung} $P$ des Intevals $[0,T]$ ist eine endliche Menge
von Teilpunkten
\[
P=\{ 0=t_0<t_1<t_2<\dots<t_m=T\}.
\]
Die {\em Maschenweite} $|P|$ der Unterteilung $P$ ist
\[
|P|=\max_{0\le k\le m-1}|t_{k+1\mathstrut}-t_{k\mathstrut}|.
\]
F"ur ein festes $0\le \lambda\le 1$ setzen wir
\[
\tau_k = (1-\lambda)t_{k\mathstrut}+\lambda t_{k+1\mathstrut}.
\]
\end{definition}
F"ur $\lambda=0$ ist $\tau_k=t_k$, f"ur $\lambda=1$ gilt $\tau_k=t_{k+1}$.
F"ur andere Werte von $\lambda$ ist $\tau_k$ ein Punkt im Inneren des
Intervals $[t_k,t_{k+1}]$.
Wie "ublich kann eine solche Unterteilung dazu verwendet werden, die 
Riemann-Summe als Approximation des Integrals zu definieren:

\begin{definition}
Die Riemann-Summe von $W$ f"ur die Unterteilung $P$ ist
\[
R=R(P,\lambda) = \sum_{k=0}^{m-1} W(\tau_k)(W(t_{k+1})-W(t_k)).
\]
\index{Riemann-Summe}
\end{definition}

\begin{hilfssatz}
\label{stochastisch:quadrvariation}
Sei
\[
P^{(n)}
=
\{0=t_0^{(n)}<t_1^{(n)}<\dots < t_{m_n}^{(n)}=T\}
\]
eine Folge von Unterteilungen des Intervals $[0,T]$ derart,
dass die Maschenweite gegen $0$ strebt, dann gilt
\[
\lim_{n\to\infty}
\sum_{k=0}^{m_n-1} \bigl(W(t_{k+1}^{(n)})-W(t_{k}^{(n)})\bigr)^2
=T
\]
in $L^2(\Omega)$.
\end{hilfssatz}

\begin{proof}[Beweis]
Konvergenz in $L^2(\Omega)$ bedeutet, dass der Erwartungswert der
quadratische Abweichung gegen $0$ streibt.
Setzen wir
\[
Q_n
= 
\sum_{k=0}^{m_n-1} \left(W(t_{k+1}^{(n)})-W(t_{k}^{(n)})\right)^2,
\]
und m"ussen untersuchen, ob $E((Q_n - T)^2)$ gegen $0$ strebt.
Wir beginnen mit der Bemerkung, dass
\begin{align*}
Q_n-T
&=
\sum_{k=0}^{m_n-1}
\left((W(t_{k+1}^{(n)})-W(t_{k}^{(n)}))^2 - (t_{k+1}^{(n)}-t_k^{(n)})\right).
\end{align*}
Die einzelnen Differenzen k"urzen wir ab als
\begin{align*}
Y_k
&=
\frac{W(t_{k+1}^{(n)})-W(t_k^{(n)})}{\sqrt{t_{k+1}^{(n)}-t_{k}^{(n)}}},
\\
\Delta_k
&=
t_{k+1}^{(n)}-t_k^{(n)}.
\end{align*}
F"ur $j\ne k$ sind $Y_j^{(n)}$ und $Y_{k}^{(n)}$ unabh"angig.
Ausserdem ist $Y_k$ standardnormalverteilt, also $E(Y_k)=0$
und $E(Y_k^2)=1$.
Die Summanden in der Summe lassen sich damit kompakter ausdr"ucken:
\begin{align*}
Q_n-T
&=
\sum_{k=0}^{m_n-1} (Y_k^2-1) \Delta_k
\end{align*}
Dies k"onnen wir in $E((Q_n-T)^2)$ einsetzen:
\begin{align*}
E((Q_n-T)^2)
&=
\sum_{k=0}^{m_n-1}
\sum_{j=0}^{m_n-1}
E\biggl(
(Y_k^2-1)\Delta_k
(Y_j^2-1)\Delta_j
\biggr)
\end{align*}
Die Doppelsumme kann zerlegt werden in Terme mit $j=k$ und $j\ne k$.

In den Terme mit $j\ne k$ sind die $Y_k$ und $Y_j$ voneinander unabh"angig,
der Erwartungswert des Produktes kann daher in das Produkt der Erwartungswerte
zerlegt werden:
\begin{align*}
E\biggl(
(Y_k^2-1)\Delta_k
(Y_j^2-1)\Delta_j
\biggr)
&=
E(Y_k^2-1)
E(Y_j^2-1)
\Delta_k
\Delta_j
\\
&=
\bigl(E(Y_k^2) -1\bigr)
\bigl(E(Y_j^2) -1\bigr)
\Delta_k
\Delta_j
=0
\end{align*}
Im letzten Schritt haben wir verwendet, dass $E(Y_k^2)=\operatorname{var}Y_k=1$
ist.

Die Terme mit $j=k$ sind 
\begin{align*}
E((Q_n-T)^2)
&=
\sum_{k=0}^{m_n-1} E\biggl((Y_k^2-1)^2\Delta_k^2\biggr)
\\
&=
\sum_{k=0}^{m_n-1} E\bigl((Y_k^2-1)^2\bigr)\Delta_k^2
\le 
\biggl(\sum_{k=0}^{m_n-1} E\bigl((Y_k^2-1)^2\bigr)\Delta_k\biggr)\, |P^{(n)}|
\end{align*}
Im letzten Schritt haben wir einen Faktor $\Delta_k$ aus der Summe
herausgenommen und durch die Maschenweite $|P^{(n)}|$ abgesch"atzt.

Da $Y_k$ standardnormalverteilt ist, ist der Erwartungswert $E((Y_k^2-1)^2)$
unabh"angig von $k$, wir nennen diesen Wert $C$, der genaue Wert ist
nicht wichtig\footnote{Man kann den Wert nat"urlich wie folgt berechnen:
\begin{align*}
E((Y_k^2-1)^2)
&=
E(Y_k^4-2Y_k^2+1)
=
E(Y_k^4)-2E(Y_k^2)+E(1)
=
6-2+1=5.
\end{align*}}.
Damit l"asst sich die Differenz jetzt unabh"angig von $k$ absch"atzen
\begin{align*}
E((Q_n-T)^2)
&
\le
C\biggl(\sum_{k=1}^{m_n}\Delta_k\biggr)\, |P^{(n)}|
=CT|P^{(n)}|.
\end{align*}
Da die Maschenweite $|P^{(n)}|\to 0$ f"ur $n\to\infty$ folgt, dass
$Q_n\to T$ in $L^2(\Omega)$.
\end{proof}

\begin{hilfssatz} Sei $P^{(n)}$ eine Folge von Unterteilungen des
Intervals $[0,T]$ derart, dass die Maschenweite gegen $0$ strebt und
sei $R_n=R(P^{(n)},\lambda)$ die zugeh"orige Riemann-Summe.
Dann gilt
\[
\lim_{n\to\infty} R_n = \frac{W(T)^2}2 + \biggl(\lambda-\frac12\biggr)T
\]
als Funktion in $L^2(\Omega)$.
\end{hilfssatz}

\begin{proof}[Beweis]
Wir m"ussen die Riemann-Summe
\begin{align*}
R_n
&=
\sum_{k=0}^{m_n-1}
W(\tau_k) (W(t_{k+1}^{(n)})-W(t_k^{(n)}))
\end{align*}
durch Inkremente von Werten von $W(t)$ ausdr"ucken, konkret durch
\[
W(t_{k+1}^{(n)})-W(t_k^{(n)}),\quad
W(t_{k+1}^{(n)})-W(\tau_k^{(n)})
\quad \text{und} \quad
W(\tau_k^{(n)})-W(t_k^{(n)}).
\]
Um herauszufinden, wie dies m"oglich sein k"onnte, k"urzen wir die Werte
von $W$ ab durch
\[
a_k=W(t_k^{(n)}),\quad
b_k=W(\tau_k^{(n)})\quad\text{und}\quad
c_k=W(t_{k+1}^{(n)}),
\]
wobei wir im folgenden zur Vereinfachung der Rechnung die Indizes auch
weglassen.
Jetzt versuchen wir $b(c-a)$ durch andere Differenzen auszudr"ucken.
\begin{align*}
(b-a)^2
&=
b^2-2ab + a^2
\\
(c-a)^2
&=
c^2-2ac+a^2
\\
(c-b)(b-a)
&=
cb-b^2-ac+ab
\end{align*}
Um den Ausdruck $b(c-a)$ zu produzieren, muss der letzte Term verwendet
werden, denn er ist der einzige, der $bc$ enth"alt.
Dann muss aber auch der erste Term verwendet werden, um den Term $b^2$ zum
Verschwinden  zu bringen.
Ebenso muss der zweite Term $\frac12$-mal subtrahiert werden, damit der
Ausdruck $ac$ wegf"allt:
\begin{align*}
(b-a)^2-\frac12(c-a)^2+(c-b)(b-a)
&=
(b^2-2ab + a^2)
-\frac12(c^2-2ac+a^2)
+(cb-b^2-ac+ab)
\\
&=
cb-ab + \frac12a^2
-\frac12c^2
\end{align*}
Die quadratischen Terme sind $\frac12c_k^2=\frac12W(t_{k+1}^{(n)})^2$ und
$\frac12a_k^2=\frac12W(t_{k}^{(n)})^2$, die sich in aufeinanderfolgenden Termen
jeweils wegheben.
Nur der erste und letzte Term bleibt in der Summe bestehen, was wir
aber leicht korrigieren k"onnen.
So finden wir daher:
\begin{align*}
R_n
&=
\underbrace{
\sum_{k=0}^{m_n-1} (b_k-a_k)^2
}_{\textstyle\to\lambda T}
-
\underbrace{
\frac12\sum_{k=0}^{m_n-1} (c_k-a_k)^2
}_{\textstyle\to T}
+\sum_{k=0}^{m_n-1} (c_k-b_k)(b_k-a_k)
- \frac12a_0^2
+ \frac12c_{m_n-1}^2
\end{align*}
Die zweite Summe wurde im Hilfssatz~\ref{stochastisch:quadrvariation}
berechnet, dort wurde gefunden, dass die Summe f"ur $n\to\infty$ gegen $T$
konvergiert.
Mit der gleichen Rechnung wie in \ref{stochastisch:quadrvariation}
kann man finden, dass die zweite Summe gegen $\lambda T$ strebt.

Die Inkremente in der dritten Summe sind unabh"angig voneinander, daher
verschwinden die Erwartunsgewerte gemischter Produkte, es bleiben nur
die Terme
\begin{align*}
E\biggl(\biggl(\sum_{k=0}^{m_n-1}(c_k-b_k)(b_k-a_k)\biggr)^2\biggr)
&=
\sum_{k=0}^{m_n-1}
E\bigl((c_k-b_k)^2\bigr)E\bigl((b_k-a_k)^2\bigr)
\\
&=
\sum_{k=0}^{m_n-1}
E\bigl((W(t_{k+1}^{(n)})-W(\tau_k^{(n)}))^2\bigr)E\bigl((W(\tau_k^{(n)})-W(t_k^{(n)})^2\bigr)
\\
&=
\sum_{k=0}^{m_n-1}
(t_{k+1}^{(n)}-\tau_k^{(n)}) (\tau_k^{(n)}-t_k^{(n)})
\\
&=
\sum_{k=0}^{m_n-1}
(1-\lambda)(t_{k+1}^{(n)}-t_k^{(n)}) \lambda(t_{k+1}^{(n)}-t_k^{(n)})
\\
&\le 
(1-\lambda)\lambda
|P^{(n)}|
\sum_{k=0}^{m_n-1}
(t_{k+1}^{(n)}-t_k^{(n)})
\le (1-\lambda)\lambda T|P^{(n})|\to 0
\end{align*}
f"ur $n\to\infty$.
Damit ist gezeigt, dass die Riemann-Summe in $L^2(\Omega)$ gegen
\[
R_n\to
\frac{W(T)^2}2+\biggl(\lambda-\frac12\biggr)T
\]
konvergiert.
\end{proof}

\subsubsection{Das It\^o-Integral}
Ausgehend von der Riemann-Summe und den f"ur sie hergeleiteten Eigenschaften
wollen wir jetzt ein Integral
\[
\int_0^T G\,dW
\]
f"ur eine breitere Klasse von stochastischen Prozessen $G$ definieren.
Der erste Schritt dazu, ist das Integral f"ur Stufenprozesse zu definieren.

\begin{definition}
\index{Stufenprozess}
Ein stochastischer Prozess $G(t)$ ist eine {\em Stufenprozess}, wenn es eine
Unterteilung $P=\{0=t_0<t_1<\dots<t_m=T\}$ gibt derart, dass
$G(t)=G(t_k)$ f"ur $t_k\le t<t_{k+1}$.
\end{definition}

\begin{definition}
Ist $G$ ein Stufenprozess, dann ist das {\em It\^o-Integral} von $G$ definiert
als
\[
\int_0^TG\,dW = \sum_{k=0}^{m-1}G(t_k)(W(t_{k+1})-W(t_k)).
\]
Das It\^o-Integral ist eine Zufallsvariable.
\end{definition}

Das It\^o-Integral hat die folgenden Eigenschaften
\begin{hilfssatz}
Wenn $G$ und $H$ Stufenprozesse sind und $a,b\in\mathbb R$, dann gilt
\begin{compactenum}
\item Das It\^o-Integral ist linear:
\[
\int_0^T aG+bH\,dW
=
a\int_0^T G\,dW + b \int_0^TH\,dW
\]
\item Der Erwartungswert des It\^o-Integrals ist $0$:
\[
E\biggl(\int_0^T G\,dW\biggr)=0.
\]
\item Die Varianz des It\^o-Integrals ist
\[
E\biggl(\biggl(\int_0^T G\,dW\biggr)^2\biggr)
=
E\biggl(\int_0^T G^2\,dt\biggr)
\]
\end{compactenum}
\end{hilfssatz}

\begin{proof}[Beweis]
F"ur die Linearit"at verwenden wir eine Unterteilung $P$, f"ur die sowohl
$G$ also auch $H$ ein ein Stufenprozess ist.
Dann gilt
\begin{align*}
\int_0^T aG+bH\,dW
&=
\sum_{k=0}^{m-1} (aG(t_k)+bH(t_k))(W(t_{k+1})-W(t_k))
\\
&=
a\sum_{k=0}^{m-1} G(t_k)(W(t_{k+1})-W(t_k))
+
b\sum_{k=0}^{m-1} H(t_k)(W(t_{k+1})-W(t_k))
\\
&=a\int_0^TG\,dW+b\int_0^TH\,dW
\end{align*}
womit die Linearit"at bewiesen ist.

F"ur den Erwartungswert berechnen wir mit Hilfe einer passenden Unterteilung
\begin{align*}
E\biggl(\int_0^T G\,dW\biggr)
&=
E\biggl(\sum_{k=0}^{m-1} G(t_k) (W(t_{k+1}) - W(t_k))\biggr)
\\
&=
\sum_{k=0}^{m-1} E(G(t_k)) \underbrace{E(W(t_{k+1}) - W(t_k))}_{\textstyle =0}=0.
\end{align*}
Damit ist 2.~beweisen.

F"ur die Berechnung der Varianz verwendet wieder die Unabh"angigkeit der
Inkremente:
\begin{align*}
E\biggl(\biggl(\int_0^T G\,dW\biggr)^2\biggr)
&=
E\biggl(\biggl(\sum_{k=0}^{m-1}G(t_k)(W(t_{k+1})-W(t_k))\biggr)^2\biggr)
\\
&=
E\biggl(
\sum_{k=0}^{m-1}G(t_k)(W(t_{k+1})-W(t_k))
\sum_{j=0}^{m-1}G(t_j)(W(t_{j+1})-W(t_j))
\biggr)
\\
&=
\sum_{k=0}^{m-1}
\sum_{j=0}^{m-1}
E(G(t_k) G(t_j)
(W(t_{k+1})-W(t_k))
(W(t_{j+1})-W(t_j))
)
\\
&=
\sum_{k=0}^{m-1}
E(G(t_k)^2) E(W(t_{k+1})-W(t_k))^2)
\\
&=
\sum_{k=0}^{m-1}
E(G(t_k)^2) (t_{k+1}-t_k)
=
E\biggl(
\sum_{k=0}^{m-1}
G(t_k)^2 (t_{k+1}-t_k)
\biggr)
=E\biggl(\int_0^T G(t)^2\,dt\biggr).
\end{align*}
Somit ist auch 3.~bewiesen.
\end{proof}

Die Bedeutung der dritten Eigenschaft besteht darin, dass eine
Folge von Stufen-Prozessen, die $L^2(0,T)$ konvergiert, zu einer
konvergenten Folge von It\^o-Integralen f"uhrt.

\begin{definition}
\index{It\^o-Integral}
Ist $G$ ein beliebiger auf $[0,T]$ quadratintegrierbarer Prozess,
und $G_n$ eine Folge von stochastischen Prozessen, die in $L^2([0,T])$
gegen $G$ konvergiert.
Dann ist des It\^o-Integral von $G$
\[
\int_0^T G\,dW
=
\
\lim_{n\to\infty} \int_0^T G_n\,dW
\]
\end{definition}

Das It\^o-Integral hat die folgenden Eigenschaften:

\begin{satz}
\label{satz:ito-integral}
Seien $G$ und $H$ auf $[0,T]$ quadratintegrierbare stochastische Prozesse
und $a,b\in\mathbb R$. Dann gilt
\begin{compactenum}
\item Das It\^o-Integral ist linear:
\[
\int_0^T aG+bH\,dW
=
a\int_0^TG\,dW
+
b\int_0^TH\,dW
\]
\item Der Erwartungswert des It\^o-Integrals ist
\[
E\biggl(\int_0^T G\,dW\biggr)=0
\]
\item Die Varianz des It\^o-Integrals ist
\[
E\biggl(\biggl(\int_0^TG\,dW\biggr)^2\biggr)
=
E\biggl(\int_0^T G^2\,dt\biggr).
\]
\end{compactenum}
\end{satz}

Mit dem It\^o-Integral k"onnen wir jetzt alle Komponenten einer
stochastischen Differentialgleichung definieren.
Zun"achst k"onnen wir f"ur jeden auf $[0,T]$ quadratintegrierbaren Prozess $G$ 
und f"ur jede integrierbare Funktion $F\in L^1([0,T])$ die Gr"ossen
\begin{align*}
\int_s^r F(t)\,dt&=\int_0^r F(t)\,dt - \int_0^s F(t)\,dt
\\
\int_s^r G\,dW&=\int_0^r G\,dW - \int_0^s G\,dW
\end{align*}
definieren.
\begin{definition}
Wenn $X$ ein stochastischer Prozess ist, der
\[
X(r)=X(s)+\int_s^rF(t)\,dt + \int_s^tG\,dW
\]
erf"ullt, dann sagt man $X$ habe das {\em stochastische Differential}
\[
dX=F\,dt + G\,dW.
\]
\end{definition}

%
%
%
\subsection{Rechenregeln}
Bisher wissen wir nur, dass das It\^o-Integral linear ist, dies reicht
aber nicht f"ur einen Kalk"ul, mit dem wir hoffen k"onnen,
Differentialgleichungen zu l"osen.u
Dazu brauchen wir mindestens noch eine Kettenregel und eine
Produktregel.

\begin{satz}[It\^o's Kettenregel]
\index{It\^o-Kettenregel}
Falls $X$ das stochastische Differential
\[
dX=F\,dt + G\,dW
\]
hat, und falls $u\colon \mathbb R\times [0,T]\to\mathbb R$ eine
stetig differenzierbare Funktion.
Dann hat die Funktion $Y(t)=u(X(t), t)$ das stochastische Differential
\begin{align*}
du(X,t)
&=\frac{\partial u}{\partial t}\,dt + \frac{\partial u}{\partial x}\,dX 
+\frac12\frac{\partial u^2}{\partial x^2}G^2\,dt
\\
&=
\biggl(
\frac{\partial u}{\partial t}+\frac{\partial u}{\partial x}F
+\frac12\frac{\partial^2u}{\partial x^2}G^2
\biggr)\,dt
+
\frac{\partial u}{\partial x}G\,dW.
\end{align*}
\end{satz}
Bis auf den Term in den zweiten Ableitungen sind die Terme genau
diejenigen, die man von der klassischen Kettenregel her erwarten
k"onnte.

\begin{beispiel}
Potenzen eines Wiener-Prozesses.
Sei $u(x)=x^m$ und $X=W$, $F=0$ und $G=1$.
Also ist $Y=W^m$.
Dann hat nach der It\^o-schen Kettenregel das stochastische Differential
\[
d(W^m)
=
\frac12m(m-1)W^{m-2}\,dt + mW^{m-1}\,dW.
\]
Im Spezialfall $m=2$ folgt
\[
dW^2 = 2W\,dW + dt.
\]
\end{beispiel}

\begin{beispiel} Wir betrachten die Funktion
$u(x,t)=e^{\lambda x-\frac12\lambda^2 t}$, und wie vorhin $X=W$, $F=0$
und $G=1$.
Es folgt
\begin{align*}
d\biggl(
e^{\lambda W(t)-\frac12\lambda^2 t}
\biggr)
&=
\biggl(
-\frac12\lambda^2 e^{\lambda W-\frac12\lambda^2 t}
+
\frac12\lambda^2 e^{\lambda W-\frac12\lambda^2 t}
\biggr)\,dt
+
\lambda e^{\lambda W -\frac12\lambda^2 t}\,dW
\\
dY&=\lambda Y\,dW
\end{align*}
also ist der Prozess $Y$ eine L"osung der stochastischen Differentialgleichung
\begin{align*}
dy&=\lambda Y\,dW
\\
Y(0)&=1.
\end{align*}
Dieses Beispiel zeigt, dass der entwickelte Kalk"ul dazu geeignet sein kann,
stochastische Differentialgleichungen zu l"osen.
\end{beispiel}

\begin{satz}[Produktregel von It\^o]
\index{It\^o-Produktregel}
\index{Produktregel von It\^o}
Falls die Prozesse $X_1$ und $X_2$ die stochastischen Differentiale
\begin{align*}
dX_1
&=
F_1\,dt + G_1\,dW
\\
dX_2
&=
F_2\,dt + G_2\,dW
\end{align*}
haben, dann ist
\begin{equation}
d(X_1X_2)
=
X_2\,dX_1 + X_1\,dX_2 + G_1G_2\,dt,
\label{stochastisch:ito-produkt}
\end{equation}
dies ist die {\em It\^o-sche Produktformel}.
\end{satz}

F"ur den Beweis der Kettenregel brauchen wir die Produktregel, wir geben
daher zuerst einen Beweis f"ur die Produktregel an.

\begin{proof}[Beweis der Produktformel]
Die integrierte Form der Produktregel besagt, dass 
\begin{equation}
X_1(r)X_2(r)-X_1(0)X_2(0)
=
\int_0^r X_2\,dX_1 + \int_0^r X_1\,dX_2 + \int_0^r G_1G_2\,dt
\label{stochastisch:ito-produkt-integriert}
\end{equation}
diese m"ussen wir nachrechnen.
Wir nehmen an der Einfachheit halber an, dass $X_1(0)=X_2(0)=0$ ist.

1.~Wir nehmen zus"atzlich an, dass die Funktion $G_i$ und $F_i$ nicht von
der Zeit abh"angen.
Dies bedeutet, dass sich die stochastische Differentialgleichung
$dX_i=F_i\,dt+G_i\,dW$ integrieren l"asst:
\[
X_i(t) = \int_0^t F_i\,d\tau + G_i\int_0^r dW =  F_it+G_iW(t).
\]
Diese Bedingung werden wir im zweiten Schritt wieder aufheben.
Wir berechnen die rechte Seite der integrierten
Produktregel~(\ref{stochastisch:ito-produkt-integriert}):
\begin{align*}
\int_0^r X_2\,dX_1 + X_1\,dX_2 + G_1G_2\,dt
&=
\int_0^rX_1F_2+X_2F_1\,dt + \int_0^r X_1G_2+X_2G_1\,dW + \int_0^r G_1G_2\,dt
\\
&=
\int_0^r(F_1t+G_1W)F_2+(F_2t+G_2W)F_1\,dt
\\
&\qquad
+ \int_0^r (F_1t+G_1W)G_2+(F_2t+G_2W)G_1\,dW + \int_0^r G_1G_2\,dt
\\
&=F_1F_2r^2+(G_1F_2+G_2F_1)\underbrace{\biggl(\int_0^rW\,dt + \int_0^rt\,dW\biggr)}_{\textstyle = rW(r)}
\\
&\qquad
+G_1G_2\cdot \underbrace{2\int_0^r W\,dW}_{\textstyle W(r)^2-r}
+ G_1G_2\underbrace{\int_0^r\,dt}_{\textstyle =r}
\\
&= 
F_1F_2r^2 + (G_1F_2+G_2F_1)rW(r) + G_1G_2W(r)^2
\\
&= 
F_1r\cdot F_2r + F_2r\cdot G_1W(r)+F_1r\cdot G_2W(r) + G_1W(r)\cdot G_2W(r)
\\
&= 
(F_1r + G_1W(r))\cdot(F_2r +  G_2W(r))
=
X_1(r) \cdot X_2(r).
\end{align*}
Dies ist die linke Seite von (\ref{stochastisch:ito-produkt-integriert}), die
damit f"ur diesen Spezialfall bewiesen ist.

2.~Wir ersetzen jetzt die Konstanten $G_i$ und $F_i$ durch Stufenprozesse.
Es gibt eine Unterteilung des Intervals $[0,T]$ so, dass die Prozesse $G_i$
und $F_i$ in jedem Teilinterval konstant sind.
In jedem solchen Teilinterval gilt daher die Ito-Produktformel, und damit
auch f"ur das ganze Interval.

3.~Im allgemeinen Fall k"onnen wir die Prozesse $G_i$ und $F_i$ mit Hilfe
einer Formel von Stufenprozessen $G_i^n$ und $F_i^n$ approximieren.
Wir k"onnen zus"atzlich fordern, dass
\[
E\biggl(\int_0^T |F_i^n-F_i|\,dt\biggr)\to 0
\qquad\text{und}\qquad
E\biggl(\int_0^T (G_i^n-G_i)^2\,dt\biggr)\to 0,
\]
was wir weiter unten brauchen.
Es gibt nach dem zweiten Schritt eine Folge von Prozessen $X_i^n$ mit
\[
X_i^n(t) = X_i(0) + \int_0^t F_i^n\,d\tau + \int_0^t G_i^n\,dW,
\]
f"ur die jeweils die Produktregel gilt.
\begin{align*}
X_i^n(r)X_i^n(r)-X_i^n(s)X_i^n(s)
&=
\int_0^r X_2^n\,dX_1^n+\int_0^r X_1^n\,dX_2^n + \int_0^r G_1^nG_2^n\,dW
\\
&=
\int_0^r X_2^nF_1^n\,dt
+
\int_0^r X_2^nG_1^n\,dW
+
\int_0^r X_1^nF_2^n\,dt
+
\int_0^r X_1^nG_2^n\,dW
+
\int_0^r G_1^nG_2^n\,dW
\\
\intertext{Der Grenz"ubergang $n\to\infty$ f"uhrt auf}
X_i(r)X_i(r)-X_i(s)X_i(s)
&=
\int_0^r X_2F_1\,dt
+
\int_0^r X_2G_1\,dW
+
\int_0^r X_1F_2\,dt
+
\int_0^r X_1G_2\,dW
+
\int_0^r G_1G_2\,dW
\end{align*}
Damit ist die Produktformel bewiesen.
\end{proof}

\begin{proof}[Beweis der Kettenregel]
Der Beweis der Kettenregel verwendet, dass die Funktion $u(x,t)$ durch
Polynome approximiert werden kann.
Wir m"ussen also zun"achst die Kettenregel f"ur Polynome in $x$ beweisen,
und dann mit Hilfe eines Grenz"ubergangs mit approximierenden Polynomen
den allgemeinen Fall gewinnen.

1. Sei $u(x)=x^m$, wir behaupten, dass
\[
d(X^m)=mX^{m-1}\,dX + \frac12m(m-1)X^{m-2}G^2\,dt,
\]
dies ist die It\^o-Produktformel f"ur $u(x)=x^m$.
Wir beweisen diese Formel durch vollst"andige Induktion.
F"ur $m=0$ ist sie trivialerweise korrekt.
Wir nehmen daher an, dass Sie f"ur $m-1$ gilt, dass also
\begin{align*}
d(X^{m-1})
&=
(m-1)X^{m-2}\,dX + \frac12(m-1)(m-2)X^{m-3}G^2\,dt
\\
&=
(m-1)X^{m-2}F\,dt + (m-1)X^{m-2}G\,dW + \frac12(m-1)(m-2)X^{m-3}G^2\,dt.
\end{align*}
Dies wenden jetzt die Produktregel auf $X^m = XX^{m-1}$ an.
F"ur $X_1=X$ ist $G_1=G$, und $X_2=X^{m-1}$ und
\[
G_2
=
(m-1)X^{m-2}G
\]
\begin{align*}
d(X^m)
&=
d(XX^{m-1})
=
X\,d(X^{m-1}) + X^{m-1}\,dX +G_1G_2\,dt
\\
&=
X\biggl( (m-1)X^{m-2}\,dX + \frac12(m-1)(m-2)X^{m-3}G^2\,dt\biggr)
+
X^{m-1}\,dX
+
(m-1)X^{m-2}G^2\,dt
\\
&=
mX^{m-1}\,dX
+\biggl(\frac12(m-2)+1\biggr)(m-1)X^{m-2}G^2\,dt
\\
&=
mX^{m-1}\,dX +\frac12m(m-1)X^{m-2}G^2\,dt,
\end{align*}
damit ist der Induktionsschritt vollzogen.

Wegen der Linearit"at der Ableitung ist die Kettenregel damit
f"ur beliebige Polynome in $x$ bewiesen.

2.~Wir approximieren die Funktion $u(x,t)$ jetzt als Produkt
$u(x,t)=f(x)g(t)$, wobei $f(x)$ und $g(t)$ Polynome sein sollen.
Dann gilt
\begin{align*}
d(u(X,t))
&=
d(f(X)g)
=
f(X)\,dg + g\,df(X)
\\
&=
f(X)g'\,dt + g\biggl(f'(X)\,dX + \frac12 f''(X)G^2\,dt\biggr)
\\
&=
\frac{\partial u}{\partial t}\,dt
+
\frac{\partial u}{\partial x}\,dX
+
\frac12\frac{\partial^2u}{\partial x^2}G^2\,dt.
\end{align*}
Die It\^o-Kettenregel ist damit gezeigt f"ur ein Produkt von Polynomen.

3.~Sei jetzt $u^n$ eine Folge von Polynomen wie im zweiten Schritt,
so dass $u^n$ sowie die Ableitungen gegen $u$ gleichm"assig konvergieren:
\[
u^n\to u,
\qquad
\frac{\partial u^n}{\partial t} \to \frac{\partial u}{\partial t},
\qquad
\frac{\partial u^n}{\partial x} \to \frac{\partial u}{\partial x},
\qquad
\frac{\partial^2 u^n}{\partial x^2} \to \frac{\partial^2 u}{\partial x^2}.
\]
F"ur jedes $u^n$ gilt die It\^o-Kettenregel:
\[
u^n(X(r),r)-u^n(X(0),0)
=
\int_0^r
\frac{\partial u^n}{\partial t}
+
\frac{\partial u^n}{\partial x}F
+
\frac12 \frac{\partial^2 u^n}{\partial x^2}G^2\,dt
+
\int_0^r\frac{\partial u^n}{\partial x}G\,dW,
\]
durch Grenz"ubergang $n\to \infty$ erhalten wir daraus die allgemeine
Form der It\^o-Kettenregel.
\end{proof}

%
% Beispiele 
%
\subsection{L"osungen von stochastischen Differentialgleichungen}
In diesem Abschnitt betrachten wir einige Beispiele von stochatischen
Differentialgleichungen und verwenden die im vorangegangenen Abschnitt
diskutierten Rechenregeln verwendet werden k"onnen, die L"osungen
zu finden.

\subsubsection{Lineare stochastische Differentialgleichung mit $F=0$}
Sei $g$ eine stetige Funktionen, man finde eine L"osung der
Differenzialgleichung
\begin{equation}
\begin{aligned}
dX&=gX\,dW\\
X(0)&=1.
\end{aligned}
\label{stochastisch:beispiel1-dgl}
\end{equation}
Man beachte, dass ohne den stochastischen Einfluss nur noch die Gleichung
$dX=0$ "ubrig bleibt, in diesem Fall ist also $X$ einfach nur eine
Konstante, $X(t)=1$.
Wir k"onnen aber auch nicht erwarten, dass der Erwartungswert der
L"osung durch die L"osung $X(t)=1$ gegeben ist.
Die Gleichung ist zwar linear, aber die Wirkung der stochastischen
Schwankungen ist proportional zum aktuellen Wert von $X$.

W"are $W$ einfach nur eine differenzierbare Funktion, dann w"urde die
Differentialgleichung~(\ref{stochastisch:beispiel1-dgl}) zu der
gew"ohnlichen Differentialgleichung
\begin{align*}
\frac{\dot X}{X} &= g\dot W,
\\
\intertext{die durch direkte Integration gel"ost werden kann:}
\frac{d}{dt}\log X&= g(t) \dot W(t)
\\
\log X(t)&=\int_0^t g(\tau)\dot W(\tau)\,d\tau,
\\
X(t)&=e^{\int_0^t g(\tau)\dot W(\tau)\,d\tau}.
\end{align*}
Daraus k"onnte man ableiten, dass f"ur einen Wiener-Prozess $W$ die L"osung
\[
X(t)=e^{\int_0^t g\,dW}
\]
sein m"usste.
Um dies zu kontrollieren, schreiben wir 
\[
Y(t)=\int_0^t g\,dW
\qquad\Rightarrow\qquad
dY = g\,dW = \underbrace{0}_{\textstyle F}\,dt
+
\underbrace{g}_{\textstyle G}\,dW
\]
f"ur das Integral im Exponenten, und beachten, dass $X(t)=e^{Y(t)}$.
Wir haben die Funktionen $F$ und $G$ wie in der Formulierung der
It\^o-Kettenregel gew"ahlt.

Um zu kontrollieren, ob $X(t)$ tats"achlich die L"osung der
Differentialgleichung~(\ref{stochastisch:beispiel1-dgl}), m"ussen wir
also $e^{Y(t)}$ ableiten, dazu ist die It\^o-Kettenregel zu verwenden,
denn $X(t)=u(Y(t))$ mit $u(x)=e^x$.
Die It\^o-Kettenregel ergibt:
\begin{align*}
dX
&=
\biggl( u'F+\frac12u''G^2 \biggr)\,dt + u' G\,dW
\\
&=
\underbrace{e^Y}_{\textstyle X}\biggl(\frac12g^2\,dt + g\,dW\biggr)
\\
&=
\frac12g^2X\,dt + gX\,dW
\end{align*}
Offensichtlich ist das weit davon entfernt, die L"osung der
Differentialgleichung zu sein.
Insbesondere der erste Term mit $g^2$, der von der It\^o-Kettenregel
beigesteuert wird, ist zu viel.

Um den Term der zweiten Ableitungen wieder los zu werden, versuchen wir
den Ansatz
\begin{equation}
X(t) = e^{-\frac12\int_0^tg^2\,d\tau+\int_0^t g\,dW}.
\label{stochastisch:beispiel1-lsg}
\end{equation}
Um dies nachzupr"ufen schreiben wir wieder 
\[
Y(t)
=
-\frac12\int_0^tg^2\,d\tau+\int_0^t g\,dW
\]
f"ur den Exponenten, dies bedeutet, dass
\[
dY
=
\underbrace{-\frac12g^2}_{\textstyle F}\,dt
+
\underbrace{g}_{\textstyle G}\,dW.
\]
Wir wenden die It\^o-Kettenregel auf die Funktion
$X(t)=u(Y(t))=e^{Y(t)}$, also $u(x)=e^x$, an.
\begin{align*}
dX
&=
\biggl( u'F + \frac12u''G^2\biggr)\,dt + u'G\,dW
\\
&=
\underbrace{\biggl(-\frac12 g^2 + \frac12 g^2\biggr)}_{\textstyle=0}\,dt
+
\underbrace{e^Y}_{\textstyle X}g\,dW
\\
&=gX\,dW.
\end{align*}
Damit haben wir nachgewiesen, dass~(\ref{stochastisch:beispiel1-lsg})
die L"osung der stochastischen
Differentialgleichung~(\ref{stochastisch:beispiel1-dgl}) ist.

Da $g^2\ge 0$ ist, ist das Integral von $g^2$ im Exponenten
immer positiv, somit ist $X(t)$ immer um einen Faktor $\le1$ 
kleiner als die urspr"unglich vermutete L"osung:
\[
X(t)
=
\underbrace{e^{-\frac12 \int_0^tg^2\,d\tau}}_{\textstyle \le 1}\cdot e^{\int_0^tg\,dW}.
\]
Im Spezialfall $g=1$ ist finden wir, dass
\[
X(t)=e^{-\frac12t}e^{W(t)}=e^{W(t)-\frac12t}
\]
die L"osung der Differentialgleichung $dX=X\,dW$ ist.

\subsubsection{Lineare stochastische Differentialgleichung}
Die allgemeinste lineare stochastische Differentialgleichung hat die Form
\begin{equation}
dX=fX\,dt+gX\,dW
\label{stochastisch:beispiel2-dgl}
\end{equation}
mit stetigen Funktionen $f$ und $g$.
Nehmen wir wieder an, dass $W$ eine differenzierbare Funktion ist,
dann m"usste $X$ die Differentialgleichung
\begin{align*}
dX&=fX\,dt + gX\dot W\,\,dt
\\
\intertext{erf"ullen, die wieder durch Integration gel"ost werden kann:}
\frac{\dot X}{X}
&=
f+g\dot W
\\
\Rightarrow\qquad
\log X(t)
&=
e^{\int_0^t f+g\dot W\,d\tau}
=
e^{\int_0^t f\,d\tau+ \int_0^t g\,dW}.
\end{align*}
Wir verzichten darauf nachzurechnen, dass dies nicht die L"osung sein kann,
denn wie vorhin wird ein Term fehlen.
Wir vermuten, dass
\begin{equation}
X(t)
=
e^{\int_0^t f -\frac12g^2\,d\tau + \int_0^t g\,dW}
\label{stochastisch:beispiel2-lsg}
\end{equation}
die L"osung ist.
Wir setzen daher wieder
\[
Y(t)=\int_0^t f-\frac12g^2\,d\tau + \int_0^tg\,dW,
\]
der stochastische Prozess $Y$ erf"ullt die stochastische Differentialgleichung
\[
dY=
\underbrace{\biggl( f-\frac12g^2\biggr)}_{\textstyle F}\,dt
+
\underbrace{g}_{\textstyle G}\,dW.
\]
Wir vermuten, dass $X(t)=u(Y(t))=e^{Y(t)}$ die L"osung der
Differentialgeichung ist.

Wir wenden die It\^o-Kettenregel auf $X(t)$ an und finden
\begin{align*}
dX
&=
\biggl(u'F+\frac12u''G^2\biggr)\,dt + u'G\,dW
\\
&=
e^Y\biggl( f\underbrace{-\frac12g^2+\frac12 g^2}_{\textstyle=0}\biggr)\,dt
+
e^Yg\,dW
\\
&=
Xf\,dt + Xg\,dW.
\end{align*}
Sind $f$ und $g$ konstante Funktionen, also $f(t)=a$ und $g(t)=b$, dann
kann man $Y$ direkt berechnen:
\[
Y(t)=\biggl(a-\frac12b^2\biggr)t + bW(t),
\]
also gilt
\[
X(t)=e^{(a-\frac12b^2)t} e^{bW(t)}.
\]

\subsubsection{Langevin-Gleichung}
Der Wiener-Prozess versucht, die Brownsche Bewegung zu modellieren.
Er kann aber nur funktionieren, wenn sich das Teilchen im Wesentlichen
reibungsfrei bewegen kann.
Ein realistischeres Modells stammt von Langevin, der stochastische 
Prozess $X$ beschreibt die Geschwindigkeit des Teilchens, und erf"ullt
die stochastische Differentialgleichung
\begin{equation}
\begin{aligned}
dX&=-bX\,dt+\sigma\,dW\\
X(0)&=X_0
\end{aligned}
\label{stochastisch:langevin-dgl}
\end{equation}
Der Koeffizient $b$ beschreibt die Reibung, $\sigma$ die Wirkung der
Diffusion.
\index{Langevin-Gleichung}
F"ur $\sigma=0$ bleibt die gew"ohnliche Differentialgleichung
\[
\frac{\dot X}{X}=-b
\qquad\Rightarrow\qquad
\log X(t)=-bt
\qquad\Rightarrow\qquad
X(t)=e^{-bt}X_0.
\]
Der stochastische Term h"angt in dieser Differentialgleichung nicht von $X$
ab, wir k"onnen daher erwarten, dass die Differentialgleichung wenigstens
formal die gleiche L"osung haben wird wie eine gew"ohnliche
inhomogene Differentialgleichung.
Wir vermuten daher, dass
\begin{equation}
X(t)
=
e^{-bt}X_0 + \sigma\int_0^te^{-b(t-\tau)}\,dW
=
e^{-bt}\biggl(X_0 + \sigma\int_0^te^{b\tau}\,dW\biggr)
\label{stochastisch:langevin-lsg}
\end{equation}
die L"osung der Langevin-Gleichung~(\ref{stochastisch:langevin-dgl})
sein wird.
Tats"achlich ist die Ableitung davon
\[
dX
=
-be^{-bt}
\underbrace{\biggl(X_0 + \sigma\int_0^te^{b\tau}\,dW\biggr)}_{\textstyle =X}\,dt
+
e^{-bt}
\sigma e^{bt}dW
=
-bX\,dt +\sigma\,dW,
\]
die Differentialgleichung ist also erf"ullt.

Die L"osung erlaubt, Erwartungswert und Varianz von $X(t)$ zu berechnen.
Dazu verwenden wir die Eigenschaften des Wiener-Prozesses und des
It\^o-Integrals in Satz~\ref{satz:ito-integral}.
F"ur den Erwartungswert erhalten wir.
\begin{align*}
E(X(t))
&=
e^{-bt}E(X_0)
+
\sigma \underbrace{E\biggl(\int_0^t e^{-b(t-\tau)}\,dW\biggr)}_{\textstyle=0}
=
e^{-bt}X_0.
\end{align*}
F"ur den Erwartungswert von $X(t)^2$ berechnen wir
\begin{align*}
E(X(t)^2)
&=
E\biggl(e^{-2bt}X_0^2+2\sigma e^{-bt}X_0\int_0^te^{-b(t-\tau)}\,dW
+\sigma^2\biggl(\int_0^t e^{-b(t-\tau)}\,dW\biggr)^2
\biggr)
\\
&=
e^{-2bt}E(X_0^2)
+
2\sigma e^{-bt}E(X_0)E\biggl( \int_0^te^{-b(t-\tau)}\,dW\biggr)
+
\sigma^2E\biggl(\biggl(\int_0^t e^{-b(t-\tau)}\,dW\biggr)^2\biggr)
\\
&=
e^{-2bt}E(X_0^2)
+
\sigma^2 \int_0^t e^{-2b(t-\tau)}\,d\tau
\\
&=
e^{-2bt}E(X_0^2)
+
\frac{\sigma^2}{2b}(1-e^{-2bt}).
\end{align*}
Die Varianz von $X(t)$ ist daher
\begin{align*}
\operatorname{var}(X(t))
&=
E(X(t)^2)-E(X(t))^2
=
e^{-2bt}E(X_0^2)
+
\frac{\sigma^2}{2b}(1-e^{-2bt}).
-
e^{-2bt}E(X_0)^2
\\
&=
e^{-2bt}\operatorname{var}(X_0)
+
\frac{\sigma^2}{2b}(1-e^{-2bt}).
\end{align*}
Die Varianz von $X(t)$ setzt sich also aus zwei Komponenten zusammen.
Sie ist das gewichtete Mittel der Varianz des Anfangsbedingung
$\operatorname{var}(X_0)$ und der Varianz des Diffusionsterms, n"amlich
$\sigma^2/2b$.
Die Gewichtsfaktoren sind $e^{-2bt}$ und $1-e^{-2bt}$, die sich zu
$1$ addieren.
Zur Zeit $t=0$ ist die Varianz nat"urlich nur die Varianz der Anfangsbedingung.
Mit zunehmender Zeit wird die Varianz der Anfangsbedingung unbedeutend,
und es bleibt nur die Varianz des Diffusionsterms "ubrig.

Der Integralterm in $X(t)$ ist normalverteilt, wie alles was aus dem
Wienerprozess entsteht.
Der Term $e^{-bt}X_0$ wird aber in $X(t)$ immer unbedeutender, somit ist
nach gen"ugend langer Zeit $t$ die Zufallsvariable $X(t)$ normalverteilt
mit Erwartungswert $0$ und Varianz $\sigma^2/2b$.

Kehren wir zur physikalischen Interpretation zur"uck, dann sehen wir,
dass die Geschwindigkeit eines dem Langenvin-Prozesses unterworfenen
Teilchens nach einiger Zeit seine Anfangsgeschwindigkeit ``vergisst'',
die Zeitkonstante daf"ur ist gegeben durch die Reibung $b$.
Die Schwangungen der Geschwindigkeit sind umso gr"osser, je geringer
die Reibung ist.

\subsection{Stochastische Differentialgleichungen 2. Ordnung}
Die meisten Bewegungsgleichungen der Physik sind Differentialgleichungen
zweiter Ordnung.
In diesem Abschnitt wollen wir daher zwei Beispiele untersichen, wie
bekannte Systeme sich unter Einfluss von Rauschen verhalten werden.

\subsubsection{Ornstein-Uhlenbeck-Prozess}
\index{Ornstein-Uhlenbeck-Prozess}
Die Langevin-Gleichung beschreibt die Geschwindigkeit eines Teilchens,
dass der Brownschen Bewegung mit Reibung unterworfen ist.
Wir m"ochten aber auch die Position des Teilchens kennen, wir braucht
also noch eine weitere Zufallsvariable $Y$, welche die Position
beschreibt.
Nat"urlich soll $X$ die Ableitung von $Y$ sein: $X=\dot Y$.
Als klassische Differentialgleichung w"urden wir die Bewegungsgleichung
schreiben als
\[
\ddot Y=-b\dot Y+\sigma\frac{dW}{dt},
\]
wobei der letzte Term nat"urlich nicht wirklich sinnvoll ist.
Um dieser Gleichung Sinn zu geben, schreiben wir sie als
Differentialgleichungssystem erster Ordnung mit den beiden Variablen
$Y_1=Y=\dot X$ und $Y_2=X$ mit den Gleichungen
\begin{align*}
\dot Y_1&=-b Y_1+\sigma \frac{dW}{dt}\\
\dot Y_2&=Y_1.
\end{align*}
Als stochastische Differentialgleichung mit Anfangsbedingungen
wird dies
\begin{equation}
\begin{aligned}
dY_1&=-bY_1\,dt +\sigma\,dW,
&&&
Y_1(0)&=Y_{10}
\\
dY_2&=Y_1\,dt
&&&
Y_2(0)&=Y_{20}
\end{aligned}
\label{stochastisch:ornstein-uhlenbeck-dgl}
\end{equation}
Die erste Gleichung f"ur $Y_1$ ist die Langevin-Gleichung, die wir bereits
in~\ref{stochastisch:langevin-lsg}) gel"ost wurde.
Durch eine weitere Integration kann man auch die Gleichung f"ur $Y_2$
l"osen:
\begin{align*}
Y_1(t)&=e^{-bt}Y_{10}+\int_0^te^{-b(t-\tau)}\,dW\\
Y_2(t)&=Y_{20}+\int_0^t Y_1(\tau)\,d\tau
\end{align*}

\subsubsection{Harmonischer Oszillator}
Der Ornstein-Uhlenbeck-Prozess war insofern einfach zu behandeln, als 
die Geschwindigkeit aus dem Langevin-Prozess einfach noch einmal
integriert werden musste.
Es gab also keine R"uckkopplung zwischen der zweiten Ableitung und
dem Prozess selbst.
Erst eine solche R"uckkopplung kann zu Schwingungen f"uhren.
Wir versuchen daher jetzt einen harmonischen Oszillator zu modellieren,
der zus"atzlich dem Rauschen aus einem Wiener-Prozess unterworfen ist.

Ein harmonischer Oszillator hat die gew"ohnliche Differentialgleichung
zweiter Ordnung
\begin{equation}
\ddot X=-\lambda^2 X-b\dot X, X(0)=X_0, \dot X(0)=X_1.
\label{stochastisch:harmosz-dgl2}
\end{equation}
Da wir zweite Ableitungen in stochastischen Differentialgleichungen
nicht verwenden k"onnen, ersetzen wir~(\ref{stochastisch:harmosz-dgl2})
wieder durch zwei gekoppelte Gleichungen erster Ordnung:
\begin{equation}
\begin{aligned}
\dot Y_1&=Y_2,                &&&Y_1(0)&=X_0, \\
\dot Y_2&=-\lambda^2 Y_1-bY_2,&&&Y_2(0)&=X_1.
\end{aligned}
\label{stochastsich:harmosz-dgl1}
\end{equation}
In dieser Form k"onnen wir jetzt in der Form eines Systems stochastischer
Differentialgleichungen formulieren:
\begin{align*}
dY_1&=Y_2\,dt,\\
dY_2&=(-\lambda^2Y_1-bY_2)\,dt.
\end{align*}
Wenn wir auf der rechten Seite von~(\ref{stochastisch:harmosz-dgl2})
einen Rauschterm, also eine Ableitung eines Wiener-Prozesses hinzuf"ugen
wollen, dann wird daraus das stochastische Differentialgleichungssystem
\begin{equation}
\begin{aligned}
dY_1&=Y_2\,dt,                                &&&Y_1(0)&=X_0,\\
dY_2&=(-\lambda^2 Y_1-bY_2)\,dt + \sigma\,dW, &&&Y_2(0)&=X_1.
\end{aligned}
\end{equation}

Wie bei den gew"ohnlichen Differentialgleichungen k"onnen wir diese
Gleichung einfacher l"osen, wenn wir sie in Matrixform schreiben:
\begin{equation}
d\begin{pmatrix}Y_1\\Y_2\end{pmatrix}
=
\underbrace{
\begin{pmatrix}
         0& 1\\
-\lambda^2&-b
\end{pmatrix}}_{\textstyle=D}
\begin{pmatrix}Y_1\\Y_2\end{pmatrix}\,dt
+
\begin{pmatrix}
\sigma\\0
\end{pmatrix}\,dW.
\end{equation}
Die L"osung wird dann
\begin{equation}
\begin{pmatrix}
Y_1(t)\\Y_2(t)
\end{pmatrix}
=
e^{Dt}\begin{pmatrix}X_0\\X_1\end{pmatrix}
+
\sigma \int_0^t e^{D(t-\tau)}\begin{pmatrix}0\\\sigma\end{pmatrix}\,dW.
\label{stochastisch:harmosz-explsg}
\end{equation}

In Abschnitt~\ref{linear:harmosz} wurde der Spezialfall $b=0$ bereits
behandelt.
Die dort gefundenen Formeln f"ur $e^{Dx}$ erm"oglichen, direkt eine
L"osung anzugeben:
%\begin{beispiel}
%Wir berechnen die L"osung f"ur den Spezialfall $b=0$.
%Die Matrix $D$ ist
%\[
%D=\begin{pmatrix}
%0&1\\-\lambda^2&0
%\end{pmatrix}.
%\]
%Die Matrix $e^{Dt}$ kann durch Diagonalisierung berechnet werden.
%Die Transformationsmatrix
%\[
T
%=
%\begin{pmatrix}
%1&1\\
%i\lambda&-i\lambda
%\end{pmatrix},
%\qquad
%T^{-1}
%=
%\frac12
%\begin{pmatrix}
%1& 1/i\lambda\\
%1&-1/i\lambda
%\end{pmatrix}
%\]
%bringt die Matrix $D$ in Diagonalform:
%\[
%T^{-1}DT
%=
%\begin{pmatrix}
%i\lambda&        0\\
%        &-i\lambda
%\end{pmatrix}.
%\]
%Daraus kann man jetzt die Exponentialfunktion berechnen:
%\begin{align*}
%T^{-1}e^{Dt}T
%&=
%\begin{pmatrix}
%\cos\lambda t+i\sin\lambda t&              0             \\
%              0             &\cos\lambda t-i\sin\lambda t
%\end{pmatrix}
%\\
%\Rightarrow\qquad\qquad
%e^{Dt}
%&=
%\begin{pmatrix}
%                \cos\lambda t&\frac1{\lambda}\sin\lambda t\\
%-\frac1{\lambda}\sin\lambda t&               \cos\lambda t
%\end{pmatrix}
%\end{align*}
%Daraus k"onnen wir jetzt die L"osung mit der
%Formel~(\ref{stochastisch:harmosz-explsg}) ablesen:
\begin{equation}
\begin{pmatrix}
Y_1(t)\\Y_2(t)
\end{pmatrix}
=
\begin{pmatrix}
                \cos\lambda t&\frac1{\lambda}\sin\lambda t\\
-\frac1{\lambda}\sin\lambda t&               \cos\lambda t
\end{pmatrix}
\begin{pmatrix}X_0\\X_1\end{pmatrix}
+
\sigma\int_0^t
\begin{pmatrix}
\frac1{\lambda}\sin\lambda(t-\tau)\\
               \cos\lambda(t-\tau)
\end{pmatrix}\,dW.
\label{stochastisch:harmosz-y2}
\end{equation}
Die erste Komponenten davon ist
\begin{equation}
Y_1(t)
=
X_0\cos\lambda t+\frac{X_1}{\lambda}\sin\lambda t
+
\frac{\sigma}{\lambda}\int_0^t\sin\lambda(t-\tau)\,dW,
\label{stochastisch:harmosz-y}
\end{equation}
die L"osung der Differentialgleichung.

Die L"osungsformel~(\ref{stochastisch:harmosz-y}) zeigt auch, dass
$Y_1(t)$ selbst dann von $0$ verschieden sein kann, wenn $X_0=X_1=0$ ist.
Wir wollen f"ur diesen Spezialfall Erwartungswert und Varianz 
von $Y_1(t)$ berechnen.
Wir verwenden wieder die Resultate von Satz~\ref{satz:ito-integral}.
Der Integralterm verschwindet bei der Berechnung des Erwartungswertes:
\begin{align*}
E(Y_1(t))
&=
E(X_0)\cos\lambda t
+
\frac{E(X_1)}{\lambda}\sin\lambda t
+
\frac{\sigma}{\lambda}E\biggl(\int_0^t\sin\lambda(t-\tau)\,dW\biggr)
\\
&=
E(X_0)\cos\lambda t+\frac{E(X_1)}{\lambda}\sin\lambda t.
\\
\intertext{Den Erwartungswert von $Y_1(t)^2$ k"onnen wir wie folgt berechnen:}
E(Y_1(t)^2)
&=
E(X_0^2)\cos^2\lambda t
+
2E(X_0X_1)\frac1{\lambda}\cos\lambda t\sin\lambda t
+
\frac{E(X_1^2)}{\lambda^2}\sin^2\lambda t
%+
%\frac{\sigma E(X_1)}{\lambda^2}E\biggl(\int_0^t\sin\lambda(t-\tau)\,dW\biggr)
%+
%2E(X_0)\frac{\sigma}{\lambda}\cos\lambda t
%E\biggl(\int_0^t\sin\lambda(t-\tau)\,dW\biggr)
%\\
%&\qquad
+
\frac{\sigma^2}{\lambda^2}E\biggl(
\biggl(\int_0^t\sin\lambda(t-\tau)\,dW\biggr)^2
\biggr)
\\
&=
E(X_0^2)\cos^2\lambda t
+
2E(X_0X_1)\frac1{\lambda}\cos\lambda t\sin\lambda t
+
\frac{E(X_1^2)}{\lambda^2}\sin^2\lambda t
+
\frac{\sigma^2}{\lambda^2}\int_0^t \sin^2\lambda(t-\tau)\,d\tau
\\
&=
E(X_0^2)\cos^2\lambda t
+
2E(X_0X_1)\frac1{\lambda}\cos\lambda t\sin\lambda t
+
\frac{E(X_1^2)}{\lambda^2}\sin^2\lambda t
+
\frac{\sigma^2}{4\lambda^3}
(2\lambda t -\sin 2\lambda t).
\\
\intertext{Die Varianz ist
$\operatorname{var}(Y_1(t))=E(Y_1(t)^2)-E(Y_1(t))^2$:}
\operatorname{var}(Y_1(t))
&=
\operatorname{var}(X_0)\cos^2\lambda t
+2\operatorname{cov}(X_0,X_1)\cos\lambda t\sin\lambda t
+
\frac{\sin^2\lambda t}{\lambda^2}\operatorname{var}(X_1)
+\frac{\sigma^2}{4\lambda^3}(2\lambda t-\sin2\lambda t).
\end{align*}
Wie im Beispiel der Langevin-Gleichung hat die Varianz
also zwei Komponenten.
Die ersten drei Komponenten beschreiben den Einfluss der Anfangsbedingungen.
Der letzte Term beschreibt die zus"atzliche Varianz, die durch
das Rauschen in den Oszillator eingebracht wird.

Der Term zu den Anfangsbedingungen kann in Matrixform geschrieben
werden:
\begin{gather*}
\operatorname{var}(X_0)\cos^2\lambda t
+2\operatorname{cov}(X_0,X_1)\cos\lambda t\sin\lambda t
+
\frac{\sin^2\lambda t}{\lambda^2}\operatorname{var}(X_1)
\qquad
\qquad
\qquad
\qquad
\\
\qquad
\qquad
\qquad
\qquad
=
\begin{pmatrix}
\cos\lambda t&\frac1\lambda\sin\lambda t
\end{pmatrix}
\begin{pmatrix}
\operatorname{var}(X_0)    &\operatorname{cov}(X_0,X_1)\\
\operatorname{cov}(X_0,X_1)&\operatorname{var}(X_1)
\end{pmatrix}
\begin{pmatrix}
\cos\lambda t\\
\frac1\lambda\sin\lambda t
\end{pmatrix}
\end{gather*}
Diese Form kann man auch erhalten, indem man die Kovarianzen mit Hilfe von
\[
\begin{pmatrix} Y_1(t)&Y_2(t) \end{pmatrix}
\begin{pmatrix}
Y_1(t)\\
Y_2(t)
\end{pmatrix}
=
\begin{pmatrix}
Y_1(t)^2&Y_1(t)Y_2(t)\\
Y_1(t)Y_2(t)&Y_2(t)^2
\end{pmatrix}
\]
berechnet, und darauf die Formel~(\ref{stochastisch:harmosz-y2}) anwendet.

%
% Partielle Differentialgleichungen
%
\section{Partielle Differentialgleichungen\label{section:pdgl}}
\rhead{Partielle Differentialgleichungen}
Die It\^osche Kettenregel stellt einen Zusammenhang her zwischen der zweiten
Ableitung von $u$ und den Werten von $u$ am Ende eines Pfades.
In diesem Abschnitt wollen wir den Zusammenhang wie folgt konkretisieren.
Wir werden zun"achst f"ur ein Gebiet $U$ und einen Punkt $x$ die
Zeit $\tau_x$ definieren, zu der eine Brownsche Bewegung das Gebiet zum
ersten Mal verl"asst.
Nat"urlich ist $\tau_x$ eine Zufallsvariable, und damit auch $u(X(\tau_x))$,
wobei der Pfad ist, der zur Zeit $\tau_x$ das Gebiet verl"asst.
Dann werden wir zeigen, dass $u(x)$ der Erwartungswert von $u(X(\tau_x))$
sein.

In der Theorie der quaslinearen partiellen Differentialgleichungen erster
Ordnung wird gezeigt, wie man die Funktionswerte der L"osungsfunktion
dadurch bestimmen kann, dass man vom Punkt $x$ aus die Charakteristik 
konstruiert, und den Wert der Randbedingung an dem Punkt ermittelt,
wo die Charakteristik das Gebiet verl"asst.
Aus diesem Wert l"asst sich dann der Wert der L"osung bestimmen.
Brownsche Bewegungen spielen also f"ur elliptische partielle
Differentialgleichungen eine "ahnliche Rolle wie Charakteristiken f"ur
quasliineare partielle Differentialgleichungen erster Ordnung.

\subsection{Stopzeiten}
Wie fr"uher betrachten wir wieder einen stochastischen Prozess $X(t)$,
der L"osung einer stochastischen Differentialgleichung
\begin{equation}
\begin{aligned}
dX(t)&=b(t,X)\,dt + B(t,X)\,dW
\\
X(0)&=X_0
\end{aligned}
\label{stochastisch:stopzeitdgl}
\end{equation}
sein soll.
\begin{figure}
\centering
\includegraphics{chapters/images/stochastisch-2.pdf}
\caption{Brownsche Bewegung in zwei Dimension und Definition der
Stopzeit $\tau$, zu der der Pfad $X(t)$ das Gebiet verl"asst.
\label{stochastisch:pfad}}
\end{figure}

\begin{definition}
Sei $E$ eine beliebige offene oder abgeschlossene nicht leere Teilmenge
von $\mathbb R^n$.
Dann setzen wir
\[
\tau = \inf\{t\ge 0\;|\;X(t)\in E\},
\]
$\tau$ ist also die fr"uheste Zeit, zu der der Weg $X(t)$ die Menge $E$
erreicht.
\end{definition}

Die charakteristische Funktion
\[
\chi_{[0,\tau]}(t)=\begin{cases}
1\qquad\qquad&t\le \tau\\
0            &t>\tau
\end{cases}
\]
ist nat"urlich auch ein stochastischer Prozess, und damit auch 
$\chi_{[0,\tau]}G$.
Damit gelten die Regeln f"ur das It\^o-Integral aus
Satz~\ref{satz:ito-integral} auch f"ur diesen Prozess, jetzt allerdings
als Integrale mit oberer Grenze $\tau$ statt $T$.

\begin{hilfssatz}
Seien $G$ und $H$ auf $[0,T]$ quadratintegrierbare stochastische Prozesse
und $a,b\in\mathbb R$
\begin{compactenum}
\item
Das It\^o-Integral ist linear:
\begin{align*}
\int_0^\tau aG+bH\,dW
&=
\int_0^T a\chi_{[0,\tau]} G + b \chi_{[0,\tau]} H\,dW
=
a\int_0^T \chi_{[0,\tau]} G\,dW + b \int_0^T\chi_{[0,\tau]} H\,dW
\\
&=
a\int_0^\tau G\,dW + b \int_0^\tau H\,dW
\end{align*}
\item Der Erwartungswert des It\^o-Integrals ist
\[
E\biggl(\int_0^\tau G\,dW\biggr)
=
E\biggl(\int_0^T \chi_{[0,\tau]}G\,dW\biggr)
=
0.
\]
\item Die Varianz des It\^o-Integrals ist
\[
E\biggl(\biggl(\int_0^\tau G\,dW\biggr)^2\biggr)
=
E\biggl(\biggl(\int_0^T \chi_{[0,\tau]}G\,dW\biggr)^2\biggr)
=
E\biggl(\int_0^T\chi_{[0,\tau]}G^2\,dt\biggr)
=
E\biggl(\int_0^\tau G^2\,dt\biggr).
\]
\end{compactenum}
\end{hilfssatz}

Die It\^o-sche Kettenregel funktioniert auch f"ur $\tau$ als obere
Grenze f"ur die stochastischen Integrale.
Wenn also $X$ als stochastischer Prozess eine L"osung der stochastischen
Differentialgleichung (\ref{stochastisch:stopzeitdgl}) ist, dann gilt
nach der der It\^o-schen Kettenregel
\[
du(X,t)
=
\frac{\partial u}{\partial t}\,dt
+
\sum_{i=1}^n\frac{\partial u}{\partial x_i}\,dX_i
+
\frac12\sum_{i,j=1}^n\frac{\partial^2 u}{\partial x_i\partial x_j}
\sum_{k=1}^n b_{ik}b_{jk}\,dt.
\]
In integrierter Form bedeutet dies
\[
u(X(t),t)-u(X(0),0)
=
\int_0^t
\frac{\partial u}{\partial t}
+
\frac12\sum_{k=1}^nb_{ik}b_{jk}
\sum_{i,j=1}^n \frac{\partial^2u}{\partial x_i\,\partial x_j} \,ds
+
\int_0^t \operatorname{grad} u\cdot B\,dW.
\]
Und nat"urlich gelten diese Formeln auch dann, wenn man $t$ durch $\tau$
ersetzt.

Im Folgenden interessiert uns nur der Fall $b=0$, $b_{ik}=\delta_{ik}$
und Funktionen $u$, die nicht von der Zeit abh"angen.
Dann vereinfacht sich die Formel zu
\begin{equation}
u(X(t))-u(X(0))
=
\int_0^t \frac12\sum_{i=1}^n\frac{\partial^2 u}{\partial x_i^2}\,ds
=
\int_0^t \frac12\Delta u\,ds
\label{stochastisch:laplaceinkrement}
\end{equation}
Ausserdem ist in diesem Fall $X$ nichts anderes als eine Brownsche Bewegung,
$X=W$.

\subsection{Brownsche Bewegung und der Laplace-Operator}
Wir wenden die Formel (\ref{stochastisch:laplaceinkrement}) jetzt in
zwei Beispielen an.

\subsubsection{Zeit bis zum Verlassen eines Gebietes}
F"ur das erste Beispiel sei $U$ ein beschr"anktes Gebiet in
$\mathbb R^n$, und $u$ eine L"osung der partiellen Differentialgleichung
\begin{equation}
\begin{aligned}
-\frac12\Delta u&=1&\qquad&\text{in $U$}\\
               u&=0&      &\text{auf $\partial U$}
\end{aligned}
\label{stochastisch:hittingtime}
\end{equation}
Wir m"ochten die L"osung $u$ dazu verwenden, die Zeit $\tau_x$ zu berechnen,
zu der eine im Punkt $x\in U$ beginnende Brownsche Bewegung zum ersten
Mal das Gebiet $U$ verl"asst.
Die Formel (\ref{stochastische:laplaceinkrement}) liefert
\[
u(X(\tau_x))-u(X(0)) = \int_0^{\tau_x} \frac12\Delta u\,ds
\]
Da $u$ eine L"osung von (\ref{stochastisch:hittingtime}) ist, ist der 
Integrand auf der rechten Seite gleich $-1$:
\[
u(X(\tau_x))-u(X(0)) = -\int_0^{\tau_x} \,ds
\]
Da der Prozess $X$ zur Zeit $\tau_x$ den Rand des Gebietes "uberquert,
ist wegen der Randbedingung $u(X(\tau_x))=0$. 
Zusammen erhalten wir
\[
-u(X(0)) = -\tau_x.
\]
Nun interessiert uns aber nicht der Wert der Zufallsvariablen, sondern
nur deren Erwartungswert:
\[
E(\tau_x)=E(u(X(0)))=u(x).
\]
Die L"osung $u$ der Differentialgleichung (\ref{stochastisch:hittingtime})
gibt die erwartete Zeit an, bis eine bei $x$ beginnende Brownsche Bewegung 
das Gebiet $U$ verlassen hat.

\subsubsection{Charakterisierung von harmonischen Funktionen}
Sei wieder $U$ ein beschr"anktes Gebiet mit glattem Rand und
$g$ eine stetige Funktion auf $\partial U$.
Ausserdem sei $u$ eine harmonische Funktion in $U$ mit den Randwerten $g$, 
also eine L"osung der partiellen Differentialgleichung
\begin{equation}
\begin{aligned}
\Delta u&=0&\qquad&\text{in $U$}\\
       u&=g&      &\text{auf $\partial U$}
\end{aligned}
\label{stochastisch:harmonisch}
\end{equation}
Wir wollen (\ref{stochastisch:laplaceinkrement}) verwenden, die Funktion
$u$ zu charakterisieren.
Dazu sei wieder $\tau_x$ die Zeit, zu der eine im Punkt $x\in U$ beginnende
Brownsche Bewegung das Gebiet $U$ verl"asst.
Aus (\ref{stochastisch:laplaceinkrement}) folgt:
\[
u(X(\tau_x))-u(X(0))
=
\int_0^{\tau_x} \frac12\underbrace{\Delta u}_{\textstyle=0}\,ds=0.
\]
Zur Zeit $\tau_x$ ist $X(\tau_x)$ ein Randpunkt des Gebiets, der mit
der Randbedingung bestimmt werden kann, es folgt:
\[
u(X(0)) = g(X(\tau_x)).
\]
Wieder interessiert uns der einzelne Wert nicht, sondern der Erwartungswert:
\[
u(x)=E(g(X(\tau_x))).
\]
Den Wert im Punkt $x$ einer harmonischen Funktion mit Randwerten $g$
kann man wie folgt finden: man l"asst eine Brownsche Bewegung von $x$ 
aus laufen, bis sie den Rand "uberquert, und nimmt den Mittelwert
der derart erreichten Randwerte.

Aus dieser Charakterisierung der harmonischen Funktionen kann man
auch deren Mittelwerteigenschaft ableiten. 
Da die Brownsche Bewegung isotrop ist, muss sich im Zentrum
eines kugelf"ormigen Gebietes immer der gleiche Wert f"ur $u$
ergeben, selbst wenn man eine beliebige Drehung auf die Randwerte 
anwendet.
Also muss der Wert von $u(x)$ der Mittelwert der Werte
auf einer Kugel um den Punkt $x$ sein.

%
%
%
\section{Kalman-Filter\label{section:kalman}}
\rhead{Kalman-Filter}






\begin{appendices}
\chapter{Newton-Verfahren}
\lhead{Newton-Verfahren}
\rhead{}

\section{Nullstellen von Funktionen}
\rhead{Nullstellen von Funktionen}
\begin{figure}
\centering
\includegraphics{chapters/images/randwert-2.pdf}
\caption{Bestimmung der Nullstelle einer Funktion $f(x)$ mit dem
Newton-Verfahren.
Die Approximation $x_{n+1}$ wird gefunden als Schnittpunkt der Tangente
im Punkt $(x_n,f(x_n))$ (mit Steigung $f'(x_n)$) mit der $x$-Achse.
\label{newton:graphik}}
\end{figure}
Das Ziel dieses Anhangs ist, die folgende Aufgabe numerisch zu l"osen:
\begin{aufgabe}
Gegen ist eine differenzierbar Funktion
$f\colon\mathbb R\to\mathbb R:x\mapsto f(x)$
und eine Zahl $y$ im Wertebereich von $f$.
Finde $\hat{x}\in\mathbb R$ so, dass $f(\hat{x})=y$.
\end{aufgabe}
Im allgemeinen kann man nicht davon ausgehen, dass sich eine L"osung der
Gleichung $f(x)=y$ in geschlossener Form finden l"asst.
Nur einige wenige Klassen von Gleichungen haben L"osungsformeln dieser Art.
Wir beschr"anken uns daher auf das Problem, eine Approximation f"ur die
L"osung zu bestimmen.

Indem wir statt der Funktion $f(x)$ die Funktion $x\mapsto g(x)=f(x)-y$
betrachten, k"onnen wir die gesuchte Zahl $x$ auch als L"osung der
Gleichung $g(x)=0$ finden:
\begin{equation}
f(x)=y
\qquad\qquad
\Rightarrow
\qquad\qquad
g(x)=f(x)-y = 0.
\label{newton:reduktion}
\end{equation}
Es gen"ugt also, ein L"osungsverfahren zu entwickeln f"ur die Aufgabe
\begin{aufgabe}
Gegen ist eine differenzierbar Funktion
$f\colon\mathbb R\to\mathbb R:x\mapsto f(x)$,
finde $\hat{x}\in\mathbb R$ so, dass $f(\hat{x})=0$.
\end{aufgabe}
Da wir nur eine numerische L"osung brauchen, versuchen wir sie dadurch
zu finden, dass wir eine Anfangssch"atzung $x_0$ wiederholt korrigieren,
bis der Fehler klein genug ist.
Es soll also eine Folge $x_0,x_1,x_2,\dots$ konstruiert werden, welche
gegen die L"osung $\hat{x}$ konvergiert.
Der Differenzenquotient ist eine Approximation f"ur die Steigung
$f'(x_n)$,
\begin{equation}
\frac{
f(x_{\mathstrut n+1})-f(x_{\mathstrut n})
}{
x_{\mathstrut n+1}-x_{\mathstrut n}
}
\simeq = f'(x_n).
\label{newton:pre}
\end{equation}
Wir m"ochten gerne, dass $f(x_{n+1})=0$ ist, und k"onnen (\ref{newton:pre})
unter dieser Annahme nach $x_{n+1}$ aufl"osen:
\begin{align*}
-f(x_n)
&\simeq
f'(x_n)\,(x_{\mathstrut n+1}-x_{\mathstrut n})
\\
x_{\mathstrut n}-\frac{f(x_n)}{f'(x_n)}
&\simeq x_{\mathstrut n+1}
\end{align*}
Damit haben wir ein L"osungsverfahren gefunden:
\begin{satz}[Newton]
Ist $f$ eine differenzierbare Funktion, deren Ableitung bei der Nullstelle
$\hat{x}$ nicht verschwindet, also $f(\hat{x})\ne 0$, und $x_0$ eine erste
Approximation f"ur $\hat{x}$, dann konvergiert die Folge
definiert durch die Rekursionsformel
\[
x_{n+1}=x_n-\frac{f(x_n)}{f'(x_n)},
\]
dann konvergiert $x_n$ gegen $\hat{x}$, falls $x_0$ nahe genug bei
$\hat{x}$ liegt.
\end{satz}

\begin{beispiel}
Man finde die Wurzel der Zahl $y$, d.~h.~man muss die Nullstellen
der Funktion $f(x)=x^2-y$ finden.
Das Newton-Verfahren ben"otigt die Ableitung von $f$, sie ist
$f'(x)=2x$, und konstruiert daraus die Folge
\begin{equation}
x_{n+1} = x_n - \frac{f(x_n)}{f'(x_n)}=x_n-\frac{x_n^2-y}{2x_n}
=
\frac{2x_n^2-x_n^2+y}{2x_n}
=
\frac12\biggl(x_n + \frac{y}{x_n}\biggr).
\label{newton:mittel}
\end{equation}
Die Quaratwurzel von $y$ erf"ullt nat"urlich
\[
\sqrt{y} = \frac12\biggl( \sqrt{y}+\frac{y}{\sqrt{y}}\biggr).
\]
Mit $x_n$ ist auch $y/x_n$ eine Approximation von $\sqrt{y}$.
Die neue Approximation $x_{n+1}$ ist das arithmetische Mittel der
beiden Approximationen $x_n$ und $y/x_n$ von $\sqrt{y}$.
Die Konvergenz dieser Folge ist sehr schnell, wie Tabelle~\ref{newton:sqrt2}
zeigt.
\begin{table}
\centering
\begin{tabular}{|>{$}r<{$}|>{$}r<{$}|}
\hline
n&x_n\\
\hline
0 &  2.00000000000000\\
1 &  1.50000000000000\\
2 &  1.\underline{41}666666666667\\
3 &  1.\underline{41421}568627451\\
4 &  1.\underline{41421356237}469\\
5 &  1.\underline{41421356237309}\\
6 &  1.\underline{41421356237309}\\
\hline
\end{tabular}
\caption{Approxmationen von $\sqrt{2}$ mit Hilfe des Newton-Algorithmus,
korrekte Stellen unterstrichen.
Die Anzahl korrekter Stellen verdoppelt sich in jedem Schritt.
\label{newton:sqrt2}}
\end{table}
In jedem Schritt verdoppelt sich die Anzahl korrekter Stellen.
Dies ist eine allgemeine Eigenschaft des Newton-Algorithmus, wie
in Abschnitt~\ref{section:newton:konvergenz} erkl"art wird.
\end{beispiel}

\section{Konvergenzgeschwindigkeit\label{section:newton:konvergenz}}

\section{L"osung von Vektorgleichungen\label{section:newton:vektor}}
\rhead{L"osung von Vektorgleichungen}
Wir m"ochten das Verfahren nun erweitern, so dass wir nicht nur eine
einzige Gleichung $f(x)=y$ nach $x$ aufl"osen k"onnen, wir m"ochten dazu 
f"ur ein Gleichungssystem von nichtlinearen Gleichungen
\begin{align*}
f_1(x_1,\dots,x_n)&=y_1\\
f_2(x_1,\dots,x_n)&=y_2\\
&\;\;\vdots\\
f_m(x_1,\dots,x_n)&=y_m
\end{align*}
ebenfalls in der Lage sein.
Wie bei einer einzigen Gleichung k"onnen wir das Problem reduzieren
auf das Finden von gleichzeitigen Nullstellen der Funktionen $g_i$ mit
\begin{align*}
g_1(x_1,\dots,x_n)&=f_1(x_1,\dots,x_n)-y_1=0\\
g_2(x_1,\dots,x_n)&=f_2(x_1,\dots,x_n)-y_2=0\\
&\;\;\vdots\\
g_m(x_1,\dots,x_n)&=f_m(x_1,\dots,x_n)-y_m=0.
\end{align*}
Wir k"onnen dies auch in Vektor-Form schreiben,
wir betrachten $x$ als Vektor der $x_1,\dots,x_n$









\input{chapters/komplexezahlen.tex}
\end{appendices}
\vfill
\pagebreak
\ifodd\value{page}\else\null\clearpage\fi
\lhead{Literatur}
\rhead{}
\printbibliography[heading=subbibliography]
\label{skript:literatur}
\end{refsection}

\part{Anwendungen und Weiterf"uhrende Themen}
\lhead{Anwendungen}
%
% uebersicht.tex -- Uebersicht ueber die Seminar-Arbeiten
%
% (c) 2015 Prof Dr Andreas Mueller, Hochschule Rapperswil
%
\chapter*{"Ubersicht}
\lhead{"Ubersicht}
\rhead{}
\label{skript:uebersicht}
Im zweiten Teil kommen die Teilnehmer des Seminars selbst zu Wort.
Sie zeigen Anwendungsbeispiele f"ur die im ersten
Teil entwickelte Theorie der gew"ohnlichen Differentialgleichungen.



\def\chapterauthor#1{{\large #1}\bigskip\bigskip}
% Artikel
\chapter{Wo steckt die zweite L"osung?\label{chapter:komplex}}
\lhead{Bessel-Funktionen zweiter Art}
\begin{refsection}
\chapterauthor{Stefan Kull und Roy Seitz}

\printbibliography[heading=subbibliography]
\end{refsection}

\chapter{Wo steckt die zweite L"osung?\label{chapter:komplex}}
\lhead{Bessel-Funktionen zweiter Art}
\begin{refsection}
\chapterauthor{Stefan Kull und Roy Seitz}

\printbibliography[heading=subbibliography]
\end{refsection}

\chapter{Wo steckt die zweite L"osung?\label{chapter:komplex}}
\lhead{Bessel-Funktionen zweiter Art}
\begin{refsection}
\chapterauthor{Stefan Kull und Roy Seitz}

\printbibliography[heading=subbibliography]
\end{refsection}

\chapter{Wo steckt die zweite L"osung?\label{chapter:komplex}}
\lhead{Bessel-Funktionen zweiter Art}
\begin{refsection}
\chapterauthor{Stefan Kull und Roy Seitz}

\printbibliography[heading=subbibliography]
\end{refsection}

\chapter{Wo steckt die zweite L"osung?\label{chapter:komplex}}
\lhead{Bessel-Funktionen zweiter Art}
\begin{refsection}
\chapterauthor{Stefan Kull und Roy Seitz}

\printbibliography[heading=subbibliography]
\end{refsection}

\chapter{Wo steckt die zweite L"osung?\label{chapter:komplex}}
\lhead{Bessel-Funktionen zweiter Art}
\begin{refsection}
\chapterauthor{Stefan Kull und Roy Seitz}

\printbibliography[heading=subbibliography]
\end{refsection}

\chapter{Wo steckt die zweite L"osung?\label{chapter:komplex}}
\lhead{Bessel-Funktionen zweiter Art}
\begin{refsection}
\chapterauthor{Stefan Kull und Roy Seitz}

\printbibliography[heading=subbibliography]
\end{refsection}

\chapter{Wo steckt die zweite L"osung?\label{chapter:komplex}}
\lhead{Bessel-Funktionen zweiter Art}
\begin{refsection}
\chapterauthor{Stefan Kull und Roy Seitz}

\printbibliography[heading=subbibliography]
\end{refsection}

\vfill
\pagebreak
\ifodd\value{page}\else\null\clearpage\fi
\lhead{Index}
\rhead{}
%
% skript.tex -- Skript ueber Differentialgleichungen
%
% (c) 2014 Prof. Dr. Andreas Mueller, HSR
%
\documentclass{book}
\usepackage{etex}
\usepackage{geometry}
\geometry{papersize={170mm,240mm},total={140mm,200mm},top=21mm,bindingoffset=10mm}
\usepackage[english,ngerman]{babel}
\usepackage{times}
\usepackage{amsmath,amscd}
\usepackage{amssymb}
\usepackage{amsfonts}
\usepackage{amsthm}
\usepackage{graphicx}
\usepackage{fancyhdr}
\usepackage{textcomp}
\usepackage[all]{xy}
\usepackage{txfonts}
\usepackage{alltt} 
\usepackage{verbatim}
\usepackage{paralist}
\usepackage{makeidx}
\usepackage{array}
\usepackage[colorlinks=true]{hyperref}
\usepackage{tikz}
\usepackage{pgfplots}
\usepackage{pgfplotstable}
\usepackage{pdftexcmds}
%\usepackage{pgfmath}
\usepackage{placeins}
\usepackage{subfigure}
\usepackage[autostyle=false,english=american]{csquotes}
\usepackage{float}
\usepackage{enumitem}
\usepackage{wasysym}
\usepackage{environ}
\usepackage{pifont}
\usepackage{feynmp}
\usepackage{appendix}
\usetikzlibrary{calc,intersections,through,backgrounds,graphs,positioning,shapes,arrows,fit}
\usetikzlibrary{patterns,decorations.pathreplacing}
\usetikzlibrary{decorations.pathreplacing}
\usetikzlibrary{external}
\usepackage[europeanvoltages,
            europeancurrents,
            europeanresistors,   % rectangular shape
            americaninductors,   % "4-bumbs" shape
            europeanports,       % rectangular logic ports
            siunitx,             % #1<#2>
            emptydiodes,
            noarrowmos,
            smartlabels]         % lables are rotated in a smart way
           {circuitikz}          %
\usepackage{siunitx}
\usepackage{tabularx}
\usetikzlibrary{arrows}

\usepackage{algpseudocode}
\usepackage{algorithm}

\usepackage{listings}
\lstdefinestyle{Matlab}{
  numbers=left,
  belowcaptionskip=1\baselineskip,
  breaklines=true,
  frame=L,
  xleftmargin=\parindent,
  language=Matlab,
  showstringspaces=false,
  basicstyle=\footnotesize\ttfamily,
  keywordstyle=\bfseries\color{green!40!black},
  commentstyle=\itshape\color{purple!40!black},
  identifierstyle=\color{blue},
  stringstyle=\color{orange},
  numberstyle=\ttfamily\tiny
}
\lstdefinelanguage{Maxima}{
  keywords={addrow,addcol,zeromatrix,ident,augcoefmatrix,ratsubst,diff,ev,tex,%
    with_stdout,nouns,express,depends,load,submatrix,div,grad,curl,matrix,%
    invert,lambda,facsum,expand,false,then,if,else,subst,%
    rootscontract,solve,part,assume,sqrt,integrate,abs,inf,exp,sin,cos,sinh,cosh},
  sensitive=true,
  comment=[n][\itshape]{/*}{*/}
}
\lstdefinestyle{Maxima}{
  numbers=left,
  belowcaptionskip=1\baselineskip,
  breaklines=true,
  frame=L,
  xleftmargin=\parindent,
  language=Maxima,
  showstringspaces=false,
  basicstyle=\footnotesize\ttfamily,
  keywordstyle=\bfseries\color{green!40!black},
  commentstyle=\itshape\color{purple!40!black},
  identifierstyle=\color{blue},
  stringstyle=\color{orange},
  numberstyle=\ttfamily\tiny
}
\usepackage{caption}
\usepackage[mode=buildnew]{standalone}
\usepackage[backend=bibtex]{biblatex}
\addbibresource{references.bib}
% Bibresources für jeden einzelnen Artikel
%\addbibresource{thema/main.bib}
\AtEndDocument{\clearpage\ifodd\value{page}\else\null\clearpage\fi}
\makeindex
%\pgfplotsset{compat=1.12}
\setlength{\headheight}{15pt} % fix headheight warning
\DeclareGraphicsRule{*}{mps}{*}{}
\begin{document}
\pagestyle{fancy}
\frontmatter
\newcommand\HRule{\noindent\rule{\linewidth}{1.5pt}}
\begin{titlepage}
\vspace*{\stretch{1}}
\HRule
\vspace*{5pt}
\begin{flushright}
{
\LARGE
Mathematisches Seminar\\
\vspace*{20pt}
\Huge
Differentialgleichungen%
}
\vspace*{5pt}
\end{flushright}
\HRule
\begin{flushright}
\vspace{60pt}
\Large
Leitung: Andreas M"uller\\
\vspace{40pt}
\Large
Reto~Christen,
Kevin~Cina,
Andri~Hartmann,
Pascal~Horat %,
Matthias~Kn"opfel,
Stefan Kull,
Daniela~Meier,
Max~Obrist %,
Hansruedi~Patzen,
Benjamin~R"aber,
Simon~Schaefer %,
Tibor~Schneider,
Tobias~Schuler,
Roy~Seitz,
Martin~Stypinski
\end{flushright}
\vspace*{\stretch{2}}
\begin{center}
Hochschule f"ur Technik, Rapperswil, 2016
\end{center}
\end{titlepage}
\hypersetup{
    linktoc=all,
    linkcolor=blue
}
\newcounter{beispiel}
\newenvironment{beispiele}{
\bgroup\smallskip\parindent0pt\bf Beispiele\egroup

\begin{list}{\arabic{beispiel}.}
  {\usecounter{beispiel}
  \setlength{\labelsep}{5mm}
  \setlength{\rightmargin}{0pt}
}}{\end{list}}
\newcounter{uebungsaufgabe}
% environment fuer uebungsaufgaben
\newenvironment{uebungsaufgaben}{
\begin{list}{\arabic{uebungsaufgabe}.}
  {\usecounter{uebungsaufgabe}
  \setlength{\labelwidth}{2cm}
  \setlength{\leftmargin}{0pt}
  \setlength{\labelsep}{5mm}
  \setlength{\rightmargin}{0pt}
  \setlength{\itemindent}{0pt}
}}{\end{list}\vfill\pagebreak}
\newenvironment{teilaufgaben}{
\begin{enumerate}
\renewcommand{\labelenumi}{\alph{enumi})}
}{\end{enumerate}}
% Loesung
\def\swallow#1{
%nothing
}
\NewEnviron{loesung}[1][L"osung]{%
\begin{proof}[#1]%
\renewcommand{\qedsymbol}{$\bigcirc$}
\BODY
\end{proof}
}
\NewEnviron{bewertung}{%
\begin{proof}[Bewertung]%
\renewcommand{\qedsymbol}{}
\BODY
\end{proof}
}
\NewEnviron{diskussion}{%
\begin{proof}[Diskussion]%
\renewcommand{\qedsymbol}{}
\BODY
\end{proof}
}
\NewEnviron{hinweis}{%
\begin{proof}[Hinweis]%
\renewcommand{\qedsymbol}{}
\BODY
\end{proof}
}
\def\keineloesungen{%
\RenewEnviron{loesung}{\relax}
\RenewEnviron{bewertung}{\relax}
\RenewEnviron{diskussion}{\relax}
}
\newenvironment{beispiel}{%
\begin{proof}[Beispiel]%
\renewcommand{\qedsymbol}{$\bigcirc$}
}{\end{proof}}

\input{linsys.tex}
\allowdisplaybreaks

\lhead{Inhaltsverzeichnis}
\rhead{}
\tableofcontents
\newtheorem{satz}{Satz}[chapter]
\newtheorem{hilfssatz}[satz]{Hilfssatz}
\newtheorem{definition}[satz]{Definition}
\newtheorem{annahme}[satz]{Annahme}
\renewcommand{\floatpagefraction}{0.75}
\mainmatter
%
% vorwort.tex -- Vorwort zum Buch zum Seminar
%
% (c) 2015 Prof Dr Andreas Mueller, Hochschule Rapperswil
%
\chapter*{Vorwort}
\lhead{Vorwort}
\rhead{}
Dieses Buch entstand im Rahmen des Mathematischen Seminars
im Fr"uhjahrssemester 2016 an der Hochschule f"ur Technik Rapperswil.
Die Teilnehmer, Studierende der Abteilungen f"ur Elektrotechnik,
Informatik und Bauingenieurwesen der
HSR, erarbeiteten nach einer Einf"uhrung in das Themengebiet jeweils
einzelne Aspekte des Gebietes in Form einer Seminararbeit, "uber
deren Resultate sie auch in einem Vortrag informierten. 

Im Fr"uhjahr 2016 war das Thema des Seminars ``Differentialgleichungen''.
Die Einf"uhrung bestand aus einigen Vorlesungsstunden, deren
Inhalt im ersten Teil dieses Skripts zusammengefasst ist.
Es ging darum, die zum Teil aus dem Analysis-Unterricht bekannte
Theorie der Differentialgleichungen zu vertiefen, mit anderen Gebieten
wie zum Beispiel der komplexen Analysis zu verkn"upfen und sie
auf die Analyse einiger relevanter Praxisprobleme anzuwenden.
Dabei ging es nicht um die analytische L"osung von Differentialgleichungen,
die meisten Differentialgleichungen lassen sich ohnehin nicht in
geschlossener Form l"osen.
Einzelne Differentialgleichungen wurden untersucht, weil sie Anlass
zu einer wichtigen Familie von Funktionen geben, zum Beispiel die
Bessel- und Airy-Funktionen.
In anderen Beispielen ging es um die Schwierigkeiten, die bei einer
numerischen L"osung zu meistern sind.
Besonders anspruchsvoll sind jedoch "Uberlegungen zum Verhalten der
L"osung f"ur lange Zeiten, zum Beispiel Stabilit"at, das Auftreten
von Schwingungen bei der Hopf-Bifurkation oder der "Ubergang zum
Chaos.

Im zweiten Teil dieses Skripts kommen dann die Teilnehmer selbst zu Wort.
Ihre Arbeiten wurden jeweils als einzelne
Kapitel mit meist nur typographischen "Anderungen "ubernommen.
Diese weiterf"uhrenden Kapitel sind sehr verschiedenartig.
Eine "Ubersicht und Einf"uhrung befindet sich in der Einleitung
zum zweiten Teil auf Seite~\pageref{skript:uebersicht}.

In einigen Arbeiten wurde auch Code zur Demonstration der 
besprochenen Methoden und Resultate geschrieben, soweit
m"oglich und sinnvoll wurde dieser Code im Github-Repository
dieses Kurses\footnote{\url{https://github.com/AndreasFMueller/SeminarDGL.git}}
abgelegt, in anderen F"allen verweisen die Artikel selbst auf
das zugeh"orige Code-Repository.

Im genannten Repository findet sich auch der Source-Code dieses
Skriptes, es wird hier unter einer Creative Commons Lizenz
zur Verf"ugung gestellt.


\part{Grundlagen}
%\keineloesungen
\begin{refsection}
\section{Einleitung}
Wellen umgeben uns st"andig, selbst wenn wir es uns dem nicht direkt bewusst 
sind. Sei es nun in Form von Lichtwellen, Schallwellen, Wasserwellen und vielen 
mehr.

Die Wasserwellen welche ans Ufer schlagen, sind wohl ein Beispiel, bei dem sich 
jeder etwas darunter vorstellen kann. Dieses an sich sch"one Naturschauspiel 
kann aber auch destruktiv sein, so kann eine pl"otzliche Absenkung am 
Meeresgrund beispielsweise einen Tsunami ausl"osen.

In diesem Kapitel wird anfangs die Titelgleichung, welche die Ausbreitung einer 
Welle in einem parabolischen Kanal beschreibt, genauer analysiert. Gegen Ende 
soll aber auch noch eine allgemeinere Kanalform untersucht werden.

Nat"urlich sind diese Gleichungen nur eine Ann"aherung und entsprechen nur 
bedingt der in der Realit"at vorkommenen Wellenausbreitung. So kann sich eine 
Wasserwelle zum Beispiel "uberschlagen, was hier so nicht abgebildet wird. 
Trotzdem kann anhand von diesen Modellrechnungen versucht werden, die 
Ausbreitung einer Welle nachzuvollziehen.

%\input{kapitel.tex}
%
% grundlagen.tex -- Grundlagen ueber Differentialgleichungen
%
% (c) 2015 Prof Dr Andreas Mueller, Hochschule Rapperswil
%
\chapter{Grundlagen der Theorie der gew"ohnlichen Differentialgleichungen
\label{chapter:grundlagen}}
\lhead{}
\rhead{Grundlagen}
\section{Differentialgleichungen\label{section:differentialgleichungen}}
Eine gew"ohnliche Differentialgleichung f"ur eine reellwertige
Funktion $y(x)$ stellt einen Zusammenhang her zwischen der Funktion
und ihren Ableitungen.
Wir schreiben die Ableitungen als $y'$, $y''$, $y'''$ und $y^{(n)}$
f"ur die $n$-te Ableitung.
Wir lassen oft das Argument der Funktion weg.
Beispiele von Differentialgleichungen sind
\begin{align*}
y'&=-Ny
&&\text{Ordnung: $1$}
\\
y''&=-\omega^2 y
&&\text{Ordnung: $2$}
\\
x^2y''+xy'+(x^2-n^2)y&=0
&&\text{Ordnung: $2$}
\end{align*}
Die Abh"angigkeit kann in expliziter Form als
\begin{equation}
y^{(n)}=f(x,y,y',\dots,y^{(n-1)})
\label{grundlagen:explizit}
\end{equation}
oder in impliziter Form
\[
F(x,y,y',\dots,y^{(n)})=0
\]
gegeben sein.
Die Ordnung einer Differentialgleichung ist die h"ochste vorkommende
Ableitung.

Insbesondere in Anwendungen in der Physik ist die Zeit die
unabh"angige Variable.
Die abh"angige Variable ist dann zum Beispiel die Ortskoordinate
$x(t)$ und wir bezeichnen ihre Ableitungen mit $\dot{x}(t)$ f"ur
die Geschwindigkeit, $\ddot{x}(t)$ f"ur die Beschleunigung.
Dieses Beispiel suggeriert auch, dass die abh"angige Variable 
ein Vektor sein kann, den man als den Ortsvektor eines Teilchens
interpretieren kann.
Die Funktion $f(t,x,\dots,x^{(n-1)})$ ist dann auch vektorwertig, und
alle Argumente ausser dem ersten von $f$ sind vektorwertig.

Eine Differentialgleichung $n$-ter Ordnung f"ur eine skalare Funktion
kann in eine Vektor-Differentialgleichung erster Ordnung f"ur eine
$n$-dimensionale vektorwertige Funktion umgewandelt werden.
Ist $y(x)$ die gesuchte Funktion in der
Differentialgleichung~(\ref{grundlagen:explizit}), dann kann man
den Vektor
\[
u(x)=\begin{pmatrix}
y(x)\\y'(x)\\\vdots\\y^{(n-1)}(x)
\end{pmatrix}
\in\mathbb R^n
\]
bilden.
Er erf"ullt die Differentialgleichung
\begin{equation}
\frac{d}{dx}\begin{pmatrix}
y\\y'\\\vdots\\y^{(n-1)}
\end{pmatrix}
=
\begin{pmatrix}
y'\\y''\\\vdots\\y^{(n)}
\end{pmatrix}
=
\begin{pmatrix}
y'\\y''\\\vdots\\f(x,y,y',\dots,y^{(n-1)}.
\end{pmatrix}
\label{grundlagen:vektordgl}
\end{equation}
Der Vektor auf der rechten Seite h"ang nur von $x$, der Funktion $y$
und ihren Ableitungen bis zur $n-1$-ten Ordnung ab, also von $u$, man
kann (\ref{grundlagen:vektordgl}) daher als
\begin{equation}
\frac{d}{dx}u=\tilde{f}(x,u)
\end{equation}
schreiben.

\section{Anfangswertprobleme\label{section:anfangswertprobleme}}

\section{Randwertprobleme\label{section:randwertprobleme}}

\section{Analytische L"osungsverfahren\label{section:analytischeverfahren}}
\subsection{Separation der Variablen}
Differentialgleichungen erster Ordnung lassen sich oft durch sogenannte
Trennung der Variablen auf die Berechnung von Integralen reduzieren.
Dank der Schreibweise der Ableitung als Differentialquotient wird
dieser L"osungsweg sehr suggestiv.
Wir betrachten als Beispiel die Differentialgleichung
\[
y'=-Ny.
\]
Schreibt man die Ableitung als Differentialquotient, wird daraus die
Gleichung
\[
\frac{dy}{dx}=-Ny.
\]
Durch Division durch $y$ und formale Multiplikation mit $dx$ wird daraus
die formale Gleichung
\begin{equation}
\frac{dy}{y}=-N\,dx.
\label{grundlagen:separiert}
\end{equation}
In dieser Gleichung kommt die Variable $y$ nur auf der linken, die Variable
$x$ nur auf der rechten Seite vor.
Man sagt, die Variablen seien {\em separiert}.
\index{separiert}
\index{Variablen, Separation der}
Man beachte, dass die Gleichung (\ref{grundlagen:separiert}) nur eine
formale Bedeutung haben kann, die Symbole $dy$ und $dx$ sind ja keine Zahlen,
mit denen man algebraische Operationen durchf"uhren k"onnte.
Mit etwas Vorsicht angewandt f"uhrt dieser Kalk"ul aber nicht auf
Widerspr"uche.

Wir integrieren jetzt beide Seiten von (\ref{grundlagen:separiert}), und
erhalten 
\[
\int\frac1y\,dy=-N\int\,dx
\]
Beide Integrale lassen sich in geschlossener Form auswerten:
\[
\log|y|=-Nx+C.
\]
Aufgel"ost nach $y$ ergibt sich
\[
y=\pm e^{C}e^{-Nx},
\]
wobei die beiden Vorzeichen $\pm$ das Betragszeichen in der Stammfunktion
von $\frac1y$ reflektieren.
Man kann den Faktor $\pm e^{C}$ in eine neue Konstante $a$ zusammenfassen,
und erh"alt somit als L"osung der urspr"unglichen Differentialgleichung
die Familie
\[
y(x)=ae^{-Nx}
\]
von Funktionen.
Der Paramter $a$ muss mit Hilfe der Anfangsbedingung festgelegt werden.

\subsection{Lineare Differentialgleichungen}
Eine Differentialgleichung der Form
\begin{equation}
a_n(x)y^{(n)}+a_{n-1}(x)y^{(n-1)}+\dots+a_2(x)y''+a_1(x)y'+a_0(x)=f(x)
\label{grundlagen:linearedgl}
\end{equation}
heisst {\em lineare Differentialgleichung}.
\index{lineare Differentialgleichung}
Ist $f(x)=0$, nennt man die Differentialgleichung {\em homogen}, die
Funktion $f(x)$ wird auch die {\em Inhomogenit"at} genannt.
\index{homogen Differentialgleichung}
\index{Inhomogenitat@Inhomogenit\"at}
Die L"osungsmenge einer homogenen linearen Differentialgleichung
bildet einen Vektorraum: jede Linearkombination von L"osungen
ist wieder eine L"osung.
Seien zum Beispiel $y_1(x)$ und $y_2(x)$ L"osungen der Differentialgleichung
(\ref{grundlagen:linearedgl}).
Wir m"ochten zeigen, dass
$y(x)=\alpha y_1(x)+\beta y_2(x)$ eine L"osung ist.
Die Ableitungen selbst sind linear:
\begin{align*}
y^{(k)}&=\alpha y_1^{(k)}(x)+\beta y_2^{(k)}(x).
\end{align*}
Setzt man dies in die Differentialgleichung ein, erh"alt man
\begin{align*}
a_ny^{(n)}+\dots+a_1y'+a_0y
&=
a_n\alpha y_1^{(n)}+a_n\beta y_2^{(n)}+\dots+a_1\alpha y_1'+a_1\beta y_2'
+ a_0\alpha y_1+a_0\beta y_2
\\
&=
\alpha(\underbrace{a_ny_1^{(n)}+\dots+a_1y_1'+a_0y_1}_{=0})
+
\beta(\underbrace{a_ny_2^{(n)}+\dots+a_1y_2'+a_0y_2}_{=0})=0,
\end{align*}
die Linearkombination $y$ erf"ullt also die homogene Differentialgleichung
ebenfalls.


\subsection{Variation der Konstanten}
\subsection{Laplace-Transformation}

%
% numerik.tex -- numerische Lösung von gewöhnlichen Differentialgleichungen
%
% (c) 2015 Prof Dr Andreas Mueller, Hochschule Rapperswil
%
\chapter{Numerische L"osung\label{chapter:numerik}}
\lhead{}
\rhead{Numerische L"osung}
\index{Numerische Loesung@Numerische L\"osung}
Im Kapitel~\ref{chapter:grundlagen} waren wir in der Lage, f"ur einige
einfache Differentialgleichungen eine L"osung in geschlossener Form
zu finden.
Zum Beispiel konnten wir lineare Differentialgleichungen mit Hilfe
der Exponentialfunktion l"osen.
Dieses Bild tr"ugt allerdings.
Die meisten Differentialgleichungen k"onnen nicht in geschlossener
Form gel"ost werden.
Wir k"onnen daher nicht erwarten, dass wir die L"osungen beliebiger
Differentialgleichungen einfach dadurch verstehen, dass wir
L"osungsfunktionen diskutieren.
Stattdessen bleiben uns nur die folgenden zwei M"oglichkeiten:
\begin{enumerate}
\item
Wir l"osen die Differentialgleichung mit Hilfe eines Computers,
und studieren den Verlauf der L"osungsfunktionen oder die Abh"angigkeit
von Parameter oder Anfangsbedingungen durch Vergleich verschiedener
numerisch gefundener L"osungen.
\item
Wir entwickeln Methoden, mit denen sich Aussagen "uber den Verlauf der
L"osungskurven studieren lassen, ohne dass man sie berechnet haben muss.
Nat"urlich kann man nicht erwarten, dass eine solche Methode genaue
Aussagen dar"uber erlaubt, wann eine L"osungskurve wo genau durchgehen
wird.
Es werden nur qualitative Aussagen m"oglich sein, zum Beispiel ob
Gleichgewichtsl"osungen stabil sind, ob es periodische L"osungen gibt
und ob L"osungskurven zu den periodischen L"osungen konvergieren.
\end{enumerate}
In diesem Kapitel entwickeln wir Methoden, Differentialgleichungen 
numerisch zu l"osen.

\section{Grundprinzip}
\lhead{Grundprinzip}
\begin{figure}
\centering
\includegraphics{chapters/images/numerik-2.pdf}
\caption{Lineare Approximation von $y(x+\Delta x)$ durch Information,
die am Punkt $x$ verf"ugbar ist.
\label{numerik:lineareapproximation}}
\end{figure}
Wir versuchen die Differentialgleichung
\begin{equation}
y'=-\alpha y,\qquad y(0)=y_0
\label{numerik:expdgl}
\end{equation}
numerisch zu l"osen. 
Dazu unterteilen wir die $x$-Achse in diskrete Abschnitte der L"ange $h$,
und bezeichnen die Teilpunkte mit $x_k=kh$.
Das Ziel ist jetzt, $y(x_k)$ n"aherungsweise zu berechnen.
Wir schreiben $y_k$ f"ur die N"aherungswerte von $y(x_k)$.
Die Ableitung liefert eine lineare Approximation f"ur $y(x)$,
n"amlich
\[
y(x+\Delta x)\simeq y(x) + y'(x)\cdot\Delta x
\]
(Abbildung~\ref{numerik:lineareapproximation}).
F"ur die Punkte $x_k$ bedeutet das
\[
y(x_{k+1})\simeq y(x_{k})+y'(x_k)h.
\]
Die Differentialgleichung liefert Werte f"ur $y'(x_k)$ aus $x_k$ und $y(x_k)$,
damit k"onnen wir aus dieser Approximation ein allgemeines
N"aherungsverfahren f"ur die L"osung einer Differentialgleichung
konstruieren.

\begin{satz}[Euler-Verfahren]
\index{Euler-Verfahren}
Die Differentialgleichung
\begin{equation}
y'=f(x,y),\qquad y(0)=y_0
\label{numerik:eulerdgl}
\end{equation}
und die Schrittweite $h$ definieren eine Folge 
\[
y_{\mathstrut k}=y_{k-1} + h\cdot f(x_{k-1}, y_{k-1}),\quad k>0,
\]
mit $x_k=kh$,
die eine N"aherung f"ur die Funktionswerte $y(x_k)$ der L"osung $y(x)$
der Differentialgleichung~(\ref{numerik:eulerdgl}) ist.
\end{satz}

Dieses Verfahren ist nicht besonders gut, wie wir im Folgenden zeigen
wollen.
Die Diskussion soll uns aber zeigen, worauf bei der Weiterentwicklung
des Verfahrens geachtet werden muss.

Im vorliegenden Beispiel liefert die
Differentialgleichung~(\ref{numerik:expdgl})
den Wert $y'(x_k)=-\alpha y(x_k)$ f"ur die Ableitung,
woraus wir die Rekursionsformel
\[
y_{k+1}=y_k - \alpha y_k \dot h.
\]
gewinnen.
Die Rekursionsgleichung kann in diesem Fall exakt gel"ost werden,
und wir finden
\begin{equation}
y(x_{k+1}) = y(x_k)-\alpha y(x_k) h=(1-\alpha h) y(x_k)=\dots
=(1-\alpha h)^{k+1}y_0
\label{numerik:rekursion}
\end{equation}
f"ur die N"aherung $y_k$ der Funktionswerte $y(x_k)$.
%Angewendet auf eine beliebige Differentialgleichung, ist dieses
%einfache numerische Verfahren bekannt als das {\em Euler-Verfahren}.
%Es ist nicht besonders genau, aber soll in diesem Abschnitt dazu
%dienen, die Anforderungen an ein gutes numerisches Verfahren
%zu illustrieren.



Wir m"ochten $y(x)$ f"ur einen ganz bestimmten $x$-Wert berechnen.
Dazu unterteilen wir das Intervall $[0,x]$ in $n$ Teilschritte der
Breite $x/n$, und wenden die Formel~(\ref{numerik:rekursion}) an:
\[
y(x)=y(x_n)=(1-\alpha h)^n y_0=\biggl(1+\frac{-\alpha x}{n}\biggr)^n y_0.
\]
F"ur eine grosse Zahl von Teilschritten erhalten wir so tats"achlich die
korrekte L"osung:
\[
\lim_{n\to\infty}y_0\biggl(1+\frac{-\alpha x}n\biggr)^n=y_0 e^{-\alpha x}.
\]
\begin{figure}
\centering
\includegraphics{chapters/images/numerik-1.pdf}
\caption{Approximationen der L"osung der Differentialgleichung $y'=-\alpha y$
mit verschiedener Anzahl Schritte (rot) n"ahern sich f"ur wachsendes
$n$ der exakten L"osung (blau).
\label{numerik:approximation}}
\end{figure}%
Abbildung~\ref{numerik:approximation} zeigt, wie die
durch~(\ref{numerik:rekursion}) gegebenen Approximationen mit zunehmendem
$n$ der exakten L"osung $y(x)=e^{-\alpha x}$ n"aher kommen.

Wir k"onnen auch den Fehler des numerischen Verfahrens berechnen.
Bei der Schrittweite $h$ ist der Fehler von $y_k$ die Differenz
\[
y(x_k)-y_k
=
y_0e^{-\alpha kh}-y_0(1-\alpha h)^k
=
y_0((e^{-\alpha h})^k - (1-\alpha h)^k)
=
y_0e^{-\alpha hk}\biggl(
1-\biggl(\frac{1-\alpha h}{e^{-\alpha h}}\biggr)^k
\biggr).
\]
Man beachte, dass der Z"ahler $1-\alpha h$ die Approximation
$y_1$ ist, als eine Approximation von $e^{-\alpha h}$, dem Nenner.
Schreiben wir
\[
q=\frac{1-\alpha h}{e^{-\alpha h}},
\]
f"ur den Quotienten zwischen der Approximation und dem korrekten Wert,
dann ist sicher immer $q<1$.
Den Fehler k"onnen wir jetzt schreiben
\[
y(x_k)-y_k = y_0e^{-\alpha hk}(1-q^k) = y(x_k)(1-q^k).
\]
Der relative Fehler des Verfahrens ist also
\[
\frac{y(x_k)-y_k}{y(x_k)}=(1-q^k).
\]
\begin{figure}
\centering
\includegraphics{chapters/images/numerik-3.pdf}
\caption{Relativer Fehler des Euler-Verfahrens f"ur die Differentialgleichung
(\ref{numerik:expdgl}) in Abh"angigkeit von der Anzahl $k$ der Schritte.
\label{numerik:relfehler}}
\end{figure}%
Ganz unabh"angig von der Schrittweite $h$ wird der relative Fehler
des Verfahrens immer gegen 1 streben, der Fehler wird also von der
gleichen Gr"ossenordnung wie die berechneten Resultate.

Die Abbildung~\ref{numerik:relfehler} zeigt, dass zu Beginn des Verfahrens
der relative Fehler ungef"ahr linear mit der Anzahl der Schritt zunimmt.
Um eine angemessene Genauigkeit "uber einen gr"osseren Bereich
zu erreichen, muss das Euler-Verfahren also sehr viel kleinere Schritte
und eine entsprechend gr"ossere Anzahl von Schritten ausf"uhren,
die entsprechend viel Rechenzeit ben"otigen.

Ein praktisch n"utzliches Verfahren muss also anstreben, mit einer
sehr viel kleineren Anzahl von Schritten eine viel gr"ossere Genaugikeit
der Approximation zu erreichen.

\section{Fehler-Entwicklung numerischer L"osungen}
\lhead{Fehler-Entwicklung}
Wir betrachten wieder die Differentialgleichung~(\ref{numerik:eulerdgl})
und versuchen, den Fehler eines N"aherungsverfahrens zu bestimmen,
welches Schritte der Gr"osse $h$ durchf"uhrt, um den Wert $y(x)$
zu approximieren.

Das Euler-Verfahren verwendet Schritte der Form
\[
y_{k+1}=y_{k\mathstrut} + hf(x_{k\mathstrut},y_{k\mathstrut}).
\]
In jedem einzelnen Schritt entsteht ein Fehler, dessen Gr"osse wir
aus der Taylor-Entwicklung
\[
y(x+\Delta x)=
y(x) + y'(x)\cdot \Delta x + R(x) \Delta x^2
\]
absch"atzen k"onnen.
Die Funktion $R(x)$ ist beschr"ankt und beschreibt den verbleibenden
Fehler.
Um $y(x)$ zu approximieren, m"ussen $n=x/h$ Schritte der Schrittweite
$h$ durchgef"uhrt werden, von denen jeder einen Fehler
von der Gr"ossenordnung $R(x)h^2$ hat.
Der Gesamtfehler ist daher von der Gr"ossenordnung
\[
y(x)-y_n=O\biggl(R(x)h^2\frac{x}h\biggr)=O(h),
\]
er ist also von erster Ordnung in $h$.
Um eine zus"atzliche Stelle Genauigkeit zu erhalten, muss man also zehnmal
so viele Schritte von zehnmal kleinerer Gr"osse durchf"uhren,
wodurch auch wieder Rundungsfehler eingef"uhrt werden.

K"onnte man den Fehler des Einzelschrittes wesentlich verkleinern, w"urde
auch die Abh"angigkeit des Fehlers des Verfahrens vorteilhafter.
W"are der Fehler des Einzelschrittes $O(h^k)$ statt $O(h^2)$, dann
w"are der Gesamtfehler des Verfahrens nur noch $O(h^{k-1})$.
F"ur $k=3$ bedeutet dies, dass eine Halbierung der Schrittweite
zwar doppelt so viele Schritte braucht, aber auch, dass in jedem
Schritt nur ein Achtel des Fehlers auftritt.
Der Gesamtfehler ist also nur ein Viertel.
Mit zehnmal mehr Arbeit kann man also nicht nur eine Stelle an
Genauigkeit gewinnen, sondern gleich deren zwei.

Man nennt ein Verfahren, bei dem der Gesamt-Fehler von der Gr"ossenordnung
$O(h^k)$ ist, von einem Verfahren $k$-ter Ordnung.
Das Euler-Verfahren ist also ein Verfahren erster Ordnung oder ein
lineares Verfahren.
In der Praxis werden Verfahren bis zu vierter und f"unfter Ordnung
verwendet, so dass eine zehnmal kleinere Schrittweite zu gleich
vier Stellen Genauigkeitsgewinn f"uhren.
Das Ziel der kommenden Abschnitte muss daher sein, einfach
berechnebare Approximationen der Funktion mit m"oglichst geringen
Einzelschrittfehlern zu finden.

\section{Einschritt-Verfahren\label{section:numerik:einschritt}}
\lhead{Einschritt-Verfahren}
Die relativ geringe Genauigkeit des Eulerschrittes beruht darauf,
dass die zu Beginn des Schrittes berechnete Ableitung $f(x_k,y_k)$
nur f"ur das linke Ende des Intervalls $[x_k, x_k+h]$ zutrifft,
weiter rechts im Intervall wird die Abweichung immer gr"osser.
Eine m"ogliche L"osung des Problems k"onnte darin bestehen, statt
nur einer linearen N"aherung zus"atzliche Glieder der Taylorreihe
\begin{equation}
y(x+\Delta x)
=
y(x)
+
y'(x)\cdot \Delta x
+
\frac12 y''(x)\cdot \Delta x^2
+
\frac16 y'''(x)\cdot \Delta x^3
+
o(\Delta x^3)
\label{numerik:taylor}
\end{equation}
zu verwenden.
In (\ref{numerik:taylor}) werden h"ohere Ableitungen von $y(x)$ ben"otigt,
w"ahrend die Differentialgleichung nur die erste Ableitung liefert.
Die h"oheren Ableitungen wurden aber bereits im
Abschnitt~\ref{grundlagen:hoehere-ableitungen} berechnet.

Wir untersuchen, wie sich das Verfahren f"ur die Beispiel-Gleichung
(\ref{numerik:expdgl}) anwenden l"asst.
Dort gilt
\begin{equation*}
\begin{aligned}
y'(x)&=f(x,y)=-\alpha y
\\
\Rightarrow\qquad
\frac{\partial f}{\partial x}&=0&\frac{\partial f}{\partial y}&=-\alpha
\end{aligned}
\end{equation*}
Alle zweiten Ableitungen verschwinden.
Die Gleichungen werden damit einfach:
\begin{align*}
y''(x)&=-\alpha f(x,y)=\alpha^2 y
\\
y'''(x)&=\alpha^2f(x,y)=-\alpha^3 y.
\end{align*}
Statt der linearen Approximation sollte daher die kubische Approximation
\begin{equation}
y_{k+1}
=
y_{k}-\alpha h y_k +\frac12\alpha^2 h^2 y_k -\frac16 \alpha^3h^3 y_k
=
y_{k}\underbrace{\biggl(1-\alpha h +\frac12\alpha^2h^2 -\frac16 \alpha^3h^3\biggr)}_{\simeq e^{-\alpha h}}
\label{numerik:kubisch}
\end{equation}
verwendet werden.
Dass man hier mit einer gr"osseren Genauigkeit rechnen darf ist schon daran
erkennbar, dass der Klammerausdruck auf der rechten Seite eine viel
bessere Approximation von $e^{-\alpha x}$ ist also der Faktor
$(1-\alpha h)$ im Euler-Verfahren.
Genauer erwarten wir, dass wir hier ein kubisches Verfahren konstruiert haben.

\begin{table}
\centering
\begin{tabular}{|r|c|r|r|r|}
\hline
$i$&$x$&$e^{-\alpha x}$&Euler&kubisch\\
\hline
 1 & 0.1 & 0.95122942 & 0.\underline{95}000000 & 0.\underline{951229}17 \\
 2 & 0.2 & 0.90483742 & 0.\underline{90}250000 & 0.\underline{904836}93 \\
 3 & 0.3 & 0.86070798 & 0.\underline{85}737500 & 0.\underline{860707}28 \\
 4 & 0.4 & 0.81873075 & 0.\underline{81}450625 & 0.\underline{818729}87 \\
 5 & 0.5 & 0.77880078 & 0.\underline{77}378094 & 0.\underline{778799}73 \\
 6 & 0.6 & 0.74081822 & 0.\underline{73}509189 & 0.\underline{74081}702 \\
 7 & 0.7 & 0.70468809 & 0.\underline{6}9833730 & 0.\underline{70468}675 \\
 8 & 0.8 & 0.67032005 & 0.\underline{6}6342043 & 0.\underline{67031}859 \\
 9 & 0.9 & 0.63762815 & 0.\underline{6}3024941 & 0.\underline{63762}660 \\
10 & 1.0 & 0.60653066 & 0.\underline{5}9873694 & 0.\underline{60652}902 \\
\hline
\end{tabular}
\caption{N"aherungswerte f"ur die L"osung $e^{-\alpha x}$ der
Beispieldifferentialgleichung (\ref{numerik:expdgl}) nach dem Euler-Verfahren
und nach dem kubischen Verfahren (\ref{numerik:kubisch}) mit einer
Schrittweite von 0.1. Unterstrichen ist jeweils die Stellen, die nach
Rundung auf die angegebene Anzahl stellen mit dem exakten Wert "ubereinstimmt.
\label{numerik:euler-kubisch}}
\end{table}%
In Tabelle~\ref{numerik:euler-kubisch} werden die Resultate des
kubischen Verfahrens denen des Euler-Verfahrens gegen"ubergestellt.
Im ersten Schritt ist der Fehler des Euler-Verfahrens kleiner als $10^{-2}$,
was einer Einheit in der zweiten Nachkommastelle entspricht.
Der Fehler des kubischen Verfahrens ist kleiner als $10^{-6}$, eine
Einheit in der sechsten Nachkommastelle, ungef"ahr die von einem
kubischen Verfahren zu erwartende Verbesserung.
Nach zehn Rechenschritten liefert das Euler-Verfahren dank Rundung
gerade noch eine korrekte Stelle, w"ahrend das kubische Verfahren immer noch
gerundet f"unf korrekte Stellen gibt.

Es wurde bereits darauf hingewiesen, dass die Terme f"ur die Ableitungen
sehr kompliziert werden.
noch viel gravierender ist allerdings, dass auch die partiellen Ableitungen
von $f$ nach $x$ und $y$ bekannt sein m"ussen.
Es ist zwar im Prinzip m"oglich, diese zu berechnen, der Rechenaufwand 
daf"ur kann aber so erheblich sein, dass er den Genauigkeitsgewinn
leicht wieder zunichte machen kann.
Praktisch n"utzliche Verfahren m"ussen daher danach streben,
die h"oheren Ableitungen von $y(x)$ ausschliesslich aus Funktionswerten
von $f(x,y)$ zu berechnen.

Wir m"ochten aber weiterhin nur $y_{k+1}$ ausschliesslich aus $x_k$ und $y_k$
berechnen, also in einem einzelnen Schritt der Form
\[
y_{k+1}=y_k + h\, F(x_k, y_k, h).
\]
Die Funktion $F(x,y,h)$ heisst die {\em Inkrement-Funktion}
\index{Inkrement-Funktion}
des Verfahrens.
F"ur das Euler-Verfahren ist $F(x,y,h)=f(x,y)$.
Es soll also eine Inkrement-Funktion gefunden werden, bei der $y(x+\Delta x)$
durch $y(x) + \Delta x\cdot F(x,y,\Delta x)$ bis auf Terme h"oherer
Ordnung approximiert werden kann.

\subsection{Quadratische Verfahren}
Ein quadratisches Verfahren verwendet eine Inkrement-Funktion $F(x,y,h)$,
welche
\[
y(x+h)=y(x)+hF(x,y,h)+O(h^3)
\]
erf"ullt.
Aus den einleitenden Bemerkungen von~\ref{section:numerik:einschritt}
folgt, dass dieses Ziel m"oglicherweise dadurch erreicht werden kann,
dass man Werte von $f$ f"ur verschiedene $x$ geeignet miteinander
kombiniert.
Ein denkbarer Ansatz daf"ur ist
\[
F(x,y,h)=af(x,y) + bf(x+\alpha h, y +\beta hf(x,y)),
\]
oder anders ausgedr"uckt: Man f"uhrt zuerst etwas "ahnliches wie einen
Eulerschritt durch, um zum Punkt $(x+\alpha h,y+\beta hf(x,y))$ zu
gelangen.
Dort berechnet man den Wert von $f$, und bildet dann einen geeigneten
Mittelwert davon  mit $f(x,y)$.
Durch geeignete Wahl von $a$, $b$, $\alpha$ und $\beta$ sollte es m"oglich
sein, dass die Inkrement-Funktion einen Fehler h"ochstens dritter Ordnung
hat, womit wir dann ein Integrationsverfahren zweiter Ordnung gewonnen
h"atten.

Wir m"ussen jetzt die Parameter $a$, $b$, $\alpha$ und $\beta$ bestimmen.
Da wir mit dem "ubereinstimmen der ersten zwei Ableitungen
nur zwei Bedingungen haben, k"onnen wir nicht erwarten, dass wir
eine eindeutige L"osung finden werden.
Vielmehr werden einzelne Parameter frei w"ahlbar sein, es wird eine
ganze Familie von quadratischen L"osungsverfahren entstehen, parametrisiert
durch eine der Variablen $a$, $b$, $\alpha$ und $\beta$.

Wir berechnen nun $F(x,y,h)$ bis zur zweiten Ordnung, damit wird 
$y(x+h)$ bis zur dritten Ordnung ausdr"ucken k"onnen.
\begin{align*}
f(x+\alpha h, y + \beta h f(x,y))
&=
f(x,y)+\alpha h\frac{\partial f(x,y)}{\partial x}
+ \beta h \frac{\partial f(x,y)}{\partial y} + O(h^2)
\end{align*}
\begin{align}
F(x,y,h)
&=
af(x,y) + bf(x+\alpha h, y + \beta h f(x,y))
\notag
\\
&=
(a+b)f(x,y) + \biggl(\alpha b\frac{\partial f(x,y)}{\partial x}
+ \beta b\frac{\partial f(x,y)}{\partial y} f(x,y))\biggr)h+O(h^2)
\label{numerik:inkrementF}
\end{align}
Damit dies bis zur zweiten Ordnung mit dem Inkrement zwischen $x$ und $x+h$
"ubereinstimmt, muss~(\ref{numerik:inkrementF}) mit der Taylorreihe
von $y(x)$ "ubereinstimmen, also mit
\begin{equation}
\frac{y(x+h)-y(x)}{h}=y'(x) + \frac12y''(x)h + O(h^2)
=f(x,y) + \frac12\frac{\partial f(x,y)}{\partial x}
+\frac12\frac{\partial f(x,y)}{\partial y}f(x,y) + O(h^2),
\label{numerik:ytaylor}
\end{equation}
wobei wir f"ur $y''(x)$ die Gleichung (\ref{grundlagen:2abl}) verwendet haben.
Durch Koeffizientenvergleich finden wir die Bedingungen
\[
\begin{aligned}
a+b&=1,&
\alpha b&=\frac12,&
\beta b&=\frac12.
\end{aligned}
\]
Einzig $b$ kommt in allen drei Gleichungen vor, und bestimmt den Wert der
jeweiligen anderen Variablen:
\[
\begin{aligned}
a&=1-b,&\alpha&= \beta=\frac{1}{2b}.
\end{aligned}
\]
Jeder Wert von $b$ zwischen $0$ und $1$ liefert ein Verfahren mit quadratischer
Genauigkeit.

Der Parameterwert $b=1$ f"uhrt auf $\alpha=\beta=1$ und $a=0$, die
Rekursionsformel ist in diesem Falle
\begin{equation}
y_{k+1}=y_{k}+hf\biggl(x_k+\frac{h}2,y_k+\frac{h}2 f(x_k,y_k)\biggr).
\label{numerik:improved-euler}
\end{equation}
Das Verfahren f"uhrt also erst einen halben Eulerschritt zum Punkt
$(x_k+\frac12h,y_k+\frac{h}2f(x_k,y_k))$ durch, berechnet dort mit Hilfe
von $f$ die Steigung, die dann f"ur einen Eulerschritt der L"ange $h$
verwendet wird.u
Daher heisst dieses Verfahren auch das {\em verbesserte Euler-Verfahren}.
\index{Euler-Verfahren!verbessertes}

Verwendet man $b=\frac12$, folgt zun"achst $a=\frac12$ und $\alpha=\beta=1$.
Daraus erh"alt man die Rekursionsformel
\begin{equation}
y_{k+1}=y_k+\frac{h}2\biggl(
f(x_k,y_k) + f(x_k+h, y_k + hf(x_k,y_k))
\biggr)
\label{numerik:simplified-runge-kutta}
\end{equation}
In diesem Verfahren f"uhrt man also zuerst einen Eulerschritt der L"ange
$h$ durch, mit dem man zum Punkt $(x_k+h, y_k+hf(x_k,y_k))$ gelangt.
Dort berechnet mit mit Hilfe von $f$ die Steigung.
Das arithmetische Mittel dieser Steigung mit der im Euler-Verfahren
verwendeten Steigung $f(x_k,y_k)$ im Punkt $x_k$ wird dann als
Steigung f"ur einen Eulerschritt verwendet.
Statt eines einzigen Steigungswertes werden hier also zwei Steigungswerte
von den Enden des Intervalls $[x_k,x_k+1]$ gemittelt.
Wegen der "Ahnlichkeit dieses Vorgehens mit dem sp"ater zu besprechenden
Runge-Kutte-Verfahren heisst diese Verfahren auch das {\em
vereinfachte Runge-Kutta-Verfahren}.
\index{Runge-Kutta-Verfahren!vereinfachtes}

\subsection{Runge-Kutta-Verfahren\label{subsection:numerik:runge-kutta}}
\index{Runge-Kutta-Verfahren}
Das {\em Runge-Kutta-Verfahren} erweitert die Inkrement-Funktion derart,
dass der Einzelschritt bis zur f"unften Ordnung mit der Taylorreihe von
$y(x)$ "ubereinstimmt.
So entsteht ein Verfahren vierter Ordnung, es stellt einen guten Kompromiss
zwischen Genauigkeit und Rechenaufwand dar.

Da vier Ableitungen korrekt dargestellt werden m"ussen, ist zu erwarten,
dass vier verschiedene Werte von $f$ an verschiedenen Punkten $(x,y)$
ausgewertet und geeignet miteinander kombiniert werden m"ussen.
Genauer: Man bestimmt zuerst die Werte
\begin{align*}
k_1&=f(x_k,y_k)\\
k_2&=f\biggl(x_k+\frac{h}2,y_k+\frac{h}2k_1\biggr)\\
k_3&=f\biggl(x_k+\frac{h}2,y_k+\frac{h}2k_2\biggr)\\
k_4&=f(x_k+h, y_k+hk_3)
\end{align*}
und setzt diese dann zusammen, um den n"achsten Wert $y_{k+1}$
zu berechnen:
\begin{equation}
y_{k+1} = y_k + h\frac{1}6(k_1 + 2k_2 + 2k_3 + k_4).
\label{numerik:runge-kutta-rekursion}
\end{equation}
Man kann die Formeln wie folgt interpretieren.
Zuerst wird ein halber Eulerschritt mit der Steigung $k_1=f(x_k,y_k)$,
durchgef"uhrt, und und am Zielpunkt die Steigung $k_2$ ermittelt.
Mit dieser Steigung wird dann erneut ein halber Schritt von $(x_k,y_k)$
aus durchgef"uhrt, und am Zielpunkt erneut die Steigung $k_3$ ermittelt.
Damit f"uhrt man einen ganzen Schritt aus, an dessen Zielpunkt man die
Steigung $k_4$ findet.
Diese vier Steigungen werden jetzt gewichtet gemittelt, wobei
$k_2$ und $k_3$ doppeltes Gewicht erhalten, und mit dieser
Steigung wird ein ganzer Schritt vorgenommen.

Die Formeln f"ur die $k_i$ sowie (\ref{numerik:runge-kutta-rekursion})
k"onnen ganz "ahnlich wie das verbesserte Euler-Verfahren bzw.~das
vereinfachte Runge-Kutta-Verfahren begr"undet werden.
Der Aufwand daf"ur ist aber betr"achtlich, so dass wir auf die
detaillierte Darstellung dieser Herleitung verzichten wollen.

\begin{table}
\centering
\begin{tabular}{|r|c|r|r|r|r|r|}
\hline
$i$& $x$ & $y(x)=e^{-\alpha x}$&Euler&verbessert&vereinfacht&Runge-Kutta\\
\hline
 0 & 0.0 & 1.00000000 & 1.000 & 1.00000000 & 1.00000000 & 1.0000000000 \\
 1 & 0.1 & 0.95122942 & 0.\underline{95}0 & 0.\underline{9512}5000 & 0.\underline{9512}5000 & 0.\underline{95122942}71 \\
 2 & 0.2 & 0.90483742 & 0.\underline{90}2 & 0.\underline{9048}7656 & 0.\underline{9048}7656 & 0.\underline{9048374}229 \\
 3 & 0.3 & 0.86070798 & 0.\underline{85}7 & 0.\underline{8607}6383 & 0.\underline{8607}6383 & 0.\underline{8607079}834 \\
 4 & 0.4 & 0.81873075 & 0.\underline{81}4 & 0.\underline{8188}0159 & 0.\underline{8188}0159 & 0.\underline{8187307}620 \\
 5 & 0.5 & 0.77880078 & 0.\underline{77}3 & 0.\underline{7788}8502 & 0.\underline{7788}8502 & 0.\underline{7788007}936 \\
 6 & 0.6 & 0.74081822 & 0.\underline{73}5 & 0.\underline{7409}1437 & 0.\underline{7409}1437 & 0.\underline{7408182}327 \\
 7 & 0.7 & 0.70468809 & 0.\underline{69}8 & 0.\underline{704}79480 & 0.\underline{704}79480 & 0.\underline{7046881}031 \\
 8 & 0.8 & 0.67032005 & 0.\underline{6}63 & 0.\underline{670}43605 & 0.\underline{670}43605 & 0.\underline{6703200}606 \\
 9 & 0.9 & 0.63762815 & 0.\underline{6}30 & 0.\underline{637}75229 & 0.\underline{637}75229 & 0.\underline{6376281}672 \\
10 & 1.0 & 0.60653066 & 0.\underline{5}98 & 0.\underline{606}66187 & 0.\underline{606}66187 & 0.\underline{6065306}762 \\
\hline
\end{tabular}
\caption{Vergleich der Genauigkeit der verbesserten numerischen Verfahren.
Unterstrichen jeweils die nach Rundung korrekten Stellen der L"osung.
\label{numerik:genauigkeit}}
\end{table}


\begin{table}
\centering
\begin{tabular}{|l|l|c|r|>{$}r<{$}|}
\hline
Verfahren                           &$h$  &Schritte&$y_n$&\text{Fehler}\\
\hline
Euler-Verfahren                     &0.025&  40    & 0.\underline{60}462232 &  0.00190834 \\
verbessertes Euler-Verfahren        &0.05 &  20    & 0.\underline{6065}6285 & -0.00003219 \\
vereinfachtes Runge-Kutta-Verfahren &0.05 &  20    & 0.\underline{6065}6285 & -0.00003219 \\
Runge-Kutta-Verfahren               &0.1  &  10    & 0.\underline{6065306}7 & -0.00000001 \\
\hline
\end{tabular}
\caption{Vergleich der verschiedenen Verfahren bei gleichbleibendem 
Rechenaufwand.
Die Schrittweite wurde jeweils so angepasst, dass in allen Verfahren bis
zum Wert $x=1$ die gleiche Anzahl von Auswertungen der Funktion $f$
notwendig wurde.
\label{numerik:vergleich-aufwand}}
\end{table}

Die Tabelle~\ref{numerik:genauigkeit} demonstriert die "uberragende
Genauigkeit des Runge-Kutta-Verfahrens.
Trotz der relativ grossen Schrittweite von $h=0.1$ erreicht das
Verfahren nach zehn Schritten eine Genauigkeit von sieben signifikanten
Stellen.
Da in jedem Schritt die Funktion $f$ viermal ausgewertet werden muss,
ist der Rechenaufwand mit dem Runge-Kutta-Verfahren viermal gr"osser
als im Euler-Verfahren, letzteres kann aber mit nur einer signifikanten
Stelle kaum als brauchbar bezeichnet werden.
Passt man in jedem Verfahren die Schrittweite so an, dass f"ur die
Berechnung der N"aherung f"ur $y(1)$ immer gleich viele Auswertungen
der Funktion $f(x,y)$ n"otig sind, ergeben sich die Resultate in
Tabelle~\ref{numerik:vergleich-aufwand}.
Bei gleichem Rechenaufwand ist das Runge-Kutta-Verfahren um viele
Gr"ossenordungen pr"aziser.
Es gibt daher eigentlich keinen praktischen Grund, "uberhaupt je etwas
anderes als das Runge-Kutta-Verfahren zu verwenden.


\section{Mehrschritt-Verfahren}
\lhead{Mehrschritt-Verfahren}
In den Einschritt-Verfahren wurde wiederholt die Funktion $f$ ausgewertet,
um die Inkrement-Funktion f"ur einen einzigen Schritt zu bestimmen.
Das Ziel dabei war, $y(x+h)$ in "Ubereinstimmung mit der Taylorreihe
bis zu m"oglichst hoher Ordnung zu bestimmen.
Im Runge-Kutta-Verfahren wurden dabei halbe Eulerschritte durchgef"uhrt,
man hat also eigentlich die Aufl"osung nochmals halbiert, um die
Inkrement-Funktion zu ermitteln.
Diese Zwischenwerte geben dem Verfahren die Information "uber die
h"oheren Ableitungen der Funktionen.

Sobald einige Werte der L"osung berechnet sind, l"asst sich die Kr"ummung
der L"osungskurve auch aus diesen Werten ablesen.
Es sollte daher auch m"oglich sein, aus mehreren bereits
ermittelten Werten $y_{n\mathstrut},y_{n+1},\dots,y_{n+s-1}$
den n"achsten Wert $y_{n+s\mathstrut}$ mit der verlangten Genauigkeit
zu berechnen.
Der Vorteil eines solchen Vorgehens ist, dass f"ur jeden Schritt nur 
eine einzige Auswertung der Funktion $f$ n"otig ist,
nicht mehrere wie bei den besprochenen Einschritt-Verfahren.

Als Beispiel versuchen wir daher ein Verfahren aufzubauen, welches
$y_{n+2}$ aus den bereits berechneten Werten $y_{n\mathstrut}$ und
$y_{n+1}$ berechnet.
Wir nehmen dabei an, dass $y_{n\mathstrut}$ und $y_{n+1}$ exakt
sind.
Der neue Datenpunkt soll mit Hilfe eines Ausdrucks der Form
\begin{equation}
y_{n+2}=y_{n+1} + h(af(x_{n+1},y_{n+1}) + b f(x_{n\mathstrut},y_{n\mathstrut}))
\label{numerik:zweischrittansatz}
\end{equation}
gefunden werden.
Die N"aherung kann wieder mit Hilfe der Ableitungen alleine
durch Werte bei $x_{n+1}$ ausgedr"uckt werden:
\begin{align*}
y_{n+2}
&=
y_{n+1}+h(af(x_{n+1}, y_{n+1}) + bf(x_{n+1}-h, y_{n\mathstrut}))
\\
&=
y_{n+1}+haf(x_{n+1}, y_{n+1}) + hbf(x_{n+1}-h, y_{n+1} - h f(x_{n+1},y_{n+1}) + O(h^2))
\\
&=
y_{n+1}+haf(x_{n+1}, y_{n+1}) + hb
\biggl(
f(x_{n+1},y_{n+1})
-h \frac{\partial f(x_{n+1},y_{n+1})}{\partial x}
\\
&\qquad
-h
\frac{\partial f(x_{n+1},y_{n+1})}{\partial y}
f(x_{n+1},y_{n+1})
+ 
\frac{\partial f(x_{n+1},y_{n+1})}{\partial y}
O(h^2)
\biggr)
\\
&=
y_{n+1}
+ (a+b)hf(x_{n+1},y_{n+1})
- bh^2\biggl(
\frac{\partial f(x_{n+1},y_{n+1})}{\partial x}
+
\frac{\partial f(x_{n+1},y_{n+1})}{\partial y}
f(x_{n+1},y_{n+1})
+O(h^3)
\biggr)
\end{align*}
Sie muss bis zur zweiten Ordnung mit der Taylorreihe "ubereinstimmen:
\begin{align*}
y(x_{n+2})
&=
y_{n+1} + hy'(x_{n+1}) + \frac12h^2 y''(x_{n+1})+O(h^3)
\\
&=
y_{n+1}+hf(x_{n+1},y_{n+1})+\frac12h^2\biggl(
\frac{\partial f(x_{n+1},y_{n+1})}{\partial x}
+
\frac{\partial f(x_{n+1},y_{n+1})}{\partial y}
f(x_{n+1},y_{n+1})
\biggr)
\end{align*}
Vergleicht man Koeffizienten, findet man
\[
\begin{aligned}
a+b&=1&-b&=\frac12&&\Rightarrow&a=\frac32
\end{aligned}
\]
Aus der Formel (\ref{numerik:zweischrittansatz}) wird somit die
Iterationsformel
\begin{equation}
y_{n+2}=y_{n+1}+h\biggl(\frac32f(x_{n+1},y_{n+1})
- \frac12 f(x_{n\mathstrut},y_{n\mathstrut})\biggr)
\end{equation}
Diese Rekursionsformel definiert ein quadratisches Verfahren, das
{\em Adams-Bashforth-Verfahren} mit $s=2$.
\index{Adams-Bashforth-Verfahren}

Das Verfahren kann "ahnlich wie das Runge-Kutta-Verfahren auf h"ohere
Ordnung erweitert werden.
Man findet nach einiger Rechnung
\begin{align*}
s&=1\colon&
y_{n+1}
&=
y_n+hf(x_n,y_n)
\\
s&=2\colon&
y_{n+2}
&=
y_{n+1}+h\biggl(\frac32f(x_{n+1},y_{n+1})-\frac12f(x_n,y_n)\biggr)
\\
s&=3\colon&
y_{n+3}
&=
y_{n+2}+h\biggl(\frac{23}{12}f(x_{n+2},y_{n+2})-\frac43f(x_{n+1},y_{n+1})+\frac{5}{12}f(x_n,y_n)\biggr)
\\
s&=4\colon&
y_{n+4}
&=
y_{n+3}+h\biggl(\frac{55}{24}f(x_{n+3},y_{n+3})
	-\frac{59}{24}f(x_{n+2},y_{n+2})
	+\frac{37}{24}f(x_{n+1},y_{n+1})
	-\frac{3}{8}f(x_n,y_n)
\biggr)
\end{align*}
Es ist also m"oglich, ausgehend von dieser Idee Verfahren beliebig hoher
Ordnung zu produzieren.

\begin{table}
\centering
\begin{tabular}{|r|c|r|r|r|r|}
\hline
$i$& $x$ & $y(x)=e^{-\alpha x}$&Euler&Adams-Bashforth&Runge-Kutta\\
\hline
 0 & 0.0 & 1.00000000 & 1.00000000 & 1.00000000 & 1.0000000000 \\
 1 & 0.1 & 0.95122942 & 0.\underline{95}000000 & 0.\underline{9512}8178 & 0.\underline{95122942}71 \\
 2 & 0.2 & 0.90483742 & 0.\underline{90}250000 & 0.\underline{904}93564 & 0.\underline{90483742}29 \\
 3 & 0.3 & 0.86070798 & 0.\underline{85}737500 & 0.\underline{860}84752 & 0.\underline{86070798}34 \\
 4 & 0.4 & 0.81873075 & 0.\underline{81}450625 & 0.\underline{818}90734 & 0.\underline{81873076}20 \\
 5 & 0.5 & 0.77880078 & 0.\underline{77}378094 & 0.\underline{779}01048 & 0.\underline{77880079}36 \\
 6 & 0.6 & 0.74081822 & 0.\underline{73}509189 & 0.\underline{741}05738 & 0.\underline{74081823}27 \\
 7 & 0.7 & 0.70468809 & 0.\underline{69}833730 & 0.\underline{704}95334 & 0.\underline{7046881}031 \\
 8 & 0.8 & 0.67032005 & 0.\underline{6}6342043 & 0.\underline{670}60827 & 0.\underline{6703200}606 \\
 9 & 0.9 & 0.63762815 & 0.\underline{63}024941 & 0.\underline{637}93648 & 0.\underline{6376281}672 \\
10 & 1.0 & 0.60653066 & 0.\underline{59}873694 & 0.\underline{606}85645 & 0.\underline{6065306}762 \\
\hline
\end{tabular}
\caption{Vergleich der Genauigkeit der Verfahren von Euler,
Adams-Bashforth und Runge-Kutta.
Als Startwerte f"ur das Adams-Bashforth-Verfahren wurden die
Werte $y(-h)=e^{-\alpha h}$ und $y(0)=1$ verwendet, um keine zus"atzlichen
Fehler aus der Durchf"uhrung des ersten Schrittes hinzuzuf"ugen.
\label{numerik:genauigkeit-adams-bashforth}}
\end{table}

In der Tabelle~\ref{numerik:genauigkeit-adams-bashforth} wird
das Adams-Bashforth-Verfahren verglichen mit dem lineare Euler-Verfahren 
und dem Verfahren vierter Ordnung von Runge-Kutta.
Die Verbesserung der Genauigkeit des Adams-Bashforth-Verfahrens
gegen"uber dem Euler-Ver\-fah\-ren ist konsistent damit, dass
das Adams-Bashforth-Verfahren ein quadratisches Verfahren ist.

Nachteilig an den Mehrschritt-Verfahren ist die Notwendigkeit,
gen"ugend viele Werte $y_{n},\dots,y_{n+s-1}$ mit ausreichend
hoher Genauigkeit zu bestimmen, bevor das Mehrschritt-Verfahren
seine Schritte der Ordnung $s$ beginnen kann.
Solange diese Werte nicht zur Verf"ugung stehen, kann ein Mehrschritt-Verfahren
nur Schritte niedrigerer Ordnung als $s$ durchf"uhren.

Bei einem Einschritt-Verfahren kann in jedem Schritt die Schrittweite $h$
ver"andert werden, zum Beispiel f"ur Bereiche von $x$-Werten, in denen
die Steigung von $y(x)$ sehr rasch "andert.

F"ur die Beispiel-Differentialgleichung (\ref{numerik:expdgl}) k"onnen
wir das Adams-Bashforth-Verfahren zweiter Ordnung ($s=2$) vollst"andig
analysieren.
Die Rekursionsformel wird zu
\[
y_{n+2}=y_{n+1}+h\biggl(\frac32 (-\alpha y_{n+1})-\frac12(-\alpha y_n)\biggr)
=
\biggl(1-\frac32\alpha h\biggr)
y_{n+1}
+\frac{\alpha h}{2}
y_{n\mathstrut}
\]
Dies ist eine Differenzengleichung mit konstanten Koeffizienten, man kann
sie mit Hilfe eines Potenzansatzes l"osen. 
Wir nehmen also an, dass $y_n=\lambda^n$, und setzen dies in die
Rekursionsformel ein.
Ausserdem k"urzen wir $\alpha h/2=\delta$  ab.
Wir erhalten
\[
\lambda^{n+2}-(1-3\delta)\lambda^{n+1}-\delta\lambda^n=0.
\]
Nach Division durch $\lambda^n$ erhalten wir die quadratische Gleichung
\[
\lambda^2-(1-3\delta )\lambda-\delta=0
\]
f"ur $\lambda$ mit den L"osungen
\[
\lambda_\pm
=
\frac12(1-3\delta) \pm \frac12\sqrt{(1-3\delta)^2+4\delta}.
\]
Da $\delta$ klein ist, wird $\lambda_-$ ebenfalls klein sein,
w"ahrend $\lambda_+$ n"aher bei $1$ sein wird.
Der dominante Einfluss auf die L"osung r"uhrt also von $\lambda_+$ her.
Um diesen Unterschied genauer zu verstehen, verwenden wir eine
lineare Approximation der Wurzel auf der rechten Seite von $\lambda_\pm$:
\begin{align*}
\sqrt{1+x}
&=
1+\frac{x}{2}-\frac{x^2}{4}+\frac{3x^3}{8}-\dots
\\
\sqrt{x}
&=
\sqrt{x_0+x-x_0}
=
\sqrt{x_0}\sqrt{1+\frac{x-x_0}{x_0}}
=
\sqrt{x_0}\biggl(1+\frac12\frac{x-x_0}{x_0}-\frac14\frac{(x-x_0)^2}{x_0^2}+\dots\biggr)
\\
&=
\sqrt{x_0}+\frac12\frac{x-x_0}{\sqrt{x_0}}-\frac14\frac{(x-x_0)^2}{\sqrt{x_0}^3}+\dots
\end{align*}
Wir verwenden diese Approximation mit $x_0=(1-3\delta)^2$ und $x-x_0=-4\delta$
\begin{align*}
\sqrt{(1-3\delta)^2+4\delta}
&=
(1-3\delta)\biggl(1+\frac12\frac{4\delta}{(1-3\delta)^2}
-\frac14\frac{16\delta^2}{(1-3\delta)^4}+\dots\biggr)
\\
&=(1-3\delta)+\frac12\frac{4\delta}{1-3\delta}
-\frac14\frac{16\delta^2}{(1-3\delta)^3}+\dots
\\
&=1-3\delta+2\delta(1+3\delta)-4\delta^2+O(\delta^3)
\\
&=1-3\delta+2\delta + 2\delta^2+O(\delta^3)
\\
&=1-3\delta+2\delta + \frac12(2\delta)^2+O(\delta^3)
\end{align*}
Damit k"onnen wir jetzt $\lambda_+$ bis zur zweiten Ordnung berechnen:
\begin{align*}
\lambda_+
&=
\frac12\biggl((1-3\delta)+ (1-3\delta)+2\delta+\frac12(2\delta)^2\biggr)
+O(\delta^3)
\\
&=
1-2\delta+\frac12(2\delta)^2+O(\delta^3)
\\
&=e^{-2\delta}+O(\delta^3).
\end{align*}
Die exakte L"osung erf"ullt $y_{n+1}=e^{-2\delta}y_n$, der Faktor
$\lambda_+$ stimmt bis auf Terme mindestens dritter Ordnung mit 
$e^{-2\delta}$ "uberein.
Damit ist erneut best"atigt, dass wir es mit einem quadratischen Verfahren zu
tun haben.

Wir k"onnen auch $\lambda_-$ berechnen, und erhalten
\[
\lambda_-=-\delta-2\delta^2+O(\delta^3).
\]
Da $\delta$ klein ist, ist eine Komponente der L"osung bereits nach
drei Schritten kleiner als $O(\delta^3)$, und spielt daher im Vergleich
zu den von $\lambda_+$ herr"uhrenden L"osungen in dritter Ordnung keine
Rolle.

\section{Software}
Die im letzten Abschnitt entwickelten numerischen Verfahren zur L"osung
einer Differentialgleichung kommen ausschliesslich mit Auswertungen der
Funktion $f$ aus, die Ableitungen der Funktion $f$ m"ussen nicht bekannt
sein.
Es sollte also ein Leichtes sein, eine Softwarebibliothek zur
Verf"ugung zu stellen, mit der eine beliebige gew"ohnliche
Differentialgleichung gel"ost werden kann.
Als Input braucht es nur die Funktion $f$ und die Anfangsbedingungen.

Als Beispiel wollen wir in diesem Abschnitt die Differentialgleichung
\[
y''+y=\sin \frac{x}{10},\qquad y(0)=y'(0)=0
\]
in verschiedenen Programmierumgebungen l"osen.
Als erstes bringen wir die Differentialgleichung wieder in die Standardform
einer Vektordifferentialgleichung erste Ordnung:
\begin{equation}
\frac{d}{dt}Y
=
\frac{d}{dx}\begin{pmatrix}y_1\\y_2\end{pmatrix}
=
\begin{pmatrix}
y_2\\
-y_1+\sin\frac{x}{10}
\end{pmatrix}
=
f(x,Y)
\label{numerik:beispieldgl}
\end{equation}
Ein numerisches Verfahren braucht also als Input eine Anfangsbedingung
sowie die Funktion $f$.
Ausserdem muss es M"oglichkeiten bereitstellen, wie man den Gang der
Rechnung beeinflussen kann, z.~B.~um die $x$-Werte anzugeben, f"ur die
die $Y(x)$ bestimmt werden sollen, oder um Genauigkeitsziele zu erreichen.

\subsection{Octave}
In Octave steht eine einzige Funktion \texttt{lsode} zur Verf"ugung, welche
auf zuverl"assige Art Differentialgleichungen l"ost.
Der Anwender muss eine Implementation der Funktion $f$ zur Verf"ugung
stellen, allerdings werden die Argument in der umgekehrten Reihenfolge
zu der erwarte, die wir in diesem Skript bisher verwendet haben.
F"ur die Beispieldifferentialgleichung (\ref{numerik:beispieldgl})
kann man sie zum Beispiel so definieren:
\verbatiminput{chapters/examples/octave-dgl-f.m}

Beim Aufruf der Funktion \texttt{lsode} muss man den {\em Namen}
der Funktion, die Anfangsbedingung, sowie einen Vektoren mit $x$-Werten,
f"ur die man die L"osung ausgegeben haben m"ochte, als Argumente
"ubergeben.
Der erste Wert im $x$-Vektor muss der $x$-Wert f"ur die Anfangsbedingung
sein, in unserem Fall also $0$.
Um die Werte von $y(x)$ f"ur ganzzahlige Werte von $x$ zu erhalten,
muss man also die Befehle
\verbatiminput{chapters/examples/octave-dgl-sol.m}
ausf"uhren.
Als R"uckgabewert erh"alt man eine Matrix, die in jeder Zeile die
Werte von $y(x)$ und $y'(x)$ zum entsprechenden Wert von $x$
aus dem \texttt{x}-Argument enth"alt.
Die Resultate sind zusammen mit den Werten der exakten
L"osung~(\ref{grundlagen:numerik-beispiel-loesung}) in der dritten Spalte 
in der Tabelle~\ref{numerik:octave-resultate} zusammengestellt.
Es ist gut erkennbar, wie der Fehler anf"anglich langsam ansteigt,
dann aber unter Kontrolle bleibt.
Die Dokumentation der Funktion \texttt{lsode} beschreibt, wie man mit
Hilfe von Optionen ihr Verhalten und insbesondere die Gr"osse der
Fehler weiter beeinflussen kann.
\begin{table}
\centering
\begin{tabular}{|>{$}r<{$}|>{$}r<{$}|>{$}r<{$}|>{$}r<{$}|}
\hline
    x&  y_{\text{numerisch}}(x)&y_{\text{exakt}}(x) & \text{Fehler}\\
\hline
    0&  0.00000000&  0.00000000&  0.00000000\\
    1&  0.00158525&  0.00158528&  0.00000003\\
    2&  0.01090682&  0.01090678&  0.00000003\\
    3&  0.02858723&  0.02858716&  0.00000007\\
    4&  0.04756207&  0.04756212&  0.00000004\\
    5&  0.05957416&  0.05957437&  0.00000020\\
    6&  0.06276426&  0.06276444&  0.00000018\\
    7&  0.06337942&  0.06337932&  0.00000010\\
    8&  0.07002849&  0.07002811&  0.00000037\\
    9&  0.08576626&  0.08576594&  0.00000032\\
   10&  0.10528405&  0.10528416&  0.00000010\\
  100&  0.84661503&  0.84661930&  0.00000427\\
 1000& -0.55228836& -0.55234514&  0.00005678\\
 2000&  0.90392063&  0.90373523&  0.00018540\\
 3000& -0.99018339& -0.99032256&  0.00013917\\
 4000&  0.75185982&  0.75202340&  0.00016358\\
 5000& -0.25298074& -0.25252044&  0.00046030\\
 6000& -0.30093757& -0.30056348&  0.00037408\\
 7000&  0.76905079&  0.76889872&  0.00015207\\
 8000& -1.00327790& -1.00396748&  0.00068958\\
 9000&  0.88860437&  0.88793099&  0.00067338\\
10000& -0.50337856& -0.50335983&  0.00001873\\
\hline
\end{tabular}
\caption{Exakte und numerische L"osung der Beispieldifferentialgleichung
berechnet mit der Funktion \texttt{lsode} von Octave.
\label{numerik:octave-resultate}}
\end{table}

\subsection{GNU Scientific Library}
W"ahrend Octave dem Benutzer die Wahl eines geeigneten Verfahrens abnimmt
und ihm "uberhaupt wenig Kontrolle "uber den Gang der Rechnung gibt,
kann ein Programmierer durch den Einsatz der GNU Scientific Library (GSL) die
volle Kontrolle "uber alle Aspekte der Iteration erhalten.
Der Preis ist eine wesentlich h"ohere Komplexit"at.
Ziel dieses Abschnitts ist, ein einfaches Beispielprogramm zu
zeigen, welches als Basis eigener Programme dienen kann.
Es verwendet eine Runge-Kutta-Verfahren achter Ordnung.

Die Funktionen zum L"osen von gew"ohnlichen Differentialgleichungen
der GSL haben alle das Pr"afix \texttt{gsl\_odeiv2\_}. 
Zun"achst braucht es nat"urlich wieder eine Implementation der
Funktion $f$. 
Die GSL "ubergibt zwei Arrays, im einen findet die Funktion die aktuellen
$Y$-Werte, im anderen soll sie die Werte der Ableitung zur"uckgeben.
F"ur die Beispiel-Differentialgleichung (\ref{numerik:beispieldgl})
sieht der Code wie folgt aus:
\verbatiminput{chapters/examples/dgl-f.c}
Der Parameter \texttt{params} dient dazu, der Funktion zus"atzliche
Parameter zu "ubergeben.
In unserem Fall ist das nur die Zahl $\omega$.
Da \texttt{params} ein \texttt{void}-Pointer ist, kann eine beliebige
Struktur zur Parameter"ubergabe verwendet werden.

Die Differentialgleichung wird beschrieben durch eine Struktur vom Typ
\texttt{gsl\_odeiv2\_system}, welche ausser Zeigern auf die Funktion
und die Parameter-Struktur auch noch die Dimension der Vektoren enth"alt.
Es kann auch noch ein Funktionszeiger f"ur eine eine Funktion "ubergeben
werden, die die Jacobi-Matrix berechnet, in unserem Beispiel wird dies
jedoch nicht ben"otigt.

Die eigentliche wird von einer ``driver''-Funktion durchgef"uhrt.
Diese sorgt im wesentlichen f"ur die Wahl der Schrittweite, verwaltet
Datenstrukturen, und ruft die Funktionen auf, die die einzelnen Schritte
durchf"uhren.
Die Treiber-Funktion f"uhrt die einzelnen Schritte (im Sinne der
in Abschnitt~\ref{section:numerik:einschritt} besprochenen
Einschritt-Verfahren) mit
Hilfe der Schritt-Funktionen durch, von denen die Bibliothek eine
ganze Reihe bereitstellt.
Die Funktion \texttt{gsl\_odeiv2\_step\_rk4} ist das klassische
Runge-Kutta-Verfahren vierter Ordnung, welches in
Abschnitt~\ref{subsection:numerik:runge-kutta}
beschrieben wurde.
Im Beispielverfahren verwenden wir \texttt{gsl\_odeiv2\_step\_rk8pd},
das Runge-Kutta Prince-Dormand Verfahren achter Ordnung.
F"ur Aufgaben allgemeiner Art ebenfalls sehr gut geeignet ist das
Runge-Kutta-Fehlberg-Verfahren f"unfter Ordnung mit dem Namen
\texttt{gsl\_odeiv2\_step\_rkf45}.
Diese Datenstrukturen werden mit dem Code
\verbatiminput{chapters/examples/dgl-init.c}
initialisiert.
Durch Austausch des zweiten Arguments der Driver-Allozierungs-Funktion
kann man leicht das Verfahren wechseln und so Zeitaufwand und Genauigkeit
f"ur verschiedene L"osungsverfahren vergleichen.

Um die Rechnung durchzuf"uhren, muss jetzt die Driver-Funktion so oft
angewendet werden, wie man Punkt der L"osungskurve ausgeben will.
Dazu dient die Funktion \texttt{gsl\_odeiv2\_driver\_apply}. 
An Argument braucht sie den eben initialisierten Driver, den aktuellen
$x$-Wert, den $x_{\text{next}}$-Wert, f"ur den der n"achste Punkt
ausgegeben werden soll, sowie einen Vektor, in dem der aktuelle Anfangswert
f"ur $Y(x)$ "ubergeben und $Y(x_{\text{next}})$ zur"uckgegeben wird.
$x$ wird als Referenz "ubergeben, wenn die Funktion zur"uckkehrt,
findet man dort den neuen aktuellen Wert von $x$, also im Erfolgsfall
$x_{\text{next}}$.
In unserem Fall brauchen wir $X(x)$ f"ur ganzzahlige $x$, die folgende
Schleife bewerkstelligt dies:
\verbatiminput{chapters/examples/dgl-loop.c}

\begin{table}
\centering
\begin{tabular}{|>{$}r<{$}|>{$}r<{$}|>{$}r<{$}|>{$}r<{$}|}
\hline
    x&  y_{\text{numerisch}}(x)&y_{\text{exakt}}(x) & \text{Fehler}\\
\hline
    1&   0.01584477&   0.01584477&  -0.00000000\\
    2&   0.10882786&   0.10882787&  -0.00000000\\
    3&   0.28425071&   0.28425071&  -0.00000001\\
    4&   0.46979656&   0.46979656&  -0.00000000\\
    5&   0.58112927&   0.58112926&   0.00000001\\
    6&   0.59856974&   0.59856972&   0.00000002\\
    7&   0.58436266&   0.58436265&   0.00000001\\
    8&   0.62466692&   0.62466694&  -0.00000001\\
    9&   0.74961115&   0.74961117&  -0.00000003\\
   10&   0.90492231&   0.90492232&  -0.00000001\\
   20&   0.82626555&   0.82626556&  -0.00000000\\
   30&   0.24234669&   0.24234664&   0.00000005\\
   40&  -0.83971103&  -0.83971092&  -0.00000011\\
   50&  -0.94210772&  -0.94210787&   0.00000015\\
   60&  -0.25144906&  -0.25144893&  -0.00000013\\
   70&   0.58545210&   0.58545205&   0.00000005\\
   80&   1.09974465&   1.09974456&   0.00000009\\
   90&   0.32597838&   0.32597860&  -0.00000023\\
  100&  -0.49836792&  -0.49836823&   0.00000031\\
  200&   1.01037929&   1.01037877&   0.00000052\\
  300&  -0.89702589&  -0.89702630&   0.00000041\\
  400&   0.83859090&   0.83859101&  -0.00000010\\
  500&  -0.21777633&  -0.21777543&  -0.00000090\\
  600&  -0.31235410&  -0.31235239&  -0.00000171\\
  700&   0.72675906&   0.72676124&  -0.00000219\\
  800&  -1.09422992&  -1.09422790&  -0.00000202\\
  900&   0.80223761&   0.80223872&  -0.00000111\\
 1000&  -0.59500324&  -0.59500363&   0.00000040\\
 2000&  -0.97606658&  -0.97606187&  -0.00000471\\
 3000&  -1.03200392&  -1.03199479&  -0.00000914\\
 4000&  -0.79047800&  -0.79047372&  -0.00000428\\
 5000&  -0.37269297&  -0.37270218&   0.00000921\\
 6000&   0.08785158&   0.08783273&   0.00001886\\
 7000&   0.49827647&   0.49826463&   0.00001184\\
 8000&   0.80219750&   0.80220742&  -0.00000992\\
 9000&   0.94568799&   0.94571577&  -0.00002778\\
10000&   0.86607968&   0.86610200&  -0.00002232\\
\hline
\end{tabular}
\caption{L"osungen der Beispieldifferentialgleichung (\ref{numerik:beispieldgl})
mit Hilfe der GNU Scientific Library (GSL).
\label{numerik:gsl-resultate}}
\end{table}

Man kann die Funktion $f$ im Programm nat"urlich auch mit einem Z"ahler
ausstatten und damit herausfinden, wie viele Aufrufe der Funktion
f"ur die numerische L"osung ben"otigt werden.
Es stellt sich heraus, dass f"ur das erste Intervall von $0$ bis $1$
die Funktion $f$ 131 mal aufgerufen wird, hier versucht die Bibliothek
die optimale Schrittweite $h$ zu bestimmen.
In allen folgenden Intervallen der L"ange $1$ von $n$ bis $n+1$ werden nur
noch jeweils 13 Aufrufe der Funktion ben"otigt.
Verwendet man stattdessen das Runge-Kutta-Fehlberg-Verfahren,
werden pro Intervall 18 Auswertungen der Funktion $f$ ben"otigt,
und die Genauigkeit sinkt auf zwei Stellen nach dem Komma.

\section{Randwertprobleme\label{section:numerik:randwertprobleme}}
Die bisher beschriebenen Verfahren gehen von einer Anfangsbedingung
aus, und berechnen die dadurch eindeutig festgelegte L"osungskurve.
Randwertproblem, beschrieben in Abschnitt~\ref{section:randwertprobleme},
verkn"upfen dagegen Werte von einzelnen Komponenten von $Y$ an den
R"andern eines Intervalls.

Wir betrachten zwei prototypische Randwertprobleme, die auch gleich
zwei v"ollig verschiedene L"osungsverfahren motivieren.

\newtheorem{aufgabe}{Aufgabe}[chapter]
\begin{aufgabe}
\label{numerik:aufgabe-ball}
Mit einem nur der Schwerkraft unterworfenen Ball, der im Ursprung des
Koordinatensystems geworfen wird, soll ein Ziel im Punkt $P$ getroffen
werden.
In welcher Richtung und mit welcher Anfangsgeschwindigkeit muss er geworfen
werden?
\end{aufgabe}

Um das Problem einfach zu halten, modellieren wir diese Aufgabe wie
folgt.
Der Ball der Masse $m$ bewegt sich in der $x$-$y$-Ebene, wobei die
Schwerkraft in negativer $y$-Richtung zeigt.
Das Newtonsche Gesetz liefert die Differentialgleichung zweiter Ordnung
\begin{equation}
m\frac{d^2}{dt^2}\begin{pmatrix}x\\y\end{pmatrix}
=
\begin{pmatrix}
0\\-mg
\end{pmatrix}
\label{numerik:ball-dgl}
\end{equation}
Die Masse $m$ kann herausgek"urzt werden.
Gesucht ist eine L"osung so, dass die Bahn durch die Punkte $(0,0)$
und $P=(p,0)$ geht.

Genau genommen ist dies nicht ein Randwertproblem wie in 
Abschnitt~\ref{section:randwertprobleme}, denn es wird nicht verlangt,
dass der Ball zu einer bestimmten Zeit $t$ beim Punkt $P$ eintrifft.
Die Differentialgleichung bedeutet aber, dass die Horizontalgeschwindigkeit
des Balls konstant ist (die horizontale Beschleunigung ist immer $0$).
Ist $v_x$ die Horizontalgeschwindigkeit, dann erreicht der Ball zur
Zeit $t_1=p/v_x$ die $x$-Koordinate des Ziels.
Gesucht ist also die anf"angliche Vertikalgeschwindigkeit, die man
dem Ball geben muss, dass zur Zeit $p/v_x$ die $y$-Komponente
der L"osung den Wert $0$ hat.
In dieser Form liegt ein Randwertproblem wie in
Abschnitt~\ref{section:randwertprobleme} vor.

Die L"osungen der Differentialgleichung~\ref{numerik:ball-dgl} sind aus
dem Physik-Unterricht bekannt:
es gilt
\begin{equation}
\begin{pmatrix}x(t)\\y(t)\end{pmatrix}
=
\begin{pmatrix}v_xt\\ v_yt-\frac12gt^2\end{pmatrix}
\end{equation}
Damit l"asst sich auch das Randwertproblem l"osen.
F"ur $t=v_x/p$ muss $y(t)=0$ sein, also
\begin{align}
y(t)=y\biggl(\frac{p}{v_x}\biggr)
=v_y\frac{p}{v_x}-\frac12g\biggl(\frac{p}{v_x}\biggr)^2&=0
\notag
\\
\Rightarrow\qquad
v_y
&=
\frac{v_x}p\frac12g\frac{p^2}{v_x^2}=\frac{gp}{2v_x}.
\label{numerik:ball-bedingung}
\end{align}
Offenbar gibt es zu jedem $v_x$ einen passenden Wert von $v_y$,
mit dem das Ziel getroffen wird.

Die Differentialgleichung (\ref{numerik:ball-dgl}) ist nicht in einer
Form, die der numerischen L"osung zug"anglich ist.
Wir schreiben Sie daher als Differentialgleichung erster Ordnung 
f"ur vierdimensionale Vektoren:
\begin{align}
\frac{d}{dt}Y
=
\frac{d}{dt}\begin{pmatrix}x\\y\\\dot x\\\dot y\end{pmatrix}
&=
\begin{pmatrix}\dot x\\\dot y\\ 0\\ -g\end{pmatrix}.
\label{numerik:ball-dgl-1}
\end{align}
Gesucht ist eine L"osung, die die Randbedingungen
\begin{equation}
Y(0)
=
\begin{pmatrix}0\\0\\v_x\\\color{red}v_y\end{pmatrix},
\qquad
Y\biggl(\frac{p}{v_x}\biggr)
=
\begin{pmatrix}p\\0\\\color{red}?\\\color{red}?\end{pmatrix}
\label{numerik:ball-dgl-2}
\end{equation}
erf"ullt.
Darin stehen die roten Eintr"age f"ur Werte, die nicht vorgegeben sind.
Aus der Symmetrie des Problems kann man nat"urlich auch die Endgeschwindigkeit
ablesen.
Zu bestimmen ist also $v_y$ so, dass die L"osungskurve durch den Punkt
$(p,0)$ geht.

Wird statt der Horizontalkomponenten der Anfangsgeschwindigkeit die
gesamte Anfangsgeschwindigkeit $v_0$ vorgegeben, dann muss der
Winkel gefunden werden, unter dem der Ball geworfen werden muss,
um das Ziel zu trefen.
Bei der Elevation $\alpha$ sind die Komponenten der Anfangsgeschwindigkeit
$v_x=v_0\cos\alpha$ und $v_y=v_0\sin\alpha$. 
Setzt man dies in die Bedingung~(\ref{numerik:ball-bedingung}) ein,
findet man
\begin{align*}
v_0 \sin \alpha &=\frac{gp}{2v_0\cos\alpha}
\\
2\sin\alpha\cos\alpha&=\frac{gp}{v_0^2}
\\
\sin2\alpha&=\frac{gp}{v_0^2}
\\
\alpha&= \frac12 \arcsin\frac{gp}{v_0^2}
\end{align*}
Im Nenner rechts steht im wesentlichen die kinetische Energie,
je mehr kinetische Energie der Ball zu Beginn hat, desto kleiner
ist der Winkel, man trifft das Ziel mit einer sehr flachen Bahn.
Kleine Winkel reichen auch f"ur geringe Schwerkraft ($g$ klein)
und kurze Distanzen ($p$ klein).
Die maximale Distanz wird erreicht, wenn das Argument des Arcussinus
den Wert $1$ erreicht, gr"osser darf $p$ nicht werden, weil es sonst
keine L"osung mehr f"ur $\alpha$ gibt.
Die Maximaldistanz ist daher
\[
p_{\text{max}} = \frac{v_0^2}{g}.
\]

\begin{aufgabe}
\label{numerik:aufgabe-seil}
Ein Seil ist zwischen zwei Punkten aufgeh"angt, welche Form nimmt es
allein unter der Wirkung seines Eigengewichtes an?
\end{aufgabe}

\subsection{Schiess-Verfahren\label{numerik:schiess-verfahren}}
Wenn man experimentell versucht, ein Ziel zu treffen, dann wird man
in wiederholten Versuchen die Richtung anpassen, so dass man dem Ziel
immer n"aher kommt.
Der $y$-Wert zur Zeit $p/v_x$ h"angt von der Vertikalgeschwindigkeit ab,
wir bezeichnen ihn mit $h(v_y)$.
Man ver"andert also $v_y$, bis die Gleichung $h(v_y)=0$ erf"ullt ist.
Um das Randwertproblem zu l"osen, muss man also die Gleichung $h(v_y)=0$
numerisch l"osen.

Man kann dies zum Beispiel dadurch machen, dass man nach zwei Werten von $v_y$
sucht, so dass die zum einen geh"orige Bahn unter dem Punkt $P$ durchgeht,
w"ahrend der Ball im anderen Fall dar"uber hinwegfliegt.
Durch wiederholte Halbierung des Intervalls kann man dann den korrekten
Wert f"ur $v_y$ immer genauer eingrenzen%
\footnote{%
Tats"achlich wird dieses Verfahren in der Artillerie verwendet.
Der Schiesskommandant beobachtet die einschlagenden Granaten und kommandiert
"Anderungen der Anfangs-Elevation an die Gesch"utzbatterien.
Dabei sucht er Einschl"age, die aus seiner Perspektive vor bzw.~hinter
dem Ziel liegen, und halbiert dann das Intervall, bis die Einschl"age dem
Ziel gen"ugend nahe kommen.}.
Der Nachteil dieses Verfahrens ist, dass mit jedem Schritt die Genauigkeit
nur um in Bit ansteigt, es sind also sehr viele Iterationen notwendig.

Schnellere Konvergenz kann mit dem Newton-Verfahren erreicht werden,
welches in Anhang~\ref{chapter:newton} beschrieben wird.
F"ur die Anwendung des Newton-Verfahrens auf das Randwert-Problem
ist die Bestimmung der Steigung der Funktion n"otig, die die Abweichung
der Kurve von der Randbedingung am rechten Rand angibt.
Wir m"ussen also berechnen, wie schnell sich $y(p/v_x)$ "andert,
wenn $v_y$ ver"andert wird.
Dies ist die Ableitung
\[
h'(v_y)= \frac{\partial y}{\partial v_y},
\]
ein Eintrag der Jacobi-Matrix.
In Abschnitt~\ref{grundlagen:XXX} wurde gezeigt, wie man auch f"ur
die Jacobi-Matrix eine Differentialgleichung aufstellen kann, die
man nat"urlich ebenfalls mit den fr"uher beschriebenen numerischen
Bibliotheken l"osen kann.

Das Randwertproblem kann daher mit folgendem Algorithmus numerisch gel"ost
werden.
\begin{enumerate}
\item Beginne mit einer Sch"atzung f"ur $v_y$
\item Finde numerisch die L"osung des Anfangswertproblems mit $v_y$
als anf"angliche Vertikalgeschwindigkeit.
Berechnet dabei auch die Jacobi-Matrix
\item Lese die $h(v_y)$ aus der L"osung zur Zeit $p/v_x$ ab, und $h'(v_y)$
aus der Jacobi-Matrix und verwendet den Newton-Algorithmus
(\ref{numerik:ball-newton}), um eine verbesserte Sch"atzung von $v_y$ 
zu bekommen.
\item Wiederhole Schritte 2 und 3 bis die Randbedingung f"ur $t=p/v_x$
gen"ugend genau erf"ullt ist.
\item Die L"osung des Anfangswertproblems mit diesem $v_y$ ist die
L"osung des gestellten Randwertproblems.
\end{enumerate}

\begin{beispiel}
\begin{figure}
\centering
\includegraphics{chapters/images/randwert-1.pdf}
\caption{L"osungen des Anfangswertproblems~(\ref{numerik:ball-dgl-1}) und
(\ref{numerik:ball-dgl-2}).
Das Newton-Verfahren korrigiert $v_y$ derart, dass $h(v_y)=0$ wird.
So wird die L"osung des Randwertproblems (rot) gefunden.
\label{numerik:randwert-bild}}
\end{figure}
Wir f"uhren den eben skizzierten Algorithmus f"ur das Ball-Problem durch.
Um die Jacobi-Matrix zu berechnen, m"ussen wir die Ableitung von $f$ berechnen:
\begin{equation}
\frac{\partial f(x,y)}{\partial y}
=
\begin{pmatrix}
0& 0& 1& 0\\
0& 0& 0& 1\\
0& 0& 0& 0\\
0& 0& 0& 0
\end{pmatrix}.
\end{equation}
Da die rechte Seite nicht von $y$ abh"angt, k"onnen wir die Gleichung f"ur
die Jacobi-Matrix ganz unabh"angig von $y$ l"osen.
Da $F$ so einfach ist, kann man das Matrizenprodukt direkt ausrechnen, 
so wird die Differentialgleichung f"ur $J$
\begin{equation}
\begin{pmatrix}
J'_{11}&J'_{12}&J'_{13}&J'_{14}\\
J'_{21}&J'_{22}&J'_{23}&J'_{24}\\
J'_{31}&J'_{32}&J'_{33}&J'_{34}\\
J'_{41}&J'_{42}&J'_{43}&J'_{44}
\end{pmatrix}
=
\begin{pmatrix}
0& 0& 1& 0\\
0& 0& 0& 1\\
0& 0& 0& 0\\
0& 0& 0& 0
\end{pmatrix}
\begin{pmatrix}
J_{11}&J_{12}&J_{13}&J_{14}\\
J_{21}&J_{22}&J_{23}&J_{24}\\
J_{31}&J_{32}&J_{33}&J_{34}\\
J_{41}&J_{42}&J_{43}&J_{44}
\end{pmatrix}
=
\begin{pmatrix}
J_{31}&J_{32}&J_{33}&J_{34}\\
J_{41}&J_{42}&J_{43}&J_{44}\\
     0&     0&     0&     0\\
     0&     0&     0&     0
\end{pmatrix}
\end{equation}
\begin{table}
\centering
\begin{tabular}{|>{$}r<{$}|>{$}r<{$}|>{$}r<{$}|>{$}r<{$}|>{$}r<{$}|>{$}r<{$}|>{$}r<{$}|>{$}r<{$}|}
\hline
n&    v_y&    t& x(t)&      y(x)&\displaystyle\frac{\partial^{\mathstrut}y}{\partial v_y}&v_{y,\text{new}}&\Delta\\
\hline
0& 7.0000&  2.5& 20.0&-13.156250&  2.5& 12.26250000& -5.2625000000\\
1&12.2625&  2.5& 20.0& -0.000004&  2.5& 12.26250145& -0.0000014458\\
2&12.2625&  2.5& 20.0&  0.000000&  2.5& 12.26250143&  0.0000000204\\
\hline
\end{tabular}
\caption{Newton-Algorithmus f"ur das Ball-Problem, Resultate der numerischen
Rechnung.
$v_y$ wird in drei Schritten mit einer Genauigkeit von mehr als 10 Stellen
gefunden.
\label{numerik:newton-resultate}}
\end{table}%
Daraus kann man ablesen, dass die Elemente $J_{3j}$ und $J_{4j}$ sich
nicht "andern, sie bleiben also konstant.
Aber auch in den ersten zwei Zeilen k"onnen sich nur die Elemente $J_{13}$
und $J_{24}$ "andern, die Differentialgleichungen f"ur diese Elemente
sind
\begin{align*}
J'_{13}&=1\\
J'_{24}&=1
\end{align*}
oder in Matrixform:
\begin{equation}
J(x) = \begin{pmatrix}
1&0&x&0\\
0&1&0&x\\
0&0&1&0\\
0&0&0&1
\end{pmatrix}
\end{equation}
Die L"osung $J_{13}(x)=x$ und $J_{24}=x$.
Damit haben wir die n"otige Information, um den Newton-Algorithmus
durchzuf"uhren.
In Tabelle~\ref{numerik:newton-resultate}
sind die Resultate der numerischen Rechnung zusammengestellt.
Es zeigt sich, dass der korrekte Wert f"ur $v_y$ in drei Iterationen
mit 10 Stellen Genauigkeit gefunden werden kann.
Damit ist das Randwertproblem numerisch gel"ost.
\end{beispiel}

%
% potenzreihen.tex -- Lösung von Differentialgleichungen mit Potenzreihen
%
% (c) 2015 Prof Dr Andreas Mueller, Hochschule Rapperswil
%
\chapter{Potenzreihen-Methode\label{chapter:potenzreihen}}
\lhead{}
\rhead{Potenzreihen-Methode}
\section{Analytische L"osungen}
\section{Trigonometrische Funktionen}

%
% linear.tex -- L"osung linearer Differentialgleichungen
%
% (c) 2015 Prof Dr Andreas Mueller, Hochschule Rapperswil
%
\chapter{Lineare Differentialgleichungen\label{chapter:linear}}
\lhead{}
\rhead{Lineare Differentialgleichungen}
Es lohnt sich, lineare Differentialgleichungen unter Zuhilfenahme
der linearen Algebra etwas genauer zu untersuchen.
\section{Definition}
\section{L"osungsmenge}
\section{Normalformen}
\subsection{Diagonalisierung}
\subsection{Jordan-Normalform}


%
% geometry.tex -- L"osung linearer Differentialgleichungen
%
% (c) 2016 Prof Dr Andreas Mueller, Hochschule Rapperswil
%
\chapter{Geometrische Eigenschaften\label{chapter:geometrie}}
\rhead{}
\lhead{Geometrische Eigenschaften}
Die Geometrie schr"ankt die m"oglichen Bahnen eines
Differentialgleichungssystems bereits wesentlich ein.
In einem eindimensionalen autonomen System sind keine
Schwingungen m"oglich, in einem zweidimensionalen
System gibt es keine chaotischen Bewegungen.
Ziel dieses Kapitels ist, in diese geometrische
Denkweise einzuf"uhren.
Das Buch \cite{skript:hirsch} f"uhrt diesen Ansatz weiter bis zu
einer Einf"uhrung in chaotische Bewegung.

\section{Autonome Systeme}
\rhead{Autonome Systeme}
\begin{figure}
\centering
\includegraphics{chapters/images/geometrie-13.pdf}
\caption{Entwicklung des Systems~(\ref{geometrie:harvest-equation})
mit $a=5$ und $h=0.8$
\label{geometrie:harvest-graph}}
\end{figure}%
Ein eindimensionales System k"onnen wir schreiben als
\[
y'=f(x,y).
\]
Die L"osungskurven dieses Systems k"onnen wir in einem $x$-$y$-Diagramm
als Graphen darstellen.
Zum Beispiel k"onnen wir die Entwicklung des Systems
\begin{equation}
y' = ay(1-y)-h(1+\sin 2\pi x)
\label{geometrie:harvest-equation}
\end{equation}
wie in Abbildung~\ref{geometrie:harvest-graph} darstellen.
Dieses nicht-autonome System hat zwei Grenzzyklen: 
Anfangswerte oberhalb der roten Kurve f"uhren zu L"osungskurven, die
sich f"ur zunehmendes $x$ der blauen Kurve ann"ahern.
Anfangswerte unterhalb der blauen Kurve f"uhren zu L"osungskurven, die
sich f"ur abnehmendes $x$ der roten Kurve n"ahern.

Startwerte in der N"ahe der roten Kurve sind nicht stabil,
die Entwicklung f"ur zunehmende $x$ f"uhrt den Wert immer weiter von
der roten Kurve weg.
Lag der Startwert unterhalb der roten Kurve, wird die L"osung gegen
$-\infty$ divergieren.
Alle anderen L"osungen konvergieren gegen die blaue Kurve.

Offenbar ist es ziemlich schwierig, das
System~(\ref{geometrie:harvest-equation}) "uber gr"ossere $x$-Intervalle
zu verstehen. 
Zum Beispiel h"angt das Verhalten davon ab, ob der Startwert zwischen den
farbigen Kurven liegt oder ausserhalb, und diese h"angt vom Start-$x$ ab.

In Kapitel~\ref{chapter:grundlagen} wurde beschrieben, wie jedes
Differentialgleichungssystem, m"oglicherweise nach Erweiterung um eine
explizite Zeitkoordinate, zu einem autonomen System gemacht und damit
durch ein Vektorfeld ersetzt werden kann,
und wie die L"osungen als Bahnen eines Teilchens in einem zeitlich
unver"anderlichen Vektorfelde verstanden werden k"onnen.
Daraus ergibt sich auch, dass das Verhalten der L"osungen einer
Differentialgleichungen studiert werden kann, ohne dass man
die Abh"angigkeit von $x$ im Detail kennt.
Die Bahnen zerlegen den Raum und k"onnen sich nicht schneiden,
so schr"ankt die Geometrie des Raumes die M"oglichkeiten f"ur das
Verhalten der L"osung "uber lange Zeiten ein.
Besonders ausgepr"agt sind diese Einschr"ankungen im zweidimensionalen
Raum.
Eine geschlossene Kurve in der Ebene unterteilt diese in zwei Bereiche,
keine L"osung kann vom einen Bereich in den anderen f"uhren.

Ein Parameter in der Differentialgleichung modifiziert das Vektorfeld,
und kann damit Eigenschaften von L"osungen "uber lange Zeiten ver"andern.
Zum Beispiel k"onnen periodische Bahnen sich aufl"osen und zu Spiralbahnen
werden.

Wir gehen in diesem Kapitel immer von einem autonomen System
\[
\frac{d}{dx}y(x)=f(y),
\]
wobei das Vektorfeld $f(y)$ nicht von $x$ abh"angt.
Uns interessieren die geometrische Eigenschaften der L"osungskurven, 
soweit sie sich direkt aus den Eigenschaften des Vektorfeldes ableiten
lassen.
Insbesondere interessieren uns also spezielle Punkte des Vektorfeldes,
zum Beispiel Nullstellen.


%
% Spezielle Punkte und Bahnen
%
\section{Spezielle Punkte und Bahnen}
\rhead{Spezielle Punkte und Bahnen}
Das Verhalten der L"osungskurven "uber lange Zeiten wird wesentlich
beeinflusst von Punkten, in denen die Bewegung auf der Bahn
zum Stillstand kommt oder sogar die Richtung umkehrt, und von Punkten,
in die ein Punkt periodisch zur"uckkehrt.

%
% Kritische Punkte
%
\subsection{Kritische Punkte}
Wir betrachten zun"achst einen Punkt $y_0$ in dem $f(y_0)\ne 0$.
Dann gilt $f(y)\ne 0$ auch f"ur Punkte $y$, die gen"ugend nahe an 
$y_0$ sind.
Eine L"osungskurve durch $y_0$ hat im Punkt $y_0$ die Tangentenrichtung
$f(y_0)$.
In einer gen"ugend kleinen Umgebung von $y_0$ werden sich verschiedene
L"osungskurven nicht schneiden.
Ein solcher Schnittpunkt w"are ein Anfangsbedingung f"ur zwei
verschiedene L"osungen, aber der Eindeutigkeitssatz f"ur die
L"osung einer Differentialgleichung sagt, dass es zu jeder 
Anfangsbedingung nur eine L"osung geben kann.
Dies bedeutet, dass in einer Umgebung des Punktes $y_0$ nichts
Spannendes passiert, die Bahnen sehen ungef"ahr aus wie in
Abbildung~\ref{geometrie:parallelebahnen}.
\begin{figure}
\centering
\includegraphics{chapters/images/geometrie-12.pdf}
\caption{Verlauf der Bahnen eines autonomen Differentialgleichungssystems
in der N"ahe eines Punktes $y_0$ mit $f(y_0)\ne 0$.
\label{geometrie:parallelebahnen}}
\end{figure}

Von besonderem Interesse sind daher Punkte, in denen das Vektorfeld
verschwindet:

\begin{definition}
$y$ heisst {\em kritischer Punkt} des Vektorfeldes $f$, wenn $f(y)=0$.
\end{definition}

Dass in einem kritischen Punkt das Vektorfeld verschwindet bedeutet nicht,
dass die Bahn nicht dar"uber hinweg fortgesetzt werden kann,
wie das folgende Beispiel zeigt.

\begin{beispiel}
Das eindimensionale Gleichungssystem
\[
y'=\sqrt{|y|}
\]
hat einen kritischen Punkt bei $y=0$.
Die Funktion
\begin{equation}
y(x)=\frac14x^2\operatorname{sign}(x)
\label{geometrie:1dimkritloesung}
\end{equation}
ist eine L"osungsfunktion, denn
\begin{align*}
y'(x)&=\frac12|x|\\
\sqrt{\left|\frac14x^2\operatorname{sign}(x)\right|}
&=
\sqrt{\frac14x^2}
=
\frac12|x|=y'(x).
\end{align*}
Trotzdem erreicht $y(x)$ jeden beliebigen Punkt der $y$-Achse, denn 
\[
x=2\sqrt{|y|}\operatorname{sign}(y)
\]
ist die Umkehrfunktion von $y(x)$. 
Die L"osungskurve bewegt sich also "uber den kritischen Punkt hinweg.
\begin{figure}
\centering
\includegraphics{chapters/images/geometrie-1.pdf}
\caption{Bahnen das Differentialgleichungssystems $y'=\sqrt{|y|}$
k"onnen den kritischen Punkt bei $y=0$ traversieren.
Die L"osung~(\ref{geometrie:1dimkritloesung}) ist etwas fetter rot
eingezeichnet.
L"osungen k"onnen aber auch beliebig lange im Punkt $y$ verweilen,
wie die blaue Kurve illustriert.
\label{geometrie:1dimkrit}}
\end{figure}
Abbildung~\ref{geometrie:1dimkrit} zeigt die Bahnen.
Die L"osung~(\ref{geometrie:1dimkritloesung}) ist nicht die einzige.
Andere L"osungen bestehen jeweils aus einem Ast der Kurve mit positiven
$y$ und einem mit negativen $y$, dazwischen kann die L"osung beliebig lange
im kritischen Punkt verweilen.
F"ur den Punkt $y=0$ ist also der Eindeutigkeitssatz verletzt, zur
Anfangsbedingung $y=0$ gibt es beliebig viele L"osungen.
\end{beispiel}

%
% Linearisierung
%
\subsection{Linearisierung}
In einem kritischen Punkt $y_0$ verschwinden alle Komponenten von $f(y_0)$.
Falls $f$ stetig differenzierbar ist, k"onnen wird $f$ in einer Umgebung
des kritischen Punktes in erster Ordnung mit Hilfe der Ableitung
approximieren:
\[
f(y)=\frac{\partial f}{\partial y}(y-y_0) + o(|y-y_0|)
\]
Die partielle Ableitung bezeichnet im mehrdimensionalen Fall
die Jacobi-Matrix, sie ist die $n\times n$-Matrix bestehend aus allen
partiellen Ableitungen der Komponenten von $f$ nach den Variablen $y_i$.
In einer Umgebung eines kritischen Punktes ist die Gestalt der Bahnen also
im wesentlichen durch die Jacobi-Matrix bestimmt.

\begin{beispiel}
Wir betrachten als Beispiel die eindimensionale ($n=1$) Differentialgleichung
\[
y'=f(y)=y,
\]
die in $y_0=0$ einen kritischen Punkt hat.
Die Jacobi-Matrix ist konstant: $f'(y)=1$.
L"osungskurven mit Anfangsbedingungen $>0$ werden daher anwachsen,
w"ahrend L"osungskurven mit Anfangsbedingungen $<0$ abnehmen werden.
Die L"osungskurven werden daher vom kritischen Punkt ``abgestossen''.

W"ahlen wir stattdessen das System
\[
y'=f(y)=-y,
\]
dann ist die Jacobi-Matrix $f'(y)=-1$, und wir k"onnen analog schliessen,
dass L"osungskurven immer zum kritischen Punkt hinstreben.
\end{beispiel}

\begin{beispiel}
Das Beispiel im letzten Abschnitt verwendet
\[
y'=f(y)=\sqrt{|y|}
\]
ist im kritischen Punkt $y=0$ nicht stetig differenzierbar, daher ist
eine lineare Approximation nicht m"oglich.
Das fr"uher beobachtete Verhalten, dass sich L"osungskurven auf der
einen Seite vom kritischen Punkt entfernen, auf der anderen aber ann"ahern,
kann nur auftreten, wenn auch $f'(0)=0$ gilt, so dass wir $f$ in zweiter
Ordnung als
\[
y'=f(y)=\frac12 f''(0)y^2.
\]
Diese Differentialgleichung hat die Funktion
\[
y(x)=-\frac{2}{f''(0)x+C}
\]
als L"osung.
Je nach Wert von $C$ bekommen wir eine L"osung, f"ur $x>0$ anwachsen
oder abfallen, wenigstens f"ur ein kleines Intervall.
Die L"osung n"ahert sich dem kritischen Punkt, oder entfernt sich, je
nachdem auf welcher Seite die L"osungskurve beginnt.
\end{beispiel}

%
% Eindimensionale Systeme
%
\section{Eindimensionale Systeme}
\rhead{Eindimensionale Systeme}
Wir untersuchen in diesem Abschnitt das Verhalten eindimensionaler autonomer
Systeme, und konzentrieren uns dabei auf Eigenschaften, die allein
auf Grund der geringen Dimension erschliessen lassen.
Ein solches System hat die Form
\[
y'=f(y),\quad y\in\mathbb R
\]
und kann sofort mit Hilfe von Separation der Variablen gel"ost werden:
\[
\int\frac{dy}{f(y)} = t+C,
\]
was nat"urlich nur funktioniert, wenn $f$ konstantes Vorzeichen hat.
Daraus folgt, dass ein Startwert $y_0$, f"ur den $f(y_0)>0$ ist,
zu einer monoton wachsenden L"osung f"uhrt, w"ahrend $f(y_0)<0$
bedeutet, dass die L"osung monoton f"allt.
Erf"ullt ein Startwert $f(y_0)=0$, dann ist $y(x)=y_0$ eine
L"osung der Differentialgleichung.
Die kritischen Punkte von $f$ strukturieren also wie erwartet die
L"osungsmenge.
Wir suchen eine "ubersichtliche Visualisierung dieser Beobachtung
und wollen verstehen, wie sich die Struktur in Abh"angigkeit
von einem Parameter in der Differentialgleichung ver"andern kann.

%
% Mögliche Phasendiagramme in einer Dimension
%
\subsection{Phasenportraits}
Seien jetzt $y_1,y_2,\dots$ Nullstellen der Funktion $f(y)$.
Dann sind die Funktion $y(x)=y_i$ L"osungen der Differentialgleichung
$y'=f(y)$.
Die Bewegung zwischen den Nullstellen wird vollst"andig dominiert durch
das Vorzeichen von $f(y)$ zwischen den Nullstellen.
Wir k"onnen dies in einem sogenannten Phasen-Portrait visualisieren.
Abbildung~\ref{geometrie:phasenportrait} zeigt die L"osungen der
Differentialgleichung
\begin{equation}
y'=f(y)=y^3-y
\label{geometrie:cube}
\end{equation}
Die Nullstellen dieser Funktion sind $-1$, $0$ und $1$.
\begin{figure}
\centering
\includegraphics{chapters/images/geometrie-14.pdf}
\caption{Fluss der Differentialgleichung~(\ref{geometrie:cube}) und
Phasenportrait
\label{geometrie:phasenportrait}}
\end{figure}
Die exakte Abh"angigkeit von $x$ ist oft nicht wichtig, entscheidend
ist nur die Tatsache, dass Punkte im blauen Gebiet sich mit zunehmendem
$x$ nach unten bewegen, w"ahrend Punkte im roten Gebiet sich nach
oben bewegen.
Diese Information wird auch durch das Phasenportrait am linken
Rand der Abbildung~\ref{geometrie:phasenportrait} dargestellt.
Der Vortail eines Phasenportraits gegen"uber der Darstellung
der L"osung ist, dass sie mit einer Dimension auskommt.

%
% Mögliche Änderungen (Bifurkationen) der Phasendiagramme
%
\subsection{Bifurkationen\label{geometrie:subsection:bifurkationen}}
Betrachten wir jetzt eine Differentialgleichung, die ausserdem von einem
Parameter $b$ abh"angt:
\[
y'=f(y,b).
\]
Das Phasenportrait wird sich ver"andern, wenn $b$ variert, wir
m"ochten die verschiedenen dabei m"oglichen "Uberg"ange verstehen
k"onnen.
Es interessiert vor allem kleine "Anderungen in der Umgebung eines
Wertes $b_0$.
Dabei kommt es vor allem darauf an, dass mit Nullstellen passiert, und
welches Vorzeichen $f$ links und rechts von der Nullstelle hat.
Wir gehen daher davon aus, dass $f$ f"ur $b_0$ an der Stelle $y=0$
hat, dass also $a_0(b_0)=0$ ist.

Wenn $a_1(b_0)\ne0$ ist, also wenn $f'(0, b_0)\ne 0$ ist, dann 
gilt dies auch in einer Umgebung von $b_0$, und daher wird die
Nullstelle in erster N"aherung nur verschoben:
\[
y_0(b) \simeq -\frac{f(0,b)}{f'(0,b)},
\]
Das Phasenportrait erlebt also keine grundds"atzlichen "Anderung.

\subsubsection{Sattel-Knoten-Bifurkation}
Wir betrachten jetzt den Fall $f(0,b_0)=0$ und $f'(0,b_0)=0$, die Funktion
$f$ ist f"ur $b=b_0$ quadratisch, mit einer Nullstelle im Scheitel.
Variert $b$, wird zwar der Graph von $y\mapsto f(y,b)$ seine
Gestalt "andern, aber er wird weiterhin "ahnlich wie eine Parabel
aussehen.
Etwas interessantes passiert nur, wenn der Scheitel von die horizontale
Achse "uberquert, wenn also beim Durchgang von $b$ durch den Wert $b_0$
die Zahl der Nullstellen von $0$ auf $2$ steigt oder umgekehrt.
Dies bedeutet, dass 
\[
f''(y_0,b_0)\ne 0
\qquad
\text{und}
\qquad
\frac{\partial f}{\partial b}(y_0, b_0)\ne 0.
\]
Modellhaft k"onnen wir dies durch das System
\[
f(y,b)=y^2+b
\]
wiedergeben.
Es hat f"ur $b<0$ die Nullstellen $\pm\sqrt{-b}$, f"ur $b>0$ jedoch
keine Nullstellen.
Das zugeh"orige Phasenportrait ist in Abbildung~\ref{geometrie:saddle-node}
dargestellt.
Bei dieser Art von Bifurkation entstehen neue kritische Punkte
paarweise, davon ist jeweils einer stabil, der andere instabil.
Diese Art der Bifurkation heisst {\em Sattel-Knoten-}
oder {\em Saddle-Node-Bifurkation}.
\index{Sattel-Knoten-Bifurkation}
\index{Saddle-Node-Bifurkation}

\begin{figure}
\centering
\includegraphics{chapters/images/bifurkation-1.pdf}
\caption{Phasendiagramm f"ur die Sattel-Knoten-Bifurkation in
Abh"angigkeit vom Parameter $b$.
\label{geometrie:saddle-node}}
\end{figure}

\subsubsection{Heugabel-Bifurkation}
\begin{figure}
\centering
\includegraphics{chapters/images/bifurkation-2.pdf}
\caption{Phasendiagramm f"ur die Heugabel-Bifurkation in Abh"angigkeit vom
Parameter $b$.
\label{geometrie:pitchfork}}
\end{figure}
Eine andere Art von Bifurkatione zeigt das Modell
\[
f(y,b)=y^3-by.
\]
F"ur $b<0$ hat die Funktion $y\mapsto f(y,b)$ nur die eine
reelle Nullstelle $y=0$.
F"ur $b>0$ dagegen liegen die drei verschiedenen Nullstellen
$-\sqrt{b}$, $0$  und $\sqrt{b}$.
Das zugeh"orige Phasendiagramm ist in Abbildung~\ref{geometrie:pitchfork}
dargestellt, diese Art von Bifurkation heisst {\em Heugabel-}
oder {\em Pitchformk-Bifurkation}.
\index{Heugabel-Bifurkation}
\index{Pitchfork-Bifurkation}
Bei dieser Art von Bifurkation werden aus einer Nullstelle deren drei.
War die Nullstelle instabil (wie in Abbildung~\ref{geometrie:pitchfork}),
entstehen zwei neue instabile Nullstellen, die bisherige Nullstellen 
wird stabil.
War die eine Nullstelle stabil, entstehen zwei stabile Nullstellen, die
eine Nullstelle wird instabil.

\subsubsection{Transkritische Bifurkation}
\begin{figure}
\centering
\includegraphics[width=\hsize]{chapters/images/bifurkation-3.pdf}
\caption{Phasendigramm f"ur die transkritische Bifurkation in Abh"angigkeit
vom Parameter $b$.
\label{geometrie:transkritisch}}
\end{figure}
Das Standard-Modell f"ur die sogenannten {\em transkritische Bifurkation} ist 
\index{transkritische Bifurkation}
\[
f(y,b)=y^2+yb = y(y+b)
\]
(Abbildung~\ref{geometrie:transkritisch}).
F"ur $b=0$ hat dieses System einen kritischen Punkt bei $y=0$.
Eine L"osungskurve, die bei negativem $y$ beginnt, wird immer n"aher
an den Punkt $y=0$ herankommen.
Liegt der Anfangswert jedoch im Gebiet $y>0$, wird sich die L"osungskurve
immer weiter von $y=0$ entfernen.
Der Fixpunkt bei $y=0$ ist also weder stabil noch instabil.

F"ur $b\ne 0$ entsteht ein weiterer kritischer Punkt bei $y=-b$.
Da f"ur $y\to\pm\infty$ die Funktion $y\mapsto f(y,b)$ immer positiv
ist, ist der rechte der beiden Fixpunkte immer stabil, der
linke immer instabil.


%
% Zweidimensionale Systeme
%
\section{Zweidimensionale Systeme}
\rhead{Zweidimensionale Systeme}
In diesem Abschnitt betrachten wir die geometrischen Einschr"ankungen, denen
zweidimensionale autonome Systeme unterliegen.
Wieder sind Punkte von besonderem Interesse, in denen $f$ verschwindet.
Bei eindimensionalen Systemen teilen solche Punkte den $y$-Bereich in
verschiedene Intervall, in denen die Bewegung nur in jeweils
eine Richtung erfolgen kann, damit ist es leicht zu entscheiden,
ob ein solcher kritischer Punkt stabil oder instabil ist.
In zwei Dimensionen legen die Punkte alleine den Charakter noch nicht
fest, insbesondere ist es ja auch m"oglich, dass eine Bahn einen solche
Punkt umschliesst.

\subsection{Nullklinen}
Kritische Punkte von $f$ sind solche, in denen beide Komponenten
des Vektors $f(y)$ verschwinden.
Die Gleichungen
\[
f_1(y)=0
\qquad\text{und}\qquad
f_2(y)=0
\]
beschreiben zwei Kurvenscharen in der Ebene, die sogenannten 
{\em Nullklinen}.
\index{Nullkline}
Die Schnittpunkte von Nullklinen sind die kritischen Punkte.

Da auf einer Nullkline jeweils eine der Komponenten von $f(y)$ verschwindet,
schneiden L"osungskurven Nullklinen immer entweder horizontal oder
vertikal.
Ausserdem trennen die Nullklinen Gebiete unterschiedlicher Bewegungsrichtung
entlang einer Achse. 
Die Nullkline $f_1(y)=0$ trennt Gebiete, in denen sich eine L"osungskurve
nach rechts (zu gr"osseren $y_1$ hin) bewegt ($f_1(y)>0$) von Gebieten,
in denen sich die L"osungskurve nach links ($f_1(y)<0$) bewegt.
Desgleichen trennt die Nullkline $f_2(y)=0$ Gebiete mit Bewegung ``nach oben''
von Gebieten mit Bewegung ``nach unten''.
Allein aus dieser Information kann man sich bereits ein recht gutes
qualitives Bild "uber den Verlauf der L"osungen verschaffen, wie wir
im Folgenden an zwei Beispiel illustrieren wollen.

\begin{beispiel}
\begin{figure}
\centering
\includegraphics{chapters/images/nullklinen-1.pdf}
\caption{Nullklinen der Differentialgleichung~(\ref{geometrie:nullklinen-dgl1}),
die $y_1$-Nullklinen in rot, die $y_2$-Nullklinen in blau.
Kritische Punkte sind die Schnittpunkte verschiedenfarbiger Nullklinen.
\label{geometrie:nullklinen1}}
\end{figure}
\begin{figure}
\centering
\includegraphics{chapters/images/nullklinen-2.pdf}
\caption{Vektorfeld der Differentialgleichung~(\ref{geometrie:nullklinen-dgl1}),
es best"atigt die Resultate der qualitiativen Diskussion aus
Abbildung~\ref{geometrie:nullklinen1}.
\label{geometrie:nullklinen-fluss}}
\end{figure}
\begin{figure}
\centering
\includegraphics{chapters/images/nullklinen-3.pdf}
\caption{Vektorfeld der Differentialgleichung~(\ref{geometrie:nullklinen-dgl1})
in einer Umgebung des instabilen kritischen Punktes $(\frac12,\frac12)$.
\label{geometrie:nullklinen-instabil}}
\end{figure}
\begin{figure}
\centering
\includegraphics{chapters/images/nullklinen-4.pdf}
\caption{Vektorfeld der Differentialgleichung~(\ref{geometrie:nullklinen-dgl1})
in einer Umgebung des stabilen kritischen Punktes $(2,0)$.
\label{geometrie:nullklinen-stabil}}
\end{figure}
Wir betrachten das nichtlineare System 
\begin{equation}
\frac{d}{dx} \begin{pmatrix}y_1\\y_2\end{pmatrix}
=
\begin{pmatrix}
2y_1\biggl(1-\displaystyle\frac{y_1}2\biggr)-3y_1y_2\\
y_2(1-y_2)-y_1y_2
\end{pmatrix}.
\label{geometrie:nullklinen-dgl1}
\end{equation}
Die direkte L"osung ist ziemlich aussichtslos, wir versuchen daher mit
Hilfe der Nullklinen ein qualitatives Bild
(Abbildung~\ref{geometrie:nullklinen1}) zu erhalten.

Die $y_2$-Nullkline hat die Gleichung
\[
0=y_2(1-y_2)-y_1y_2=y_2(1-y_2-y_1)
\qquad\Rightarrow\qquad
y_2=0
\quad\text{oder}\quad
y_1+y_2=1.
\]
L"osungskurven schneiden also die beiden Geraden $y_2=0$ (die $y_1$-Achse)
und $y_1+y_2=1$ horizontal.
Da die $y_1$-Achse bereits horizontal ist, bedeutet dies, dass sich die
L"osungskurven der $y_1$-Achse anschmiegen.

Die $y_1$-Nullkline hat die Gleichung
\[
0=2y_1\biggl(1-\frac{y_1}2\biggr)-3y_1y_2=y_1(2-y_1-3y_2),
\qquad\Rightarrow\qquad
y_1=0
\quad\text{oder}\quad
y_1+3y_2=2.
\]
Wir schliessen wieder, dass die L"osungskurven die $y_2$-Achse und
die Gerade $y_1+3y_2=3$ vertikal schneiden.
Da die $y_2$-Achse schon vertikal ist, m"ussen sich auch dort
die L"osungskurven anschmiegen.

Wir k"onnen aus den Nullklinen auch die kritischen Punkte ableiten.
Da sind zun"achst die Punkte auf den Achsen, also zum Beispiel
der Schnittpunkt der $y_2$-Nullkline $y_2=0$ mit der $y_1$-Nullkline
$y_1+3y_2=2$, also $(2,0)$, oder der Schnittpunkt der
$y_1$-Nullkline $y_1=0$ mit der $y_2$-Nullkline $y_1+y_2=0$, also $(0,1)$.
Ausserdem ist nat"urlich $(0,0)$ ein kritischer Punkt.
ein vierter kritischer Punkt entsteht als Schnittpunkt
der $y_1$-Nullkline $y_1+3y_2=2$ mit der $y_2$-Nullkline $y_1+y_2=1$,
also als L"osung des linearen Gleichungssystems
\[
\begin{linsys}{2}
y_1&+&3y_2&=&2\phantom{.}\\
y_1&+& y_2&=&1,
\end{linsys}
\]
es hat die L"osung $(\frac12,\frac12)$.
Die kritischen Punkte sind also
\begin{equation}
(0,0),\quad
(2,0),\quad
(0,1)\quad\text{und}\quad
\biggl(\frac12,\frac12\biggr).
\label{geometrie:nullklinen-krit}
\end{equation}

An den aus den Nullklinen l"asst sich auch die Bewegungsrichtung der
L"osungen im Bezug auf die kritischen Punkte ablesen.
\label{geometrie:nullklinen-stabilitaet}
Aus dem Gebiet oben rechts und unten links in der N"ahe des Ursprungs
bewegen sich die L"osungen zun"achst auf den kritischen Punkt
bei $(\frac12,\frac12)$ zu, weichen dann aber ab in Richtung auf die
kritischen Punkte $(0,1)$ und $(2,0)$ zu.
Die kritischen Punkte $(0,0)$ und $(\frac12,\frac12)$ sind also
instabil, w"ahrend $(2,0)$ und $(0,1)$ stabil sind.
Dies wird auch von dem genaueren Vektorfeld in
Abbildung~\ref{geometrie:nullklinen-fluss} best"atigt.
In Abschnitt~\ref{geometrie:umgebung-kritisch} wird gezeigt, dass man 
die Bewegung in der Umgebung eines kritischen Punktes mit Hilfe der
Eigenwerte der Ableitungen klassifizieren kann.
\end{beispiel}


\begin{beispiel}
Das {\em Fitzhugh-Nagumo-Modell} wird verwendet, um das Verhalten eines Neurons
zu simulieren.
\index{Fitzhugh-Nagumo-Modell}
\begin{figure}
\centering
\includegraphics{chapters/images/nullklinen-5.pdf}
\caption{Nullklinen des Fitzhugh-Nagumo-Modells bei nur einem kritischen Punkt,
$a=-0.8$, $b=0.9$.
\label{geometrie:nullklinen-fh-1}}
\end{figure}
\begin{figure}
\centering
\includegraphics{chapters/images/nullklinen-6.pdf}
\caption{Nullklinen des Fitzhugh-Nagumo-Modells mit drei kritischen Punkten,
$b=4$, $a=0$.
\label{geometrie:nullklinen-fh-2}}
\end{figure}
Es verwendet das Differentialgleichungssystem
\begin{equation}
\begin{aligned}
    \dot v&= v-\frac13v^3-w\\
\tau\dot w&= v-a-bw.
\end{aligned}
\label{geometrie:fitzhugh-dgl}
\end{equation}
Wir versuchen, uns wieder mit Hilfe der Nullklinen ein Bild von den
L"osungskurven zu verschaffen.
Die $v$-Nullkline hat die Gleichung
\[
0=v-\frac13v^3-w
\qquad\Rightarrow\qquad
w=v-\frac13v^3 = v(1-{\textstyle\frac1{\sqrt{3}}}v)(1+{\textstyle\frac1{\sqrt{3}}}v)
\]
Diese kubische Parabel hat im Nullpunkt die Steigung $1$.
Die $w$-Nullkline ist die Gerade
\[
0=v-a-bw
\qquad\Rightarrow\qquad
v=bw+a
\qquad\Rightarrow\qquad
w = \frac{v-a}{b}.
\]
Wenn die Steigung $1/b$ dieser Geraden gr"osser als $1$ ist, dann schneidet
die Gerade die kubische Parabel nur in einem Punkt, es gibt dann nur
einen kritischen Punkt.
Ist die Steigung $1/b<1$, gibt es f"ur nicht zu grosses $|a|$ drei
Schnittpunkte.

In Abbildung~\ref{geometrie:nullklinen-fh-1} sind die Nullklinen des
Fitzhugh-Nagumo-Modells mit nur einem kritischen Punkt dargestellt.
Die L"osungskurven bewegen sich in Spiralen im Gegenurzeigersinn
um den Fixpunkt herum.
In diesem Fall erlauben die Nullklinen keine abschliessende Beurteilung
der Stabilit"at des Fixpunktes.
Die Untersuchung der Eigenwerte der Jacobi-Matrix, die im nächsten
Abschnitt erkl"art wird, erm"oglicht die Entscheidung, und wird in
der Fortsetzung dieses Beispiels auf Seite~\pageref{geometrie:fh-fortsetzung}
durchgef"uhrt.

In Abbildung~\ref{geometrie:nullklinen-fh-2} sind die Nullklinen des
Fitzhugh-Nagumo-Modells dargstellt mit drei kritschen Punkten.
Man kann sofort ablesen, dass $(0,0)$ ein instabiler kritischer Punkt ist.
Die L"osungskurven, die von diesem Punkt abgestossen werden, n"ahern sich
einem der beiden anderen kritischen Punkte und bewegen sich
in einer Spirale um diesen Punkt herum.
Da sich die L"osungskurven nicht schneiden d"urfen kann man folgern,
dass die beiden anderen kritschen Punkte stabil sein m"ussen,
die L"osungskurven werden sich ihnen in Spiralbahnen n"ahern.
\label{geometrie:fh-diskussion}
\end{beispiel}

%
% Bewegung in der Umgebung eines kritischen Punktes
%
\subsection{Bewegung in der Umgebung eines kritischen Punktes
\label{geometrie:umgebung-kritisch}}
Wir gehen jetzt davon aus, dass 
\[
y'=f(y)
\]
einen kritischen Punkt hat, wir k"onnen die Koordinaten immer so w"ahlen,
dass der kritische Punkt im Nullpunkt des Koordinatensystems liegt.
Die Bewegung in unmittelbarer Umgebung des Nullpunktes kann dann approximiert
werden durch die Bewegung des linearisierten Systems
\[
y'=\frac{\partial f(0)}{\partial y}y
\]
Die m"oglichen Bewegungsformen in der Umgebung des kritischen Punktes
sind also bestimmt durch die Jacobi-Matrix.
Jede beliebige $2\times 2$-Matrix kann auch tats"achlich als Jacobi-Matrix
vorkommen, denn das System
$
y'=Ay
$
hat die Matrix $A$ als Jacobi-Matrix.

Wir interessieren uns im Moment nur f"ur eine qualitative Beschreibung
der L"osungen, wir k"onnen also immer eine Koordinatentransformation
vornehmen, um die Situation zu vereinfachen.
Die gesuchte qualitative Klassifizierung von zweidimensionalen
Differentialgleichungssystemen l"auft also auf eine Klassifizierung
von reellen $2\times 2$-Matrizen bis auf Koordinatentransformation
hinaus.

Eine solche Klassifikation kann auf der Basis von Eigenwerten und
Eigenvektoren erfolgen.
Dazu ben"otigen wir eine "Ubersicht "uber die Eigenwerte einer
Matrix
\[
A=\begin{pmatrix}a&b\\c&d\end{pmatrix},
\]
die wir als Nullstellen des charakteristischen Polynoms bestimmen
k"onnen.
Das charakteristische Polynom ist
\[
\chi_A(\lambda)
=
\det(A-\lambda E)
=
\left|\,\begin{matrix}a-\lambda&b\\c&d-\lambda\end{matrix}\,\right|
=
(a-\lambda)(d-\lambda)-bc
=
\lambda^2-(a+b)\lambda + ad-bc,
\]
es ist bestimmt durch die Spur und die Determinante der Matrix
\[
\begin{aligned}
\det A&=ad -bc,
&
\operatorname{Spur}A&=a+d
&
&\Rightarrow
&
\chi_A(\lambda)&=\lambda^2-\lambda \operatorname{Spur}A+\det A
\end{aligned}
\]
Die L"osungsformel f"ur die quadratische Gleichung liefert die
Eigenwerte
\[
\lambda_{1,2}
=
\frac{\operatorname{Spur}A}2\pm\sqrt{\Delta}
\qquad
\qquad
\text{mit}\quad
\Delta = \biggl(\frac{\operatorname{Spur}A}2\biggr)^2 - \det A
\]
Falls die Diskriminanten $\Delta > 0$ ist, sind die beiden Eigenwerte
verschieden, und folglich gibt es zwei verschiedene Eigenvektoren,
die Matrix $A$ kann diagonalisiert werden mit Diagonalelementen
$\lambda_{1,2}$.  Das Differentialgleichungssystem zerf"allt dann
in zwei unabh"angige eindimensionale Systeme
\begin{align*}
y_1'&= \lambda_1 y_1\\
y_2'&= \lambda_2 y_2,
\end{align*}
die auch durch
\begin{align*}
y_1(x)&=y_{10} e^{\lambda_1 x}\\
y_2(x)&=y_{20} e^{\lambda_2 x}
\end{align*}
sofort gel"ost werden k"onnen.
F"ur die Diskussion der Form der L"osungskurven brauchen wir aber die
Abh"angigkeit der beiden Koordinaten untereinander, nicht von $x$.
Wir stellen daher $y_2$ als Funktion von $y_1$ dar:
\[
y_1=y_{10} e^{\lambda_1 x}
\qquad\Rightarrow\qquad
x=\frac1{\lambda_1}\log\frac{y_1}{y_{10}}
\qquad\Rightarrow\qquad
y_2
=
y_{20} e^{\frac{\lambda_2}{\lambda_1}\log\frac{y_1}{y_{10}}}
=
y_{20}\biggl(\frac{y_1}{y_{10}}\biggr)^{\frac{\lambda_2}{\lambda_1}}
=
Cy^{\frac{\lambda_2}{\lambda_1}}
\]
\begin{figure}
\centering
\begin{tabular}{ccc}
\includegraphics{chapters/images/geometrie-2.pdf}&%
\includegraphics{chapters/images/geometrie-3.pdf}&%
\includegraphics{chapters/images/geometrie-4.pdf}\\
$\displaystyle \frac{\lambda_2}{\lambda_1}>1$&%
$\displaystyle \frac{\lambda_2}{\lambda_1}=1$&%
$\displaystyle \frac{\lambda_2}{\lambda_1}<1$
\end{tabular}
\caption{L"osungskurven des linearisierten Systems f"ur
$\frac{\lambda_2}{\lambda_1}>0$.
\label{geometrie:posportraits}}
\end{figure}

\begin{figure}
\centering
\begin{tabular}{ccc}
\includegraphics{chapters/images/geometrie-5.pdf}&%
\includegraphics{chapters/images/geometrie-6.pdf}&%
\includegraphics{chapters/images/geometrie-7.pdf}\\
$\displaystyle \frac{\lambda_2}{\lambda_1}>-1$&%
$\displaystyle \frac{\lambda_2}{\lambda_1}=-1$&%
$\displaystyle \frac{\lambda_2}{\lambda_1}<-1$
\end{tabular}
\caption{L"osungskurven des linearisierten Systems f"ur
$\frac{\lambda_2}{\lambda_1}<0$.
\label{geometrie:negportraits}}
\end{figure}
Die Gestalt der L"osungskurven sind also im Wesentlichen durch den
Quotienten $\lambda_2/\lambda_1$ bestimmt.
Die Abbildung~\ref{geometrie:posportraits} zeigt die L"osungskurven
f"ur positive Werte des Quotienten, w"ahrend
Abbildung~\ref{geometrie:negportraits} die L"osungskurven f"ur
negative Werte des Quotienten zeigt.
F"ur positive Werte von $\lambda_2/\lambda_1$ bewegen sich die 
Punkte entweder immer auf den kritischen Punkt zu, oder entfernen
sich.
F"ur negative Werte von $\lambda_2/\lambda_1$ n"ahert sich ein
Punkt in der N"ahe der einen Achse zun"achst immer mehr dem kritischen
Punkt, um sich dann in Richtung der anderen Achse zu entfernen.

Die Matrix $A$ muss jedoch nicht diagonalisierbar sein, wenn
$\lambda_1=\lambda_2$.
Wenn die Matrix nur einen Eigenvektor hat, dann kann die Matrix
durch eine geeignete Koordinatentransformation in die Form
\[
\begin{pmatrix}
\lambda&      1\\
      0&\lambda
\end{pmatrix}
\]
bringen.
Die Differentialgleichungen lauten in diese Fall
\begin{align*}
y_1'&=\lambda y_1 + y_2\\
y_2'&=\lambda y_2
\end{align*}
Die zweite Gleichung kann wie vorher gel"ost werden, die L"osung f"ur $y_2$
ist
\[
y_2(x)=y_{20}e^{\lambda x},
\]
dies k"onnen wir in die erste Gleichung einsetzen, sie lautet jetzt
\[
y_1' = \lambda y_1 + y_{20}e^{\lambda x},
\]
dies ist wieder eine lineare Differentialgleichung, diesmal jedoch
eine inhomogene. 
Die L"osung der homogenen Gleichung ist $Ce^{\lambda x}$, die L"osung
der inhomogenen Gleichung kann durch Variation der Konstanten gefunden
werden, also $y_1(x)=C(x)e^{\lambda x}$.
Setzen wir dies in die Differentialgleichung 
\[
y_1'(x)
=
C'(x)e^{\lambda x}+C(x)\lambda e^{\lambda x}
=
\lambda y_1(x) + C'(x)e^{\lambda x}
=
\lambda y_1(x) + y_{20}e^{\lambda x}
\]
ein.
Diese Gleichung kann nur erf"ullt sein, wenn
\[
C'(x)=y_{20}
\qquad\Rightarrow\qquad
C(x)=y_{20}x+y_{10},
\]
die L"osung der Gleichung ist also
\[
y(x)=\begin{pmatrix}
y_{20}x+y_{10}\\
y_{20}
\end{pmatrix}e^{\lambda x}.
\]
\begin{figure}
\centering
\includegraphics{chapters/images/geometrie-8.pdf}
\caption{L"osungskurven f"ur den Fall nicht diagonalisierbarer Jacobi-Matrix
mit zwei gleichen Eigenwerten.
\label{geometrie:jnf-kurven}}
\end{figure}%
In Abbildung~\ref{geometrie:jnf-kurven} sind die L"osungskurven dargestellt.
F"ur $\lambda >0$ streben die L"osungskurven gegen den Nullpunkt, aber auf
eine Art, sie im Grenzfall die $y_1$-Achse ber"uhren.

Falls die Diskriminante $\Delta$ negativ ist, gibt es keine reellen
Eigenwerte, also auch keine reellen Eigenvektoren.
Wir k"onnen aber trotzdem eine Basis finden, in der die Geometrie
dar Bahnkurven leichter verst"andlich ist.
Dazu schreiben wir 
\[
\alpha = \frac{\operatorname{Spur}A}2=\frac{a+d}2
\qquad
\text{und}
\qquad
\beta = \sqrt{-\Delta},
\]
und betrachten die beiden Vektoren
\[
w_1 = \begin{pmatrix}0\\\beta\end{pmatrix}.
\qquad\text{und}\qquad
w_2 = \begin{pmatrix}b\\\alpha-a\end{pmatrix}
\]
Wir berechnen die Wirkung der Matrix $A$ auf diesen beiden Vektoren,
und zerlegen jeweils das Resultat wieder in $w_1$ und $w_2$:
\begin{align*}
Aw_1
&=
\begin{pmatrix}a&b\\c&d\end{pmatrix}
\begin{pmatrix}0\\\beta\end{pmatrix}
=
\begin{pmatrix}
\beta b\\
\beta d
\end{pmatrix}
=
v
\begin{pmatrix}0\\\beta\end{pmatrix}
+
\beta
\begin{pmatrix}b\\\alpha-a\end{pmatrix}
\\
Aw_2
&=
\begin{pmatrix}a&b\\c&d\end{pmatrix}
\begin{pmatrix}b\\\alpha-a\end{pmatrix}
=
\begin{pmatrix}
ab-ab+b\alpha\\
bc-ad+\alpha d
\end{pmatrix}
=
\alpha
\begin{pmatrix}b\\\alpha-a\end{pmatrix}
+u
\begin{pmatrix}0\\\beta\end{pmatrix}
\end{align*}
Wir m"ussen nur noch die Konstanten $u$ und $v$ bestimmen:
\begin{align*}
\beta d
&=
v\beta
+
\alpha\beta
-
\beta a
&&\Rightarrow&
v
&=
\alpha a+d-\alpha=2\alpha-\alpha=\alpha
\\
-\det A +\alpha d
&=
\alpha^2- \alpha a+u\beta
&&\Rightarrow&
u\beta&=-\det A +\alpha(a+d)-\alpha^2
=\alpha^2-\det A
=\Delta=-\beta^2
\end{align*}
Aus der zweiten Gleichung folgt $ u=-\beta$.
Damit haben wir die Wirkung der Matrix $A$ auf den Vektoren $w_1$ und $w_2$
bestimmt, und wir k"onnen daraus die Matrix von $A$ in der Basis
$\{w_1,w_2\}$ ablesen, wir bezeichnen sie mit $A'$
\[
\begin{aligned}
Aw_1&=\alpha w_1 + \beta w_2\\
Aw_2&=-\beta w_1 + \alpha w_2
\end{aligned}
\qquad\Rightarrow\qquad
A'=\begin{pmatrix}
\alpha&\beta\\
-\beta&\alpha
\end{pmatrix}
\]
Dies ist die gesuchte Form der Matrix, in der sich die L"osungskurven
leichter beschreiben lassen.
Eine L"osung daf"ur l"asst sich angeben, wenn man ber"ucksichtigt, dass
$A'$ einer Drehmatrix "ahnelt.
Wir vermuten daher, dass die L"osungskurve im wesentlichen den kritischen
Punkt umkreist, m"oglicherweise mit einer "Anderung des Abstandes
zum kritischen Punkt, und schreiben daher
\begin{equation}
y(x)
=
\begin{pmatrix}
r_0e^{ux}\cos(vx+\delta_0)\\
r_0e^{ux}\sin(vx+\delta_0)
\end{pmatrix}
=
r_0e^{ux}
\begin{pmatrix}
\cos(vx+\delta_0)\\
\sin(vx+\delta_0)
\end{pmatrix}
,
\label{geometrie:rotsol}
\end{equation}
wobei wird $r_0$ und $\delta_0$ so w"ahlen, dass
\[
y_{10}=r_0\cos\delta_0
\qquad\text{und}\qquad
y_{20}=r_0\sin\delta_0.
\]
Setzen wir jetzt den Ansatz~(\ref{geometrie:rotsol}) in die
Differentialgleichung ein, erhalten wir
\begin{align*}
y'(x)
&=
\begin{pmatrix}
r_0ue^{ux}\cos(vx+\delta_0)-r_0e^{ux}v\sin(vx+\delta_0)\\
r_0ue^{ux}\sin(vx+\delta_0)+r_0e^{ux}v\cos(vx+\delta_0)
\end{pmatrix}
\\
&=
\begin{pmatrix}
 \alpha&\beta\\
-\beta &\alpha
\end{pmatrix}
\begin{pmatrix}
r_0e^{ux}\cos(vx+\delta_0)\\
r_0e^{ux}\sin(vx+\delta_0)
\end{pmatrix}
=
\begin{pmatrix}
r_0e^{ux}\alpha\cos(vx+\delta_0)+r_0e^{ux}\beta\sin(vx +\delta_0)\\
-r_0e^{ux}\beta\cos(vx+\delta_0)+r_0e^{ux}\alpha\sin(vx+\delta_0)
\end{pmatrix}
\end{align*}
Diese Gleichung ist genau dann korrekt, wenn 
\[
u=\alpha
\qquad\text{und}\qquad
v=-\beta.
\]
Die Zahlen $\alpha$ und $\beta$ charakterisieren also wieder die L"osung.
F"ur $\alpha < 0$ n"ahern sich die L"osungen dem kritischen Punkt, f"ur
$\alpha>0$ entfernen sie sich.
Die Zahl $\beta$ ist die Winkelgeschwindigkeit, mit der die L"osung
um den kritischen Punkt rotiert.
Die L"osungskurven sind daher Spiralen um den kritischen Punkt, sie
sind in Abbildung~\ref{geometrie:rotkurv} dargestellt.
\begin{figure}
\centering
\begin{tabular}{ccc}
\includegraphics{chapters/images/geometrie-9.pdf}&%
\includegraphics{chapters/images/geometrie-11.pdf}&%
\includegraphics{chapters/images/geometrie-10.pdf}\\
$\alpha > 0$&$\alpha = 0$&$\alpha < 0$
\end{tabular}
\caption{L"osungskurven des linearisierten Systems im Falle $\Delta < 0$
sind Spiralen um den kritischen Punkt
\label{geometrie:rotkurv}}
\end{figure}

\begin{beispiel}
Wir kehren nochmals zum Beispiel~(\ref{geometrie:nullklinen-dgl1})
von Seite~\pageref{geometrie:nullklinen-dgl1} zur"uck.
Die kritischen Punkte wurden in (\ref{geometrie:nullklinen-krit}) bereits
zusammengestellt.
Die Ableitung von $f$, also die Jacobi-Matrix, ist
\[
\frac{\partial f}{\partial y}
=
\frac{\partial}{\partial y}
\begin{pmatrix}
2y_1-y_1^2-3y_1y_2\\
y_2-y_2^2-y_1y_2
\end{pmatrix}
=
\begin{pmatrix}
2-2y_1-3y_2 & -3y_1\\
-y_2        &1-2y_2-y_1
\end{pmatrix}
\]
In den vier kritischen Punkten finden wir die folgenden Matrizen und
Eigenwerte
\begin{align*}
&(0,0)
	&\frac{\partial f}{\partial y}&=\begin{pmatrix}2&0\\0&1\end{pmatrix}
		&&\lambda_1=2,\;\lambda_2=1\\
&(2,0)
	&\frac{\partial f}{\partial y}&=\begin{pmatrix}-2&-6\\0&-1\end{pmatrix}
		&&\lambda_1=-2,\;\lambda_2=-1\\
&(0,1)
	&\frac{\partial f}{\partial y}&=\begin{pmatrix}-1&0\\-1&-1\end{pmatrix}
		&&\lambda_1=-1,\;\lambda_2=-1\\
&\textstyle(\frac12,\frac12)
	&\frac{\partial f}{\partial y}&=\begin{pmatrix}-\frac12&-\frac32\\-\frac12&-\frac12\end{pmatrix}
		&&\lambda_1=\frac{\sqrt{3}-1}2,\;\lambda_2=-\frac{\sqrt{3}+1}2
\end{align*}
Nur bei den Punktein $(2,0)$ und $(0,1)$ sind beide Eigenwerte negativ,
nur diese beiden Punkte sind stabil, wie bereits die  Diskussion der
Nullklinen auf Seite~\pageref{geometrie:nullklinen-stabilitaet}
gezeigt hat.
\end{beispiel}

%
% Komplexe Eigenwerte
%
\subsection{Komplexe Eigenwerte}
Die Darstellung im vorangegangenen Abschnitt war darum bem"uht,
komplexe Zahlen zu vermeiden.
Die Darstellung im Falle $\Delta<0$ wurde dadurch unn"otig verkompliziert,
in diesem Abschnitt soll gezeigt werden, wie die Formeln f"ur die Vektoren
$w_1$ und $w_2$ mit Hilfe komplexer Zahlen hergeleitet werden k"onnen.

Zun"achst halten wir fest, dass im Falle $\Delta<0$ zwei konjugiert
komplexe Eigenwerte
$
\lambda= \alpha + i\beta
$
und
$
\overline{\lambda}= \alpha - i\beta
$
existieren.
Nehmen wir an, dass $v$ ein Eigenvektor zum Eigenwert $\lambda$ ist,
dann ist $\overline{v}$, dessen Komponenten die konjugiert komplexen
Komponenten von $v$ sind, ein Eigenvektor zum Eigenwert $\overline{\lambda}$.
Grund daf"ur ist die Tatsache, dass die Matrix $A$ nur reelle Matrixelemente
hat, also gilt
\[
A\overline{v}
=
\overline{Av}
=
\overline{\lambda v}=\overline{\lambda}\overline{v}.
\]
Der Vektor $v$ ist nat"urlich nicht geeignet f"ur eine reelle Beschreibung
der L"osungskurven des linearisierten Systems.
Wir konstruieren daher die Vektoren
\begin{equation}
\begin{aligned}
w_1&=\frac{i}2(v-\overline v)
&&\qquad
&
v&=-iw_1+w_2
\\
w_2&=\frac12(v+\overline v)
&&\qquad
&
\overline{v}&=iw_1+w_2
\end{aligned}
\label{geometrie:wv}
\end{equation}
und untersuchen, wie die Matrix $A$ darauf wirkt:
\begin{align*}
Aw_1
&=
\frac{i}2(Av-A\overline v)
=
\frac{i}2(\lambda v-\overline{\lambda}\overline{v})
\\
Aw_2
&=
\frac12(Av+A\overline{v})
=
\frac12(\lambda v+\overline{\lambda}\overline{v}).
\end{align*}
Setzen wir die Darstellungen von $v$ und $\overline{v}$ durch $w_i$ aus 
(\ref{geometrie:wv}) ein, und erhalten:
\begin{align*}
\frac{i}2(\lambda v-\overline{\lambda}\overline{v})
&=
\frac{i}2(\lambda(-iw_1+w_2) -\overline{\lambda}(iw_1+w_2))
=
\frac{1}2(\lambda+\overline{\lambda}) w_1
+
\frac{i}2(\lambda-\overline{\lambda}) w_2
=\alpha w_1-\beta w_2
\\
\frac12(\lambda v+\overline{\lambda}\overline{v})
&=
\frac12(\lambda(-iw_1+w_2)+\overline{\lambda}(iw_1+w_2))
=
-\frac{i}2(\lambda-\overline{\lambda}) w_1
+
\frac12(\lambda+\overline{\lambda}) w_2.
=\beta w_1+\alpha w_2
\end{align*}
Verwendet man also $\{w_1,w_2\}$ als Basis, dann bekommt die Matrix die
Form
\begin{equation}
A'=\begin{pmatrix}
\alpha&-\beta\\
\beta &\alpha
\end{pmatrix}
\label{geometrie:drehmatrix}
\end{equation}
Auf Grund der Konstruktion haben die Vektoren $w_1$ und $w_2$ reelle
Komponenten, $w_1$ ist der Realteil des Vektors $v$, $w_2$ ist
der Imagin"arteil.
Damit haben wir ein Rezept, wie wir eine Basis von reellen Vektoren
konstruieren k"onnen, in denen das System die
Form~(\ref{geometrie:drehmatrix}) hat.

Die Komponenten eines Eigenvektors $v$ erf"ullen die Gleichung
\[
(a-\lambda)v_1 + bv_2=0
\]
eine L"osung daf"ur ist
\[
v=\begin{pmatrix}
-b\\
a-\alpha-i\beta
\end{pmatrix},
\]
dessen Real- und Imagin"arteile
\[
\begin{pmatrix}
-b\\a-\alpha
\end{pmatrix}
\qquad\text{und}\qquad
\begin{pmatrix}
0\\\beta
\end{pmatrix},
\]
dies sind die Vektoren, die im vorangegangenen Abschnitt aus dem "Armel
gesch"uttelt worden waren, um die Matrix des Systems in die
Form~(\ref{geometrie:drehmatrix}) zu bringen.

\begin{beispiel}
\label{geometrie:fh-fortsetzung}
Wir wenden obige Analyse auf das
Fitzhugh-Nagumo-Modell~(\ref{geometrie:fitzhugh-dgl}) von
Seite~\pageref{geometrie:fitzhugh-dgl} an.
Um die Diskussion etwas zu vereinfachen, untersuchen wir nur den Fall
$\tau = 1$.
Wir m"ussen die Jacobi-Matrix in einem kritischen Punkt bestimmen, sie ist
\begin{equation}
J=
\begin{pmatrix}
1-v^2 &  -1 \\
  1   &  -b
\end{pmatrix}.
\end{equation}
Das charakteristische Polynom ist
\begin{align*}
\det(J-\lambda E)
&=
\left|
\begin{matrix}
1-v^2-\lambda&-1\\
1&-b-\lambda
\end{matrix}
\right|
\\
&=
(1-v^2-\lambda)(-b-\lambda)+1
\\
&=
(\lambda+v^2-1)(\lambda+b)+1
\\
&=
\lambda^2 + (\underbrace{v^2 + b - 1}_{\textstyle p})\lambda
+ \underbrace{b(v^2 - 1)+1}_{\textstyle q}.
\end{align*}
Es hat die Nullstellen
\begin{equation}
\lambda_{1,2}
=
-\frac{p}{2}\pm\sqrt{\frac{p^2}4-q}
=
-\frac{v^2+b-1}2\pm\sqrt{\frac{(v^2+b-1)^2}4-b(v^2-1)-1}.
\label{geometrie:fn-eigenwerte}
\end{equation}
Der erste Term in der Wurzel ist das Quadrat des Terms vor der Wurzel.
Ohne $q$ in der Wurzel g"abe es einen Eigenwert mit dem gleichen Vorzeichen
wie $p$, der andere ist $0$.
Die Vorzeichen von $p$ und $q$ bestimmen also weitgehen, ob ein 
kritischer Punkt stabil oder instabil ist.

\begin{enumerate}
\item
Wenn $q<0$ ist, dann ist die Wurzel gr"osser als $p/2$, die beiden
Eigenwerte haben verschiedenes Vorzeichen, und der kritische Punkt
ist instabil unabh"angig vom Vorzeichen von $p$. 
Dieser Fall tritt ein, wenn 
\[
b(v^2 -1) + 1 < 0
\qquad\Rightarrow\qquad
1-\frac1b > v^2.
\]
F"ur $b<1$ wird die linke Seite negativ, dann kann dieser Fall also gar nicht
eintreten.
\item
Ist $q>0$ und von gen"ugend grossem absolutem Betrag,
dann wird der Radikand negativ, beide Eigenwerte
sind komplex und der kritische Punkt ist genau dann stabil, wenn $p>0$ ist.
Ist $q>0$, aber nicht von gen"ugend grossem absolutem Betrag, dann
bleibt der Radikand positiv.
In diesem Fall haben beide Eigenwerte das gleiche Vorzeichen wie $p$,
auch in diesem Fall hat man also Stabilit"at genau dann, wenn $p>0$ ist.
\end{enumerate}
Wir k"onnen die Resultate der obigen Diskussion in der folgenden
Entscheidungstabelle zusammenfassen:
\begin{center}
\begin{tabular}{|>{$}c<{$}|>{$}c<{$}|l|}
\hline
q=b(v^2-1)+1 & p= v^2 + b -1 &           \\
\hline
     <0      &               &  instabil \\
     >0      &       <0      &  instabil \\
     >0      &       >0      &  stabil   \\
\hline
\end{tabular}
\end{center}

F"ur $b>1$ und $a=0$ ist $(0,0)$ ein kritischer Punkt, und es gilt $q=-b+1<0$
und $p=b-1>0$, wir sind daher im Fall~1, der Ursprung ist instabil.
Es ist aber auch $p=v^2+b-1>v^2\ge 0$, der Fall $p<0$ kann also 
gar nicht eintreten, ein solcher kritischer Punkt muss also immer
stabil sein.
Dies deckt sich mit den Resultaten der Diskussion von
Seite~\pageref{geometrie:fh-diskussion}.
\end{beispiel}

%
% Hopf-Bifurkation
%
\subsection{Hopf-Bifurkation}
\begin{figure}
\centering
\includegraphics{chapters/images/hopf-1.pdf}
\caption{Fluss f"ur $b<0$, der kritische Punkt ist stabil,
Bahnkurven konvergieren gegen $0$.
\label{geometrie:hopf1}}
\end{figure}%
\begin{figure}
\centering
\includegraphics{chapters/images/hopf-2.pdf}
\caption{Fluss f"ur $b=0$, der kritische Punkt ist immer noch stabil,
die Bahnkurven n"ahern sich jedoch nicht mehr exponentiell schnell
dem Nullpunkt.
\label{geometrie:hopf2}}
\end{figure}%
\begin{figure}
\centering
\includegraphics{chapters/images/hopf-3.pdf}
\caption{Fluss f"ur $b>0$, der Nullpunkt ist nicht mehr stabil, daf"ur
ist der Kreis mit Radius $\sqrt{b}$ eine stabile periodische Bahn (blau),
gegen die alle Bahnkurven exponentiell schnell konvergieren.
\label{geometrie:hopf3}}
\end{figure}%
\begin{figure}
\centering
\begin{tabular}{ccc}
\includegraphics{chapters/images/hopf-4.pdf}&%
\includegraphics{chapters/images/hopf-5.pdf}&%
\includegraphics{chapters/images/hopf-6.pdf}%
\end{tabular}
\caption{Vorzeichen von $\dot r$ in Abh"angigkeit von $b$.
Punkte mit $\dot r <0$ sind blau gef"arbt, Punkte mit $\dot r >0$ rot.
\label{geometrie:hopfvorzeichen}}
\end{figure}%
Die in Abschnitt~\ref{geometrie:subsection:bifurkationen} untersuchten
Bifurkationen eindimensionaler Differentialgleichungen k"onnen in
analoger Form auch bei zweidimensionalen Differentialgleichungen auftreten.
Sie sind jedoch immer eindimensionale Bifurkationen, die entlang der
durch die Eigenvektoren der Linearisierung gegebenen Richtungen
auftreten.

Die h"ohere Dimensionszahl erlaubt aber auch eine Bifurkation, bei der
ein stabiler Fixpunkt in stabil wird und einen stabilen Zyklus ``abwirft''
(Abbildungen~\ref{geometrie:hopf1}, \ref{geometrie:hopf2} and
\ref{geometrie:hopf3}).
Sie heisst die Hopf-Bifurkation.
\index{Hopf-Bifurkation}
Wir betrachten dazu das System 
\begin{equation}
\begin{aligned}
\dot r      &= r(b-r^2)\\
\dot \varphi&= -1
\end{aligned}
\label{geometrie:hopfsystem}
\end{equation}
in Polarkoordinaten.
Offenbar ist $r=0$ ein Fixpunkt.
F"ur $b>0$ gibt es ausserdem eine periodische Bahn mit $r=\sqrt{b}$
(Abbildung~\ref{geometrie:hopf3}).
Wir wollen die Stabilit"at des Fixpunktes sowie der periodischen Bahn
untersuchen.
Die Abbildung~\ref{geometrie:hopfvorzeichen} fasst die f"ur das Bahnverhalten
entscheidenden Vorzeichen in den drei F"allen $b<0$, $b=0$ und $b>0$
zusammen.

In Polarkoordinaten beschreibt die Gleichung f"ur $r$ eine
Heugabel-Bifurkation.
Der kritische Punkt $r$ ist f"ur $b<0$ stabil, er wird f"ur $b>0$
instabil, daf"ur entstehen zwei neue stabile kritische Punkte
$r=\pm\sqrt{b}$.

Unsere bisherige Theorie zur Beurteilung von Fixpunkten ging von
kartesischen Koordinaten aus, wir f"uhren daher die Analyse auch noch
in kartesischen Koordinaten durch.
Die Umrechnungsformeln von Polarkoordinaten in kartesische Koordinaten
und ihre Ableitungen
\[
\begin{aligned}
x&=r\cos\varphi&&\qquad&\dot x&=\dot r\cos\varphi-r\sin\varphi\cdot\dot\varphi\\
y&=r\sin\varphi&&\qquad&\dot y&=\dot r\sin\varphi+r\cos\varphi\cdot\dot\varphi
\end{aligned}
\]
erlauben uns,
das System~(\ref{geometrie:hopfsystem}) in kartesische Koordinaten
umzurechnen:
\begin{equation}
\begin{aligned}
\dot x&=r(b-r^2)\frac{x}{r}+y=(b-x^2-y^2)x+y\\
\dot y&=r(b-r^2)\frac{y}{r}-x=(b-x^2-y^2)y-x.
\end{aligned}
\label{geometrie:hopf-kartesisch}
\end{equation}
Zur Beurteilung der Stabilit"at des Nullpunktes berechnen wir die
Jacobi-Matrix
\[
J(x,y)=
\begin{pmatrix}
b-3x^2-y^2&1-2xy\\
-1-2xy&b-x^2-3y^2
\end{pmatrix}
\quad\Rightarrow\quad
J(0,0)=\begin{pmatrix}
b&1\\-1&b
\end{pmatrix}.
\]
Die Matrix $J(0,0)$ hat das charakteristische Polynom
\[
(b-\lambda)^2+1=0
\]
mit den Nullstellen
\[
\lambda=b\mp i.
\]
Stabilit"at wird durch das Vorzeichen des Realteils der Eigenwerte
bestimmt, wir lesen daher ab, dass der kritische Punkt $0$ stabil
ist f"ur $b<0$ und instabil f"ur $b>0$.

\section{"Ubungsaufgaben}
\rhead{"Ubungsaufgaben}
\uebungsaufgabe{601}
\uebungsaufgabe{602}
\uebungsaufgabe{603}


%
% komplex.tex -- Komplexe Differentialgleichungen
%
% (c) 2015 Prof Dr Andreas Mueller, Hochschule Rapperswil
%
\chapter{Komplexe Differentialgleichungen\label{chapter:komplexeanalysis}}
\rhead{}
\lhead{Komplexe Differentialgleichungen}
Die bisher betrachteten Differentialgleichungen waren immer f"ur
$x\in\mathbb R$ definiert.
Bei der L"osung mit Hilfe von Potenzreihen haben wir L"osungsfunktionen
gefunden, die man auch f"ur komplexe $x$-Werte auswerten kann.
Definiert man die Ableitung einer Funktionen einer komplexen Variablen
$z$ rein formal als
\[
\frac{d}{dz}z^n= nz^{n-1},
\]
dann kann man auch Potenzreihen in der Variablen $z$ formal differenzieren,
indem man jeden Term der Potenzreihe ableitet.
Und die mit der Potenzreihen-Methode gefunden L"osungen erf"ullen dann
auch die urspr"ungliche Differentialgleichung.
Dies ist aber eine rein formale "Uberlegung, da die Ableitung nach einer
komplexen Variablen noch gar nicht definiert ist.

%
% Komplex differenzierbare Funktion
%
\section{Komplex differenzierbare Funktionen}
\rhead{Komplex differenzierbare Funktionen}
Wir betrachten in diesem Kapitel komplexwertige Funktionen,
\index{komplexwertige Funktion}%
die ein einem Teilgebiet der komplexen Ebene definiert sind.
Ein {\em Gebiet} ist eine offene Teilmenge $\Omega\subset \mathbb C$.
\index{Gebiet}%
{\em Offen} heisst, dass mit jedem Punkt $z_0\in\Omega$ eine Umgebung
\index{offen}%
\index{Umgebung}%
\[
U=\{z\in\mathbb Z\,|\,|z-z_0|<\varepsilon\}
\]
ebenfalls in $\Omega$ enthalten ist, also $U\subset \Omega$ f"ur gen"ugen
kleines $\varepsilon$.
Sei also $f(z)$ eine in $\Omega\subset\mathbb C$ definierte
Funktion $f\colon\Omega\to\mathbb C$. 

Eine komplexwertige Funktion $f(z)$ kann betrachtet werden als zwei
reellwertige Funktionen von zwei Variablen $x$ und $y$:
\[
f(z)=\operatorname{Re}f(x+iy) + i \operatorname{Im}f(x+iy)
\]
Schreibt man
$\operatorname{Re}f(x+iy)=u(x,y)$
und
$\operatorname{In}f(x+iy)=v(x,y)$,
dann ist die komplexe Funktion vollst"andig durch reelle Funktionen
beschrieben.
Und nat"urlich wissen wir auch, was unter den Ableitungen der Funktionen 
$u(x,y)$ und $v(x,y)$ zu verstehen ist.
Der Funktion $f(z)$ entspricht eine Abbildung $\mathbb R^2\to\mathbb R^2$
\index{Abbildung}%
\[
(x,y)\mapsto\begin{pmatrix}u(x,y)\\v(x,y)\end{pmatrix}.
\]
Die Ableitung einer solchen Funktion im Punkt $(x_0,y_0)$
ist eine lineare Abbildung von Vektoren, die in linearer N"aherung
\index{lineare Naherung@lineare N\"aherung}
\index{Naherung@N\"aherung, lineare}
den Funktionswert bei $f(z_0 + \Delta z)$ 
\[
\begin{pmatrix}
u(x+\Delta x, y +\Delta y)\\
v(x+\Delta x, y +\Delta y)
\end{pmatrix}
=
\begin{pmatrix}
\frac{\partial u}{\partial x}&\frac{\partial u}{\partial y}\\
\frac{\partial v}{\partial x}&\frac{\partial v}{\partial y}
\end{pmatrix}
\begin{pmatrix} \Delta x\\\Delta y \end{pmatrix}
+o(\Delta x, \Delta y).
\]
In dieser Sicht einer komplexen Funktion gibt es keine einzelne Zahl, die
die Funktion einer Ableitung "ubernehmen k"onnte, die Ableitung
ist eine $2\times 2$-Matrix.

%
% Definition der komplexen Ableitungen
%
\subsection{Komplexe Ableitung}
Die Ableitung einer Funktion einer reellen Variablen wird mit Hilfe des
Grenzwertes
\[
f'(x_0)=\lim_{x\to x_0}\frac{f(x)-f(x_0)}{x-x_0}
\]
definiert, oder als diejenige Zahl $f'(x_0)\in\mathbb R$ mit der Eigenschaft,
dass
\begin{equation}
f(x)=f(x_0)+f'(x_0)(x-x_0) + o(x-x_0)
\label{komplex:abldef}
\end{equation}
gilt.
Der Term $x-x_0$ und die Gleichung \eqref{komplex:abldef} sind aber auch
f"ur komplexe Argument sinnvoll, wir definieren daher

\begin{definition}
Die komplexe Funktion $f(z)$ heisst im Punkt $z_0$ komplex differenzierbar
und hat die komplexe Ableitung $f'(z_0)\in\mathbb C$, wenn 
\index{komplex differenzierbar}%
\index{komplexe Ableitung}%
\index{Ableitung!komplexe}%
\begin{equation}
f(z)=f(z_0) + f'(z_0)(z-z_0) +o(z-z_0)
\label{komplex:defkomplabl}
\end{equation}
gilt.
\end{definition}

\begin{beispiel}
Die Funktion $z\mapsto f(z)=z^n$ ist "uberall komplex differenzierbar
und hat die Ableitung $nz^{n-1}$.
Um dies nachzupr"ufen, m"ussen wir die Bedingung~\eqref{komplex:defkomplabl}
verifizieren.
Aus einer wohlbekannten Faktorisierung von $z^n - z_0^n$ k"onnen wir den
Differenzenquotienten finden:
\begin{align*}
\frac{f(z)-f(z_0)}{z-z_0}
&=
\frac{z^n-z_0^n}{z-z_0}
=
\frac{(z-z_0)(z^{n-1}+z^{n-2}z_0+z^{n-3}z_0^2+\dots+z^{n-1})}{z-z_0}
\\
&=
\underbrace{z^{n-1}+z^{n-2}z_0+z^{n-3}z_0^2+\dots+z^{n-1}
}_{\displaystyle \text{$n$ Summanden}}.
\end{align*}
Lassen wir jetzt $z$ gegen $z_0$ gehen, wird die rechte Seite
zu $nz_0^{n-1}$.
\end{beispiel}

\begin{beispiel}
Die Funktion $z\mapsto f(z)=\bar z=x-iy$ ist nicht differenzierbar.
Wenn $f(z)=\bar z$ differenzierbar w"are, dann m"usste es eine Zahl
$a\in\mathbb C$ geben, so dass 
\[
\bar z-\bar z_0=a(z-z_0)+o(z-z_0)
\]
gilt.
w"ahlen wir $z=z_0+x$ bzw.~$z=z_0+iy$, dann erhalten wir
\[
\begin{aligned}
z-z_0&=x:&
\bar z-\bar z_0&=x
&&\Rightarrow&
\bar z-\bar z_0&=1\cdot x
&&\Rightarrow&
a&=1
\\
z-z_0&=iy:&
\bar z-\bar z_0&=-iy
&&\Rightarrow&
\bar z-\bar z_0&=-1\cdot iy
&&\Rightarrow&
a&=-1
\end{aligned}
\]
Es ist also nicht m"oglich, eine einzige Zahl $a$ zu finden, die als
die Ableitung der Funktion $z\mapsto \bar z$ betrachtet werden k"onnte.
\end{beispiel}

Das letzte Beispiel zeigt, dass
selbst Funktionen, deren Real- und Imagin"arteil beliebig oft stetig
differenzierbare Funktionen sind, nicht komplex differenzierbar
sein m"ussen.
Komplexe Differenzierbarkeit ist eine wesentlich st"arkere Bedingung
an eine Funktion, komplex differenzierbare Funktionen bilden eine
echte Teilmenge aller Funktionen, deren Real- und Imagin"arteil
differenzierbar ist.

%
% Cauchy-Riemann-Differentialgleichungen
%
\subsection{Die Cauchy-Riemann-Differentialgleichungen}
Komplexe Funktionen k"onnen nur differenzierbar sein, wenn sich die vier
partiellen Ableitungen zu einer einzigen komplexen Zahl zusammenfassen
lassen.
Um diese Beziehung zu finden, gehen wir von einer komplexen Funktion
\[
f(x+iy) = u(x,y) + iv(x,y)
\]
aus, und berechnen die Ableitung auf zwei verschiedene Arten, indem
wir sowohl nach $x$ als auch nach $iy$ ableiten:
\begin{align*}
f'(z)&
=
\lim_{x\to 0}\frac{f(z+x)-f(z)}{x}
=
\frac{\partial u}{\partial x}+i\frac{\partial v}{\partial x}
\\
f'(z)&
=
\lim_{y\to 0}\frac{f(z+iy)-f(z)}{iy}
=
\frac1{i}
\frac{\partial u}{\partial y}+\frac{\partial v}{\partial y}
=
\frac{\partial v}{\partial y}
-i
\frac{\partial u}{\partial y}.
\end{align*}
Dies ist nur m"oglich, wenn Real- und Imagin"arteile "ubereinstimmen.
Es folgt also

\begin{satz}
\label{komplex:satz:cauchy-riemann}
Real- und Imagin"arteil $u(x,y)$ und $v(x,y)$ einer
komplex differenzierbaren Funktion $f(z)$ mit $f(x+iy)=u(x,y)+iv(x,y)$
erf"ullen die Cauchy-Riemannschen Differentialgleichungen
\index{Cauchy-Riemann-Differentialgleichungen}
\begin{equation}
\begin{aligned}
\frac{\partial u}{\partial x}
&=
\frac{\partial v}{\partial y},
&
\frac{\partial u}{\partial y}
&=
-
\frac{\partial v}{\partial x}.
\end{aligned}
\label{komplex:dgl:cauchy-riemann}
\end{equation}
\end{satz}

Leitet man die Cauchy-Riemann-Differentialgleichungen nochmals nach
$x$ und $y$ ab, erh"alt man
\begin{equation*}
\begin{aligned}
\frac{\partial^2 u}{\partial x^2}
&=
\frac{\partial^2 v}{\partial x\,\partial y},
&
\frac{\partial^2 u}{\partial x\,\partial y}
&=
-\frac{\partial^2 v}{\partial x^2},
&
\frac{\partial^2 u}{\partial y\,\partial x}
&=
\frac{\partial^2 v}{\partial y^2},
&
\frac{\partial^2 u}{\partial y^2}
&=
-\frac{\partial^2 v}{\partial y\,\partial x}.
\end{aligned}
\end{equation*}
Die erste und die letzte sowie die mittleren zwei k"onnen zu jeweils
einer Differentialgleichung f"ur die Funktionen $u$ und $v$ zusammengefasst
werden, n"amlich
\begin{equation*}
\frac{\partial^2 u}{\partial x^2}
+
\frac{\partial^2 u}{\partial y^2}
=
0
\qquad\text{und}\qquad
\frac{\partial^2 v}{\partial x^2}
+
\frac{\partial^2 v}{\partial y^2}
=
0.
\end{equation*}

\begin{definition}
Der Operator 
\[
\Delta =
\frac{\partial^2}{\partial x^2}
+
\frac{\partial^2}{\partial y^2}
\]
heisst der {\em Laplace-Operator} in zwei Dimensionen.
\index{Laplace-Operator}%
\end{definition}

\begin{definition}
Eine Funktion $h(x,y)$ von zwei Variablen heisst {\em harmonisch}, wenn sie
die Gleichung
\[
\Delta h=0
\]
erf"ullt.
\index{harmonische Funktion}%
\index{harmonisch}%
\end{definition}

\begin{satz}
Real- und Imagin"arteil einer komplexen Funktion sind harmonische Funktionen.
\end{satz}

Die Cauchy-Riemann-Differentialgleichungen schr"anken also einerseits stark
ein, welche Funktionen "uberhaupt als Real- und Imagin"arteil einer
komplex differenzierbaren Funktion in Frage kommen.
Andererseits koppeln sie auch Real- und Imagin"arteil stark zusammen.

\begin{beispiel}
Von einer komplex differenzierbaren Funktion $f(z)$ sei nur der Realteil
$u(x,y)=x^3 -3xy^2$ bekannt.
Man finde alle m"oglichen Funktionen $f(z)$.

Zun"achst kontrollieren wir, ob dies "uberhaupt ein Realteil sein kann,
indem wir nachrechnen, ob $u(x,y)$ harmonisch ist.
\begin{equation*}
\begin{aligned}
\frac{\partial u}{\partial x}
&=
3x^2-3y^2
&&\Rightarrow&
\frac{\partial^2 u}{\partial x^2}
&=
6x
\\
\frac{\partial u}{\partial y}
&=
-6xy
&&\Rightarrow&
\frac{\partial^2 u}{\partial y^2}
&=
-6x
\\
&&&&\Delta u&=\frac{\partial^2u}{\partial x^2}+\frac{\partial^2u}{\partial y^2}=6x-6x=0,
\end{aligned}
\end{equation*}
$u$ ist also harmonisch.

Um die Funktion $f$ zu finden, brauchen wir jetzt noch den Imagin"arteil.
Wir finden ihn mit Hilfe der Cauchy-Riemann-Differentialgleichungen.
Es gilt
\begin{equation}
\begin{aligned}
\frac{\partial v}{\partial x}
&=
-\frac{\partial u}{\partial y}=6xy,
&
\frac{\partial v}{\partial y}
&=
\frac{\partial u}{\partial x}=3x^2-3y^2
\end{aligned}
\label{komplex:crbeispiel}
\end{equation}
Aus der ersten Gleichung erh"alt man durch Integrieren nach $x$ 
\[
v(x,y)=-3x^2y + C(y),
\]
die Integrations-``Konstante'' ist eine Funktion, die aber nur von $y$
abh"angen darf.
Die zweite Cauchy-Riemann-Gleichung verwendet die Ableitung von $v$ nach $y$,
sie ist
\[
\frac{\partial v}{\partial y}=3x^2+C'(y).
\]
Aus der zweiten Gleichung von \eqref{komplex:crbeispiel} liest man
ab, dass
\[
C'(y)=-3y^2
\qquad\Rightarrow\qquad
C(y)=-y^3+k
\]
sein muss.
Damit ist $v$ bis auf eine Konstante bestimmt.
Die zugeh"orige Funktion $f(z)$ ist daher
\[
f(z)=f(x+iy)=x^3-3xy^2+i(3x^2y-y^3)+ik
=x^3 + 3x^2iy + 3x(iy)^2+(iy)^3+ik=z^3+ik.
\]
Wir haben die Funktion $f(z)$ bis auf eine Konstanten $ik$ 
aus ihrem Realteil rekonstruiert.
\end{beispiel}

Die Cauchy-Riemann-Differentialgleichungen besagen auch, dass man nur
die Ableitungen nach $x$ zu berechnen braucht, um die Ableitung $f'(x)$
zu bestimmen.
Die Rechenregeln f"ur die Ableitung lassen sich daher direkt auf
komplexe Funktionen "ubertragen:
\begin{align*}
\frac{d}{dz}z^n
&=
nz^{n-1}
\\
\frac{d}{dz}e^z
&=
e^z
\\
\frac{d}{dz}f(g(z))
&=
f'(g(z)) g'(z)
\\
\frac{d}{dz}\bigl(f(z)g(z)\bigr)
&=
f'(z)g(z)+f(z)g'(z)
\end{align*}

%
% Analytische Funktionen
%
\subsection{Analytische Funktionen}
Als wichtiges Beispiel komplex differenzierbarer Funktionen betrachten
wir die analytischen Funktionen.
\begin{definition}
Eine Funktion $f\colon\mathbb C\to\mathbb C$ heisst analytisch im Punkt
\index{analytisch}%
$z_0$, wenn sie in eine konvergente Potenzreihe
\begin{equation}
f(z)=\sum_{k=0}^\infty a_k(z-z_0)^k
\label{komplex:freihe}
\end{equation}
entwickelt werden kann.
\end{definition}

Da die Ableitungsregeln f"ur komplex differenzierbare Funktionen nicht
anders sind als die Ableitungsregeln f"ur relle Funktionen, muss gelten
\begin{equation}
f'(z)=\sum_{k=1}^\infty ka_k(z-z_0)^{k-1},
\label{komplex:fpreihe}
\end{equation}
und es stellt sich nur die Frage, ob diese Potenzreihe ebenfalls konvergent
ist.
Man kann dies mit der Formel f"ur den Konvergenzradius pr"ufen.
\index{Konvergenzradius}%
F"ur die Potenzreihe \eqref{komplex:freihe} f"ur $f(z)$ liefert sie den
Konvergenzradius
\[
\frac{1}{\varrho} = \limsup_{k\to\infty} \root{k}\of{|a_k|}.
\]
Dieselbe Formel f"ur die Reihe~\eqref{komplex:fpreihe} liefert
f"ur den Konvergenzradius der Reihenentwicklung der Ableitung $f'(z)$
\[
\limsup_{k\to\infty} \root{k}\of{k|a_k|}
=
\underbrace{\lim_{k\to\infty} \root{k}\of{k}}_{\textstyle=1}
\cdot
\underbrace{ \limsup_{k\to\infty} \root{k}\of{|a_k|}}_{\textstyle=1/\varrho}
=
\frac1{\varrho}.
\]
Die Reihe f"ur die Ableitung $f'(z)$ hat also den gleichen Konvergenzradius
wie die Reihe f"ur die Funktion $f(z)$, dies gilt nat"urlich auch f"ur
die h"oheren Ableitungen.

Eine analytische Funktion ist somit beliebig oft komplex differenzierbar.
Aus den Ableitungen kann wie bei reellen Funktionen die Taylor-Reihe
gebildet werden, sie muss mit der Potenzreihe \eqref{komplex:freihe}
"ubereinstimmen.
\index{Taylor-Reihe}%
Analytische Funktionen haben also eine konvergente Taylor-Reihe,
\[
f(z) = \sum_{k=0}^\infty \frac{f^{(k)}(z_0}{k!}(z-z_0)^k.
\]

%
% Wegintegrale und die Cauchy-Formel
%
\subsection{Wegintegrale\label{subsection:wegintegrale}}
Das Finden einer Stammfunktion, die Integration, ist die Grundtechnik,
\index{Stammfunktion}%
mit der man den "Ubergang von lokaler Information in Form von Ableitungen,
zu globaler Information "uber reelle Funktionen vollzieht.
Sie liefert aus der Steigung zwischen zwei Punkten $x_0$ und $x$ den
Funktionswert mittels
\[
f(x)=f(x_0)+\int_{x_0}^xf'(\xi)\,d\xi.
\]
Bei einer reellen Funktion gibt es nur eine Richtung, entlang der man
integrieren k"onnte.

Auch in der komplexen Ebene erwarten wir eine Formel
\[
f(z) = f(z_0) + \int_{z_0}^z f'(\zeta)\,d\zeta.
\]
In der komplexen Ebene gibt es aber beliebig viele Wege, mit denen die
Punkte $z_0$ und $z$ verbunden werden k"onnen. 
Der Wert von $f(z)$ muss also durch Integration entlang eines speziell
gew"ahlten Weges $\gamma$
\[
f(z) = f(z_0) + \int_{\gamma} f'(\zeta)\,d\zeta
\]
bestimmt werden.
Es muss also zun"achst gekl"art werden, wie ein solches Wegintegral
"uberhaupt zu verstehen und zu berechnen ist.
Dann gilt es zu untersuchen, inwieweit diese Konstruktion unabh"angig
von der Wahl des Weges ist.
F"ur komplex differenzierbare Funktionen wird sich eine sehr erfolgreiche
Theorie ergeben.

%
% Wegintegrale
%
\subsubsection{Definition des Wegintegrals}
Ein Weg in der komplexen Ebene ist eine Abbildung 
\index{Abbildung}%
\[
\gamma\colon [a,b]\to\mathbb C: t\mapsto \gamma(t).
\]
Wir verlangen f"ur unsere Zwecke zus"atzlich, dass $\gamma$ differenzierbar
ist.
Dann k"onnen wir f"ur jede beliebige Funktion das Wegintegral definieren.

\begin{definition}
Sei $\gamma\colon[a,b]\to\mathbb C$ ein Weg in $\mathbb C$ und $f(z)$
eine stetige komplexe Funktion, dann heisst
\[
\int_{\gamma} f(z)\,dz = \int_a^bf(\gamma(t)) \gamma'(t)\,dt
\]
das {\em Wegintegral} von $f(z)$ entlang der Kurve $\gamma$.
\index{Wegintegral}
\end{definition}

\begin{beispiel}
Man berechne das Wegintegral der Funktion $f(z)=z^n$ entlang des
Weges 
$\gamma(t)=1+t+it^2$
f"ur $t\in[0,1]$.

Die Definition besagt
\begin{align*}
\int_\gamma f(z)\,dz
&=
\int_0^1 f(\gamma(t))\gamma'(t)\,dt
=
\int_0^1 \gamma(t)^n \gamma'(t)\,dt
=
\int_0^1 \frac{d}{dt}\frac{\gamma(t)^{n+1}}{n+1}\,dt
\\
&=
\biggl[\frac{\gamma(t)^{n+1}}{n+1}\biggr]_0^1
=
\frac{(2+i)^{n+1}}{n+1}-\frac{1^{n+1}}{n+1}
=
\frac{(2+i)^{n+1}-1}{n+1}.
\end{align*}
Man stellt in diesem Beispiel auch fest, dass das Integral offenbar
unabh"angig ist von der Wahl des Weges, es kommt einzig auf die
beiden Endpunkte an:
\[
\int_\gamma z^n \,dz = \frac1{n+1}\bigl(\gamma(1)^{n+1}-\gamma(0)^{n+1}\bigr).
\]
\end{beispiel}

\begin{beispiel}
Wir berechnen als Beispiel das Wegintegral der Funktion $f(z)=1/z$ entlang
eines Halbkreises von $1$ zu $-1$. 
Es gibt zwei verschiedene solche Halbkreise:
\begin{equation*}
\begin{aligned}
\gamma_+(t)&=e^{it},&t&\in[0,\pi]
\\
\gamma_-(t)&=e^{-it},&t&\in[0,\pi]
\end{aligned}
\end{equation*}
Wir finden f"ur die Wegintegrale
\begin{align*}
\int_{\gamma_+}\frac1z\,dz
&=
\int_0^\pi \frac1{e^{it}}ie^{it}\,dt=i\int_0^\pi\,dt=i\pi
\\
\int_{\gamma_-}\frac1z\,dz
&=
-\int_0^\pi \frac1{e^{-it}}ie^{-it}\,dt=-i\int_0^\pi\,dt=-i\pi
\end{align*}
Das Wegintegral zwischen $1$ und $-1$ h"angt also mindestens f"ur diese
spezielle Funktion $f(z)=1/z$ von der Wahl des Weges ab.
\end{beispiel}

Wie Wahl der Parametrisierung der Kurve hat keinen Einfluss auf den 
Wert des Wegintegrals.

\begin{satz}
Seien $\gamma_1(t), t\in[a,b],$ und $\gamma_2(s),s\in[c,d]$
verschiedene Parametrisierungen
\index{Parametrisierung}%
der gleichen Kurve, es gebe also eine Funktion $t(s)$ derart, dass
$\gamma_1(t(s))=\gamma_2(s)$.
Dann ist
\[
\int_{\gamma_1}f(z)\,dz
=
\int_{\gamma_2}f(z)\,dz.
\]
\end{satz}

\begin{proof}[Beweis]
Wir verwenden die Definition des Wegintegrals
\begin{align*}
\int_{\gamma_1} f(z)\,dz
&=
\int_a^b f(\gamma_1(t))\,\gamma_1'(t)\,dt
=
\int_c^d f(\gamma_1(t(s))\,\underbrace{\gamma_1'(t(s)) t'(s)}_{\displaystyle
=\frac{d}{ds}\gamma_1(t(s))}\,ds
\\
&=
\int_c^d f(\gamma_2(s)\,\gamma_2'(s)\,ds
=
\int_{\gamma_2}f(z)\,dz
\end{align*}
Beim zweiten Gleichheitszeichen haben wir die Formel f"ur die
Variablentransformation $t=t(s)$ in einem Integral verwendet.
\index{Variablentransformation}%
\end{proof}

Wir erwarten, dass das Wegintegral "ahnlich wie das Integral reeller
Funktionen eine Art ``Umkehroperation'' zur Ableitung ist.
Wir untersuchen daher den Fall, dass $f(z)$ eine komplexe Stammfunktion $F(z)$
hat, also $f(z)=F'(z)$.
Wir berechnen das Wegintegral entlang des Weges $\gamma$:
\begin{align*}
\int_{\gamma}f(z)\,dz
&=
\int_a^bf(\gamma(t))\,\gamma'(t)\,dt
=
\int_a^bF'(\gamma(t))\,\gamma'(t)\,dt
=
\int_a^b\frac{d}{dt}F(\gamma(t))\,dt
=
F(\gamma(a))-F(\gamma(b))
\end{align*}
Dies ist genau die Formel, die man als den Hauptsatz der Infinitesimalrechnung
kennt.
Trotzdem ist die Situation hier etwas anders.
In der reellen Infinitesimalrechnung war die Existenz einer Stammfunktion
durch das Integral gesichert, man konnte mit
\[
F(x)=\int_a^xf(\xi)\,d\xi
\]
immer eine Stammfunktion angeben.
Im komplexen Fall k"onnen wir nat"urlich auch versuchen, eine Stammfunktion
mit Hilfe von 
\[
F(z)=\int_{\gamma_z} f(\zeta)\,d\zeta
\]
zu definieren.
Dabei muss allerdings $\gamma_z$ ein Weg sein, der im Punkt $z$ endet,
und wir wissen noch nicht einmal, ob die Wahl des Weges eine Rolle
spielt.
Bevor wir also sicher sein k"onnen, dass eine Stammfunktion existiert,
m"ussen wir zeigen, dass das Wegintegral einer komplex differenzierbaren
Funktion zwischen zwei Punkten nicht von der Wahl des Weges abh"angt,
der die beiden Punkte verbindet.
Dazu ist notwendig, geschlossene Wege genauer zu betrachten.

%
% Wegintegrale führen auf analytische Funktionen
%
\subsubsection{Wegintegrale f"uhren auf analytische Funktionen}
\begin{figure}
\centering
\includegraphics{chapters/images/komplex-4.pdf}
\caption{Pfad und Konvergenzradius f"ur den Nachweis, dass Wegintegrale
auf analytische Funktionen f"uhren (Satz~\ref{komplex:integralanalytisch})
\label{komplex:integralanalytischpfad}}
\end{figure}
Mit Wegintegralen kann man aus stetigen Funktionen neue Funktionen
konstruieren.
Die folgende Konstruktion liefert "uberraschenderweise immer
analytische Funktionen.
\begin{satz}
\label{komplex:integralanalytisch}
Sei $\gamma\colon [a,b]\to\mathbb C$ ein Weg in $\mathbb C$, der nicht
durch den Nullpunkt verl"auft, und $g$ eine stetige Funktion
auf $\gamma([a,b])$ (Abbildung~\ref{komplex:integralanalytischpfad}).
Dann ist die Funktion
\[
f(z) = \frac1{2\pi i}\int_\gamma \frac{g(x)}{x-z}\,dx
\]
in einer Umgebung des Nullpunktes analytisch:
\[
f(z) = \sum_{k=0}^\infty c_k z^k,\qquad
\text{mit\quad}
c_k=\frac1{2\pi i}\int_\gamma \frac{g(x)}{x^{k+1}}\,dx.
\]
Der Konvergenzradius $\varrho$ dieser Reihe ist der minimale Abstand der
Kurve $\gamma$ vom Nullpunkt.
\end{satz}

\begin{proof}[Beweis]
Zun"achst schreiben wir
\begin{equation}
\frac{1}{x-z}
=
\frac1x\cdot \frac{1}{1-\displaystyle\frac{z}{x}}
=
\frac1x\cdot \sum_{k=0}^\infty \biggl(\frac{z}{x}\biggr)^k
=
\sum_{k=0}^\infty \frac{z^k}{x^{k+1}}.
\label{komplex:georeihe}
\end{equation}
Damit k"onnen wir jetzt die Funktion $f(z)$ berechnen:
\begin{align*}
f(z)
&=
\frac1{2\pi i} \int_{\gamma} \frac{g(x)}{x-z}\,dx
=
\frac1{2\pi i} \int_{\gamma} \sum_{k=0}^\infty \frac{z^k}{x^{k+1}}g(x)\,dx
=
\sum_{k=0}^\infty
\underbrace{\biggl(\frac1{2\pi i} \int_{\gamma} \frac{g(x)}{x^{k+1}}\,dx\biggr)}_{\textstyle =c_k}
z^k
=
\sum_{k=0}^\infty c_kz^k.
\end{align*}
Wir m"ussen uns noch die Konvergenz dieser Reihen "uberlegen.
Wenn $z<\varrho$ ist, dann ist 
\[
\biggl|\frac{z}{x}\biggr| 
=
\frac{|z|}{|x|}
<1,
\]
so dass die geometrische Reihe \eqref{komplex:georeihe} konvergent ist,
daraus lesen wir ab, dass der Konvergenzradius mindestens $\varrho$
ist.
Gr"osser kann er allerdings auch nicht sein, da f"ur $|z|\ge \varrho$
das Integral nicht mehr definiert sein muss.
Nimmt man n"amlich einen Punkt von $g([a,b])$ f"ur $z$ wird der Integrand
unendlich gross.
\end{proof}

Der Satz~\ref{komplex:integralanalytisch} ist nur f"ur Potenzreihen
im Punkt $0$ formuliert, was im Wesentlichen durch die
Umformung~\eqref{komplex:georeihe} bedingt war.
Man kann dies aber auch als Potenzreihe
\[
\frac1{x-z}
=
\frac1{x-z_0-(z-z_0)}
=
\frac1{x-z_0}\cdot\frac1{1-\displaystyle\frac{z-z_0}{x-z_0}}
=
\frac1{x-z_0}\sum_{k=0}^\infty\biggl(\frac{z-z_0}{x-z_0}\biggr)^k
=
\sum_{k=0}^\infty\frac1{(x-z_0)^{k+1}}(z-z_0)^k
\]
im Punkt $z_0$ ausdr"ucken.
Man bekommt dann die Potenzreihe
\[
f(z) = \sum_{k=1}^\infty c_k(z-z_0)^k,\qquad
\text{mit}\quad
c_k=\frac1{2\pi i}\oint_\gamma\frac{g(x)}{(x-z_0)^{k+1}}\,dx
\]
f"ur das Wegintegral.

\subsubsection{Laurent-Reihen}
\label{sssec:LaurentReihen}
\begin{figure}
\centering
\includegraphics{chapters/images/komplex-3.pdf}
\caption{Pfad zur Herleitung der Laurent-Reihe einer Funktion $f(z)$
mit einer Singularit"at $z_0$.
\label{komplex:laurentpfad}}
\end{figure}%
\index{Laurent-Reihe}%
In Satz~\ref{komplex:integralanalytisch} konnten wir eine Potenzreihe f"ur
solche $z$ konstruieren, deren Betrag kleiner ist als der kleinste Abstand
der Kurve $\gamma$ vom Ursprung.
Dies war notwendig, weil in~\eqref{komplex:georeihe} die geometrische Reihe
nur konvergiert, wenn der Quotient $<1$ ist.
Wenn die Funktion $f(z)$ jedoch eine Singularit"at im Punkt $z_0$ hat, dann
kann es nicht m"oglich sein, die Funktion mit einer Potenzreihe zu
beschreiben.

Wir verwenden daher den speziellen Pfad in Abbildung~\ref{komplex:laurentpfad}.
Er f"uhrt in einem grossen Kreis $\gamma_1$ um den Punkt $z_0$ herum,
dann folgt ein zur $x$-Achse paralleler Abschnitt, der bis zum kleinen
Kreis $\gamma_2$ f"uhrt.
Nach Durchlaufen des kleinen Kreises $\gamma_2$ im Uhrzeigersinn folgt wieder
ein zur $x$-Achse paralleles St"uck zur"uck zum grossen Kreis.
Da die geraden St"ucke zweimal in entgegegengesetzer Richtung durchlaufen
werden, heben sie sich weg.
Ein Wegintegral entlang $\gamma$ zerf"allt daher in eine Differenz
\[
\oint_\gamma\dots\,dz
=
\oint_{\gamma_1}\dots\,dz
-
\oint_{\gamma_2}\dots\,dz
\]
von Wegintegralen entlang $\gamma_1$ und $\gamma_2$.

Der "aussere Pfad $\gamma_1$ gibt wie in Satz~\ref{komplex:integralanalytisch}
Anlass zu einer Potenzreihe in $(z-z_0)$.
Der innere Pfad $\gamma_2$ kann aber nicht so behandelt werden, da $z$ immer
weiter von $z_0$ entfernt als die Punkte auf $\gamma_2$.
Allerdings ist $|x/z| < 1$ f"ur Punkte auf $\gamma_2$, wir m"ussen daher
die geometrische Reihe auf $x/z$ anwenden:
\begin{align*}
\frac{1}{x-z}
&=
\frac{1}{x-z_0-(z-z_0)}
=
\frac{1}{z-z_0}
\cdot
\frac{1}{\displaystyle\frac{x-z_0}{z-z_0}-1}
=
-\sum_{k=0}^\infty \frac{(x-z_0)^k}{(z-z_0)^{k+1}}.
\end{align*}
Das Integral entlang der Kurve $\gamma_2$ kann also als Reihe in $1/(z-z_0)$
entwickelt werden:
\begin{align*}
f_2(z)
&=
\frac{1}{2\pi i}\int_{\gamma_2} \frac{g(x)}{x-z}\,dx
=
\frac{1}{2\pi i}\int_{\gamma_2}\sum_{k=0}^\infty
\frac{(x-z_0)^k}{(z-z_0)^{k+1}}\,dx
\\
&=
\sum_{k=0}^\infty
\biggl(
\underbrace{\frac1{2\pi i}\int_{\gamma_2} (x-z_0)^kg(x)\,dx
}_{\textstyle =d_{k+1}}
\biggr)
\frac1{(z-z_0)^{k+1}}
=\sum_{k=1}^\infty \frac{d_k}{(z-z_0)^k}.
\end{align*}
Zusammen mit der vom Integral entlang $\gamma_1$ herr"uhrenden Reihe finden
wir den Satz
\begin{satz}
\label{komplex:laurentreihe}
Ist $g(z)$ eine entlang der Kurve $\gamma$ wie in
Abbildung~\ref{komplex:laurentpfad} definierte stetige Funktion, dann gilt
\[
f(z)=\frac1{2\pi i}\oint_{\gamma} \frac{f(x)}{x-z}\,dx
=
\sum_{k=0}^{\infty} c_k(z-z_0)^k-\sum_{k=1}^\infty \frac{d_k}{(z-z_0)^k},
\]
wobei die Koeffizienten $c_k$ und $d_k$ gegeben sind durch
\[
\begin{aligned}
c_k&=\frac1{2\pi i}\oint_{\gamma_1} \frac{g(x)}{x-z_0}\,dx
&&
\text{und}
&
d_k&=\frac1{2\pi i}\oint_{\gamma_2} g(x)x^{k-1}\,dx.
\end{aligned}
\]
\end{satz}

\begin{definition}
Eine Reihe der Form
\[
\sum_{k=-\infty}^\infty a_k(z-z_0)^k
\]
heisst {\em Laurent-Reihe }
im Punkt $z_0$.
\end{definition}


%
% Geschlossene Wege
%
\subsubsection{Geschlossene Wege}
\begin{definition}
Ein Weg $\gamma\colon[a,b]\to\mathbb C$ heisst {\em geschlossen}, wenn
$\gamma(a)=\gamma(b)$.
\index{geschlossener Weg}
Das Integral entlang eines geschlossenen Weges h"angt nicht von der
Parametrisierung ab und wird zur Verdeutlichung mit
\[
\int_{\gamma}f(z)\,dz
=
\oint_{\gamma}f(z)\,dz
\]
bezeichnet.
\end{definition}

\begin{beispiel}
Wir berechnen das Integral von $f(z)=z^n$ entlang des Einheitskreises,
den wir mit $\gamma(t)=e^{it},t\in[0,2\pi]$ parametrisieren.
Die Definition liefert:
\begin{align*}
\oint_{\gamma}f(z)\,dz
&=
\int_0^{2\pi}e^{int}ie^{it}\,dt
=
i\int_0^{2\pi}e^{i(n+1)t}\,dt
\end{align*}
F"ur $n=-1$ ist dies das Integral einer konstanten Funktion, also
\[
\oint_{\gamma}\frac1z\,dz=2\pi i.
\]
F"ur $n\ne -1$ kann man eine Stammfunktion von $e^{i(n+1)t}$
verwenden:
\[
\oint_{\gamma}f(z)\,dz
=
i\left[\frac1{i(n+1)}e^{i(n+1)t}\right]_0^{2\pi}
=0,
\]
weil $e^{i(n+1)t}$ periodisch ist mit Periode $2\pi$.
\end{beispiel}
Das Beispiel zeigt, dass ein Wegintegral der Potenzfunktionen,
aller Polynome und schliesslich aller konvergenten Potenzreihen
"uber einen geschlossenen Weg verschwinden.
Es zeigt aber auch, dass das Wegintegral "uber einen geschlossenen
Weg nicht zu verschwinden braucht, wie das Beispiel $f(z)=1/z$ 
zeigt.
Letztere Funktion unterscheidet sich von den Potenzfunktionen allerdings
dadurch, dass sie im Nullpunkt nicht definiert ist.

\begin{satz}
Sei $f(z)$ eine in einem zusammenh"angenden Gebiet $\Omega\subset\mathbb C$
definierte komplexe Funktion, f"ur die das Wegintegral "uber jeden
geschlossenen Weg verschwindet.
Dann hat $f(z)$ eine komplexe Stammfunktion $F(z)$.
\end{satz}

\begin{proof}[Beweis]
Wir w"ahlen einen beliebigen Punkt $z_0\in\Omega$ definieren die
komplexe Stammfunktion mit Hilfe des Wegintegrals
\[
F(z)=\int_{\gamma_z} f(\zeta)\,d\zeta,
\]
wobei $\gamma_z$ ein beliebiger Weg ist, der $z_0$ mit $z$ verbindet.

Wir m"ussen uns davon "uberzeugen, dass die Wahl des Weges keinen Einfluss
auf $F(z)$ hat.
Dazu seien $\gamma_1$ und $\gamma_2$ zwei verschiedene Wege, die
$z_0$ mit $z$ verbinden.
Da die Parametrisierung der Wege keinen Einfluss auf das Wegintegral haben,
nehmen wir an, dass beide Wege auf dem Intervall $[0,1]$ definiert sind.

Jetzt konstruieren wir einen geschlossene Weg $\gamma$ durch die
Definition:
\[
\gamma\colon[0,2]\to\mathbb C:t\mapsto
\begin{cases}
\gamma_1(t)&\qquad 0\le t\le 1\\
\gamma_2(2-t)&\qquad 1\le t\le 2
\end{cases}
\]
Der Weg $\gamma$ besteht aus $\gamma_1$ und dem in umgekehrter Richtung
durchlaufenen Weg $\gamma_2$, denn an der Stelle $t=1$ passen die
beiden Teilwege nahtlos zusammen: $\gamma_1(1)=\gamma_2(1)=\gamma_2(2-1)$.
Wegen $\gamma(2)=\gamma_2(2-2)=\gamma_2(0)=\gamma_1(0)$ ist der
Weg geschlossen.
Nach Voraussetzung ist verschwindet das Wegintegral "uber $\gamma$.
Es folgt
\begin{align*}
0
&=
\int_{\gamma}f(z)\,dz
\\
&=
\int_0^1 f(\gamma_1(t))\gamma_1'(t)\,dt
+ \int_1^2f(\gamma_2(2-t))\frac{d}{dt}\gamma_2(2-t)\,dt
\\
&=
\int_0^1 f(\gamma_1(t))\gamma_1'(t)\,dt
- \int_1^2f(\gamma_2(2-t))\gamma_2'(2-t)\,dt
\\
&=
\int_0^1 f(\gamma_1(t))\gamma_1'(t)\,dt
- \int_0^1f(\gamma_2(s))\gamma_2'(s)\,ds
\\
&=
\int_{\gamma_1}f(z)\,dz - \int_{\gamma_2}f(z)\,dz
\\
\Rightarrow\qquad
\int_{\gamma_2}f(z)\,dz&=\int_{\gamma_1}f(z)\,dz.
\end{align*}
Da die Wahl des Weges keine Rolle spielt, ist $F(z)$ wohldefiniert.
\end{proof}

Die Bedingung des eben bewiesenen Satzes ist nicht wirklich n"utzlich,
sie ist kaum nachpr"ufbar.
Es braucht also zus"atzliche Anstrengungen um gen"ugend viele
Funktionen zu finden, welche die Eigenschaft haben, dass Wegintegrale
"uber geschlossene Wege verschwinden.
Wir zielen dabei auf den folgenden Satz hin:
\begin{satz}[Cauchy]
Ist $f(z)$ eine in einem Gebiet $\Omega\subset\mathbb C$ definierte
komplex differenzierbare Funktion, und ist $\gamma$ ein im Gebiet
$\Omega$ auf einen Punkt zusammenziehbarer geschlossener Weg, dann gilt
\[
\oint_{\gamma}f(z)\,dz=0.
\]
Ist insbesondere $\Omega$ {\em einfach zusammenh"angend}
\index{einfach zusammenhangend@einfach zusammenh\"angend}%
\index{zusammenziehbar}%
(d.~h.~jeder geschlossene Weg l"asst sich in einen Punkt zusammenziehen),
dann verschwindet das Wegintegral von $f(z)$ "uber jeden geschlossenen
Weg in $\Omega$.
\index{einfach zusammenhangend@einfach zusammenh\"angend}
\end{satz}

\begin{proof}[Beweis]
Wir verwenden f"ur den folgenden Beweis den Satz von Green "uber
\index{Green, Satz von}%
Wegintegrale in der Ebene.
Er besagt, dass f"ur einen geschlossenen Weg $\gamma$ der in der Ebene
das Gebiet $D$ berandet, und zwei Funktionen $L(x,y)$ und $M(x,y)$, gilt
\[
\oint_\gamma(L\,dx + M\,dy)
=
\int_D \biggl(\frac{\partial M}{\partial x}
-\frac{\partial L}{\partial y}\biggr)\,dx\,dy.
\]
Wir berechnen jetzt das Integral einer komplex differenzierbaren Funktion
$f(z)$
\begin{align*}
\oint_\gamma f(z)\,dz
&=
\int (u(x,y)+iv(x,y))(\dot x(t)+i\dot y(t))\,dt
\\
&=
\int u(x,y)\dot x(t) -v(x,y)\dot y(t)\,dt
+
i \int u(x,y)\dot y(t)+v(x,y)\dot x(t)\,dt
\\
&=\oint_\gamma(u\,dx - v\,dy) + i\oint_\gamma(v\,dx + u\,dy)
\\
&=
\int_D
\underbrace{-\frac{\partial v}{\partial x}}_{\displaystyle=\frac{\partial u}{\partial y}}
-\frac{\partial u}{\partial y}
\,dx\,dy
+i
\int_D
\underbrace{\frac{\partial u}{\partial x}}_{\displaystyle=\frac{\partial v}{\partial y}}
-\frac{\partial v}{\partial y}\,dx\,dy
=0.
\end{align*}
Dabei haben wir auf der dritten Zeile den Satz von Green angewendet,
und auf der letzten Zeile die Cauchy-Riemann-Differentialgleichungen.
\end{proof}

\subsection{Beispiel: Airy-Differentialgleichung}
\label{komplex:airydgl}
\index{Airy-Differentialgleichung}%
\index{Differentialgleichung!Airy-}
\begin{figure}
\centering
\includegraphics{chapters/images/airy-1.pdf}
\caption{Kurven zur Berechnung der Airy-Funktionen
\label{komplex:airy}}
\end{figure}
Als Beispiel soll die Airy-Differentialgleichung
\[
y''(x)-xy(x)=0
\]
gel"ost werden.
Wir nehmen an, dass die Differentialgleichung mit der Laplace-Transformation
\index{Laplace-Transformation}%
gel"ost werden kann.
Allerdings verwenden wir nicht die Integration entlang der positiven
reellen Achse, wie das bei der traditionellen Laplace-Transformation
"ublich ist, sondern Integration eintlang eines beliebigen Weges in
der komplexen Ebene.
Wir nehmen also an, dass sich $y(x)$ schreiben l"asst als
\[
y(x)=\int_\gamma e^{xz}v(z)\,dz
\]
f"ur eine komplex differenzierbare Funktion $v(z)$ und eine Kurve
$\gamma\colon\mathbb R\to \mathbb C$.
Setzen wir diesen Ansatz in die Airy-Differentialgleichung ein,
erhalten wir
\begin{align*}
\int_\gamma z^2e^{xz}v(z)\,dz-\int_\gamma xe^{xz}v(z)\,dz&=0
\end{align*}
Das zweite Integral kann mit partieller Integration umgeformt werden,
man erh"alt
\begin{align*}
\int_\gamma xe^{xz}v(z)\,dz
&=
\int_{-\infty}^{\infty} xe^{x\gamma(t)}v(\gamma(t))\dot\gamma(t)\,dt
=
\left[e^{xz}v(z)\right]_{\gamma(-\infty)}^{\gamma(\infty)}
-\int_\gamma e^{xz}v'(z)\,dz
\end{align*}
Dieser Ausdruck ist nur dann sinnvoll, wenn der Weg $\gamma$ so gew"ahlt
wird, dass an den Endpunkte des Weges der Integrand beliebig klein wird,
dies wird sp"ater die Wahl des Weges einschr"anken.
Andererseits erhalten wir die Differentialgleichung erster Ordnung
\begin{align*}
\int_\gamma e^{xz}(z^2v(z)+v'(z))\,dz=0
\qquad
\Rightarrow
\qquad
v'(z)+z^2v(z)=0
\end{align*}
f"ur $v(z)$, die mit Separation gel"ost werden kann:
\[
\int \frac{dv}{v}=-\int z^2\,dz
\qquad\Rightarrow\qquad
v(z)=-\frac13z^3.
\]
Setzen wir dies in den Ansatz f"ur $y(x)$ ein, finden wir
\[
y(x)=\int_\gamma e^{xz-\frac13z^3}\,dz.
\]
Jetzt muss die Kurve $\gamma$ so gew"ahlt werden, dass das Integral
wohldefiniert ist. 
Dazu muss der Realteil des Exponenten f"ur jedes beliebige reelle $x$ 
gegen $-\infty$ gehen, wenn $z$ gegen unendlich geht.
In Abbildung~\ref{komplex:airy} sind die Gebiete grau eingezeichnet,
in denen $-\frac13z^3$ negativen Realteil hat.
Zul"assige Wege m"ussen daher ``Enden'' in den grauen Gebieten haben,
d.~h.~$\gamma(t)$ muss in diesen Gebieten gegen Unendlich gehen, wenn
$t\to\pm\infty$.
Damit sind im Wesentlichen die drei Kurven $\gamma_0$, $\gamma_+$ und
$\gamma_-$ aus Abbildung~\ref{komplex:airy} w"ahlbar.

F"ur $x\to\infty$ w"achst $e^{xz}$ im positiven grauen Teilgebiet
"uber alle Grenzen, die Kurven $\gamma_+$ und $\gamma_-$ ergeben daher
eine unbeschr"ankte L"osung.
Nur die Kurve $\gamma_0$ kann ein beschr"ankte L"osung der Airy-Gleichung
geben.
Man nennt
\[
\operatorname{Ai}(x)
=
\frac{1}{2\pi i}\int_{\gamma_0} e^{xz-\frac13z^3}\,dz
\]
die Airy-Funktion.
\index{Airy-Funktion}%
\index{Funktion!Airy-}%
\index{Ai}%
Auf die Wahl der Kurve $\gamma_0$ kommt es nicht an, solange sie in den
beiden grauen Gebieten links der imagin"aren Achse endet.
Deformiert man die Kurve $\gamma_0$ in die imagin"are Achse, erh"alt
man daher die folgende Integraldarstellung der Airy-Funktion:
\begin{align*}
\operatorname{Ai}(x)
&=
\frac1{2\pi i}\int_{-\infty}^{\infty} e^{xit-\frac13(it)^3}i\,dt
\\
&=\frac{1}{2\pi i}\int_{-\infty}^{\infty} ie^{i(xt+\frac13t^3)}\,dt
\\
&=
\frac{1}{2\pi i}\int_{-\infty}^{\infty}
i\cos\biggl(xt+\frac13t^3\biggr)-\sin\biggl(xt+\frac13t^3\biggr)\,dt
\\
&=\frac{1}{\pi}\int_0^{\infty}\cos\biggl(xt+\frac13t^3\biggr)\,dt.
\end{align*}
Dabei haben wir im letzten Schritt verwendet, dass das Integral der
ungeraden Funktion $\sin(xt+\frac13t^3)$ "uber ein symmetrisches
Interval verschwindet.

Weitere Information "uber die Airy-Funktionen sind in \cite{skript:airy}
zusammengefasst.

\subsection{Die Cauchy-Integralformel}
\index{Cauchy-Integralformel}%
Sei jetzt $f(z)$ eine komplex differenzierbare Funktion.
Dann ist auch die Funktion
\[
g(z)=\frac{f(z)}{z-a}
\]
komplex differenzierbar f"ur $z\ne a$.
Insbesondere ist der Wert des Wegintegrals von $g(z)$ entlang
eines geschlossenen Pfades um den Punkt $a$ unabh"angig von der Wahl
des Weges.
Zum Beispiel k"onnten wir das Wegintegral mit Hilfe eines kleinen Kreises
um $a$ mit Radius $r$ mit der Parametrisierung
\[
t\mapsto \gamma(t)=a+re^{it},\quad t\in[0,2\pi]
\]
berechnen.
Die Rechnung ergibt
\begin{align*}
\oint_\gamma \frac{f(z)}{z-a}\,dz
&=
\int_0^{2\pi} \frac{f(a+re^{it})}{re^{it}}ire^{it}\,dt
=
i\int_0^{2\pi} f(a+re^{it})\,dt
\end{align*}
Da $f(z)$ komplex differenzierbar ist, k"onnen wir $f(z)$ approximieren
durch $f(z)=f(a)+f'(a)(z-a)+o(z-a)$, also
\begin{align*}
\oint_{\gamma} \frac{f(z)}{z-a}\,dz
&=
i\int_0^{2\pi}f(a) + f'(a)re^{it}+o(r)\,dt
\\
&=
f(a)i\int_0^{2\pi}\,dt
+ irf'(a)\int_0^{2\pi} e^{it}\,dt + i\int_0^{2\pi}o(r)\,dt
\\
&=
2\pi i f(a) + irf'(a)\underbrace{\left[\frac1{i}e^{it}\right]_0^{2\pi}}_{=0}+o(r)
\\
&=2\pi i f(a)+o(r).
\end{align*}
Da das Wegintegral einer komplex differenzierbaren Funktion aber unabh"angig
vom Weg und damit vom Radius $r$ sein muss, folgt
\[
\oint_\gamma \frac{f(z)}{z-a}\,dz=2\pi i f(a).
\]
Wir haben damit den folgenden Satz bewiesen:

\begin{satz}[Cauchy]
Ist $\gamma$ ein geschlossener Weg in der komplexen Ebene, die ein
Gebiet umrandet, in dem die komplexe Funktion $f(z)$ komplex
differenzierbar ist, dann gilt
\[
f(a)=\frac{1}{2\pi i}\oint_{\gamma}\frac{f(z)}{z-a}\,dz.
\]
Insbesondere sind die Werte einer komplex differenzierbaren Funktion 
im Inneren eines Gebietes durch die Werte auf dem Rand bereits vollst"andig
bestimmt.
\end{satz}

\subsubsection{Ableitungen und Cauchy-Formel}
Sei $f(z)$ eine komplex differenzierbare Funktion, als Definitionsgebiet
nehmen wir der Einfachheit halber einen Kreis vom Radius $r$ um den Nullpunkt,
sein Rand ist die Kurve $\gamma$.
Durch Ableiten der Cachyschen Integralformel finden wir
\begin{align*}
f(z)
&=
\frac1{2\pi i}\oint_{\gamma}\frac{f(\zeta)}{\zeta-z}\,d\zeta
\\
f'(z)
&=
\frac1{2\pi i}\oint_{\gamma}\frac{f(\zeta)}{(\zeta-z)^2}\,d\zeta
\\
f'' (z)
&=
\frac1{2\pi i}\oint_{\gamma}2\frac{f(\zeta)}{(\zeta-z)^3}\,d\zeta
\\
f'''(z)
&=
\frac1{2\pi i}\oint_{\gamma}2\cdot 3\frac{f(\zeta)}{(\zeta-z)^4}\,d\zeta
\\
&\vdots
\\
f^{(k)}(z)
&=
\frac{k!}{2\pi i}\oint_{\gamma}\frac{f(\zeta)}{(\zeta-z)^{k+1}}\,d\zeta
\end{align*}
Es folgt

\begin{satz}
Eine komplex differenzierbare Funktion ist beliebig oft differenzierbar.
\end{satz}

\subsubsection{Komplex differenzierbare Funktionen sind analytisch}
Wir haben fr"uher gesehen, dass Wegintegrale auf analytische Funktionen
f"uhren.
Andererseits zeigt das Cauchy-Integral, dass komplex differenzierbare
Funktionen durch genau die Integrale bestimmt sind, die in den
Reihenentwicklungen in Satz~\ref{komplex:integralanalytisch} auftraten.
Diese Resultate k"onnen wir im folgenden Satz zusammenfassen.

\begin{satz}
Eine komplex differenzierbare Funktion $f(z)$, die in einer Kreisscheibe
vom Radius $r$ um den Punkt $z_0$ definiert ist, ist analytisch.
Ihre Potenzreihenentwicklung
\[
f(z)=\sum_{k=0}^na_k(z-z_0)^k
\]
hat die Koeffizienten
\[
a_k=\frac1{2\pi i}\int_{\gamma}\frac{f(z)}{(z-z_0)^{k+1}}\,dz,\quad
k\ge 0
\]
\end{satz}

\begin{proof}[Beweis]
Da $f$ komplex differenzierbar ist, gilt
\[
f(z)=\frac1{2\pi i}\oint_\gamma \frac{f(\zeta)}{\zeta-z}\,d\zeta.
\]
In Satz~\ref{komplex:integralanalytisch} wurde gezeigt, dass $f(z)$
analytisch ist, und dass die Koeffizienten der Potenzreihe von
der verlangten Form sind.
\end{proof}

F"ur eine komplexe Funktion, die im Punkt $z_0$ eine Singularit"at hat,
also in einer Umgebung von $z_0$ ohne den Punkt $z_0$ definiert ist,
k"onnen wir das Resultat aus Satz~\ref{komplex:laurentreihe} verwenden,
und zum folgenden analogen Resultat gelangen:

\begin{satz}
Eine komplex differenzierbare Funktion $f(z)$, die in einer Kreisscheibe
vom Radius $r$ um den Punkt $z_0$ mit Ausnahme des Punktes $z_0$
definiert ist, kann in eine konvergente Laurent-Reihe
\[
f(z)=\sum_{k=-\infty}^{\infty} c_k(z-z_0)^k
\]
entwickelt werden, deren Koeffizienten durch
\[
c_k = \frac1{2\pi i}\oint_\gamma \frac{f(\zeta)}{(z-z_0)^{k+1}}\,d\zeta,\qquad k\in\mathbb Z
\]
gegeben sind.
\end{satz}

%
% Analytische Fortsetzung
%
\section{Analytische Fortsetzung}
\rhead{Analytische Fortsetzung}
\label{sec:fortsetzung}
Wir haben schon gesehen, dass eine reelle Funktion, die in einem
Punkte eine konvergente
Potenzreihe besitzt, auf nat"urliche Weise auch als komplexe Funktion
betrachtet werden kann, indem man komplexe Argumente in der Potenzreihe
zul"asst.
Die neue komplexe Funktion ist ein einem Kreis um den Punkt
konvergent.
Mit Hilfe der Potenzreihe kann man also immer eine Funktion auf ein
Kreisgebiet ausdehen.
Dieser Abschnitt untersucht die Frage, ob man diese Idee auch auf 
noch gr"ossere Gebiete ausdehnen kann.
\subsection{Analytische Fortsetzung mit Potenzreihen}
\begin{figure}
\centering
\includegraphics{chapters/images/komplex-1.pdf}
\caption{Analytische Fortsetzung einer komplexen Funktion entlang einer
Kurve $\gamma$.
\label{komplex:fortsetzung}}
\end{figure}
Eine komplex differenzierbare Funktion $f(z)$ ist immer darstellbar als
Potenzreihe, und ist daher analytisch.
So kann zum Beispiel die Funktion $1/z$ als Potenzreihe um jeden 
beliebigen Punkt $z_0$ entwickelt werden:
\begin{align}
f(z)
&=
\frac1z
=
\frac1{z_0-(z_0-z)}
=
\frac1{z_0}\cdot
\frac1{1-\displaystyle\frac{z_0-z}{z_0}}
=
\frac1{z_0}\sum_{k=0}^{\infty} \biggl(\frac{z_0-z}{z_0}\biggr)^k
=
\sum_{k=0}^{\infty} \frac{(-1)^k}{z_0^{k+1}} (z-z_0)^k,
\label{komplex:1durchreihe}
\end{align}
Die Koeffizienten dieser Potenzreihe sind
\[
a_k=\frac{(-1)^k}{z_0^{k+1}},
\]
und man kann den Konvergenzradius ausrechnen:
\[
\frac1{\varrho}
=
\limsup_{k\to\infty} \root{k}\of{|a_k|} = \lim_{k\to\infty}\frac1{|z_0|^{\frac{k+1}{k}}}
=
\frac1{|z_0|}.
\]
Der Konvergenzradius ist limitiert durch die Singularit"at bei an der Stelle
$z=0$.

Es gibt also keine einzelne Potenzreihe, die die Funktion $f(z)=\frac1z$ in der
ganzen komplexen Ebene darstellen kann.
W"ahlt man aber einzelne Punkte $z_0$ und $z_1$ derart, dass der Kreis
um $z_0$ mit Radius $|z_0|$ und der Kreis um $z_1$ mit Radius $|z_1|$
"uberlappen, dann werden die beiden Potenzreihen im "Uberlappungsgebiet
die gleichen Werte annehmen.

Man k"onnte allso eine Kurve $\gamma$ in der komplexen Ebene w"ahlen,
entlang der man in jedem Punkt die Funktion $f(z)$ in eine Potenzreihe 
entwickelt.
Liegen zwei Punkte nahe genug auf der Kurve $\gamma$, werden die
Konvergenzkreise der Potenzreihen "uberlappen, und die Potenzreihen
werden im "Uberlappungsgebiet die gleichen Werte liefern.

Selbst wenn man eine Funktion $f(z)$ nur in einem Kreis um den Punkt $z_0$
kennt, zum Beispiel durch eine Potenzreihe im Punkt $z_0$, kann man entlang
einer Kurve, die $z_0$ mit $z_1$ verbindet, in jedem Punkt eine Potenzreihe
finden, die mit der Potenzreihe in den Nachbarpunkten "ubereinstimmt, und
so die Definition der Funktion entlang dieser Kurve auf ein gr"osseres
Gebiet ausweiten, wie in Abbildung~\ref{komplex:fortsetzung} dargestellt.
Man nennt dies die {\em analytische Fortsetzung} der Funktion $f(z)$ 
entlange der Kurve $\gamma$.
\index{analytische Fortsetzung}
\index{Fortsetzung, analytische}

\begin{beispiel}
Wir haben bereits gesehen, dass sich die Funktion $f(z)=1/z$ in jedem
Punkt $z_0$ der komplexen Ebene in die Potenzreihe~\eqref{komplex:1durchreihe}
entwickeln l"asst.
Diese Reihe l"asst sich integrieren
\[
F(z,z_0)
=
\sum_{k=0}^\infty\frac{(-1)^k}{(k+1)z_0^{k+1}}z^{k+1},
\]
diese Reihe ist ebenfalls auf einem Kreis vom Radius $|z_0|$ um den
Punkt $z_0$ konvergent.
Wir vermuten nat"urlich, dass dies eine Darstellung des nat"urlichen
Logarithmus einer komplexen Zahl ist.
Nat"urlich ist das immer nur auf einem Kreisgebiet m"oglich, die Reihe
f"ur $z=1$ ist zum Beispiel im Punkt $z=-1$ nicht konvergent.

Um eine in der ganzen komplexen Ebene definierte Funktion $\log(z)$ zu
konstruieren, m"ussen wir also eine analytische Fortsetzung aufbauen.
Bei der Integration haben wir eine frei w"ahlbare Integrationskonstante
$C(z_0)$, die wir so w"ahlen m"ussen, dass die Reihen im "Uberlappungsgebiet
"ubereinstimmen:
\[
F(z,z_0) + C(z_0) = F(z,z_1)  + C(z_1)
\]
f"ur jedes $z$ im "Uberlappungsgebiet.
Dadurch wird aber nur die Differenz $C(z_1)-C(z_0)$ der Werte festgelegt.
Da wir "Ubereinstimmung mit der "ublichen Definition des Logarithmus
erreichen m"ochten, k"onnen wir $C(1)=0$ festlegen.

\begin{figure}
\centering
\includegraphics{chapters/images/komplex-2.pdf}
\caption{Analytische Fortsetzung f"ur die Funktion $\frac1z$ 
entlang der Pfade $\gamma_+$ und $\gamma_-$
\label{komplex:logfortsetzung}}
\end{figure}
Wir konstruieren jetzt die analytische Forstsetzung entlang der Kurven
$\gamma_+$ und $\gamma_-$ wie in Abbildung~\ref{komplex:logfortsetzung}
dargestellt.
Um die Differenz $C(z_1)-C(z_0)$ zu bestimmen, Werten wir die Funktionen
$F(z,z_0)$ und $F(z,z_1)$ jeweils im rot eingezeichneten Punkt aus.
Die exakte Berechnung ist etwas m"uhsam, da es sich ja nur um ein Beispiel
handelt, k"onnen wir die Reihen auch numerisch ausrechnen, und so die
Differenzen bestimmen:
\begin{align*}
&\text{Startpunkt $z_0=1$:}& C(1)&=0             &       &       \\
&\text{entlang $\gamma_+$:}& C(i)&= i\frac{\pi}2 & C(-1) &=  i\pi\\
&\text{entlang $\gamma_-$:}&C(-i)&=-i\frac{\pi}2 & C(-1) &= -i\pi
\end{align*}
Wir stellen fest, dass die analytische Fortsetzung der Logarthmusfunktion
entlang der Kurve $\gamma_+$ die Potenzreihe
\[
\log_+(z)
=
i\pi +\sum_{k=1}^\infty \frac{(-1)^{k+1}}{k(-1)^k}(z+1)^k
=
i\pi
-
\sum_{k=1}^\infty \frac{(z+1)^k}{k}
\]
ergibt, w"ahrend man entlang der  Kurve $\gamma_-$
\[
\log_-(z)
=
-i\pi +\sum_{k=1}^\infty \frac{(-1)^{k+1}}{k(-1)^k}(z+1)^k
=
-i\pi
-
\sum_{k=1}^\infty \frac{(z+1)^k}{k}
\]
findet.
Die beiden analytischen Fortsetzungen entlang der Kurven $\gamma_+$ und
$\gamma_-$ stimmen auf der negativen reellen Achse nicht "uberein,
sie unterscheiden sich um $2\pi i$:
\[
\log_+(z)-\log_-(z)=2\pi i.
\]
\end{beispiel}

Das Beispiel zeigt, dass es im Allgmeinen eine auf der ganzen komplexen
Ebene definierte komplexe Entsprechung einer reellen Funktion nicht
zu geben braucht.
Dieses Ph"anomen tritt zum Beispiel auch bei der Wurzelfunktion $f(z)=\sqrt{z}$
auf.
Diese Funktion ist im Punkt $z=0$ nicht differenzierbar, man muss diesen
Punkt also aus dem Definitionsbereich ausschliessen.
F"uhrt man man analog zum Beispiel eine analytische Fortsetzung durch,
findet man, dass sich die Werte von $f(z)$ f"ur die beiden Wege $\gamma_+$
und $\gamma_-$ durch das Vorzeichen unterscheiden.

\subsection{Analytische Fortsetzung mit Differentialgleichungen
\label{komplex:analytische-fortsetzung-dgl}}
In Abschnitt~\ref{subsection:wegintegrale} wurde gezeigt, wie Wegintegrale
Stammfunktionen komplexer Funktionen liefern k"onnen.
Im vorangegangenen Abschnitt wurde untersucht, wie eine komplex differenzierbare
Funktion mit Hilfe von analytischer Fortsetzung entlang einer Kurve
ausgedehnt werden kann.

Sei $f(z)$ eine komplex differenzierbare Funktion.
In jedem beliebigen Punkt des Definitionsbereichs k"onnen wir $f(z)$
in eine Potenzreihe entwickeln, und nat"urlich auch termweise integrieren.
Es gibt also in jedem Punkt $z_0$ des Definitionsbereichs eine
Funktion $F_{z_0}(z)$, die $F'_{z_0}(z)=f(z)$ erf"ullt.
Durch analytische Fortsetzung entlang einer Kurve $\gamma$ k"onnen
wir eine komplex differenzierbare Funktion $f(z)$ finden, die in einer
Umgebung der Kurve $F'(z)=f(z)$ erf"ullt.

Sei andererseits $\gamma\colon[a,b]\to\mathbb C$ eine Kurve in $\mathbb C$.
Dann k"onnen wir die Werte der Stammfunktion im Punkt $\gamma(b)$ durch
\[
F(\gamma(b)) = F(\gamma(a))+\int_\gamma f(z)\,dz
\]
berechnen.

\begin{beispiel}
\begin{figure}
\centering
\includegraphics{chapters/images/komplex-5.pdf}
\caption{Analytische Fortsetzung des Logarithmus als L"osung der
Differentialgleichung $y'=\frac1z$.
Bei einem Umlauf um den Nullpunkt nimmt der Wert von $y(z)$ um
$2\pi i$ zu.
\label{komplex:analytische-fortsetzung-log}
}
\end{figure}
Wir bestimmen die Stammfunktion von $f(z)=1/z$.
Entlang der reellen Achse weiss man bereits, dass die Stammfunktion
der nat"urliche Logarithmus ist, also $F(x)=\log x$.
Um diese Stammfunktion auf $\mathbb C$ auszudehnen, verwenden wir einen
kreisf"ormigen Pfad von der reellen Achse bis zum Punkt $z$.
Liegt $z$ in der oberen Halbebene, w"ahlen wir einen Pfad in der
oberen Halbebene, und umgekehrt.
Wir k"onnen die Zahl $z$ in Polarkoordinaten darstellen als $z=re^{i\varphi}$.
Ein Pfad von der reellen Achse kann mit
\[
\gamma\colon [0,1]\to\mathbb C: t\mapsto re^{it\varphi}
\]
parametrisiert werden.
Der Zuwachs der Stammfunktion entlang dieses Pfades ist
\[
F(z)-F(r)
=
\int_\gamma\frac1z\,dz
=
\int_0^1 \frac1{e^{it\varphi}}i\varphi e^{it\varphi}\,dt
=
i\varphi \int_0^1\,dt
=
i\varphi.
\]
Der Wert der Stammfunktion am Anfang der Kurve ist $\log r$, somit
folgt, dass
\[
\log z = \log r + i\varphi
\]
(Abbildung~\ref{komplex:analytische-fortsetzung-log}).
\end{beispiel}

\section{"Ubungsaufgaben}
\rhead{"Ubungsaufgaben}
\uebungsaufgabe{701}
\uebungsaufgabe{702}
\uebungsaufgabe{703}
\uebungsaufgabe{704}
\uebungsaufgabe{705}
\uebungsaufgabe{706}


%
% stabilitaet.tex -- Stabilität der Lösungen von Differentialgleichungen
%
% (c) 2015 Prof Dr Andreas Mueller, Hochschule Rapperswil
%
\chapter{Stabilit"at\label{chapter:stabilitaet}}
\lhead{}
\rhead{Stabilit"at}


%
% chaos.tex -- Grundlagen des "Ubergangs zum Chaos
%
% (c) 2015 Prof Dr Andreas Mueller, Hochschule Rapperswil
%
\chapter{Chaos\label{chapter:chaos}}
\lhead{}
\rhead{Chaos}


%
% stochastisch.tex -- Kapitel ueber stochastische Differentialgleichungen
%
\chapter{Stochastische Differentialgleichungen\label{chapter:stochastisch}}
\lhead{Stochastische Differentialgleichungen}
\rhead{}
In vielen Anwendungen wird die Bewegung eines Systems auch von
zuf"alligen Einfl"ussen bestimmt, die man oft auch Rauschen nennt.
Die Natur des Rauschen bedeutet, dass aufeinanderfolgende inkremente
v"ollig unkorreliert sind, w"ahrend Inkremente einer differenzierbaren
Funktion voneinander abh"angig sind.
Die L"osung einer Differentialgleichung unter Einfluss von Rauschen 
kann daher niemals eine differenzierbare Funktion sein, und sie kann
niemals eine L"osung der Differentialgleichung im bisher verwendeten
Sinn sein.
Um der Idee einen mathematischen Sinn zu geben, der auch erlaubt,
solche Differentialgleichungen zu l"osen und in Anwendungen
einzusetzen, muss daher zuerst gekl"art werden, was Rauschen genau ist.
Anschliessend muss das Konzept einer Differentialgleichung so formuliert
werden, dass es auch f"ur nicht differenzierbare Funktionen und Rauschen
anwendbar ist.

Die Darstellung in diesem Kapitel orientiert sich in vielen Punkten
an dem hervorragenden und leicht lesbaren Buch \cite{skript:evans}.
Eine mathematisch vertieftere Entwicklung ist in \cite{skript:oksendal}
zu finden.

%
% Ein Modell f"ur Rauschen
%
\section{Modell f"ur Rauschen: der Wiener-Prozess\label{section:wiener}}
\rhead{Wiener-Prozess}
Rauschen ist ein Zufallsph"anomen, die Wiederholung eines Experimentes
wird im Allgemeinen einen anderen Verlauf ergeben.
Der Pfad eines Teilchens $W(t)$ in Abh"angikeit ist daher ein Zufallsresultat.
Wir brauchen daher einen Wahrscheinlichkeitsraum $\Omega$ und ein
Wahrscheinlichkeitsmass $P$, und die Wege $W$ sind abh"angig von 
der Durchf"uhrung $\omega\in\Omega$ des Experiments. 
Genau genommen m"ussen wir also sagen, dass f"ur jedes $\omega\in\Omega$
der Weg $W(\omega)$ eine Funktion
\[
W(\omega)\colon\mathbb R \to\mathbb R:t\mapsto W(\omega)(t)
\]
ist.
Wir nennen eine solche Funktion einen {\em stochastischen Prozess}.
\index{stochastischer Prozess}
Im Folgenden werden wir die etwas schwerf"allige Notation etwas
vereinfachen, und das $\omega$ weglassen.

Wir m"ochten die Position eines Teilchens berechnen, dessen Geschwindigkeit
ein solches ``Rauschen'' ist.
Diese Position $W(t)$ ist ein stochastischer Prozess im eben erkl"arten Sinne.
Die Brownsche Bewegung ist ein solcher Prozess, die Position $W(t)$ eines
Teilchens unter dem Einfluss der thermischen Bewegung der Teilchen
in einer Fl"ussigkeit als Funktion der Zeit wird eine nicht
differenzierbare Funktion sein.
Das beste, was wir erwarten k"onnen, ist dass die Positionsunterschiede
\[
W(t+\Delta t)-W(t),
\quad
W(t + 2\Delta t)-W(t+\Delta t),
\quad
W(t + 3\Delta t)-W(t+2\Delta t),\quad\dots
\]
voneinander unabh"angig sind, und nicht beliebig gross sind.
Wir erwarten, dass diese Differenzen, die sich aus vielen kleinen
St"ossen zusammensetzen, normalverteilt sind.
Wir definieren daher

\begin{definition}
Ein stochastischer Prozess $W(t)$ heisst {\em Brownsche Bewegung} oder
{\em Wiener Prozess}, wenn gilt
\begin{compactenum}
\item $W(0)=0$
\item $W(t)-W(s)$ ist normalverteilt mit Erwartungswert $0$ und
Varianz $t-s$, f"ur beliebige $t\ge s\ge 0$.
\item F"ur beliebige Werte $t_i$ mit $0<t_1<t_2<\dots<t_n$, dann sind
die Zufallsvariablen
$W(t_1), W(t_2)-W(t_1),\dots,W(t_n)-W(t_{n-1})$ unabh"angig.
\end{compactenum}
\index{Brownsche Bewegung}
\index{Wiener Prozess}
\end{definition}

A priori ist nicht klar, dass es so einen Prozess "uberhaupt gibt, wir
m"ussen daher zeigen, dass sich eine solche Funktion konstruieren l"asst.
Eine solche Funktion ist aber sicher nicht differenzierbar, denn
der Differenzenquotient "uber ein Interval der L"ange $2\Delta t$
\begin{align*}
\frac{W(t+2\Delta)-W(t)}{2\Delta t}
&=
\frac{W(t+2\Delta t)-W(t+\Delta t)}{2\Delta t}
+
\frac{W(t+\Delta t)-W(t)}{2\Delta t}
\\
&=
\frac12\biggl(
\frac{W(t+2\Delta t)-W(t+\Delta t)}{\Delta t}
+
\frac{W(t+\Delta t)-W(t)}{\Delta t}
\biggr)
\end{align*}
ist Mittelwert aus zwei Differenzenquotienten "uber k"urzere Intervalle,
aber diese beiden Differenzenquotienten sind voneinander unabh"angig.
Ein Grenzwert des Differenzenquotienten kann daher nicht existieren.

\subsection{Eigenschaften des Wiener-Prozesses}
Wir brauchen Rechenregeln, wie man mit Wiener-Prozessen Funktionen
rechnen kann.
Zum Beispiel ist $W(t)$ wegen Eigenschaft~2 normalverteilt mit Erwartungswert
$0$ und Varianz $t$, also gilt
\[
E(W(t))=0,\qquad E(W(t)^2)=t,\qquad \forall\;t>0.
\]
Etwas weniger offensichtlich ist
\begin{hilfssatz}
Wenn $W(t)$ eine Brownsche Bewegung ist, dann ist
\[
E(W(t)W(s)) = t\wedge s = \min\{s,t\}
\]
f"ur beliebige $t,s\ge 0$.
\end{hilfssatz}

\begin{proof}[Beweis]
Nehmen wir an, dass $t\ge s$, dass also $t\wedge s = s$.
Dann k"onnen wir $E(W(t)W(s))$ berechnen
\begin{align*}
E(W(t)W(s))
&=
E((W(t)-W(s)+W(s))W(s))
=
E((W(t)-W(s))W(s))+E(W(s)^2)
\end{align*}
Eigenschaften~1 und~2 zeigen, dass $E(W(s)^2)=s$ ist.
Eigenschaft~3 besagt, dass $W(t)-W(s)$ und $W(s)$ unabh"angig sind,
der Erwartungswert ihres Produktes ist daher das Produkt der Erwartungswerte:
\begin{align*}
E(W(t)W(s))
&=
E((W(t)-W(s))W(s))+E(W(s)^2)
=
\underbrace{E(W(t)-W(s))}_{\textstyle =0} \underbrace{E(W(s))}_{\textstyle =0} + s
\end{align*}
wobei wir erneut Eigenschaft~2 verwendet haben.
\end{proof}
Man k"onnte diese Eigenschaft umschreiben als die Beobachtung,
dass die weitere Entwicklung von $W(t)$ nach der Zeit $s$ bedeutungslos ist.

\begin{figure}
\centering
\includegraphics{chapters/images/stochastisch-1.pdf}
\caption{Wiener-Prozess $W(t)$ in Abh"angigkeit von der Zeit
\label{stochastisch:wiener}}
\end{figure}

\subsection{Konstruktion des Wiener-Prozesses}
Wir m"ussen eine Konstruktion angeben, mit der wir zu einem gegebenen
Interval $[0,T]$ eine Funktion $W(t)$ konstruieren k"onnen, die
die Eigenschaften des Wiener-Prozesses erf"ullt.

Zun"achst verlangen die Eigenschaften~1 und 2 des Wiener-Prozesses,
dass $X(T)$ eine normalverteilte Zufallsvariable ist mit Erwartungswert
$X(0)=0$ und Varianz $T$.
Da $X(t)$ ausserdem stetig sein soll, verwenden wir als erste Iteration
die lineare Funktion:
\[
W_1(t) = X(T)\frac{t}{T}
\]
Die Eigenschaft~3 verlangt, dass auch $X(T/2)$ normalverteilt ist mit
Erwartungswert $0$ und Varianz $T/2$.
Dies kann dadurch erreicht werden, dass wir $W_1$ durch einen 
Polygonzug $W_2$ ersetzen, der bei $t=T/2$ einen zus"atzlichen
Eckpunkt besitzt.
Wir bezeichnen den Unterschied zwischen $W_2$ und $W_1$ an der
Stelle $t=T/2$ mit
\[
Y=W_2(T/2)-W_1(T/2)=W_2(T/2) - W_1(T)/2.
\]
$Y$ muss so gew"ahlt werden, dass $W_2(T/2)$ eine normalverteilte
Zufallsvariable mit Erwartungswert $0$ und Varianz $T/2$ wird,
somit ist $Y$ auch normalverteilt.
Der Erwartungswert von $Y$ ist
\begin{align*}
E(Y)&=E(W_2(T/2) - W_1(T)/2)=E(W_2(T/2))-E(W_1(T))/2=0,
\end{align*}
es ist also nur noch die Varianz $\sigma^2_Y$ w"ahlbar.
Sie muss so gew"ahlt werden, dass $W_2(T/2)$ Varianz $T/2$
bekommt:
\begin{align*}
\operatorname{var}\biggl(W_2\biggl(\frac{T}2\biggr))\biggr))
&=
\operatorname{var}\biggl(W_1\biggl(\frac{T}2\biggr) + Y\biggr)
=
\frac{\operatorname{var}(W_1)}4 + \operatorname{var}Y
=
\frac{T}4 +\sigma_Y^2
\end{align*}
Es folgt $\sigma_Y^2=\frac{T}4$.

Wir m"ussen kontrollieren, ob die Inkremente unabh"angig sind.
Dazu berechnen wir der Erwartungswert des Produktes der
beiden Inkremente
\begin{align*}
W_2(T)-W_2\biggl(\frac{T}2\biggr)
&=
W_1(T)-\biggl(\frac{W_1(T)}2 + Y\biggr)
=
Y-\frac{W_1(T)}2
\\
W_2\biggl(\frac{T}2\biggr)-W_2(0)
&=
Y+
\frac{W_1(T)}2
\\
\biggl(W_2(T)-W_2\biggl(\frac{T}2\biggr)\biggr)
\biggl(W_2\biggl(\frac{T}2\biggr)-W_2(0)\biggr)
&=
\biggl(Y-\frac{W_1(T)}2\biggr)\biggl(Y+\frac{W_1(T)}2\biggr)
=
Y^2-\frac{W_1(T)^2}4
\\
E\biggl(
\biggl(W_2(T)-W_2\biggl(\frac{T}2\biggr)\biggr)
\biggl(W_2\biggl(\frac{T}2\biggr)-W_2(0)\biggr)
\biggr)
&=
E(Y^2)-E\biggl(\frac{W_1(T)^2}4\biggr)
=
\sigma_Y^2 - \frac14\operatorname{var}(W_1(T))
=
0
\end{align*}
Somit sind die Inkremente unabh"angig.

\begin{figure}
\centering
\includegraphics{chapters/images/stochastisch-3.pdf}\\
\includegraphics{chapters/images/stochastisch-4.pdf}\\
\includegraphics{chapters/images/stochastisch-5.pdf}\\
\includegraphics{chapters/images/stochastisch-6.pdf}\\
\includegraphics{chapters/images/stochastisch-7.pdf}
\caption{Schrittweise Konstruktion des Wiener-Prozesses als Grenzewert
der Folge $W_n(t)$, $n=1,\dots,5$
\label{stochastisch:folge1}}
\end{figure}
\begin{figure}
\centering
\includegraphics{chapters/images/stochastisch-8.pdf}\\
\includegraphics{chapters/images/stochastisch-9.pdf}\\
\includegraphics{chapters/images/stochastisch-10.pdf}\\
\includegraphics{chapters/images/stochastisch-11.pdf}\\
\includegraphics{chapters/images/stochastisch-12.pdf}
\caption{Schrittweise Konstruktion des Wiener-Prozesses als Grenzewert
der Folge $W_n(t)$, $n=6,\dots,10$
\label{stochastisch:folge2}}
\end{figure}

Diesen Prozess k"onnen wir fortsetzen: f"ur $W_3$ nehmen wir einen
Polygonzug mit zus"atzlichen Eckpunkten an den Stellen $T/4$ und $3T/4$.
Die Differenz zwischen $W_3$ und $W_2$ an diesen Stellen m"ussen
normalverteilt sein mit Erwartungswert $0$ und Varianz $T/8$.
Auf diese Weise k"onnen wir eine Folge $W_n$ von Prozessen konstruieren,
wie in den Abbildungen~\ref{stochastisch:folge1} und \ref{stochastisch:folge2}
dargestellt.

Dies reicht aber nicht.
F"ur eine vollst"andige Konstruktion muss man noch die folgenden zwei
Dinge zeigen.
\begin{compactenum}
\item
Der Grenzwert existiert tats"achlich.
Weil die einzelnen Zufallsvariablen $Y$ normalverteilt sind, k"onnen
sie beliebig grosse Werte annehmen.
Dies bedeutet auch, dass wir im Allgemeinen nicht davon ausgehen 
k"onnen, dass die Folge $W_n(t)$ konvergiert.
Zwar sind die Werte von $W_m$ an den Stellen $t_{k,n}=kT/2^{n-1}$ f"ur
$k=0,\dots,2^{n-1}$ fest, sobald $m\ge n$.
Zwischen diesen Werten k"onnen aber immer wieder grosse Werte auftreten,
so dass die Folge $W_n$ weder gleichm"assig noch punktweise konvergieren
kann.
Die Frage der Konvergenz muss daher als Konvergenz in einer Art Mittel 
angegangen werden.
\item
Die Forderung, dass die Inkremente $W(t)-W(s)$ und $W(s)-W(r)$ f"ur jedes 
Tripel $t \ge s\ge r$ unabh"angig sein m"ussen.
Die Konstruktion stellt nur sicher, dass dies gilt, wenn die Zeitpunkte
des Tripels Eckpunkte der Polygonkonstruktion sind.
\end{compactenum}

%
%
%
\section{Stochastische Differentialgleichungen\label{section:stochdgl}}
\rhead{Differentialgleichungen}
Wir m"ochten gerne eine Differentialgleichung f"ur den Zustand
$X(t)$ eines Systems l"osen, welches von Rauschen mit beeinfluss wird.
Eine solche Differentialgleichung k"onnten wir schreiben als
\[
\frac{dX(t)}{dt}
=
b(X(t)) + B(X(t))\frac{dW(t)}{dt}
\]
wobei $b$ die Rolle der Funktion $f$ aus
Abschnitt~\ref{section:anfangswertprobleme} spielt.
Die Ableitung von $W$ spielt die Rolle des Rauschens, wir wissen aber
bereits, dass $W$ nicht differenzierbar sein kann, die Gleichung
in dieser Form kann daher gar nicht sinnvoll sein.

Formal kann man die Gleichung mit $dt$ multiplizieren, so dass man
das formale Gleichungssystem
\begin{align*}
dX(t)
&=
b(X(t)) + B(X(t),t)\,dW(t)
\\
X(0)
&=
x_0
\end{align*}
erh"alt, aber auch in diesr Form sind die Ausdr"ucke $dX$ und $dW$ nicht
ohne zus"atzliche Definition sinnvoll.
Am ehesten hat man eine Chance, dieser Gleichung einen Sinn zu geben,
wenn man integriert:
\begin{align*}
X(t)=x_0+\int_0^t b(X(s))\,ds + \int_0^t B(X(s), s)\, dW,\qquad t>0.
\end{align*}
Das erste Integral ist ein gew"ohnliches Integral, denn wir
gehen davon aus, dass die Funktion $X(t)$ stetig ist.
Wenn $B=0$ ist, dann liefert diese Formel genau L"osungen der
urspr"unglichen Differentialgleichung.
Wir brauchen aber immer noch eine Interpretation des zweiten Integrals,
diese werden wir in Abschnitt~\ref{subsection:stochint} geben.

Nehmen wir f"ur den Moment an, dass die L"osung $X(t)$ gefunden werden
kann, und sei $u$ eine beliebig oft differenzierbare Funktion.
Wir erwarten, dass die Ableitung von $Y(t)=u(X(t))$ nach der
Kettenregel
\[
dY = u'\,dX = u'b\,dt + u'\,dW
\]
sein sollte.
Tats"achlich ist das Rauschen so stark, dass die Kr"ummung der Funktion
$u$ bereits eine Rolle spielt.
Bei einer gew"ohnlichen Differentialgleichung sind auf kleine Entfernungen
die zweiten Ableitungen von $u$ vernachl"assigbar.
Die schnellen, vom Rauschen verursachten Abweichungen f"uhren sind aber nicht
klein, so dass die korrekte Kettenregel bei Anwesenheit von Rauschen die
It\^o-sche Kettenregel wird, die stattdessen den Ausdruck
\[
dY =  \biggl(u'b + \frac12u''\biggr)\,dt + u'\,dW
\]
liefert.
Der Term $\frac12u''$ ist in der klassischen Analysis nicht vorhanden.
Wir m"ussen daher auch alle gewohnten Rechenregeln der Analysis "uberpr"ufen.

\subsection{Stochastische Integrale\label{subsection:stochint}}
\index{stochastisches Integral}
Wir m"ochten eine Definition f"ur ein Integral der Form
\[
\int_0^T G\,dW
\]
f"ur zwei stochastische Prozesse $G(t)$ und $W(t)$.

\subsubsection{Das Paley-Wiener-Zygmund Integral}
\index{Paley-Wiener-Zygmund Integral}
F"ur differenzierbare Funktionen $g(t)$ und $w(t)$ ist klar, was 
mit dem Integral gemeint ist:
\[
\int_0^T g(t)\,dw = \int_0^T g(t) w'(t)\,dt.
\]
Wenn $g(0)=g(T)=0$ ist, dann kann man dies mit Hilfe partieller Integration
vereinfachen:
\[
\int_0^T g(t)\,dw
=
\int_0^T g(t) w'(t)\,dt
=
[g(t)w(t)]_0^T
-
\int_0^T g'(t) w(t)\,dt
=
-\int_0^T g'(t) w(t)\,dt
\]
Diese letzte Formel ist aber auch geeignet als Definition des Integrals
f"ur eine brownsche Bewegung $W(t)$ an Stelle von $w(t)$.

\begin{definition}
F"ur eine stetig differenzierbare Funktion $g\colon[0,T]\to \mathbb R$ 
mit $g(0)=g(T)=0$, setze
\[
\int_0^T g\,dW = -\int g'(t) W(t)\,dt.
\]
\end{definition}

\begin{hilfssatz}
Erwartungswert und Varianz der Zufallsvariablen
\[
Z=\int_0^T g\,dW
\]
ist
\begin{compactenum}
\item $E(Z)=0$
\item $\operatorname{var}(Z)=\int_0^Tg(t)^2\,dt$
\end{compactenum}
\end{hilfssatz}

\begin{proof}[Beweis]
F"ur den Erwartungswert finden wir mit Hilfe der Rechenregeln f"ur den
Erwartungswert
\begin{align*}
E(Z)
&=
E\biggl(\int_0^T g\,dW\biggr)
=
E\biggl(
\int_0^T g'(t) W(t)\,dt
\biggr)
=
\int_0^T g'(t) \underbrace{E(W(t))}_{\textstyle =0}\,dt=0
\end{align*}
Die Varianz ist daher der Erwartungswert $E(Z^2)$, die wir ebenfalls unter
Verwendung der Rechenregeln berechnen k"onnen
\begin{align*}
E(Z^2)
&=
E\biggl(\biggl(\int_0^T g\,dW\biggr)^2\biggr)
=
E\biggl(
\biggl(
\int_0^T g'(t)W(t)\,dt
\biggr)^2
\biggr)
\\
&=
E\biggl(
\int_0^T g'(t)W(t)\,dt
\int_0^T g'(s)W(s)\,ds
\biggr)
=
E\biggl(
\int_0^T \int_0^T g'(t)W(t) g'(s)W(s) \,ds \,dt
\biggr)
\\
&=
\int_0^T\int_0^Tg'(t)g'(s) \underbrace{E(W(t)W(s))}_{\textstyle =t\wedge s}\,ds\,dt
\\
&=
\int_0^Tg'(t) \biggl(\int_0^t g'(s) s\,ds + \int_t^T g'(s)t\,ds\biggr)
\,dt
\\
&=
\int_0^T g'(t)\biggl(
\underbrace{[g(s)s]_0^t}_{\textstyle =tg(t)}-\int_0^t g(s)\,ds+t(\underbrace{g(T)}_{\textstyle =0}-g(t))
\biggr)\,dt
\\
&=
\int_0^T g'(t)\biggl(-\int_0^t g(s)\,ds\biggr)\,dt
\\
&=
\biggl[
g(t)\biggl(-\int_0^t\int_0^tg(s)\,ds\biggr)
\biggr]_0^T
+
\int_0^Tg(t)^2\,dt
\end{align*}
Damit ist die Behauptung bewiesen.
\end{proof}
Der Hilfssatz kann dazu verwendet werden, die Definition des Integrals auf
weitere Funktionen auszudehnen.
Eine Folge von Funktionen $g_n$ f"uhrt auf eine Folge von Werten des
Integrals. 
Um daraus eine Erweiterung des Integrals zu konstruieren, m"ussen wir 
definieren, was es heissen soll, dass die Folge $g_n$ konvergiert.
Wir verwenden die Norm
\[
\|f\|_2^2=\int_0^T f(t)^2 \,dt,
\]
um den Abstand zwischen Funktionen zu messen.
Eine Cauchy-Folge von Funktionen in dieser Norm ist dann eine Folge so,
dass f"ur jedes $\varepsilon>0$ ein $N>0$ existiert, so dass aus
$n,m>N$ folgt
\[
\|g_n-g_m\|_2^2=\int_0^T (g_n(t)-g_m(t))^2\,dt<\varepsilon.
\]
In diesem Fall gilt, dass
\[
E\biggl(
\biggl(
\int_0^T g_n\,dW
-
\int_0^T g_m\,dW
\biggr)^2
\biggr)
=
\int_0^T(g_m(t)-g_n(t))^2\,dt<\varepsilon
\]
Die Zufallsvariablen
\[
Z_n = \int_0^T g_n\,dW
\]
bildet daher eine Cauchy-Folge von quadratintegrierbaren Funktionen
in $L^2(\Omega)$, und wir k"onnen deren Grenzwert
\[
\int_0^T g\,dW
=
\lim_{n\to\infty}\int_0^T g_n\,dW
\]
als den verallgemeinerten Wert f"ur das Integral einer beliebigen Funktion
$g\in L^2([0,T])$ definieren.

\subsubsection{Riemann-Summen}
Die Verallgemeinerung des Paley-Wiener-Zygmund-Integral auf
quadratintegrierbare Funktionen ist allerdings nicht ausreichend, um
\[
\int_0^T W\,dW
\]
zu definieren.
Wir k"onnen aber noch weiter zur"uck gehen zur Definition des
Riemann-Integrals, und sie verallgemeinern auf einen stochastischen
Prozess.

\begin{definition}
Eine {\em Unterteilung} $P$ des Intevals $[0,T]$ ist eine endliche Menge
von Teilpunkten
\[
P=\{ 0=t_0<t_1<t_2<\dots<t_m=T\}.
\]
Die {\em Maschenweite} $|P|$ der Unterteilung $P$ ist
\[
|P|=\max_{0\le k\le m-1}|t_{k+1\mathstrut}-t_{k\mathstrut}|.
\]
F"ur ein festes $0\le \lambda\le 1$ setzen wir
\[
\tau_k = (1-\lambda)t_{k\mathstrut}+\lambda t_{k+1\mathstrut}.
\]
\end{definition}
F"ur $\lambda=0$ ist $\tau_k=t_k$, f"ur $\lambda=1$ gilt $\tau_k=t_{k+1}$.
F"ur andere Werte von $\lambda$ ist $\tau_k$ ein Punkt im Inneren des
Intervals $[t_k,t_{k+1}]$.
Wie "ublich kann eine solche Unterteilung dazu verwendet werden, die 
Riemann-Summe als Approximation des Integrals zu definieren:

\begin{definition}
Die Riemann-Summe von $W$ f"ur die Unterteilung $P$ ist
\[
R=R(P,\lambda) = \sum_{k=0}^{m-1} W(\tau_k)(W(t_{k+1})-W(t_k)).
\]
\index{Riemann-Summe}
\end{definition}

\begin{hilfssatz}
\label{stochastisch:quadrvariation}
Sei
\[
P^{(n)}
=
\{0=t_0^{(n)}<t_1^{(n)}<\dots < t_{m_n}^{(n)}=T\}
\]
eine Folge von Unterteilungen des Intervals $[0,T]$ derart,
dass die Maschenweite gegen $0$ strebt, dann gilt
\[
\lim_{n\to\infty}
\sum_{k=0}^{m_n-1} \bigl(W(t_{k+1}^{(n)})-W(t_{k}^{(n)})\bigr)^2
=T
\]
in $L^2(\Omega)$.
\end{hilfssatz}

\begin{proof}[Beweis]
Konvergenz in $L^2(\Omega)$ bedeutet, dass der Erwartungswert der
quadratische Abweichung gegen $0$ streibt.
Setzen wir
\[
Q_n
= 
\sum_{k=0}^{m_n-1} \left(W(t_{k+1}^{(n)})-W(t_{k}^{(n)})\right)^2,
\]
und m"ussen untersuchen, ob $E((Q_n - T)^2)$ gegen $0$ strebt.
Wir beginnen mit der Bemerkung, dass
\begin{align*}
Q_n-T
&=
\sum_{k=0}^{m_n-1}
\left((W(t_{k+1}^{(n)})-W(t_{k}^{(n)}))^2 - (t_{k+1}^{(n)}-t_k^{(n)})\right).
\end{align*}
Die einzelnen Differenzen k"urzen wir ab als
\begin{align*}
Y_k
&=
\frac{W(t_{k+1}^{(n)})-W(t_k^{(n)})}{\sqrt{t_{k+1}^{(n)}-t_{k}^{(n)}}},
\\
\Delta_k
&=
t_{k+1}^{(n)}-t_k^{(n)}.
\end{align*}
F"ur $j\ne k$ sind $Y_j^{(n)}$ und $Y_{k}^{(n)}$ unabh"angig.
Ausserdem ist $Y_k$ standardnormalverteilt, also $E(Y_k)=0$
und $E(Y_k^2)=1$.
Die Summanden in der Summe lassen sich damit kompakter ausdr"ucken:
\begin{align*}
Q_n-T
&=
\sum_{k=0}^{m_n-1} (Y_k^2-1) \Delta_k
\end{align*}
Dies k"onnen wir in $E((Q_n-T)^2)$ einsetzen:
\begin{align*}
E((Q_n-T)^2)
&=
\sum_{k=0}^{m_n-1}
\sum_{j=0}^{m_n-1}
E\biggl(
(Y_k^2-1)\Delta_k
(Y_j^2-1)\Delta_j
\biggr)
\end{align*}
Die Doppelsumme kann zerlegt werden in Terme mit $j=k$ und $j\ne k$.

In den Terme mit $j\ne k$ sind die $Y_k$ und $Y_j$ voneinander unabh"angig,
der Erwartungswert des Produktes kann daher in das Produkt der Erwartungswerte
zerlegt werden:
\begin{align*}
E\biggl(
(Y_k^2-1)\Delta_k
(Y_j^2-1)\Delta_j
\biggr)
&=
E(Y_k^2-1)
E(Y_j^2-1)
\Delta_k
\Delta_j
\\
&=
\bigl(E(Y_k^2) -1\bigr)
\bigl(E(Y_j^2) -1\bigr)
\Delta_k
\Delta_j
=0
\end{align*}
Im letzten Schritt haben wir verwendet, dass $E(Y_k^2)=\operatorname{var}Y_k=1$
ist.

Die Terme mit $j=k$ sind 
\begin{align*}
E((Q_n-T)^2)
&=
\sum_{k=0}^{m_n-1} E\biggl((Y_k^2-1)^2\Delta_k^2\biggr)
\\
&=
\sum_{k=0}^{m_n-1} E\bigl((Y_k^2-1)^2\bigr)\Delta_k^2
\le 
\biggl(\sum_{k=0}^{m_n-1} E\bigl((Y_k^2-1)^2\bigr)\Delta_k\biggr)\, |P^{(n)}|
\end{align*}
Im letzten Schritt haben wir einen Faktor $\Delta_k$ aus der Summe
herausgenommen und durch die Maschenweite $|P^{(n)}|$ abgesch"atzt.

Da $Y_k$ standardnormalverteilt ist, ist der Erwartungswert $E((Y_k^2-1)^2)$
unabh"angig von $k$, wir nennen diesen Wert $C$, der genaue Wert ist
nicht wichtig\footnote{Man kann den Wert nat"urlich wie folgt berechnen:
\begin{align*}
E((Y_k^2-1)^2)
&=
E(Y_k^4-2Y_k^2+1)
=
E(Y_k^4)-2E(Y_k^2)+E(1)
=
6-2+1=5.
\end{align*}}.
Damit l"asst sich die Differenz jetzt unabh"angig von $k$ absch"atzen
\begin{align*}
E((Q_n-T)^2)
&
\le
C\biggl(\sum_{k=1}^{m_n}\Delta_k\biggr)\, |P^{(n)}|
=CT|P^{(n)}|.
\end{align*}
Da die Maschenweite $|P^{(n)}|\to 0$ f"ur $n\to\infty$ folgt, dass
$Q_n\to T$ in $L^2(\Omega)$.
\end{proof}

\begin{hilfssatz} Sei $P^{(n)}$ eine Folge von Unterteilungen des
Intervals $[0,T]$ derart, dass die Maschenweite gegen $0$ strebt und
sei $R_n=R(P^{(n)},\lambda)$ die zugeh"orige Riemann-Summe.
Dann gilt
\[
\lim_{n\to\infty} R_n = \frac{W(T)^2}2 + \biggl(\lambda-\frac12\biggr)T
\]
als Funktion in $L^2(\Omega)$.
\end{hilfssatz}

\begin{proof}[Beweis]
Wir m"ussen die Riemann-Summe
\begin{align*}
R_n
&=
\sum_{k=0}^{m_n-1}
W(\tau_k) (W(t_{k+1}^{(n)})-W(t_k^{(n)}))
\end{align*}
durch Inkremente von Werten von $W(t)$ ausdr"ucken, konkret durch
\[
W(t_{k+1}^{(n)})-W(t_k^{(n)}),\quad
W(t_{k+1}^{(n)})-W(\tau_k^{(n)})
\quad \text{und} \quad
W(\tau_k^{(n)})-W(t_k^{(n)}).
\]
Um herauszufinden, wie dies m"oglich sein k"onnte, k"urzen wir die Werte
von $W$ ab durch
\[
a_k=W(t_k^{(n)}),\quad
b_k=W(\tau_k^{(n)})\quad\text{und}\quad
c_k=W(t_{k+1}^{(n)}),
\]
wobei wir im folgenden zur Vereinfachung der Rechnung die Indizes auch
weglassen.
Jetzt versuchen wir $b(c-a)$ durch andere Differenzen auszudr"ucken.
\begin{align*}
(b-a)^2
&=
b^2-2ab + a^2
\\
(c-a)^2
&=
c^2-2ac+a^2
\\
(c-b)(b-a)
&=
cb-b^2-ac+ab
\end{align*}
Um den Ausdruck $b(c-a)$ zu produzieren, muss der letzte Term verwendet
werden, denn er ist der einzige, der $bc$ enth"alt.
Dann muss aber auch der erste Term verwendet werden, um den Term $b^2$ zum
Verschwinden  zu bringen.
Ebenso muss der zweite Term $\frac12$-mal subtrahiert werden, damit der
Ausdruck $ac$ wegf"allt:
\begin{align*}
(b-a)^2-\frac12(c-a)^2+(c-b)(b-a)
&=
(b^2-2ab + a^2)
-\frac12(c^2-2ac+a^2)
+(cb-b^2-ac+ab)
\\
&=
cb-ab + \frac12a^2
-\frac12c^2
\end{align*}
Die quadratischen Terme sind $\frac12c_k^2=\frac12W(t_{k+1}^{(n)})^2$ und
$\frac12a_k^2=\frac12W(t_{k}^{(n)})^2$, die sich in aufeinanderfolgenden Termen
jeweils wegheben.
Nur der erste und letzte Term bleibt in der Summe bestehen, was wir
aber leicht korrigieren k"onnen.
So finden wir daher:
\begin{align*}
R_n
&=
\underbrace{
\sum_{k=0}^{m_n-1} (b_k-a_k)^2
}_{\textstyle\to\lambda T}
-
\underbrace{
\frac12\sum_{k=0}^{m_n-1} (c_k-a_k)^2
}_{\textstyle\to T}
+\sum_{k=0}^{m_n-1} (c_k-b_k)(b_k-a_k)
- \frac12a_0^2
+ \frac12c_{m_n-1}^2
\end{align*}
Die zweite Summe wurde im Hilfssatz~\ref{stochastisch:quadrvariation}
berechnet, dort wurde gefunden, dass die Summe f"ur $n\to\infty$ gegen $T$
konvergiert.
Mit der gleichen Rechnung wie in \ref{stochastisch:quadrvariation}
kann man finden, dass die zweite Summe gegen $\lambda T$ strebt.

Die Inkremente in der dritten Summe sind unabh"angig voneinander, daher
verschwinden die Erwartunsgewerte gemischter Produkte, es bleiben nur
die Terme
\begin{align*}
E\biggl(\biggl(\sum_{k=0}^{m_n-1}(c_k-b_k)(b_k-a_k)\biggr)^2\biggr)
&=
\sum_{k=0}^{m_n-1}
E\bigl((c_k-b_k)^2\bigr)E\bigl((b_k-a_k)^2\bigr)
\\
&=
\sum_{k=0}^{m_n-1}
E\bigl((W(t_{k+1}^{(n)})-W(\tau_k^{(n)}))^2\bigr)E\bigl((W(\tau_k^{(n)})-W(t_k^{(n)})^2\bigr)
\\
&=
\sum_{k=0}^{m_n-1}
(t_{k+1}^{(n)}-\tau_k^{(n)}) (\tau_k^{(n)}-t_k^{(n)})
\\
&=
\sum_{k=0}^{m_n-1}
(1-\lambda)(t_{k+1}^{(n)}-t_k^{(n)}) \lambda(t_{k+1}^{(n)}-t_k^{(n)})
\\
&\le 
(1-\lambda)\lambda
|P^{(n)}|
\sum_{k=0}^{m_n-1}
(t_{k+1}^{(n)}-t_k^{(n)})
\le (1-\lambda)\lambda T|P^{(n})|\to 0
\end{align*}
f"ur $n\to\infty$.
Damit ist gezeigt, dass die Riemann-Summe in $L^2(\Omega)$ gegen
\[
R_n\to
\frac{W(T)^2}2+\biggl(\lambda-\frac12\biggr)T
\]
konvergiert.
\end{proof}

\subsubsection{Das It\^o-Integral}
Ausgehend von der Riemann-Summe und den f"ur sie hergeleiteten Eigenschaften
wollen wir jetzt ein Integral
\[
\int_0^T G\,dW
\]
f"ur eine breitere Klasse von stochastischen Prozessen $G$ definieren.
Der erste Schritt dazu, ist das Integral f"ur Stufenprozesse zu definieren.

\begin{definition}
\index{Stufenprozess}
Ein stochastischer Prozess $G(t)$ ist eine {\em Stufenprozess}, wenn es eine
Unterteilung $P=\{0=t_0<t_1<\dots<t_m=T\}$ gibt derart, dass
$G(t)=G(t_k)$ f"ur $t_k\le t<t_{k+1}$.
\end{definition}

\begin{definition}
Ist $G$ ein Stufenprozess, dann ist das {\em It\^o-Integral} von $G$ definiert
als
\[
\int_0^TG\,dW = \sum_{k=0}^{m-1}G(t_k)(W(t_{k+1})-W(t_k)).
\]
Das It\^o-Integral ist eine Zufallsvariable.
\end{definition}

Das It\^o-Integral hat die folgenden Eigenschaften
\begin{hilfssatz}
Wenn $G$ und $H$ Stufenprozesse sind und $a,b\in\mathbb R$, dann gilt
\begin{compactenum}
\item Das It\^o-Integral ist linear:
\[
\int_0^T aG+bH\,dW
=
a\int_0^T G\,dW + b \int_0^TH\,dW
\]
\item Der Erwartungswert des It\^o-Integrals ist $0$:
\[
E\biggl(\int_0^T G\,dW\biggr)=0.
\]
\item Die Varianz des It\^o-Integrals ist
\[
E\biggl(\biggl(\int_0^T G\,dW\biggr)^2\biggr)
=
E\biggl(\int_0^T G^2\,dt\biggr)
\]
\end{compactenum}
\end{hilfssatz}

\begin{proof}[Beweis]
F"ur die Linearit"at verwenden wir eine Unterteilung $P$, f"ur die sowohl
$G$ also auch $H$ ein ein Stufenprozess ist.
Dann gilt
\begin{align*}
\int_0^T aG+bH\,dW
&=
\sum_{k=0}^{m-1} (aG(t_k)+bH(t_k))(W(t_{k+1})-W(t_k))
\\
&=
a\sum_{k=0}^{m-1} G(t_k)(W(t_{k+1})-W(t_k))
+
b\sum_{k=0}^{m-1} H(t_k)(W(t_{k+1})-W(t_k))
\\
&=a\int_0^TG\,dW+b\int_0^TH\,dW
\end{align*}
womit die Linearit"at bewiesen ist.

F"ur den Erwartungswert berechnen wir mit Hilfe einer passenden Unterteilung
\begin{align*}
E\biggl(\int_0^T G\,dW\biggr)
&=
E\biggl(\sum_{k=0}^{m-1} G(t_k) (W(t_{k+1}) - W(t_k))\biggr)
\\
&=
\sum_{k=0}^{m-1} E(G(t_k)) \underbrace{E(W(t_{k+1}) - W(t_k))}_{\textstyle =0}=0.
\end{align*}
Damit ist 2.~beweisen.

F"ur die Berechnung der Varianz verwendet wieder die Unabh"angigkeit der
Inkremente:
\begin{align*}
E\biggl(\biggl(\int_0^T G\,dW\biggr)^2\biggr)
&=
E\biggl(\biggl(\sum_{k=0}^{m-1}G(t_k)(W(t_{k+1})-W(t_k))\biggr)^2\biggr)
\\
&=
E\biggl(
\sum_{k=0}^{m-1}G(t_k)(W(t_{k+1})-W(t_k))
\sum_{j=0}^{m-1}G(t_j)(W(t_{j+1})-W(t_j))
\biggr)
\\
&=
\sum_{k=0}^{m-1}
\sum_{j=0}^{m-1}
E(G(t_k) G(t_j)
(W(t_{k+1})-W(t_k))
(W(t_{j+1})-W(t_j))
)
\\
&=
\sum_{k=0}^{m-1}
E(G(t_k)^2) E(W(t_{k+1})-W(t_k))^2)
\\
&=
\sum_{k=0}^{m-1}
E(G(t_k)^2) (t_{k+1}-t_k)
=
E\biggl(
\sum_{k=0}^{m-1}
G(t_k)^2 (t_{k+1}-t_k)
\biggr)
=E\biggl(\int_0^T G(t)^2\,dt\biggr).
\end{align*}
Somit ist auch 3.~bewiesen.
\end{proof}

Die Bedeutung der dritten Eigenschaft besteht darin, dass eine
Folge von Stufen-Prozessen, die $L^2(0,T)$ konvergiert, zu einer
konvergenten Folge von It\^o-Integralen f"uhrt.

\begin{definition}
\index{It\^o-Integral}
Ist $G$ ein beliebiger auf $[0,T]$ quadratintegrierbarer Prozess,
und $G_n$ eine Folge von stochastischen Prozessen, die in $L^2([0,T])$
gegen $G$ konvergiert.
Dann ist des It\^o-Integral von $G$
\[
\int_0^T G\,dW
=
\
\lim_{n\to\infty} \int_0^T G_n\,dW
\]
\end{definition}

Das It\^o-Integral hat die folgenden Eigenschaften:

\begin{satz}
\label{satz:ito-integral}
Seien $G$ und $H$ auf $[0,T]$ quadratintegrierbare stochastische Prozesse
und $a,b\in\mathbb R$. Dann gilt
\begin{compactenum}
\item Das It\^o-Integral ist linear:
\[
\int_0^T aG+bH\,dW
=
a\int_0^TG\,dW
+
b\int_0^TH\,dW
\]
\item Der Erwartungswert des It\^o-Integrals ist
\[
E\biggl(\int_0^T G\,dW\biggr)=0
\]
\item Die Varianz des It\^o-Integrals ist
\[
E\biggl(\biggl(\int_0^TG\,dW\biggr)^2\biggr)
=
E\biggl(\int_0^T G^2\,dt\biggr).
\]
\end{compactenum}
\end{satz}

Mit dem It\^o-Integral k"onnen wir jetzt alle Komponenten einer
stochastischen Differentialgleichung definieren.
Zun"achst k"onnen wir f"ur jeden auf $[0,T]$ quadratintegrierbaren Prozess $G$ 
und f"ur jede integrierbare Funktion $F\in L^1([0,T])$ die Gr"ossen
\begin{align*}
\int_s^r F(t)\,dt&=\int_0^r F(t)\,dt - \int_0^s F(t)\,dt
\\
\int_s^r G\,dW&=\int_0^r G\,dW - \int_0^s G\,dW
\end{align*}
definieren.
\begin{definition}
Wenn $X$ ein stochastischer Prozess ist, der
\[
X(r)=X(s)+\int_s^rF(t)\,dt + \int_s^tG\,dW
\]
erf"ullt, dann sagt man $X$ habe das {\em stochastische Differential}
\[
dX=F\,dt + G\,dW.
\]
\end{definition}

%
%
%
\subsection{Rechenregeln}
Bisher wissen wir nur, dass das It\^o-Integral linear ist, dies reicht
aber nicht f"ur einen Kalk"ul, mit dem wir hoffen k"onnen,
Differentialgleichungen zu l"osen.u
Dazu brauchen wir mindestens noch eine Kettenregel und eine
Produktregel.

\begin{satz}[It\^o's Kettenregel]
\index{It\^o-Kettenregel}
Falls $X$ das stochastische Differential
\[
dX=F\,dt + G\,dW
\]
hat, und falls $u\colon \mathbb R\times [0,T]\to\mathbb R$ eine
stetig differenzierbare Funktion.
Dann hat die Funktion $Y(t)=u(X(t), t)$ das stochastische Differential
\begin{align*}
du(X,t)
&=\frac{\partial u}{\partial t}\,dt + \frac{\partial u}{\partial x}\,dX 
+\frac12\frac{\partial u^2}{\partial x^2}G^2\,dt
\\
&=
\biggl(
\frac{\partial u}{\partial t}+\frac{\partial u}{\partial x}F
+\frac12\frac{\partial^2u}{\partial x^2}G^2
\biggr)\,dt
+
\frac{\partial u}{\partial x}G\,dW.
\end{align*}
\end{satz}
Bis auf den Term in den zweiten Ableitungen sind die Terme genau
diejenigen, die man von der klassischen Kettenregel her erwarten
k"onnte.

\begin{beispiel}
Potenzen eines Wiener-Prozesses.
Sei $u(x)=x^m$ und $X=W$, $F=0$ und $G=1$.
Also ist $Y=W^m$.
Dann hat nach der It\^o-schen Kettenregel das stochastische Differential
\[
d(W^m)
=
\frac12m(m-1)W^{m-2}\,dt + mW^{m-1}\,dW.
\]
Im Spezialfall $m=2$ folgt
\[
dW^2 = 2W\,dW + dt.
\]
\end{beispiel}

\begin{beispiel} Wir betrachten die Funktion
$u(x,t)=e^{\lambda x-\frac12\lambda^2 t}$, und wie vorhin $X=W$, $F=0$
und $G=1$.
Es folgt
\begin{align*}
d\biggl(
e^{\lambda W(t)-\frac12\lambda^2 t}
\biggr)
&=
\biggl(
-\frac12\lambda^2 e^{\lambda W-\frac12\lambda^2 t}
+
\frac12\lambda^2 e^{\lambda W-\frac12\lambda^2 t}
\biggr)\,dt
+
\lambda e^{\lambda W -\frac12\lambda^2 t}\,dW
\\
dY&=\lambda Y\,dW
\end{align*}
also ist der Prozess $Y$ eine L"osung der stochastischen Differentialgleichung
\begin{align*}
dy&=\lambda Y\,dW
\\
Y(0)&=1.
\end{align*}
Dieses Beispiel zeigt, dass der entwickelte Kalk"ul dazu geeignet sein kann,
stochastische Differentialgleichungen zu l"osen.
\end{beispiel}

\begin{satz}[Produktregel von It\^o]
\index{It\^o-Produktregel}
\index{Produktregel von It\^o}
Falls die Prozesse $X_1$ und $X_2$ die stochastischen Differentiale
\begin{align*}
dX_1
&=
F_1\,dt + G_1\,dW
\\
dX_2
&=
F_2\,dt + G_2\,dW
\end{align*}
haben, dann ist
\begin{equation}
d(X_1X_2)
=
X_2\,dX_1 + X_1\,dX_2 + G_1G_2\,dt,
\label{stochastisch:ito-produkt}
\end{equation}
dies ist die {\em It\^o-sche Produktformel}.
\end{satz}

F"ur den Beweis der Kettenregel brauchen wir die Produktregel, wir geben
daher zuerst einen Beweis f"ur die Produktregel an.

\begin{proof}[Beweis der Produktformel]
Die integrierte Form der Produktregel besagt, dass 
\begin{equation}
X_1(r)X_2(r)-X_1(0)X_2(0)
=
\int_0^r X_2\,dX_1 + \int_0^r X_1\,dX_2 + \int_0^r G_1G_2\,dt
\label{stochastisch:ito-produkt-integriert}
\end{equation}
diese m"ussen wir nachrechnen.
Wir nehmen an der Einfachheit halber an, dass $X_1(0)=X_2(0)=0$ ist.

1.~Wir nehmen zus"atzlich an, dass die Funktion $G_i$ und $F_i$ nicht von
der Zeit abh"angen.
Dies bedeutet, dass sich die stochastische Differentialgleichung
$dX_i=F_i\,dt+G_i\,dW$ integrieren l"asst:
\[
X_i(t) = \int_0^t F_i\,d\tau + G_i\int_0^r dW =  F_it+G_iW(t).
\]
Diese Bedingung werden wir im zweiten Schritt wieder aufheben.
Wir berechnen die rechte Seite der integrierten
Produktregel~(\ref{stochastisch:ito-produkt-integriert}):
\begin{align*}
\int_0^r X_2\,dX_1 + X_1\,dX_2 + G_1G_2\,dt
&=
\int_0^rX_1F_2+X_2F_1\,dt + \int_0^r X_1G_2+X_2G_1\,dW + \int_0^r G_1G_2\,dt
\\
&=
\int_0^r(F_1t+G_1W)F_2+(F_2t+G_2W)F_1\,dt
\\
&\qquad
+ \int_0^r (F_1t+G_1W)G_2+(F_2t+G_2W)G_1\,dW + \int_0^r G_1G_2\,dt
\\
&=F_1F_2r^2+(G_1F_2+G_2F_1)\underbrace{\biggl(\int_0^rW\,dt + \int_0^rt\,dW\biggr)}_{\textstyle = rW(r)}
\\
&\qquad
+G_1G_2\cdot \underbrace{2\int_0^r W\,dW}_{\textstyle W(r)^2-r}
+ G_1G_2\underbrace{\int_0^r\,dt}_{\textstyle =r}
\\
&= 
F_1F_2r^2 + (G_1F_2+G_2F_1)rW(r) + G_1G_2W(r)^2
\\
&= 
F_1r\cdot F_2r + F_2r\cdot G_1W(r)+F_1r\cdot G_2W(r) + G_1W(r)\cdot G_2W(r)
\\
&= 
(F_1r + G_1W(r))\cdot(F_2r +  G_2W(r))
=
X_1(r) \cdot X_2(r).
\end{align*}
Dies ist die linke Seite von (\ref{stochastisch:ito-produkt-integriert}), die
damit f"ur diesen Spezialfall bewiesen ist.

2.~Wir ersetzen jetzt die Konstanten $G_i$ und $F_i$ durch Stufenprozesse.
Es gibt eine Unterteilung des Intervals $[0,T]$ so, dass die Prozesse $G_i$
und $F_i$ in jedem Teilinterval konstant sind.
In jedem solchen Teilinterval gilt daher die Ito-Produktformel, und damit
auch f"ur das ganze Interval.

3.~Im allgemeinen Fall k"onnen wir die Prozesse $G_i$ und $F_i$ mit Hilfe
einer Formel von Stufenprozessen $G_i^n$ und $F_i^n$ approximieren.
Wir k"onnen zus"atzlich fordern, dass
\[
E\biggl(\int_0^T |F_i^n-F_i|\,dt\biggr)\to 0
\qquad\text{und}\qquad
E\biggl(\int_0^T (G_i^n-G_i)^2\,dt\biggr)\to 0,
\]
was wir weiter unten brauchen.
Es gibt nach dem zweiten Schritt eine Folge von Prozessen $X_i^n$ mit
\[
X_i^n(t) = X_i(0) + \int_0^t F_i^n\,d\tau + \int_0^t G_i^n\,dW,
\]
f"ur die jeweils die Produktregel gilt.
\begin{align*}
X_i^n(r)X_i^n(r)-X_i^n(s)X_i^n(s)
&=
\int_0^r X_2^n\,dX_1^n+\int_0^r X_1^n\,dX_2^n + \int_0^r G_1^nG_2^n\,dW
\\
&=
\int_0^r X_2^nF_1^n\,dt
+
\int_0^r X_2^nG_1^n\,dW
+
\int_0^r X_1^nF_2^n\,dt
+
\int_0^r X_1^nG_2^n\,dW
+
\int_0^r G_1^nG_2^n\,dW
\\
\intertext{Der Grenz"ubergang $n\to\infty$ f"uhrt auf}
X_i(r)X_i(r)-X_i(s)X_i(s)
&=
\int_0^r X_2F_1\,dt
+
\int_0^r X_2G_1\,dW
+
\int_0^r X_1F_2\,dt
+
\int_0^r X_1G_2\,dW
+
\int_0^r G_1G_2\,dW
\end{align*}
Damit ist die Produktformel bewiesen.
\end{proof}

\begin{proof}[Beweis der Kettenregel]
Der Beweis der Kettenregel verwendet, dass die Funktion $u(x,t)$ durch
Polynome approximiert werden kann.
Wir m"ussen also zun"achst die Kettenregel f"ur Polynome in $x$ beweisen,
und dann mit Hilfe eines Grenz"ubergangs mit approximierenden Polynomen
den allgemeinen Fall gewinnen.

1. Sei $u(x)=x^m$, wir behaupten, dass
\[
d(X^m)=mX^{m-1}\,dX + \frac12m(m-1)X^{m-2}G^2\,dt,
\]
dies ist die It\^o-Produktformel f"ur $u(x)=x^m$.
Wir beweisen diese Formel durch vollst"andige Induktion.
F"ur $m=0$ ist sie trivialerweise korrekt.
Wir nehmen daher an, dass Sie f"ur $m-1$ gilt, dass also
\begin{align*}
d(X^{m-1})
&=
(m-1)X^{m-2}\,dX + \frac12(m-1)(m-2)X^{m-3}G^2\,dt
\\
&=
(m-1)X^{m-2}F\,dt + (m-1)X^{m-2}G\,dW + \frac12(m-1)(m-2)X^{m-3}G^2\,dt.
\end{align*}
Dies wenden jetzt die Produktregel auf $X^m = XX^{m-1}$ an.
F"ur $X_1=X$ ist $G_1=G$, und $X_2=X^{m-1}$ und
\[
G_2
=
(m-1)X^{m-2}G
\]
\begin{align*}
d(X^m)
&=
d(XX^{m-1})
=
X\,d(X^{m-1}) + X^{m-1}\,dX +G_1G_2\,dt
\\
&=
X\biggl( (m-1)X^{m-2}\,dX + \frac12(m-1)(m-2)X^{m-3}G^2\,dt\biggr)
+
X^{m-1}\,dX
+
(m-1)X^{m-2}G^2\,dt
\\
&=
mX^{m-1}\,dX
+\biggl(\frac12(m-2)+1\biggr)(m-1)X^{m-2}G^2\,dt
\\
&=
mX^{m-1}\,dX +\frac12m(m-1)X^{m-2}G^2\,dt,
\end{align*}
damit ist der Induktionsschritt vollzogen.

Wegen der Linearit"at der Ableitung ist die Kettenregel damit
f"ur beliebige Polynome in $x$ bewiesen.

2.~Wir approximieren die Funktion $u(x,t)$ jetzt als Produkt
$u(x,t)=f(x)g(t)$, wobei $f(x)$ und $g(t)$ Polynome sein sollen.
Dann gilt
\begin{align*}
d(u(X,t))
&=
d(f(X)g)
=
f(X)\,dg + g\,df(X)
\\
&=
f(X)g'\,dt + g\biggl(f'(X)\,dX + \frac12 f''(X)G^2\,dt\biggr)
\\
&=
\frac{\partial u}{\partial t}\,dt
+
\frac{\partial u}{\partial x}\,dX
+
\frac12\frac{\partial^2u}{\partial x^2}G^2\,dt.
\end{align*}
Die It\^o-Kettenregel ist damit gezeigt f"ur ein Produkt von Polynomen.

3.~Sei jetzt $u^n$ eine Folge von Polynomen wie im zweiten Schritt,
so dass $u^n$ sowie die Ableitungen gegen $u$ gleichm"assig konvergieren:
\[
u^n\to u,
\qquad
\frac{\partial u^n}{\partial t} \to \frac{\partial u}{\partial t},
\qquad
\frac{\partial u^n}{\partial x} \to \frac{\partial u}{\partial x},
\qquad
\frac{\partial^2 u^n}{\partial x^2} \to \frac{\partial^2 u}{\partial x^2}.
\]
F"ur jedes $u^n$ gilt die It\^o-Kettenregel:
\[
u^n(X(r),r)-u^n(X(0),0)
=
\int_0^r
\frac{\partial u^n}{\partial t}
+
\frac{\partial u^n}{\partial x}F
+
\frac12 \frac{\partial^2 u^n}{\partial x^2}G^2\,dt
+
\int_0^r\frac{\partial u^n}{\partial x}G\,dW,
\]
durch Grenz"ubergang $n\to \infty$ erhalten wir daraus die allgemeine
Form der It\^o-Kettenregel.
\end{proof}

%
% Beispiele 
%
\subsection{L"osungen von stochastischen Differentialgleichungen}
In diesem Abschnitt betrachten wir einige Beispiele von stochatischen
Differentialgleichungen und verwenden die im vorangegangenen Abschnitt
diskutierten Rechenregeln verwendet werden k"onnen, die L"osungen
zu finden.

\subsubsection{Lineare stochastische Differentialgleichung mit $F=0$}
Sei $g$ eine stetige Funktionen, man finde eine L"osung der
Differenzialgleichung
\begin{equation}
\begin{aligned}
dX&=gX\,dW\\
X(0)&=1.
\end{aligned}
\label{stochastisch:beispiel1-dgl}
\end{equation}
Man beachte, dass ohne den stochastischen Einfluss nur noch die Gleichung
$dX=0$ "ubrig bleibt, in diesem Fall ist also $X$ einfach nur eine
Konstante, $X(t)=1$.
Wir k"onnen aber auch nicht erwarten, dass der Erwartungswert der
L"osung durch die L"osung $X(t)=1$ gegeben ist.
Die Gleichung ist zwar linear, aber die Wirkung der stochastischen
Schwankungen ist proportional zum aktuellen Wert von $X$.

W"are $W$ einfach nur eine differenzierbare Funktion, dann w"urde die
Differentialgleichung~(\ref{stochastisch:beispiel1-dgl}) zu der
gew"ohnlichen Differentialgleichung
\begin{align*}
\frac{\dot X}{X} &= g\dot W,
\\
\intertext{die durch direkte Integration gel"ost werden kann:}
\frac{d}{dt}\log X&= g(t) \dot W(t)
\\
\log X(t)&=\int_0^t g(\tau)\dot W(\tau)\,d\tau,
\\
X(t)&=e^{\int_0^t g(\tau)\dot W(\tau)\,d\tau}.
\end{align*}
Daraus k"onnte man ableiten, dass f"ur einen Wiener-Prozess $W$ die L"osung
\[
X(t)=e^{\int_0^t g\,dW}
\]
sein m"usste.
Um dies zu kontrollieren, schreiben wir 
\[
Y(t)=\int_0^t g\,dW
\qquad\Rightarrow\qquad
dY = g\,dW = \underbrace{0}_{\textstyle F}\,dt
+
\underbrace{g}_{\textstyle G}\,dW
\]
f"ur das Integral im Exponenten, und beachten, dass $X(t)=e^{Y(t)}$.
Wir haben die Funktionen $F$ und $G$ wie in der Formulierung der
It\^o-Kettenregel gew"ahlt.

Um zu kontrollieren, ob $X(t)$ tats"achlich die L"osung der
Differentialgleichung~(\ref{stochastisch:beispiel1-dgl}), m"ussen wir
also $e^{Y(t)}$ ableiten, dazu ist die It\^o-Kettenregel zu verwenden,
denn $X(t)=u(Y(t))$ mit $u(x)=e^x$.
Die It\^o-Kettenregel ergibt:
\begin{align*}
dX
&=
\biggl( u'F+\frac12u''G^2 \biggr)\,dt + u' G\,dW
\\
&=
\underbrace{e^Y}_{\textstyle X}\biggl(\frac12g^2\,dt + g\,dW\biggr)
\\
&=
\frac12g^2X\,dt + gX\,dW
\end{align*}
Offensichtlich ist das weit davon entfernt, die L"osung der
Differentialgleichung zu sein.
Insbesondere der erste Term mit $g^2$, der von der It\^o-Kettenregel
beigesteuert wird, ist zu viel.

Um den Term der zweiten Ableitungen wieder los zu werden, versuchen wir
den Ansatz
\begin{equation}
X(t) = e^{-\frac12\int_0^tg^2\,d\tau+\int_0^t g\,dW}.
\label{stochastisch:beispiel1-lsg}
\end{equation}
Um dies nachzupr"ufen schreiben wir wieder 
\[
Y(t)
=
-\frac12\int_0^tg^2\,d\tau+\int_0^t g\,dW
\]
f"ur den Exponenten, dies bedeutet, dass
\[
dY
=
\underbrace{-\frac12g^2}_{\textstyle F}\,dt
+
\underbrace{g}_{\textstyle G}\,dW.
\]
Wir wenden die It\^o-Kettenregel auf die Funktion
$X(t)=u(Y(t))=e^{Y(t)}$, also $u(x)=e^x$, an.
\begin{align*}
dX
&=
\biggl( u'F + \frac12u''G^2\biggr)\,dt + u'G\,dW
\\
&=
\underbrace{\biggl(-\frac12 g^2 + \frac12 g^2\biggr)}_{\textstyle=0}\,dt
+
\underbrace{e^Y}_{\textstyle X}g\,dW
\\
&=gX\,dW.
\end{align*}
Damit haben wir nachgewiesen, dass~(\ref{stochastisch:beispiel1-lsg})
die L"osung der stochastischen
Differentialgleichung~(\ref{stochastisch:beispiel1-dgl}) ist.

Da $g^2\ge 0$ ist, ist das Integral von $g^2$ im Exponenten
immer positiv, somit ist $X(t)$ immer um einen Faktor $\le1$ 
kleiner als die urspr"unglich vermutete L"osung:
\[
X(t)
=
\underbrace{e^{-\frac12 \int_0^tg^2\,d\tau}}_{\textstyle \le 1}\cdot e^{\int_0^tg\,dW}.
\]
Im Spezialfall $g=1$ ist finden wir, dass
\[
X(t)=e^{-\frac12t}e^{W(t)}=e^{W(t)-\frac12t}
\]
die L"osung der Differentialgleichung $dX=X\,dW$ ist.

\subsubsection{Lineare stochastische Differentialgleichung}
Die allgemeinste lineare stochastische Differentialgleichung hat die Form
\begin{equation}
dX=fX\,dt+gX\,dW
\label{stochastisch:beispiel2-dgl}
\end{equation}
mit stetigen Funktionen $f$ und $g$.
Nehmen wir wieder an, dass $W$ eine differenzierbare Funktion ist,
dann m"usste $X$ die Differentialgleichung
\begin{align*}
dX&=fX\,dt + gX\dot W\,\,dt
\\
\intertext{erf"ullen, die wieder durch Integration gel"ost werden kann:}
\frac{\dot X}{X}
&=
f+g\dot W
\\
\Rightarrow\qquad
\log X(t)
&=
e^{\int_0^t f+g\dot W\,d\tau}
=
e^{\int_0^t f\,d\tau+ \int_0^t g\,dW}.
\end{align*}
Wir verzichten darauf nachzurechnen, dass dies nicht die L"osung sein kann,
denn wie vorhin wird ein Term fehlen.
Wir vermuten, dass
\begin{equation}
X(t)
=
e^{\int_0^t f -\frac12g^2\,d\tau + \int_0^t g\,dW}
\label{stochastisch:beispiel2-lsg}
\end{equation}
die L"osung ist.
Wir setzen daher wieder
\[
Y(t)=\int_0^t f-\frac12g^2\,d\tau + \int_0^tg\,dW,
\]
der stochastische Prozess $Y$ erf"ullt die stochastische Differentialgleichung
\[
dY=
\underbrace{\biggl( f-\frac12g^2\biggr)}_{\textstyle F}\,dt
+
\underbrace{g}_{\textstyle G}\,dW.
\]
Wir vermuten, dass $X(t)=u(Y(t))=e^{Y(t)}$ die L"osung der
Differentialgeichung ist.

Wir wenden die It\^o-Kettenregel auf $X(t)$ an und finden
\begin{align*}
dX
&=
\biggl(u'F+\frac12u''G^2\biggr)\,dt + u'G\,dW
\\
&=
e^Y\biggl( f\underbrace{-\frac12g^2+\frac12 g^2}_{\textstyle=0}\biggr)\,dt
+
e^Yg\,dW
\\
&=
Xf\,dt + Xg\,dW.
\end{align*}
Sind $f$ und $g$ konstante Funktionen, also $f(t)=a$ und $g(t)=b$, dann
kann man $Y$ direkt berechnen:
\[
Y(t)=\biggl(a-\frac12b^2\biggr)t + bW(t),
\]
also gilt
\[
X(t)=e^{(a-\frac12b^2)t} e^{bW(t)}.
\]

\subsubsection{Langevin-Gleichung}
Der Wiener-Prozess versucht, die Brownsche Bewegung zu modellieren.
Er kann aber nur funktionieren, wenn sich das Teilchen im Wesentlichen
reibungsfrei bewegen kann.
Ein realistischeres Modells stammt von Langevin, der stochastische 
Prozess $X$ beschreibt die Geschwindigkeit des Teilchens, und erf"ullt
die stochastische Differentialgleichung
\begin{equation}
\begin{aligned}
dX&=-bX\,dt+\sigma\,dW\\
X(0)&=X_0
\end{aligned}
\label{stochastisch:langevin-dgl}
\end{equation}
Der Koeffizient $b$ beschreibt die Reibung, $\sigma$ die Wirkung der
Diffusion.
\index{Langevin-Gleichung}
F"ur $\sigma=0$ bleibt die gew"ohnliche Differentialgleichung
\[
\frac{\dot X}{X}=-b
\qquad\Rightarrow\qquad
\log X(t)=-bt
\qquad\Rightarrow\qquad
X(t)=e^{-bt}X_0.
\]
Der stochastische Term h"angt in dieser Differentialgleichung nicht von $X$
ab, wir k"onnen daher erwarten, dass die Differentialgleichung wenigstens
formal die gleiche L"osung haben wird wie eine gew"ohnliche
inhomogene Differentialgleichung.
Wir vermuten daher, dass
\begin{equation}
X(t)
=
e^{-bt}X_0 + \sigma\int_0^te^{-b(t-\tau)}\,dW
=
e^{-bt}\biggl(X_0 + \sigma\int_0^te^{b\tau}\,dW\biggr)
\label{stochastisch:langevin-lsg}
\end{equation}
die L"osung der Langevin-Gleichung~(\ref{stochastisch:langevin-dgl})
sein wird.
Tats"achlich ist die Ableitung davon
\[
dX
=
-be^{-bt}
\underbrace{\biggl(X_0 + \sigma\int_0^te^{b\tau}\,dW\biggr)}_{\textstyle =X}\,dt
+
e^{-bt}
\sigma e^{bt}dW
=
-bX\,dt +\sigma\,dW,
\]
die Differentialgleichung ist also erf"ullt.

Die L"osung erlaubt, Erwartungswert und Varianz von $X(t)$ zu berechnen.
Dazu verwenden wir die Eigenschaften des Wiener-Prozesses und des
It\^o-Integrals in Satz~\ref{satz:ito-integral}.
F"ur den Erwartungswert erhalten wir.
\begin{align*}
E(X(t))
&=
e^{-bt}E(X_0)
+
\sigma \underbrace{E\biggl(\int_0^t e^{-b(t-\tau)}\,dW\biggr)}_{\textstyle=0}
=
e^{-bt}X_0.
\end{align*}
F"ur den Erwartungswert von $X(t)^2$ berechnen wir
\begin{align*}
E(X(t)^2)
&=
E\biggl(e^{-2bt}X_0^2+2\sigma e^{-bt}X_0\int_0^te^{-b(t-\tau)}\,dW
+\sigma^2\biggl(\int_0^t e^{-b(t-\tau)}\,dW\biggr)^2
\biggr)
\\
&=
e^{-2bt}E(X_0^2)
+
2\sigma e^{-bt}E(X_0)E\biggl( \int_0^te^{-b(t-\tau)}\,dW\biggr)
+
\sigma^2E\biggl(\biggl(\int_0^t e^{-b(t-\tau)}\,dW\biggr)^2\biggr)
\\
&=
e^{-2bt}E(X_0^2)
+
\sigma^2 \int_0^t e^{-2b(t-\tau)}\,d\tau
\\
&=
e^{-2bt}E(X_0^2)
+
\frac{\sigma^2}{2b}(1-e^{-2bt}).
\end{align*}
Die Varianz von $X(t)$ ist daher
\begin{align*}
\operatorname{var}(X(t))
&=
E(X(t)^2)-E(X(t))^2
=
e^{-2bt}E(X_0^2)
+
\frac{\sigma^2}{2b}(1-e^{-2bt}).
-
e^{-2bt}E(X_0)^2
\\
&=
e^{-2bt}\operatorname{var}(X_0)
+
\frac{\sigma^2}{2b}(1-e^{-2bt}).
\end{align*}
Die Varianz von $X(t)$ setzt sich also aus zwei Komponenten zusammen.
Sie ist das gewichtete Mittel der Varianz des Anfangsbedingung
$\operatorname{var}(X_0)$ und der Varianz des Diffusionsterms, n"amlich
$\sigma^2/2b$.
Die Gewichtsfaktoren sind $e^{-2bt}$ und $1-e^{-2bt}$, die sich zu
$1$ addieren.
Zur Zeit $t=0$ ist die Varianz nat"urlich nur die Varianz der Anfangsbedingung.
Mit zunehmender Zeit wird die Varianz der Anfangsbedingung unbedeutend,
und es bleibt nur die Varianz des Diffusionsterms "ubrig.

Der Integralterm in $X(t)$ ist normalverteilt, wie alles was aus dem
Wienerprozess entsteht.
Der Term $e^{-bt}X_0$ wird aber in $X(t)$ immer unbedeutender, somit ist
nach gen"ugend langer Zeit $t$ die Zufallsvariable $X(t)$ normalverteilt
mit Erwartungswert $0$ und Varianz $\sigma^2/2b$.

Kehren wir zur physikalischen Interpretation zur"uck, dann sehen wir,
dass die Geschwindigkeit eines dem Langenvin-Prozesses unterworfenen
Teilchens nach einiger Zeit seine Anfangsgeschwindigkeit ``vergisst'',
die Zeitkonstante daf"ur ist gegeben durch die Reibung $b$.
Die Schwangungen der Geschwindigkeit sind umso gr"osser, je geringer
die Reibung ist.

\subsection{Stochastische Differentialgleichungen 2. Ordnung}
Die meisten Bewegungsgleichungen der Physik sind Differentialgleichungen
zweiter Ordnung.
In diesem Abschnitt wollen wir daher zwei Beispiele untersichen, wie
bekannte Systeme sich unter Einfluss von Rauschen verhalten werden.

\subsubsection{Ornstein-Uhlenbeck-Prozess}
\index{Ornstein-Uhlenbeck-Prozess}
Die Langevin-Gleichung beschreibt die Geschwindigkeit eines Teilchens,
dass der Brownschen Bewegung mit Reibung unterworfen ist.
Wir m"ochten aber auch die Position des Teilchens kennen, wir braucht
also noch eine weitere Zufallsvariable $Y$, welche die Position
beschreibt.
Nat"urlich soll $X$ die Ableitung von $Y$ sein: $X=\dot Y$.
Als klassische Differentialgleichung w"urden wir die Bewegungsgleichung
schreiben als
\[
\ddot Y=-b\dot Y+\sigma\frac{dW}{dt},
\]
wobei der letzte Term nat"urlich nicht wirklich sinnvoll ist.
Um dieser Gleichung Sinn zu geben, schreiben wir sie als
Differentialgleichungssystem erster Ordnung mit den beiden Variablen
$Y_1=Y=\dot X$ und $Y_2=X$ mit den Gleichungen
\begin{align*}
\dot Y_1&=-b Y_1+\sigma \frac{dW}{dt}\\
\dot Y_2&=Y_1.
\end{align*}
Als stochastische Differentialgleichung mit Anfangsbedingungen
wird dies
\begin{equation}
\begin{aligned}
dY_1&=-bY_1\,dt +\sigma\,dW,
&&&
Y_1(0)&=Y_{10}
\\
dY_2&=Y_1\,dt
&&&
Y_2(0)&=Y_{20}
\end{aligned}
\label{stochastisch:ornstein-uhlenbeck-dgl}
\end{equation}
Die erste Gleichung f"ur $Y_1$ ist die Langevin-Gleichung, die wir bereits
in~\ref{stochastisch:langevin-lsg}) gel"ost wurde.
Durch eine weitere Integration kann man auch die Gleichung f"ur $Y_2$
l"osen:
\begin{align*}
Y_1(t)&=e^{-bt}Y_{10}+\int_0^te^{-b(t-\tau)}\,dW\\
Y_2(t)&=Y_{20}+\int_0^t Y_1(\tau)\,d\tau
\end{align*}

\subsubsection{Harmonischer Oszillator}
Der Ornstein-Uhlenbeck-Prozess war insofern einfach zu behandeln, als 
die Geschwindigkeit aus dem Langevin-Prozess einfach noch einmal
integriert werden musste.
Es gab also keine R"uckkopplung zwischen der zweiten Ableitung und
dem Prozess selbst.
Erst eine solche R"uckkopplung kann zu Schwingungen f"uhren.
Wir versuchen daher jetzt einen harmonischen Oszillator zu modellieren,
der zus"atzlich dem Rauschen aus einem Wiener-Prozess unterworfen ist.

Ein harmonischer Oszillator hat die gew"ohnliche Differentialgleichung
zweiter Ordnung
\begin{equation}
\ddot X=-\lambda^2 X-b\dot X, X(0)=X_0, \dot X(0)=X_1.
\label{stochastisch:harmosz-dgl2}
\end{equation}
Da wir zweite Ableitungen in stochastischen Differentialgleichungen
nicht verwenden k"onnen, ersetzen wir~(\ref{stochastisch:harmosz-dgl2})
wieder durch zwei gekoppelte Gleichungen erster Ordnung:
\begin{equation}
\begin{aligned}
\dot Y_1&=Y_2,                &&&Y_1(0)&=X_0, \\
\dot Y_2&=-\lambda^2 Y_1-bY_2,&&&Y_2(0)&=X_1.
\end{aligned}
\label{stochastsich:harmosz-dgl1}
\end{equation}
In dieser Form k"onnen wir jetzt in der Form eines Systems stochastischer
Differentialgleichungen formulieren:
\begin{align*}
dY_1&=Y_2\,dt,\\
dY_2&=(-\lambda^2Y_1-bY_2)\,dt.
\end{align*}
Wenn wir auf der rechten Seite von~(\ref{stochastisch:harmosz-dgl2})
einen Rauschterm, also eine Ableitung eines Wiener-Prozesses hinzuf"ugen
wollen, dann wird daraus das stochastische Differentialgleichungssystem
\begin{equation}
\begin{aligned}
dY_1&=Y_2\,dt,                                &&&Y_1(0)&=X_0,\\
dY_2&=(-\lambda^2 Y_1-bY_2)\,dt + \sigma\,dW, &&&Y_2(0)&=X_1.
\end{aligned}
\end{equation}

Wie bei den gew"ohnlichen Differentialgleichungen k"onnen wir diese
Gleichung einfacher l"osen, wenn wir sie in Matrixform schreiben:
\begin{equation}
d\begin{pmatrix}Y_1\\Y_2\end{pmatrix}
=
\underbrace{
\begin{pmatrix}
         0& 1\\
-\lambda^2&-b
\end{pmatrix}}_{\textstyle=D}
\begin{pmatrix}Y_1\\Y_2\end{pmatrix}\,dt
+
\begin{pmatrix}
\sigma\\0
\end{pmatrix}\,dW.
\end{equation}
Die L"osung wird dann
\begin{equation}
\begin{pmatrix}
Y_1(t)\\Y_2(t)
\end{pmatrix}
=
e^{Dt}\begin{pmatrix}X_0\\X_1\end{pmatrix}
+
\sigma \int_0^t e^{D(t-\tau)}\begin{pmatrix}0\\\sigma\end{pmatrix}\,dW.
\label{stochastisch:harmosz-explsg}
\end{equation}

In Abschnitt~\ref{linear:harmosz} wurde der Spezialfall $b=0$ bereits
behandelt.
Die dort gefundenen Formeln f"ur $e^{Dx}$ erm"oglichen, direkt eine
L"osung anzugeben:
%\begin{beispiel}
%Wir berechnen die L"osung f"ur den Spezialfall $b=0$.
%Die Matrix $D$ ist
%\[
%D=\begin{pmatrix}
%0&1\\-\lambda^2&0
%\end{pmatrix}.
%\]
%Die Matrix $e^{Dt}$ kann durch Diagonalisierung berechnet werden.
%Die Transformationsmatrix
%\[
T
%=
%\begin{pmatrix}
%1&1\\
%i\lambda&-i\lambda
%\end{pmatrix},
%\qquad
%T^{-1}
%=
%\frac12
%\begin{pmatrix}
%1& 1/i\lambda\\
%1&-1/i\lambda
%\end{pmatrix}
%\]
%bringt die Matrix $D$ in Diagonalform:
%\[
%T^{-1}DT
%=
%\begin{pmatrix}
%i\lambda&        0\\
%        &-i\lambda
%\end{pmatrix}.
%\]
%Daraus kann man jetzt die Exponentialfunktion berechnen:
%\begin{align*}
%T^{-1}e^{Dt}T
%&=
%\begin{pmatrix}
%\cos\lambda t+i\sin\lambda t&              0             \\
%              0             &\cos\lambda t-i\sin\lambda t
%\end{pmatrix}
%\\
%\Rightarrow\qquad\qquad
%e^{Dt}
%&=
%\begin{pmatrix}
%                \cos\lambda t&\frac1{\lambda}\sin\lambda t\\
%-\frac1{\lambda}\sin\lambda t&               \cos\lambda t
%\end{pmatrix}
%\end{align*}
%Daraus k"onnen wir jetzt die L"osung mit der
%Formel~(\ref{stochastisch:harmosz-explsg}) ablesen:
\begin{equation}
\begin{pmatrix}
Y_1(t)\\Y_2(t)
\end{pmatrix}
=
\begin{pmatrix}
                \cos\lambda t&\frac1{\lambda}\sin\lambda t\\
-\frac1{\lambda}\sin\lambda t&               \cos\lambda t
\end{pmatrix}
\begin{pmatrix}X_0\\X_1\end{pmatrix}
+
\sigma\int_0^t
\begin{pmatrix}
\frac1{\lambda}\sin\lambda(t-\tau)\\
               \cos\lambda(t-\tau)
\end{pmatrix}\,dW.
\label{stochastisch:harmosz-y2}
\end{equation}
Die erste Komponenten davon ist
\begin{equation}
Y_1(t)
=
X_0\cos\lambda t+\frac{X_1}{\lambda}\sin\lambda t
+
\frac{\sigma}{\lambda}\int_0^t\sin\lambda(t-\tau)\,dW,
\label{stochastisch:harmosz-y}
\end{equation}
die L"osung der Differentialgleichung.

Die L"osungsformel~(\ref{stochastisch:harmosz-y}) zeigt auch, dass
$Y_1(t)$ selbst dann von $0$ verschieden sein kann, wenn $X_0=X_1=0$ ist.
Wir wollen f"ur diesen Spezialfall Erwartungswert und Varianz 
von $Y_1(t)$ berechnen.
Wir verwenden wieder die Resultate von Satz~\ref{satz:ito-integral}.
Der Integralterm verschwindet bei der Berechnung des Erwartungswertes:
\begin{align*}
E(Y_1(t))
&=
E(X_0)\cos\lambda t
+
\frac{E(X_1)}{\lambda}\sin\lambda t
+
\frac{\sigma}{\lambda}E\biggl(\int_0^t\sin\lambda(t-\tau)\,dW\biggr)
\\
&=
E(X_0)\cos\lambda t+\frac{E(X_1)}{\lambda}\sin\lambda t.
\\
\intertext{Den Erwartungswert von $Y_1(t)^2$ k"onnen wir wie folgt berechnen:}
E(Y_1(t)^2)
&=
E(X_0^2)\cos^2\lambda t
+
2E(X_0X_1)\frac1{\lambda}\cos\lambda t\sin\lambda t
+
\frac{E(X_1^2)}{\lambda^2}\sin^2\lambda t
%+
%\frac{\sigma E(X_1)}{\lambda^2}E\biggl(\int_0^t\sin\lambda(t-\tau)\,dW\biggr)
%+
%2E(X_0)\frac{\sigma}{\lambda}\cos\lambda t
%E\biggl(\int_0^t\sin\lambda(t-\tau)\,dW\biggr)
%\\
%&\qquad
+
\frac{\sigma^2}{\lambda^2}E\biggl(
\biggl(\int_0^t\sin\lambda(t-\tau)\,dW\biggr)^2
\biggr)
\\
&=
E(X_0^2)\cos^2\lambda t
+
2E(X_0X_1)\frac1{\lambda}\cos\lambda t\sin\lambda t
+
\frac{E(X_1^2)}{\lambda^2}\sin^2\lambda t
+
\frac{\sigma^2}{\lambda^2}\int_0^t \sin^2\lambda(t-\tau)\,d\tau
\\
&=
E(X_0^2)\cos^2\lambda t
+
2E(X_0X_1)\frac1{\lambda}\cos\lambda t\sin\lambda t
+
\frac{E(X_1^2)}{\lambda^2}\sin^2\lambda t
+
\frac{\sigma^2}{4\lambda^3}
(2\lambda t -\sin 2\lambda t).
\\
\intertext{Die Varianz ist
$\operatorname{var}(Y_1(t))=E(Y_1(t)^2)-E(Y_1(t))^2$:}
\operatorname{var}(Y_1(t))
&=
\operatorname{var}(X_0)\cos^2\lambda t
+2\operatorname{cov}(X_0,X_1)\cos\lambda t\sin\lambda t
+
\frac{\sin^2\lambda t}{\lambda^2}\operatorname{var}(X_1)
+\frac{\sigma^2}{4\lambda^3}(2\lambda t-\sin2\lambda t).
\end{align*}
Wie im Beispiel der Langevin-Gleichung hat die Varianz
also zwei Komponenten.
Die ersten drei Komponenten beschreiben den Einfluss der Anfangsbedingungen.
Der letzte Term beschreibt die zus"atzliche Varianz, die durch
das Rauschen in den Oszillator eingebracht wird.

Der Term zu den Anfangsbedingungen kann in Matrixform geschrieben
werden:
\begin{gather*}
\operatorname{var}(X_0)\cos^2\lambda t
+2\operatorname{cov}(X_0,X_1)\cos\lambda t\sin\lambda t
+
\frac{\sin^2\lambda t}{\lambda^2}\operatorname{var}(X_1)
\qquad
\qquad
\qquad
\qquad
\\
\qquad
\qquad
\qquad
\qquad
=
\begin{pmatrix}
\cos\lambda t&\frac1\lambda\sin\lambda t
\end{pmatrix}
\begin{pmatrix}
\operatorname{var}(X_0)    &\operatorname{cov}(X_0,X_1)\\
\operatorname{cov}(X_0,X_1)&\operatorname{var}(X_1)
\end{pmatrix}
\begin{pmatrix}
\cos\lambda t\\
\frac1\lambda\sin\lambda t
\end{pmatrix}
\end{gather*}
Diese Form kann man auch erhalten, indem man die Kovarianzen mit Hilfe von
\[
\begin{pmatrix} Y_1(t)&Y_2(t) \end{pmatrix}
\begin{pmatrix}
Y_1(t)\\
Y_2(t)
\end{pmatrix}
=
\begin{pmatrix}
Y_1(t)^2&Y_1(t)Y_2(t)\\
Y_1(t)Y_2(t)&Y_2(t)^2
\end{pmatrix}
\]
berechnet, und darauf die Formel~(\ref{stochastisch:harmosz-y2}) anwendet.

%
% Partielle Differentialgleichungen
%
\section{Partielle Differentialgleichungen\label{section:pdgl}}
\rhead{Partielle Differentialgleichungen}
Die It\^osche Kettenregel stellt einen Zusammenhang her zwischen der zweiten
Ableitung von $u$ und den Werten von $u$ am Ende eines Pfades.
In diesem Abschnitt wollen wir den Zusammenhang wie folgt konkretisieren.
Wir werden zun"achst f"ur ein Gebiet $U$ und einen Punkt $x$ die
Zeit $\tau_x$ definieren, zu der eine Brownsche Bewegung das Gebiet zum
ersten Mal verl"asst.
Nat"urlich ist $\tau_x$ eine Zufallsvariable, und damit auch $u(X(\tau_x))$,
wobei der Pfad ist, der zur Zeit $\tau_x$ das Gebiet verl"asst.
Dann werden wir zeigen, dass $u(x)$ der Erwartungswert von $u(X(\tau_x))$
sein.

In der Theorie der quaslinearen partiellen Differentialgleichungen erster
Ordnung wird gezeigt, wie man die Funktionswerte der L"osungsfunktion
dadurch bestimmen kann, dass man vom Punkt $x$ aus die Charakteristik 
konstruiert, und den Wert der Randbedingung an dem Punkt ermittelt,
wo die Charakteristik das Gebiet verl"asst.
Aus diesem Wert l"asst sich dann der Wert der L"osung bestimmen.
Brownsche Bewegungen spielen also f"ur elliptische partielle
Differentialgleichungen eine "ahnliche Rolle wie Charakteristiken f"ur
quasliineare partielle Differentialgleichungen erster Ordnung.

\subsection{Stopzeiten}
Wie fr"uher betrachten wir wieder einen stochastischen Prozess $X(t)$,
der L"osung einer stochastischen Differentialgleichung
\begin{equation}
\begin{aligned}
dX(t)&=b(t,X)\,dt + B(t,X)\,dW
\\
X(0)&=X_0
\end{aligned}
\label{stochastisch:stopzeitdgl}
\end{equation}
sein soll.
\begin{figure}
\centering
\includegraphics{chapters/images/stochastisch-2.pdf}
\caption{Brownsche Bewegung in zwei Dimension und Definition der
Stopzeit $\tau$, zu der der Pfad $X(t)$ das Gebiet verl"asst.
\label{stochastisch:pfad}}
\end{figure}

\begin{definition}
Sei $E$ eine beliebige offene oder abgeschlossene nicht leere Teilmenge
von $\mathbb R^n$.
Dann setzen wir
\[
\tau = \inf\{t\ge 0\;|\;X(t)\in E\},
\]
$\tau$ ist also die fr"uheste Zeit, zu der der Weg $X(t)$ die Menge $E$
erreicht.
\end{definition}

Die charakteristische Funktion
\[
\chi_{[0,\tau]}(t)=\begin{cases}
1\qquad\qquad&t\le \tau\\
0            &t>\tau
\end{cases}
\]
ist nat"urlich auch ein stochastischer Prozess, und damit auch 
$\chi_{[0,\tau]}G$.
Damit gelten die Regeln f"ur das It\^o-Integral aus
Satz~\ref{satz:ito-integral} auch f"ur diesen Prozess, jetzt allerdings
als Integrale mit oberer Grenze $\tau$ statt $T$.

\begin{hilfssatz}
Seien $G$ und $H$ auf $[0,T]$ quadratintegrierbare stochastische Prozesse
und $a,b\in\mathbb R$
\begin{compactenum}
\item
Das It\^o-Integral ist linear:
\begin{align*}
\int_0^\tau aG+bH\,dW
&=
\int_0^T a\chi_{[0,\tau]} G + b \chi_{[0,\tau]} H\,dW
=
a\int_0^T \chi_{[0,\tau]} G\,dW + b \int_0^T\chi_{[0,\tau]} H\,dW
\\
&=
a\int_0^\tau G\,dW + b \int_0^\tau H\,dW
\end{align*}
\item Der Erwartungswert des It\^o-Integrals ist
\[
E\biggl(\int_0^\tau G\,dW\biggr)
=
E\biggl(\int_0^T \chi_{[0,\tau]}G\,dW\biggr)
=
0.
\]
\item Die Varianz des It\^o-Integrals ist
\[
E\biggl(\biggl(\int_0^\tau G\,dW\biggr)^2\biggr)
=
E\biggl(\biggl(\int_0^T \chi_{[0,\tau]}G\,dW\biggr)^2\biggr)
=
E\biggl(\int_0^T\chi_{[0,\tau]}G^2\,dt\biggr)
=
E\biggl(\int_0^\tau G^2\,dt\biggr).
\]
\end{compactenum}
\end{hilfssatz}

Die It\^o-sche Kettenregel funktioniert auch f"ur $\tau$ als obere
Grenze f"ur die stochastischen Integrale.
Wenn also $X$ als stochastischer Prozess eine L"osung der stochastischen
Differentialgleichung (\ref{stochastisch:stopzeitdgl}) ist, dann gilt
nach der der It\^o-schen Kettenregel
\[
du(X,t)
=
\frac{\partial u}{\partial t}\,dt
+
\sum_{i=1}^n\frac{\partial u}{\partial x_i}\,dX_i
+
\frac12\sum_{i,j=1}^n\frac{\partial^2 u}{\partial x_i\partial x_j}
\sum_{k=1}^n b_{ik}b_{jk}\,dt.
\]
In integrierter Form bedeutet dies
\[
u(X(t),t)-u(X(0),0)
=
\int_0^t
\frac{\partial u}{\partial t}
+
\frac12\sum_{k=1}^nb_{ik}b_{jk}
\sum_{i,j=1}^n \frac{\partial^2u}{\partial x_i\,\partial x_j} \,ds
+
\int_0^t \operatorname{grad} u\cdot B\,dW.
\]
Und nat"urlich gelten diese Formeln auch dann, wenn man $t$ durch $\tau$
ersetzt.

Im Folgenden interessiert uns nur der Fall $b=0$, $b_{ik}=\delta_{ik}$
und Funktionen $u$, die nicht von der Zeit abh"angen.
Dann vereinfacht sich die Formel zu
\begin{equation}
u(X(t))-u(X(0))
=
\int_0^t \frac12\sum_{i=1}^n\frac{\partial^2 u}{\partial x_i^2}\,ds
=
\int_0^t \frac12\Delta u\,ds
\label{stochastisch:laplaceinkrement}
\end{equation}
Ausserdem ist in diesem Fall $X$ nichts anderes als eine Brownsche Bewegung,
$X=W$.

\subsection{Brownsche Bewegung und der Laplace-Operator}
Wir wenden die Formel (\ref{stochastisch:laplaceinkrement}) jetzt in
zwei Beispielen an.

\subsubsection{Zeit bis zum Verlassen eines Gebietes}
F"ur das erste Beispiel sei $U$ ein beschr"anktes Gebiet in
$\mathbb R^n$, und $u$ eine L"osung der partiellen Differentialgleichung
\begin{equation}
\begin{aligned}
-\frac12\Delta u&=1&\qquad&\text{in $U$}\\
               u&=0&      &\text{auf $\partial U$}
\end{aligned}
\label{stochastisch:hittingtime}
\end{equation}
Wir m"ochten die L"osung $u$ dazu verwenden, die Zeit $\tau_x$ zu berechnen,
zu der eine im Punkt $x\in U$ beginnende Brownsche Bewegung zum ersten
Mal das Gebiet $U$ verl"asst.
Die Formel (\ref{stochastische:laplaceinkrement}) liefert
\[
u(X(\tau_x))-u(X(0)) = \int_0^{\tau_x} \frac12\Delta u\,ds
\]
Da $u$ eine L"osung von (\ref{stochastisch:hittingtime}) ist, ist der 
Integrand auf der rechten Seite gleich $-1$:
\[
u(X(\tau_x))-u(X(0)) = -\int_0^{\tau_x} \,ds
\]
Da der Prozess $X$ zur Zeit $\tau_x$ den Rand des Gebietes "uberquert,
ist wegen der Randbedingung $u(X(\tau_x))=0$. 
Zusammen erhalten wir
\[
-u(X(0)) = -\tau_x.
\]
Nun interessiert uns aber nicht der Wert der Zufallsvariablen, sondern
nur deren Erwartungswert:
\[
E(\tau_x)=E(u(X(0)))=u(x).
\]
Die L"osung $u$ der Differentialgleichung (\ref{stochastisch:hittingtime})
gibt die erwartete Zeit an, bis eine bei $x$ beginnende Brownsche Bewegung 
das Gebiet $U$ verlassen hat.

\subsubsection{Charakterisierung von harmonischen Funktionen}
Sei wieder $U$ ein beschr"anktes Gebiet mit glattem Rand und
$g$ eine stetige Funktion auf $\partial U$.
Ausserdem sei $u$ eine harmonische Funktion in $U$ mit den Randwerten $g$, 
also eine L"osung der partiellen Differentialgleichung
\begin{equation}
\begin{aligned}
\Delta u&=0&\qquad&\text{in $U$}\\
       u&=g&      &\text{auf $\partial U$}
\end{aligned}
\label{stochastisch:harmonisch}
\end{equation}
Wir wollen (\ref{stochastisch:laplaceinkrement}) verwenden, die Funktion
$u$ zu charakterisieren.
Dazu sei wieder $\tau_x$ die Zeit, zu der eine im Punkt $x\in U$ beginnende
Brownsche Bewegung das Gebiet $U$ verl"asst.
Aus (\ref{stochastisch:laplaceinkrement}) folgt:
\[
u(X(\tau_x))-u(X(0))
=
\int_0^{\tau_x} \frac12\underbrace{\Delta u}_{\textstyle=0}\,ds=0.
\]
Zur Zeit $\tau_x$ ist $X(\tau_x)$ ein Randpunkt des Gebiets, der mit
der Randbedingung bestimmt werden kann, es folgt:
\[
u(X(0)) = g(X(\tau_x)).
\]
Wieder interessiert uns der einzelne Wert nicht, sondern der Erwartungswert:
\[
u(x)=E(g(X(\tau_x))).
\]
Den Wert im Punkt $x$ einer harmonischen Funktion mit Randwerten $g$
kann man wie folgt finden: man l"asst eine Brownsche Bewegung von $x$ 
aus laufen, bis sie den Rand "uberquert, und nimmt den Mittelwert
der derart erreichten Randwerte.

Aus dieser Charakterisierung der harmonischen Funktionen kann man
auch deren Mittelwerteigenschaft ableiten. 
Da die Brownsche Bewegung isotrop ist, muss sich im Zentrum
eines kugelf"ormigen Gebietes immer der gleiche Wert f"ur $u$
ergeben, selbst wenn man eine beliebige Drehung auf die Randwerte 
anwendet.
Also muss der Wert von $u(x)$ der Mittelwert der Werte
auf einer Kugel um den Punkt $x$ sein.

%
%
%
\section{Kalman-Filter\label{section:kalman}}
\rhead{Kalman-Filter}






\begin{appendices}
\chapter{Newton-Verfahren}
\lhead{Newton-Verfahren}
\rhead{}

\section{Nullstellen von Funktionen}
\rhead{Nullstellen von Funktionen}
\begin{figure}
\centering
\includegraphics{chapters/images/randwert-2.pdf}
\caption{Bestimmung der Nullstelle einer Funktion $f(x)$ mit dem
Newton-Verfahren.
Die Approximation $x_{n+1}$ wird gefunden als Schnittpunkt der Tangente
im Punkt $(x_n,f(x_n))$ (mit Steigung $f'(x_n)$) mit der $x$-Achse.
\label{newton:graphik}}
\end{figure}
Das Ziel dieses Anhangs ist, die folgende Aufgabe numerisch zu l"osen:
\begin{aufgabe}
Gegen ist eine differenzierbar Funktion
$f\colon\mathbb R\to\mathbb R:x\mapsto f(x)$
und eine Zahl $y$ im Wertebereich von $f$.
Finde $\hat{x}\in\mathbb R$ so, dass $f(\hat{x})=y$.
\end{aufgabe}
Im allgemeinen kann man nicht davon ausgehen, dass sich eine L"osung der
Gleichung $f(x)=y$ in geschlossener Form finden l"asst.
Nur einige wenige Klassen von Gleichungen haben L"osungsformeln dieser Art.
Wir beschr"anken uns daher auf das Problem, eine Approximation f"ur die
L"osung zu bestimmen.

Indem wir statt der Funktion $f(x)$ die Funktion $x\mapsto g(x)=f(x)-y$
betrachten, k"onnen wir die gesuchte Zahl $x$ auch als L"osung der
Gleichung $g(x)=0$ finden:
\begin{equation}
f(x)=y
\qquad\qquad
\Rightarrow
\qquad\qquad
g(x)=f(x)-y = 0.
\label{newton:reduktion}
\end{equation}
Es gen"ugt also, ein L"osungsverfahren zu entwickeln f"ur die Aufgabe
\begin{aufgabe}
Gegen ist eine differenzierbar Funktion
$f\colon\mathbb R\to\mathbb R:x\mapsto f(x)$,
finde $\hat{x}\in\mathbb R$ so, dass $f(\hat{x})=0$.
\end{aufgabe}
Da wir nur eine numerische L"osung brauchen, versuchen wir sie dadurch
zu finden, dass wir eine Anfangssch"atzung $x_0$ wiederholt korrigieren,
bis der Fehler klein genug ist.
Es soll also eine Folge $x_0,x_1,x_2,\dots$ konstruiert werden, welche
gegen die L"osung $\hat{x}$ konvergiert.
Der Differenzenquotient ist eine Approximation f"ur die Steigung
$f'(x_n)$,
\begin{equation}
\frac{
f(x_{\mathstrut n+1})-f(x_{\mathstrut n})
}{
x_{\mathstrut n+1}-x_{\mathstrut n}
}
\simeq = f'(x_n).
\label{newton:pre}
\end{equation}
Wir m"ochten gerne, dass $f(x_{n+1})=0$ ist, und k"onnen (\ref{newton:pre})
unter dieser Annahme nach $x_{n+1}$ aufl"osen:
\begin{align*}
-f(x_n)
&\simeq
f'(x_n)\,(x_{\mathstrut n+1}-x_{\mathstrut n})
\\
x_{\mathstrut n}-\frac{f(x_n)}{f'(x_n)}
&\simeq x_{\mathstrut n+1}
\end{align*}
Damit haben wir ein L"osungsverfahren gefunden:
\begin{satz}[Newton]
Ist $f$ eine differenzierbare Funktion, deren Ableitung bei der Nullstelle
$\hat{x}$ nicht verschwindet, also $f(\hat{x})\ne 0$, und $x_0$ eine erste
Approximation f"ur $\hat{x}$, dann konvergiert die Folge
definiert durch die Rekursionsformel
\[
x_{n+1}=x_n-\frac{f(x_n)}{f'(x_n)},
\]
dann konvergiert $x_n$ gegen $\hat{x}$, falls $x_0$ nahe genug bei
$\hat{x}$ liegt.
\end{satz}

\begin{beispiel}
Man finde die Wurzel der Zahl $y$, d.~h.~man muss die Nullstellen
der Funktion $f(x)=x^2-y$ finden.
Das Newton-Verfahren ben"otigt die Ableitung von $f$, sie ist
$f'(x)=2x$, und konstruiert daraus die Folge
\begin{equation}
x_{n+1} = x_n - \frac{f(x_n)}{f'(x_n)}=x_n-\frac{x_n^2-y}{2x_n}
=
\frac{2x_n^2-x_n^2+y}{2x_n}
=
\frac12\biggl(x_n + \frac{y}{x_n}\biggr).
\label{newton:mittel}
\end{equation}
Die Quaratwurzel von $y$ erf"ullt nat"urlich
\[
\sqrt{y} = \frac12\biggl( \sqrt{y}+\frac{y}{\sqrt{y}}\biggr).
\]
Mit $x_n$ ist auch $y/x_n$ eine Approximation von $\sqrt{y}$.
Die neue Approximation $x_{n+1}$ ist das arithmetische Mittel der
beiden Approximationen $x_n$ und $y/x_n$ von $\sqrt{y}$.
Die Konvergenz dieser Folge ist sehr schnell, wie Tabelle~\ref{newton:sqrt2}
zeigt.
\begin{table}
\centering
\begin{tabular}{|>{$}r<{$}|>{$}r<{$}|}
\hline
n&x_n\\
\hline
0 &  2.00000000000000\\
1 &  1.50000000000000\\
2 &  1.\underline{41}666666666667\\
3 &  1.\underline{41421}568627451\\
4 &  1.\underline{41421356237}469\\
5 &  1.\underline{41421356237309}\\
6 &  1.\underline{41421356237309}\\
\hline
\end{tabular}
\caption{Approxmationen von $\sqrt{2}$ mit Hilfe des Newton-Algorithmus,
korrekte Stellen unterstrichen.
Die Anzahl korrekter Stellen verdoppelt sich in jedem Schritt.
\label{newton:sqrt2}}
\end{table}
In jedem Schritt verdoppelt sich die Anzahl korrekter Stellen.
Dies ist eine allgemeine Eigenschaft des Newton-Algorithmus, wie
in Abschnitt~\ref{section:newton:konvergenz} erkl"art wird.
\end{beispiel}

\section{Konvergenzgeschwindigkeit\label{section:newton:konvergenz}}

\section{L"osung von Vektorgleichungen\label{section:newton:vektor}}
\rhead{L"osung von Vektorgleichungen}
Wir m"ochten das Verfahren nun erweitern, so dass wir nicht nur eine
einzige Gleichung $f(x)=y$ nach $x$ aufl"osen k"onnen, wir m"ochten dazu 
f"ur ein Gleichungssystem von nichtlinearen Gleichungen
\begin{align*}
f_1(x_1,\dots,x_n)&=y_1\\
f_2(x_1,\dots,x_n)&=y_2\\
&\;\;\vdots\\
f_m(x_1,\dots,x_n)&=y_m
\end{align*}
ebenfalls in der Lage sein.
Wie bei einer einzigen Gleichung k"onnen wir das Problem reduzieren
auf das Finden von gleichzeitigen Nullstellen der Funktionen $g_i$ mit
\begin{align*}
g_1(x_1,\dots,x_n)&=f_1(x_1,\dots,x_n)-y_1=0\\
g_2(x_1,\dots,x_n)&=f_2(x_1,\dots,x_n)-y_2=0\\
&\;\;\vdots\\
g_m(x_1,\dots,x_n)&=f_m(x_1,\dots,x_n)-y_m=0.
\end{align*}
Wir k"onnen dies auch in Vektor-Form schreiben,
wir betrachten $x$ als Vektor der $x_1,\dots,x_n$









\input{chapters/komplexezahlen.tex}
\end{appendices}
\vfill
\pagebreak
\ifodd\value{page}\else\null\clearpage\fi
\lhead{Literatur}
\rhead{}
\printbibliography[heading=subbibliography]
\label{skript:literatur}
\end{refsection}

\part{Anwendungen und Weiterf"uhrende Themen}
\lhead{Anwendungen}
%
% uebersicht.tex -- Uebersicht ueber die Seminar-Arbeiten
%
% (c) 2015 Prof Dr Andreas Mueller, Hochschule Rapperswil
%
\chapter*{"Ubersicht}
\lhead{"Ubersicht}
\rhead{}
\label{skript:uebersicht}
Im zweiten Teil kommen die Teilnehmer des Seminars selbst zu Wort.
Sie zeigen Anwendungsbeispiele f"ur die im ersten
Teil entwickelte Theorie der gew"ohnlichen Differentialgleichungen.



\def\chapterauthor#1{{\large #1}\bigskip\bigskip}
% Artikel
\chapter{Wo steckt die zweite L"osung?\label{chapter:komplex}}
\lhead{Bessel-Funktionen zweiter Art}
\begin{refsection}
\chapterauthor{Stefan Kull und Roy Seitz}

\printbibliography[heading=subbibliography]
\end{refsection}

\chapter{Wo steckt die zweite L"osung?\label{chapter:komplex}}
\lhead{Bessel-Funktionen zweiter Art}
\begin{refsection}
\chapterauthor{Stefan Kull und Roy Seitz}

\printbibliography[heading=subbibliography]
\end{refsection}

\chapter{Wo steckt die zweite L"osung?\label{chapter:komplex}}
\lhead{Bessel-Funktionen zweiter Art}
\begin{refsection}
\chapterauthor{Stefan Kull und Roy Seitz}

\printbibliography[heading=subbibliography]
\end{refsection}

\chapter{Wo steckt die zweite L"osung?\label{chapter:komplex}}
\lhead{Bessel-Funktionen zweiter Art}
\begin{refsection}
\chapterauthor{Stefan Kull und Roy Seitz}

\printbibliography[heading=subbibliography]
\end{refsection}

\chapter{Wo steckt die zweite L"osung?\label{chapter:komplex}}
\lhead{Bessel-Funktionen zweiter Art}
\begin{refsection}
\chapterauthor{Stefan Kull und Roy Seitz}

\printbibliography[heading=subbibliography]
\end{refsection}

\chapter{Wo steckt die zweite L"osung?\label{chapter:komplex}}
\lhead{Bessel-Funktionen zweiter Art}
\begin{refsection}
\chapterauthor{Stefan Kull und Roy Seitz}

\printbibliography[heading=subbibliography]
\end{refsection}

\chapter{Wo steckt die zweite L"osung?\label{chapter:komplex}}
\lhead{Bessel-Funktionen zweiter Art}
\begin{refsection}
\chapterauthor{Stefan Kull und Roy Seitz}

\printbibliography[heading=subbibliography]
\end{refsection}

\chapter{Wo steckt die zweite L"osung?\label{chapter:komplex}}
\lhead{Bessel-Funktionen zweiter Art}
\begin{refsection}
\chapterauthor{Stefan Kull und Roy Seitz}

\printbibliography[heading=subbibliography]
\end{refsection}

\vfill
\pagebreak
\ifodd\value{page}\else\null\clearpage\fi
\lhead{Index}
\rhead{}
%
% skript.tex -- Skript ueber Differentialgleichungen
%
% (c) 2014 Prof. Dr. Andreas Mueller, HSR
%
\documentclass{book}
\usepackage{etex}
\usepackage{geometry}
\geometry{papersize={170mm,240mm},total={140mm,200mm},top=21mm,bindingoffset=10mm}
\usepackage[english,ngerman]{babel}
\usepackage{times}
\usepackage{amsmath,amscd}
\usepackage{amssymb}
\usepackage{amsfonts}
\usepackage{amsthm}
\usepackage{graphicx}
\usepackage{fancyhdr}
\usepackage{textcomp}
\usepackage[all]{xy}
\usepackage{txfonts}
\usepackage{alltt} 
\usepackage{verbatim}
\usepackage{paralist}
\usepackage{makeidx}
\usepackage{array}
\usepackage[colorlinks=true]{hyperref}
\usepackage{tikz}
\usepackage{pgfplots}
\usepackage{pgfplotstable}
\usepackage{pdftexcmds}
%\usepackage{pgfmath}
\usepackage{placeins}
\usepackage{subfigure}
\usepackage[autostyle=false,english=american]{csquotes}
\usepackage{float}
\usepackage{enumitem}
\usepackage{wasysym}
\usepackage{environ}
\usepackage{pifont}
\usepackage{feynmp}
\usepackage{appendix}
\usetikzlibrary{calc,intersections,through,backgrounds,graphs,positioning,shapes,arrows,fit}
\usetikzlibrary{patterns,decorations.pathreplacing}
\usetikzlibrary{decorations.pathreplacing}
\usetikzlibrary{external}
\usepackage[europeanvoltages,
            europeancurrents,
            europeanresistors,   % rectangular shape
            americaninductors,   % "4-bumbs" shape
            europeanports,       % rectangular logic ports
            siunitx,             % #1<#2>
            emptydiodes,
            noarrowmos,
            smartlabels]         % lables are rotated in a smart way
           {circuitikz}          %
\usepackage{siunitx}
\usepackage{tabularx}
\usetikzlibrary{arrows}

\usepackage{algpseudocode}
\usepackage{algorithm}

\usepackage{listings}
\lstdefinestyle{Matlab}{
  numbers=left,
  belowcaptionskip=1\baselineskip,
  breaklines=true,
  frame=L,
  xleftmargin=\parindent,
  language=Matlab,
  showstringspaces=false,
  basicstyle=\footnotesize\ttfamily,
  keywordstyle=\bfseries\color{green!40!black},
  commentstyle=\itshape\color{purple!40!black},
  identifierstyle=\color{blue},
  stringstyle=\color{orange},
  numberstyle=\ttfamily\tiny
}
\lstdefinelanguage{Maxima}{
  keywords={addrow,addcol,zeromatrix,ident,augcoefmatrix,ratsubst,diff,ev,tex,%
    with_stdout,nouns,express,depends,load,submatrix,div,grad,curl,matrix,%
    invert,lambda,facsum,expand,false,then,if,else,subst,%
    rootscontract,solve,part,assume,sqrt,integrate,abs,inf,exp,sin,cos,sinh,cosh},
  sensitive=true,
  comment=[n][\itshape]{/*}{*/}
}
\lstdefinestyle{Maxima}{
  numbers=left,
  belowcaptionskip=1\baselineskip,
  breaklines=true,
  frame=L,
  xleftmargin=\parindent,
  language=Maxima,
  showstringspaces=false,
  basicstyle=\footnotesize\ttfamily,
  keywordstyle=\bfseries\color{green!40!black},
  commentstyle=\itshape\color{purple!40!black},
  identifierstyle=\color{blue},
  stringstyle=\color{orange},
  numberstyle=\ttfamily\tiny
}
\usepackage{caption}
\usepackage[mode=buildnew]{standalone}
\usepackage[backend=bibtex]{biblatex}
\addbibresource{references.bib}
% Bibresources für jeden einzelnen Artikel
%\addbibresource{thema/main.bib}
\AtEndDocument{\clearpage\ifodd\value{page}\else\null\clearpage\fi}
\makeindex
%\pgfplotsset{compat=1.12}
\setlength{\headheight}{15pt} % fix headheight warning
\DeclareGraphicsRule{*}{mps}{*}{}
\begin{document}
\pagestyle{fancy}
\frontmatter
\newcommand\HRule{\noindent\rule{\linewidth}{1.5pt}}
\begin{titlepage}
\vspace*{\stretch{1}}
\HRule
\vspace*{5pt}
\begin{flushright}
{
\LARGE
Mathematisches Seminar\\
\vspace*{20pt}
\Huge
Differentialgleichungen%
}
\vspace*{5pt}
\end{flushright}
\HRule
\begin{flushright}
\vspace{60pt}
\Large
Leitung: Andreas M"uller\\
\vspace{40pt}
\Large
Reto~Christen,
Kevin~Cina,
Andri~Hartmann,
Pascal~Horat %,
Matthias~Kn"opfel,
Stefan Kull,
Daniela~Meier,
Max~Obrist %,
Hansruedi~Patzen,
Benjamin~R"aber,
Simon~Schaefer %,
Tibor~Schneider,
Tobias~Schuler,
Roy~Seitz,
Martin~Stypinski
\end{flushright}
\vspace*{\stretch{2}}
\begin{center}
Hochschule f"ur Technik, Rapperswil, 2016
\end{center}
\end{titlepage}
\hypersetup{
    linktoc=all,
    linkcolor=blue
}
\newcounter{beispiel}
\newenvironment{beispiele}{
\bgroup\smallskip\parindent0pt\bf Beispiele\egroup

\begin{list}{\arabic{beispiel}.}
  {\usecounter{beispiel}
  \setlength{\labelsep}{5mm}
  \setlength{\rightmargin}{0pt}
}}{\end{list}}
\newcounter{uebungsaufgabe}
% environment fuer uebungsaufgaben
\newenvironment{uebungsaufgaben}{
\begin{list}{\arabic{uebungsaufgabe}.}
  {\usecounter{uebungsaufgabe}
  \setlength{\labelwidth}{2cm}
  \setlength{\leftmargin}{0pt}
  \setlength{\labelsep}{5mm}
  \setlength{\rightmargin}{0pt}
  \setlength{\itemindent}{0pt}
}}{\end{list}\vfill\pagebreak}
\newenvironment{teilaufgaben}{
\begin{enumerate}
\renewcommand{\labelenumi}{\alph{enumi})}
}{\end{enumerate}}
% Loesung
\def\swallow#1{
%nothing
}
\NewEnviron{loesung}[1][L"osung]{%
\begin{proof}[#1]%
\renewcommand{\qedsymbol}{$\bigcirc$}
\BODY
\end{proof}
}
\NewEnviron{bewertung}{%
\begin{proof}[Bewertung]%
\renewcommand{\qedsymbol}{}
\BODY
\end{proof}
}
\NewEnviron{diskussion}{%
\begin{proof}[Diskussion]%
\renewcommand{\qedsymbol}{}
\BODY
\end{proof}
}
\NewEnviron{hinweis}{%
\begin{proof}[Hinweis]%
\renewcommand{\qedsymbol}{}
\BODY
\end{proof}
}
\def\keineloesungen{%
\RenewEnviron{loesung}{\relax}
\RenewEnviron{bewertung}{\relax}
\RenewEnviron{diskussion}{\relax}
}
\newenvironment{beispiel}{%
\begin{proof}[Beispiel]%
\renewcommand{\qedsymbol}{$\bigcirc$}
}{\end{proof}}

\input{linsys.tex}
\allowdisplaybreaks

\lhead{Inhaltsverzeichnis}
\rhead{}
\tableofcontents
\newtheorem{satz}{Satz}[chapter]
\newtheorem{hilfssatz}[satz]{Hilfssatz}
\newtheorem{definition}[satz]{Definition}
\newtheorem{annahme}[satz]{Annahme}
\renewcommand{\floatpagefraction}{0.75}
\mainmatter
\input{vorwort.tex}
\part{Grundlagen}
%\keineloesungen
\begin{refsection}
\input{chapters/einleitung.tex}
%\input{kapitel.tex}
\input{chapters/grundlagen.tex}
\input{chapters/numerik.tex}
\input{chapters/potenzreihen.tex}
\input{chapters/linear.tex}
\input{chapters/geometrie.tex}
\input{chapters/komplex.tex}
\input{chapters/stabilitaet.tex}
\input{chapters/chaos.tex}
\input{chapters/stochastisch.tex}
\begin{appendices}
\input{chapters/newton.tex}
\input{chapters/komplexezahlen.tex}
\end{appendices}
\vfill
\pagebreak
\ifodd\value{page}\else\null\clearpage\fi
\lhead{Literatur}
\rhead{}
\printbibliography[heading=subbibliography]
\label{skript:literatur}
\end{refsection}

\part{Anwendungen und Weiterf"uhrende Themen}
\lhead{Anwendungen}
\input{uebersicht.tex}
\def\chapterauthor#1{{\large #1}\bigskip\bigskip}
% Artikel
\input{wellen/main.tex}
\input{kreis/main.tex}
\input{schrittlaenge/main.tex}
\input{licht/main.tex}
\input{sir/main.tex}
\input{friedmann/main.tex}
\input{komplex/main.tex}
\input{trafo/main.tex}
\vfill
\pagebreak
\ifodd\value{page}\else\null\clearpage\fi
\lhead{Index}
\rhead{}
\input{skript.ind}

\end{document}


\end{document}


\end{document}


\end{document}
