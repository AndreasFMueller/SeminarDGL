%
% vorwort.tex -- Vorwort zum Buch zum Seminar
%
% (c) 2015 Prof Dr Andreas Mueller, Hochschule Rapperswil
%
\chapter*{Vorwort}
\lhead{Vorwort}
\rhead{}
Dieses Buch entstand im Rahmen des Mathematischen Seminars
im Fr"uhjahrssemester 2016 an der Hochschule f"ur Technik Rapperswil.
Die Teilnehmer, Studierende der Abteilungen f"ur Elektrotechnik,
Informatik und Bauingenieurwesen der
HSR, erarbeiteten nach einer Einf"uhrung in das Themengebiet jeweils
einzelne Aspekte des Gebietes in Form einer Seminararbeit, "uber
deren Resultate sie auch in einem Vortrag informierten. 

Im Fr"uhjahr 2016 war das Thema des Seminars ``Differentialgleichungen''.
Die Einf"uhrung bestand aus einigen Vorlesungsstunden, deren
Inhalt im ersten Teil dieses Skripts zusammengefasst ist.
Es ging darum, die zum Teil aus dem Analysis-Unterricht bekannte
Theorie der Differentialgleichungen zu vertiefen, mit anderen Gebieten
wie zum Beispiel der komplexen Analysis zu verkn"upfen und sie
auf die Analyse einiger relevanter Praxisprobleme anzuwenden.
Dabei ging es nicht um die analytische L"osung von Differentialgleichungen,
die meisten Differentialgleichungen lassen sich ohnehin nicht in
geschlossener Form l"osen.
Einzelne Differentialgleichungen wurden untersucht, weil sie Anlass
zu einer wichtigen Familie von Funktionen geben, zum Beispiel die
Bessel- und Airy-Funktionen.
In anderen Beispielen ging es um die Schwierigkeiten, die bei einer
numerischen L"osung zu meistern sind.
Besonders anspruchsvoll sind jedoch "Uberlegungen zum Verhalten der
L"osung f"ur lange Zeiten, zum Beispiel Stabilit"at, das Auftreten
von Schwingungen bei der Hopf-Bifurkation oder der "Ubergang zum
Chaos.

Im zweiten Teil dieses Skripts kommen dann die Teilnehmer selbst zu Wort.
Ihre Arbeiten wurden jeweils als einzelne
Kapitel mit meist nur typographischen "Anderungen "ubernommen.
Diese weiterf"uhrenden Kapitel sind sehr verschiedenartig.
Eine "Ubersicht und Einf"uhrung befindet sich in der Einleitung
zum zweiten Teil auf Seite~\pageref{skript:uebersicht}.

In einigen Arbeiten wurde auch Code zur Demonstration der 
besprochenen Methoden und Resultate geschrieben, soweit
m"oglich und sinnvoll wurde dieser Code im Github-Repository
dieses Kurses\footnote{\url{https://github.com/AndreasFMueller/SeminarDGL.git}}
abgelegt, in anderen F"allen verweisen die Artikel selbst auf
das zugeh"orige Code-Repository.

Im genannten Repository findet sich auch der Source-Code dieses
Skriptes, es wird hier unter einer Creative Commons Lizenz
zur Verf"ugung gestellt.

