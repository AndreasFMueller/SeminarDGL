\section{Diagonalisierbarer Fall}
%Matrix Diagonalisierbar
Ist die Matrix A diagonalisierbar, gilt:
$$\sk w_j = \lambda_j w_j\, ,j=1,2$$
Damit die Funktion als Laurent-Reihe dargestellt werden kann, muss der Faktor $\lambda_j$ kompensiert werden. Betrachten wir folgende Zahl
$$\varrho_j = \frac{1}{2\pi i}\log(\lambda_j)$$
und bilden damit die Funktion
$$
z^{\varrho_j}.
$$
Um genauer zu verstehen, was mit dieser Funktion bei der analytischen Fortsetzung geschieht, soll die äquivalente Form
$$
z^{\varrho_j} = e^{\log(z^{\varrho_j})} = e^{\varrho_j \log(z)}
$$
betrachtet werden. Diese Funktion wird nun analytisch fortgesetzt. Die Exponentialfunktion geht in sich selbst über, der Logarithmus transformiert sich wie früher erläutert in $\sk\log(z) = \log(z) + 2\pi i$, und es folgt
$$
\sk e^{\varrho_j \log(z)} 
= e^{\varrho_j (\log(z) + 2\pi i)}
= e^{\varrho_j \log(z)}  e^{2\pi i\varrho_j}
= z^{\varrho_j}  e^{2\pi i\frac{1}{2\pi i}\log(\lambda_j)} 
= z^{\varrho_j}  \lambda_j
$$
Die Funktion $z^{\varrho_j}$ wird also beim Umfahren der Singularit"at ebenfalls mit dem Faktor $\lambda_j$ multipliziert. Nun wird das Verh"altnis $\frac{w_j}{z^{\varrho_j}}$ untersucht:
$$
\sk \frac{w_j}{z^{\varrho_j}} 
= \frac{\lambda_j w_j}{\lambda_jz^{\varrho_j}} = \frac{w_j}{z^{\varrho_j}}
$$
Durch die Funktion $z^\varrho_j$ ist es also gelungen, eine eindeutige Funktion zu finden, respektive einen konstanten Faktoren zu kompensieren. Die Funktion $\frac{w_j}{z^{\varrho_j}}$ wird durch die analytische Fortsetzung nicht verändert und kann als Laurent-Reihe dargestellt werden.
$$
\frac{w_j}{z^{\varrho_j}} = \mathcal{L}_j(z)
$$
Daraus folgt:
$$
\begin{matrix}
w_1 = z^{\varrho_1} \mathcal{L}_1(z) \\
w_2 = z^{\varrho_2} \mathcal{L}_2(z)
\end{matrix}
$$
Ist also die Matrix A diagonalisierbar, k"onnen zwei L"osungen dieser Form gefunden werden.

\section{Nicht diagonalisierbarer Fall}
%Matrix nicht diagonalisierbar
Ist die Matrix A nicht diagonalisierbar, gilt:
$$
\begin{array}{l}
\sk w_1 = \lambda w_1 \\
\sk w_2 = \lambda w_2 + w_1
\end{array}
$$
Die Gleichung für $w_1$ ist also identisch mit derjenigen aus dem diagonalisierbaren Fall, und auch die erste Lösung lässt sich in gleicher Weise finden. Für die zweite Lösung funktioniert das nicht. Nun betrachten wir das Verh"altnis $\frac{w_2}{w_1}$
$$
\sk \frac{w_2}{w_1} = \frac{\lambda w_2+w_1}{\lambda w_1} = \frac{w_2}{w_1} + \frac{1}{\lambda}
$$
Hier muss anstelle eines Faktors ein Summand kompensiert werden. Gemacht wird dies mit der Funktion
$$ \frac{1}{2\pi i\lambda} \log(z)$$
$$\sk \frac{1}{2\pi i\lambda} \log(z) 
= \frac{1}{2\pi i\lambda} (\log(z) + 2\pi i) 
=\frac{1}{2\pi i\lambda} \log(z)+  \frac{1}{\lambda}
$$
Wird die Differenz des Verhältnisses $\frac{w_2}{w_1}$ und dieser Funktionen gebildet, folgt daraus
$$
\sk\left( \frac{w_2}{w_1} - \frac{1}{2\pi i\lambda} \log(z) \right)
= \frac{w_2}{w_1} +\frac{1}{\lambda} - \frac{1}{2\pi i\lambda} \log(z) - \frac{1}{\lambda} 
= \frac{w_2}{w_1} - \frac{1}{2\pi i\lambda} \log(z).
$$
Dieser Term wird durch die analytische Fortsetzung nicht verändert und kann folglich als Laurent-Reihe dargestellt werden. Mit $a=\frac{1}{2\pi i\lambda}$ folgt
$$\frac{w_2}{w_1} - a\log(z) =\mathcal{L}_2'(z)$$
Um die Lösung $w_2$ zu erhalten, muss lediglich umgeformt werden
$$
w_2 = \mathcal{L}_2'(z) w_1 +a w_1 \log(z) 
= \mathcal{L}_2'(z)\mathcal{L}_1(z)z^\varrho + a w_1 \log(z)
$$
Eine letzte Vereinfachung ergibt sich aus der Multiplikation der Laurent-Reihen $\mathcal{L}_1$ und $\mathcal{L}_2'$. Die Multiplikation zweier Potenzreihen führt wieder auf eine Potenzreihe, also $\mathcal{L}_1 \mathcal{L}_2'=\mathcal{L}_2$ und folglich
$$
w_2 = z^\varrho\mathcal{L}_2(z) + a w_1 \log(z)
$$
Die zweite Lösung findet sich also als Kombination einer Laurent-Reihe multipliziert mit $z^\varrho$ und der ersten Lösung mal den Logarithmus.