%Matrix Diagonalisierbar
Ist die Matrix A diagonalisierbar, gilt:
$$
\begin{matrix}
w_1 \, \sk \, \lambda_1 w_1  \\
w_2 \, \sk \, \lambda_2 w_2
\end{matrix}
$$
Der Faktor $\lambda_{1,2}$ muss kompensiert werden, damit die L"osung als Laurent-Reihe dargestellt werden kann. Dazu wird folgende Funktion verwendet: 
$$
(z-z_0)^{\varrho_{1,2}} \, \text{mit} \, \varrho_{1,2} = \frac{1}{2\pi i}log(\lambda_{1,2})
$$
Es wird noch eine kleine Umformung vorgenommen:
$$
(z-z_0)^{\varrho_{1,2}} = e^{log((z-z_0)^{\varrho_{1,2}})} = e^{\varrho_{1,2} log((z-z_0))}
$$                                 
Diese Funtion wird nun analytisch fortgesetzt:
$$
e^{\varrho_{1,2} log((z-z_0))} \, \sk \,	  e^{\varrho_{1,2} (log((z-z_0)) + 2\pi i)}
= e^{\varrho_{1,2} log((z-z_0))}  e^{2\pi i\frac{1}{2\pi i}log(\lambda_{1,2})}
= (z-z_0)^{\varrho_{1,2}}  \lambda_{1,2}
$$
Die Funktion $(z-z_0)^{\varrho_{1,2}}$ wird also beim Umfahren der Singularit"at ebenfalls mit dem Faktor $\lambda_{1,2}$ multipliziert. Nun wird das Verh"altnis $\frac{w_{1,2}}{(z-z_0)^{\varrho_{1,2}}}$ untersucht:
$$
\frac{w_{1,2}}{(z-z_0)^{\varrho_{1,2}}} \, \sk \, \frac{\lambda_{1,2} w_{1,2}}{\lambda_{1,2}(z-z_0)^{\varrho_{1,2}}} = \frac{w_{1,2}}{(z-z_0)^{\varrho_{1,2}}}
$$
Die Funktion $\frac{w_{1,2}}{(z-z_0)^{\varrho_{1,2}}}$ ist also eindeutig und kann deshalb als Laurent-Reihe dargestellt werden.
$$
\frac{w_{1,2}}{(z-z_0)^{\varrho_{1,2}}} = \mathcal{L}_{1,2}(z)
$$
Daraus folgt:
$$
\begin{matrix}
w_1 = (z-z_0)^{\varrho_1} \mathcal{L}_1(z-z_0) \\
w_2 = (z-z_0)^{\varrho_2} \mathcal{L}_2(z-z_0)
\end{matrix}
$$
Ist also die Matrix A diagonalisierbar, k"onnen zwei L"osungen dieser Form gefunden werden.

%Matrix nicht diagonalisierbar
Ist die Matrix A nicht diagonalisierbar, gilt:
$$
\begin{array}{l}
w_1 \, \sk \, \lambda w_1 \\
w_2 \, \sk \, \lambda w_2 + w_1
\end{array}
$$
Nun bilden wir das Verh"altnis $\frac{w_2}{w_1}$
$$
\frac{w_2}{w_1} \, \sk \, \frac{\lambda w_2}{\lambda w_1} = \frac{w_2}{2_1} + \frac{1}{\lambda}
$$
Diesmal muss also ein Summand, kein Faktor kompensiert werden. Gemacht wird dies mit der Funktion
$$
\frac{1}{2\pi i\lambda} log(z-z_0) \, \sk \, \frac{1}{2\pi i\lambda} log(z-z_0)
+  \frac{1}{\lambda}
$$
Nun wird die Differenz dieser beiden Funktionen gebildet:
$$
\frac{w_2}{w_1} - \frac{1}{2\pi i\lambda} log(z-z_0) \, \sk \, \frac{w_2}{w_1} +\frac{1}{\lambda} - (\frac{1}{2\pi i\lambda} log(z-z_0) + \frac{1}{\lambda}) = \frac{w_2}{w_1} - \frac{1}{2\pi i\lambda} log(z-z_0) = \mathcal{L}_2(z-z_0)
$$
Umformen nach $w_2$ ergibt:
$$
w_2 = \mathcal{L}_2(z-z_0) w_1 +\frac{1}{2\pi i\lambda} w_1 log(z-z_0) =  \mathcal{L}_2(z-z_0)\mathcal{L}_1(z-z_0)(z-z_0)^\varrho - a w_1 log(z-z_0) \, \text{mit} \, a=\frac{1}{2\pi i\lambda}
$$
Eine Laurent-Reihe multipliziert mit einer anderen Laurent-Reihe gibt wieder eine Laurent-Reihe. Damit ergibt sich:
$$
\begin{array}{l}
w_1 = (z-z_0)^{\varrho} \mathcal{L}_1(z-z_0) \\
w_2 = (z-z_0)^\varrho\mathcal{L}_3(z-z_0) - a w_1 log(z-z_0)
\end{array}
$$
Ist also die Matrix A nicht diagonalisierbar, findet man eine L"osung wie im diagonalisierbaren Fall und eine zweite L"osung, welche die erste L"osung und einen Logarithmus-Term beinhaltet.