\section{Einleitung}

In diesem Kapitel möchten wir ein Problem aufgreiffen, welches wir bei der Bessel'schen Differentialgleichung
$$ z^2w''+zw'+(z^2 - \nu^2)w=0$$
angetroffen haben. Wir fanden Lösungen der Form 
$$J_\nu(z)=z^\varrho\sum_{k=0}^{\infty}a_kz^k.$$
Die Indexgleichung lieferte $\varrho=\pm\nu.$

Im Falle von ganzzahligen $\nu\in\mathbb{Z}\setminus\{0\}$ erhielten wir zwar zwei Lösungen, allerdings war die zweite Lösung linear abhängig von der ersten $$J_\nu(z) = J_{-\nu}(-z).$$
Im Falle von $\nu=0$ erhielten wir sogar nur eine Lösung (mit $\varrho=0$). Eine Differentialgleichung 2. Ordnung besitzt aber stets genau zwei linear unabhängige Lösungen. Wo also steckt die zweite Lösung?

Bei der Analyse dieses Problems ist es nicht wichtig, welche Differentialgleichung 2. Ordnung betrachtet wird. Die Herleitung gilt für alle DGL 2. Ordung.


\subsection{Benötigte Werkzeuge}
Wir werden in diesem Kapitel verschiedene Werkzeuge verwenden. Insbesondere die analytischen Fortsetzung (siehe \ref{komplex:fortsetzung}) und die Laurent-Reihen
$$\mathcal{L}(z)=\sum_{k=-\infty}^{\infty}a_k(z-z_0)^k.$$

Die analytische Fortsetzung $f^+$ einer Funktion $f$ entlang eines geschlossenen Weges $\gamma$ in positiver Drehrichtung wird in diesem Kapitel sehr oft verwendet. Wir definieren deshalb folgenden Operator:
$$f(z)\sk f^+(z)$$
Das heisst, wenn man $f(z)$ einmal entlang eines geschlossenen Weges in positiver Richtung analytisch fortsetzt, resultiert $f^+(z)$.

\subsection{Analythische Fortsetzung und Singularitäten}
Ist das Gebiet $B$, welches durch $\gamma$ eingeschlossen wird einfach zusammenhängend, oder besitzt es Singularitäten vom Typ $z^{-n}$, so ist diese Funktion überall in diesem Gebiet als endliche Laurent-Reihe darstellbar und eindeutig. Umgekehrt gilt, dass die Analytische Fortsetzung einer Laurent-Reihge entlang eines geschlossenen Weges wieder auf dieselbe Laurent-Reihe führt. Laurent-Reihen sind eindeutig.

Besitzt dieses Gebiet $B$ allerdings eine (oder mehrere) logarithmische Singularitäten, so findet sich einerseits keine endliche Laurent-Reihe, und andererseits ist die Funktion nach der analytischen Fortsetzung nicht mehr eindeutig. Letzteres folgt aus der Mehrdeutigkeit des Logarithmus
$$\log z = \log|z| + \arg z + 2k\pi i\forall k\in\mathbb{Z}.$$
Betrachten wir beispielhaft eine Funktion mit einer solchen Singularität, den Logarithmus selbst, so sehen wir 
$$\log z\sk \log z + 2\pi i.$$
Funktionen, die logarithmische Singularitäten beinhalten, sind also nicht als reine Laurent-Reihen darstellbar.

Laurent-Reihen sind jedoch sowohl nummerisch als auch analytisch von Vorteil, da sie oftmals einfacher zu berechnen und handhaben sind. Es wäre also schön, wenn es eine Darstellung ähnlich den Laurent-Reihen gäbe, die auch mit Funktionen klar kommt, die logarithmische Singularitäten beinhalten.

\subsubsection*{Besselgleichung mit $\nu=0$}
Setzt man in der Besselgleichung $\nu=0$ und dividiert die ganze Gleichung durch $z^2$, so resultiert
$$ w''+\frac{1}{z}w'+w=0.$$
Dieses $\frac{1}{z}$ beim einfach abgeleiteten Term führt in der Lösung zu einer logarithmischen Singularität. ($\int\frac{1}{z}\text{d}z = \log z.$)
Die Besselfunktion mit $\nu=0$ besitzt im Ursprung also genau eine solche Singularität, die in der analytischen Fortsetzung zu neuen Lösungen führen kann.

\subsection{Lineare Algebra}
Wir wissen, dass es zwei linear unabhängige Lösungen geben muss. Wenn diese Lösungen existieren, dann können wir sie auch analytisch fortsetzen und schauen, was passiert.
$$\begin{pmatrix}
w_1 \\ w_2
\end{pmatrix}=w\sk w^+=\begin{pmatrix}
w_1^+\\ w_2^+
\end{pmatrix}
$$
Die analythische Fortsetzung linear unabhängiger Lösungen $w$ führt dabei stets auf neue, linear unabhängige Lösungen $w^+$. Da diese neuen Lösungen untereinander linear unabhängig sind, und eine DGL 2. Ordnung insgesammt nur zwei linear unabhängige Lösungen besitzt, muss eine Abhängigkeit in Form von 
$$ w^+=Aw$$
existieren. Insbesondere ist die Matrix 
$$A = \begin{pmatrix}a_{11} & a_{12} \\ a_{21} & a_{22}\end{pmatrix}$$
regulär und eindeutig für bestimmte $w$ und $w^+$. Es gilt $\det(A)\ne0$.
Die Matrix $A$ beschreibt also, was bei der analytischen Fortsetzung mit den Lösungen passiert.

Aus der linearen Algebra ist bekannt, dass durch geeignete lineare Abbildungen immer eine Basis gefunden werden kann, in welcher $A$ entweder Diagonalform oder Dreiecksform besitzt.
$$A\rightsquigarrow\begin{pmatrix}\lambda_1 & 0 \\ 0 & \lambda_2\end{pmatrix}\vee\begin{pmatrix}\lambda & 0 \\ 1 & \lambda\end{pmatrix}$$
Gilt $\lambda_{1,2}=1$, so ändern sich die Funktionen bei der analytischen Fortsetzung nicht und können als Laurent-Reihen dargestellt werden. Sind $\lambda_{1,2}\ne1$ oder existiert lediglich die Dreiecksform, so ist es etwas komplexer. Als nächstes werden also die beiden interessanten Fälle der diagonalisierbaren und nicht-diagonalisierbaren Matrix betrachtet.



