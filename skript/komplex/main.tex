\chapter{Wo steckt die zweite L"osung?\label{chapter:thema}}
\lhead{Bessel-Funktionen zweiter Art}
\begin{refsection}
\chapterauthor{Stefan Kull und Roy Seitz}

%Matrix Diagonalisierbar
Is die Matrix A diagonalisierbar, gilt:
$$
\begin{matrix}
w_1 @ \lambda_1 w_1  \\
w_2 @ \lambda_2 w_2
\end{matrix}
$$
Der Faktor $\lambda_{1,2}$ muss kompensiert werden, damit die L"osung als Laurent-Reihe dargestellt werden kann. Dazu wird folgende Funktion verwendet: 
$$
\begin{matrix}
(z-z_0)^{\varrho_{1,2}} & | \varrho_{1,2} = \frac{1}{2\pi i}log(\lambda_{1,2})
\end{matrix}
$$
Es wird noch eine kleine Umformung vorgenommen:
$$
(z-z_0)^{\varrho_{1,2}} = e^{log((z-z_0)^{\varrho_{1,2}})} = e^{\varrho_{1,2} log((z-z_0))}
$$
Diese Funtion wird nun analytisch Fortgesetzt:
$$
e^{\varrho_{1,2} log((z-z_0))}  \text{@}  e^{\varrho_{1,2} (log((z-z_0)) + 2\pi i)}
= e^{\varrho_{1,2} log((z-z_0))}  e^{2\pi i\frac{1}{2\pi i}log(\lambda_{1,2})}
= (z-z_0)^{\varrho_{1,2}}  \lambda_{1,2}
$$
Die Funktion $(z-z_0)^{\varrho_{1,2}}$ wird also beim Umfahren der Singularit"at ebenfalls mit dem Faktor $\lambda_{1,2}$ multipliziert. Nun wird das Verh"altnis $\frac{w_{1,2}}{(z-z_0)^{\varrho_{1,2}}}$ untersucht:
$$
\frac{w_{1,2}}{(z-z_0)^{\varrho_{1,2}}} @ \frac{\lambda_{1,2} w_{1,2}}{\lambda_{1,2}(z-z_0)^{\varrho_{1,2}}} = \frac{w_{1,2}}{(z-z_0)^{\varrho_{1,2}}}
$$
Die Funktion $\frac{w_{1,2}}{(z-z_0)^{\varrho_{1,2}}}$ ist also eindeutig und kann deshalb als Laurent-Reihe dargestellt werden.
$$
\frac{w_{1,2}}{(z-z_0)^{\varrho_{1,2}}} = \mathcal{L}_{1,2}
$$
Daraus folgt:
$$
\begin{matrix}
w_1 = (z-z_0)^{\varrho_1} \mathcal{L}_1 \\
w_2 = (z-z_0)^{\varrho_2} \mathcal{L}_2
\end{matrix}
$$
Ist also die Matrix A diagonalisierbar, k"onnen zwei Lösungen dieser Form gefunden werden.

%Matrix nicht diagonalisierbar

\printbibliography[heading=subbibliography]
\end{refsection}

