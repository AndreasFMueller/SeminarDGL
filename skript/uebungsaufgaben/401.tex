L"osen Sie die Airy-Differentialgleichung
\[
y''-xy=0
\]
mit den Anfangsbedingung $y(0)=1$ und $y'(0)=0$.

\begin{loesung}
Wir verwenden einen Potenzreihenansatz
\[
y(x)=\sum_{k=0}^\infty a_kx^k,
\]
aus der Anfangsbedingung k"onnen wir bereits die Werte von $a_0$ und
$a_1$ ablesen:
\begin{align*}
a_0&=y(0)&
a_1&=y'(0).
\end{align*}
Wir setzen die Funktion und ihre Ableitungen
\begin{align*}
y(x)&=a_0+a_1x+a_2x^2+a_3x^3+\dots+a_kx^k+\dots
\\
y'(x)&=a_1+2a_2x+3a_3x^2+4a_4x^3+\dots+ka_kx^{k-1}+\dots
\\
y''(x)&=2a_2+2\cdot 3a_3x + 3\cdot 4d_4x^2+\dots +k(k-1)a_kx^{k-2}+\dots
\end{align*}
in die Differentialgleichung ein.
Damit der Koeffizientenvergleich effizient von statten gehen kann,
m"ussen wir die rechten Seiten so schreiben, dass die gleichen Potenzen
von $x$ auftreten:
\begin{align*}
y''(x)
&=
\sum_{k=0}^\infty (k+2)(k+1)a_{k+2}x^k
\\
xy(x)
&=
\sum_{k=1}^\infty a_{k-1}x^k
\end{align*}
Den Koeffizienten $k=0$ werden wir gesondert behandeln m"ussen.
Eingesetzt in die Differentialgleichung erhalten wir
\[
y''(x)-xy(x)
=
a_2 + \sum_{k=1}^\infty \bigl((k+2)(k+1)a_{k+2}-a_{k-1}\bigr)x^k=
\]
Alle Koeffizienten m"ussen verschwinden, insbesondere folgt $a_2=0$,
die ersten drei Koeffizienten sind damit vollst"andig bestimmt.

Die h"oheren Koeffizienten erf"ullen die Gleichung
\[
a_{k+3}=\frac{a_k}{(k+3)(k+2)}.
\]
Insbesonderen bestimmt $a_0$ alle Koeffizienten, deren Index durch
drei teilbar ist, und $a_1$ bestimmt alle Koeffizienten, deren Index
Rest $1$ hat bei Teilung durch drei.
Wegen $a_2$ verschwinden alle "ubrigen Koeffizienten.

F"ur die gegebene Anfangsbedingung ist $a_1=0$, und damit kommen in der
resultierenden Reihe nur Term in $x^3$ vor.
Die Koeffizienten sind
\begin{align*}
a_3&=\frac1{3\cdot 2}a_0=\frac1{3\cdot 2}=\frac16\\
a_6&=\frac1{6\cdot 5}a_3=\frac1{6\cdot 5\cdot 3\cdot 2}=\frac1{180}\\
a_9&=\frac1{9\cdot 8}a_6=\frac1{9\cdot 8\cdot 6\cdot 5\cdot 3\cdot 2}=\frac1{12960}\\
a_{12}&=\frac1{12\cdot 11}a_9=\frac1{12\cdot 11\cdot 9\cdot 8\cdot 6\cdot 5\cdot 3\cdot 2}=\frac1{1710720}
\\
&\dots
\end{align*}
Daraus kann man jetzt die Potenzreihe ableiten:
\begin{align*}
y(x)
&=
\sum_{k=0}^\infty \frac{x^{3k}}{3k\cdot(3k-1)\cdot (3k-3)\cdot(3k-4)\cdot \dots 3\cdot 2}
\\
&=
1 + \frac16x^3+\frac1{180}x^6+\frac1{12960}x^9+\frac1{1710720}x^{12}+\dots
\qedhere
\end{align*}
\end{loesung}

