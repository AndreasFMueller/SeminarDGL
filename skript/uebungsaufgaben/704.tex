Die Potenzreihe
\[
f(z)
=
2\biggl(z+\frac{1}{3}z^3+\frac{1}{5}z^5+\frac{1}{7}z^7+\dots\biggr)
\]
ist f"ur $|z|<1$ konvergent.
Finden Sie einen einfachen Ausdruck f"ur $f(z)$.

\begin{hinweis}
Leiten Sie die Funktion ab, und versuchen Sie $f(z)$ mit Hilfe der
geometrischen Reihe als Bruch zu schreiben.
\end{hinweis}

\begin{loesung}
Beim Ableiten entsteht die Reihe
\[
f'(z) = 2(1+z^2+z^4+z^6+\dots)
\]
In der Klammer rechts steht die geometrische Reihe mit Quotient $q=z^2$,
also ist die Summe der Reihe auch
\begin{equation}
f'(z)=2\frac{1}{1-z^2}.
\label{704:fprime}
\end{equation}
Um $f$ zu bestimmen muss man jetzt davon eine Stammfunktion finden.
Dazu findet man erst die Partialbruchzerlegung von von (\ref{704:fprime})
\begin{align*}
f'(z)
&=
\frac{2}{(1+z)(1-z)}
=
\frac1{1+z}+\frac1{1-z}
\end{align*}
Die einzelnen Summanden haben je einen Logarithmus als Stammfunktion:
\[
f(z) = \log(1+z)-\log(1-z)=\log\biggl(\frac{1+z}{1-z}\biggr).
\qedhere
\]
\end{loesung}

