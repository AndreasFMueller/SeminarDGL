Betrachten Sie die in der ganzen komplexen Ebene $\mathbb C$ definierte
komplexe Funktion
\[
f(z)=\bar z=x-iy.
\]
\begin{teilaufgaben}
\item
Verwenden Sie die Cauchy-Riemann-Gleichungen um zu zeigen, dass diese
Funktion nicht komplex differenzierbar ist.
\item
Berechnen Sie die Wegintegrale
\[
\int_{\gamma_1}f(z)\,dz
\qquad\text{und}\qquad
\int_{\gamma_2}f(z)\,dz
\]
f"ur die zwei Wege
\[
\begin{aligned}
\gamma_1(t)&=1-t    &&\text{f"ur $t\in[0,2]$ und }\\
\gamma_2(t)&=e^{it} &&\text{f"ur $t\in[0,\pi]$.}
\end{aligned}
\]
\end{teilaufgaben}

\begin{loesung}
\begin{teilaufgaben}
\item
Es ist $u=x$ und $v=y$ mit den Ableitungen
\[
\begin{aligned}
\frac{\partial u}{\partial x}&=1,&
\frac{\partial v}{\partial x}&=0,\\
\frac{\partial u}{\partial y}&=0&
\frac{\partial v}{\partial y}&=-1
\end{aligned}
\]
Die Cauchy-Riemann-Gleichungen verlangen, dass
\[
\begin{aligned}
\frac{\partial u}{\partial x}&=\frac{\partial v}{\partial y}
&&\text{und}&
\frac{\partial u}{\partial y}&=-\frac{\partial v}{\partial x}
\\
\Rightarrow\qquad
1&=-1&&&0&=0
\end{aligned}
\]
Die erste Gleichung ist offenbar nicht erf"ullt, also kann die Funktion
nicht komplex differenzierbar sein.
\item
Die beiden Wegintegrale werden mit der Formel
\[
\int\gamma f(z)\,dz = \int_a^b f(\gamma(t))\cdot \dot\gamma(t)\,dt
\]
berechnet:
\begin{align*}
\int_{\gamma_1}f(z) \,dz
&=
\int_0^2 f(1-t)\cdot (-1)\,dt
=
\int_0^2 t-1\,dt
=
\biggl[\frac12t^2-t\biggl]_0^2
=
2-2=0
\\
\int_{\gamma_2}f(z) \,dz
&=
\int_0^\pi f(e^{it})\cdot ie^{it}\,dt
=
\int_0^\pi e^{-it}\cdot ie^{it}\,dt
=
\int_0^\pi \,dt
=
\pi
\end{align*}
Im Gegensatz zu einer komplex differenzierbaren Funktion ist also das
Wegintegral einer nicht komplex differenzierbaren Funktion nicht
unabh"angig vom Weg.
\qedhere
\end{teilaufgaben}
\end{loesung}

