Man m"ochte wissen, ob die Differentialgleichung
\[
zw'(z)+(az+b)w(z)=0
\]
mit $b\ne 0$ eine L"osung $w(z)$, die an der Stelle $z=0$
nicht singul"ar ist.

\begin{hinweis}
L"osen Sie nach $w'(z)$ auf und berechnen Sie das Wegintegral
entlang eines Kreises um den Nullpunkt.
\end{hinweis}

Pr"ufen Sie, ob die Funktion
\[
w(z) = z^{-b}e^{-az}
\]
eine L"osung der Differentialgleichung
\[
zw'(z)+(az+b)w(z)=0
\]
ist.

\begin{loesung}
Wir nehmen an, $w(z)$ sei eine in ganz $\mathbb C$ definierte L"osung.
L"ost man die Differentialgleichung nach $w'(z)$ auf, erh"alt man
\begin{equation}
w'(z) = -\biggl(a+\frac{b}z\biggr)w(z)
\label{705:eq}
\end{equation}
f"ur $z\ne 0$.
Nun integriert man "uber einen geschlossenen Pfad $\gamma$ um den 
Nullpunkt.
Das Integral auf der linken Seite verschwindet, weil der Pfad geschlossen
ist.
Das Integral auf der rechten Seite ist 
\[
-\oint_\gamma 
\biggl(a+\frac{b}z\biggr)w(z)\,dz
=
-a\oint_\gamma w(z)\,wz
-
b\oint_\gamma \frac{w(z)}z\,dz
\]
Das erste Integral auf der rechten Seite verschwindet, wenn $w(z)$ eine
in ganz $\mathbb C$ definierte komplex differenzierbare Funktion ist.
Das zweite Integral auf der rechten Seite kann mit der Cauchy-Integralformel
ausgerechnet werden:
\[
\oint_\gamma \frac{w(z)}z\,dz
=
2\pi i w(0)
\]
Das Integral der Gleichung~(\ref{705:eq}) ist daher
\[
0=2\pi i b w(0)
\]
Damit sie erf"ullt ist, muss $b=0$ sein oder $w(0)=0$.

Im ersten Fall, $b=0$, wird die Differentialgleichung zu
\[
zw'(z)+azw(z)=0
\qquad\Rightarrow\qquad
w'(z)+aw(z)=0,
\]
eine lineare Differentialgleichung mit konstanten Koeffizienten, von der
bekannt ist, dass Sie eine in ganz $\mathbb C$ definierte L"osung hat.

Im zweiten Fall, $w(0)=0$, k"onnen wir in der Gleichung~(\ref{705:eq}) 
den Grenz"ubergang $z\to 0$ durchf"uhren, bei dem wir auf der
linken Seite $w'(0)$ erhalten.
Auf der rechten Seite erhalten wir aber
\[
 \lim_{z\to 0}\frac{w(z)}{z}=w'(0),
\]
mithin die Gleichung
\[
w'(0)=-aw(0)-bw'(0)=-bw'(0).
\]
Diese kann nur erf"ullt sein, wenn $b=-1$ ist, oder wenn $w'(0)=0$.
Im Fall $w'(0)=0$ muss die Potenzreihe von $w(z)$ mit dem quadratischen
Term beginnen.
\end{loesung}

